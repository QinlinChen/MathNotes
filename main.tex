\documentclass[hyperref,a4paper,UTF8]{ctexbook}

\usepackage{amsmath}
\usepackage{amsfonts}
\usepackage{amssymb}
\usepackage{amsthm}
\usepackage{textcomp}
\usepackage{makeidx}
\makeindex

\theoremstyle{definition} \newtheorem{definition}{定义}[chapter]
\theoremstyle{definition} \newtheorem{theorem}[definition]{定理}
\theoremstyle{definition} \newtheorem{lemma}[definition]{引理}
\theoremstyle{definition} \newtheorem{corollary}[definition]{推论}
\theoremstyle{remark} \newtheorem*{remark}{Remark}

% math field begin
\newcommand{\mf}[1]{\mathbb{#1}}
\newcommand{\mfN}{\mf{N}}
\newcommand{\mfZ}{\mf{Z}}
\newcommand{\mfR}{\mf{R}}
\newcommand{\mfQ}{\mf{Q}}
\newcommand{\mfC}{\mf{C}}
\newcommand{\mfF}{\mf{F}}
% math field end

% math matrix begin
\newcommand{\mm}[1]{\mathbf{#1}}
\newcommand{\mmA}{\mm{A}}
\newcommand{\mmB}{\mm{B}}
\newcommand{\mmC}{\mm{C}}
\newcommand{\mmE}{\mm{E}}
\newcommand{\mmI}{\mm{I}}
\newcommand{\mmP}{\mm{P}}
\newcommand{\mmQ}{\mm{Q}}
\newcommand{\mmZero}{\mm{0}}
\newcommand{\mmat}[2]{\left(\begin{array}{#1} #2 \end{array}\right)}
\newcommand{\mdet}[2]{\left|\begin{array}{#1} #2 \end{array}\right|}
\newcommand{\meqs}[2]{\left\{\begin{array}{#1} #2 \end{array}\right.}
% three elementry row transformations and matrices
\newcommand{\mTfRowSwi}[2]{#1 \leftrightarrow #2}
\newcommand{\mTfRowMul}[2]{#1(#2)}
\newcommand{\mTfRowAdd}[3]{#1 #2 (#3)}
\newcommand{\mmRowSwi}[2]{\mmP(\mTfRowSwi{#1}{#2})}
\newcommand{\mmRowMul}[2]{\mmP(\mTfRowMul{#1}{#2})}
\newcommand{\mmRowAdd}[3]{\mmP(\mTfRowAdd{#1}{#2}{#3})}
% Basis matrix %
\newcommand{\mmBasis}[1]{\mmB_{#1}}
% math matrix end

% math vector begin
\newcommand{\mv}[1]{\mathbf{#1}}
\newcommand{\mva}{\mv{a}}
\newcommand{\mvb}{\mv{b}}
\newcommand{\mvk}{\mv{k}}
\newcommand{\mvx}{\mv{x}}
\newcommand{\mvy}{\mv{y}}
\newcommand{\mvZero}{\mv{0}}
% math vector end

% math function begin
\newcommand{\mfuncname}[1]{\textrm{#1}}
\newcommand{\mdiag}{\mfuncname{diag}}
\newcommand{\mdim}{\mfuncname{dim}}
\newcommand{\mspan}{\mfuncname{span}}
\newcommand{\mim}{\mfuncname{im}}
\newcommand{\mker}{\mfuncname{ker}}
\newcommand{\mtr}{\mfuncname{tr}}
\newcommand{\mconj}[1]{\overline{#1}}
\newcommand{\mT}{\mfuncname{T}}
% math function end

\begin{document}
\author{陈钦霖}
\title{数学笔记}
\date{\today}

\maketitle
\tableofcontents


\part{微积分}
\chapter{分析基础}

\section{基础概念}
\subsection{邻域}
对于实数$a\in\mfR$,定义它的邻域为
$N(a,\delta) = \{ x: |x-a| < \delta \}$,
其中,实数$\delta > 0$。

同样,可以定义$a$的左邻域为
$N(a,\delta)_+ = \{ x: 0 \le x - a < \delta \}$,
右邻域为
$N(a,\delta)_- = \{ x: -\delta < x - a \le 0  \}$。

\subsection{常用等式与不等式}
\begin{enumerate}
  \item 
  $1^2+2^2+\dots +n^2 = n(n+1)(2n+1)/6$
  \item
  $(\sum_{i}a_i b_i)^2 \le (\sum_{i}a_i^2)(\sum_{i}b_i)^2$
  \item
  $\sin x < x < \tan x \quad (0 < x < \pi/2)$
  \item
  若$x_1,\dots,x_n$符号相同且都大于$-1$,那么
  \[(1+x_1)(1+x_2)\dots(1+x_n)\ge 1+x_1+x_2+\dots+x_n\]
\end{enumerate}

\section{极限的概念}
\chapter{一元函数的微分学}

\section{导数与微分的概念}
本小节给出导数与微分的定义。
并且介绍导数与连续的关系、导数与微分的关系。

\subsection{导数的概念}
\begin{definition}[导数]
  设函数$f$在点$a$的某一邻域内有定义。
  若极限
  \begin{displaymath}
    \lim_{\Delta x\to 0}\frac{f(a+\Delta x)-f(a)}{\Delta x}
  \end{displaymath}
  存在有限,则称函数$f$在点$a$\textbf{可导},
  并称此极限值为$f$在点$a$的\textbf{导数},记为
  \begin{displaymath}
    f'(a)\quad\textrm{或}\quad \left.\frac{\md f}{\md x}\right|_a
  \end{displaymath}
\end{definition}

\begin{definition}[左导数和右导数]
  极限
  \begin{displaymath}
    \lim_{\Delta x\to 0-}\frac{f(a+\Delta x)-f(a)}{\Delta x}
    \quad\textrm{以及}\quad
    \lim_{\Delta x\to 0+}\frac{f(a+\Delta x)-f(a)}{\Delta x}
  \end{displaymath}
  被称为函数$f$在点$a$的\textbf{左导数}和\textbf{右导数}
\end{definition}
\begin{remark}
  $f$在点$a$可导$\iff$ $f$在点$a$的左导数和右导数存在且相等。
\end{remark}

\begin{theorem}[可导与连续的关系]
  若函数$f$在点$a$可导,则$f$在点$a$连续。
\end{theorem}
\begin{remark}
  反之不成立。即在点$a$连续的函数未必在点$a$可导。
\end{remark}

\subsection{微分的概念}
\begin{definition}[微分]
  设函数$f$在区间$I$上有定义。
  对点$a\in I$,当自变量有增量$\Delta x$时,
  相应地,因变量$y$有增量$\Delta y=f(a+\Delta x)-f(a)$。
  若当$\Delta x\to 0$时,有
  \begin{displaymath}
    \Delta y = A\Delta x +o(\Delta x)
  \end{displaymath}
  其中$A$与$\Delta x$无关(一般与$a$有关),
  则称函数$f$在点$a$\textbf{可微},
  且称$A\Delta x$为$f$在点$a$的\textbf{微分},记作
  \begin{displaymath}
    \md f(a)\quad\textrm{或}\quad \left.\md y\right|_a
  \end{displaymath}
\end{definition}

\begin{remark}
  当$\Delta x\to 0$时,无穷小$\md y$是$\Delta y$的主部。
  又因为$\md y$关于$\Delta x$是一次的,
  故称$\md y$是$\Delta y$的\textbf{线性主部}。
\end{remark}

\begin{theorem}[可导与可微等价]
  函数$f$在点$a$可微的充要条件是
  $f$在点$a$存在有限导数$f'(a)$,
  于是就有
  \begin{displaymath}
    \md y = f'(a)\Delta x
  \end{displaymath}
\end{theorem}

\begin{remark}
  我们约定自变量$x$的增量$\Delta x$为自变量的微分,
  即$\md x = \Delta x$,
  于是上面的微分可以写成
  \begin{displaymath}
    \md y = f'(a)\md x
  \end{displaymath}
  需要注意的是,
  $\frac{\md y}{\md x} = f'(x)$中的左半部分应当视为一个整体,
  而不是一个分子除以分母的形式。
  只是因为上述定理的存在,
  才使得看起来只要把等式左边的分母移到右边就能得到$f$的微分形式。
  事实上,\textbf{该定理只对一元函数成立}。
\end{remark}

\section{导数与微分的计算}
本小节首先说明了导数和微分的四则运算法则。
然后讨论了复合函数的求导法则以及一阶微分形式的不变性。
接着讨论了反函数导数的求法。
最后,我们列出一些常用且容易遗忘的函数的导数。

\subsection{四则运算}
\begin{theorem}[导数的四则运算]
  设函数$u,v$在点$x$可导,那么
  \begin{enumerate}
    \item 
    $y=u\pm v$在点$x$可导,且有
    \begin{displaymath}
      (u\pm v)'=u'\pm v'
    \end{displaymath}
    \item 
    $y=uv$在点$x$可导,且有
    \begin{displaymath}
      (uv)' = u'v+uv'
    \end{displaymath}
    \item 
    当$v(x)\neq 0$时,$y=u/v$在点$x$可导,且有
    \begin{displaymath}
      \left(\frac{u}{v}\right)'=\frac{u'v-uv'}{v^2}
    \end{displaymath}
  \end{enumerate}
\end{theorem}

\begin{theorem}[微分的四则运算]
  设函数$u,v$在点$x$可微,那么
  \begin{enumerate}
    \item 
    $\md (u\pm v) = \md u \pm \md v$
    \item 
    $\md (uv) = v\md u + u \md v$
    \item 
    $\md (\frac{u}{v}) = \frac{v\md u - u\md v}{v^2}$
  \end{enumerate}
\end{theorem}

\subsection{复合函数的导数}
\begin{theorem}[复合函数的导数]
  设函数$y=f(u)$与$u=u(x)$在$x_0$的邻域能构成复合函数,
  且$u=u(x)$在点$x_0$可导,$y=f(x)$在点$u_0=u(x_0)$可导。
  那么,复合函数$f\circ u$在点$x_0$可导,且有
  \begin{displaymath}
    (f\circ u)'(x_0) = f'(u_0)u'(x_0)
  \end{displaymath}
\end{theorem}

\begin{corollary}[一阶微分形式不变性]
  设$y=f(u)$,则无论$u$是自变量还是中间变量$u=u(x)$,
  其微分形式不变,都是
  \begin{displaymath}
    \md y = f'(u)\md u
  \end{displaymath}
\end{corollary}

\begin{remark}
  该定理对高阶微分不成立:若$y=f(x)$,但$x$是中间变量,
  即$x=u(t)$,那么有
  \begin{align*}
    \md^2 y
    &= \md (f'(x)\md x) \\
    &= f''(x)\md x^2 + f'(x)\md^2 x
      & (\text{不是$f''(x)\md x^2$}) \\
    &= f''(x)\left(u'(t)\md t\right)^2 + f'(x)(u''(t)\md t^2)\\
    &= \left(f''(x)u'(t)^2 + f'(x)u''(t)\right)\md t^2
  \end{align*}
\end{remark}

\subsection{反函数的导数}
\begin{theorem}[反函数的导数]
  设函数$x=\phi(y)$在某一区间$I$内严格单调,
  有在区间$I$内一点$y$出导数$\phi(y)$存在且不为零,
  则反函数$y=f(x)$在对应点$x$出具有导数$f'(x)$,且
  \begin{displaymath}
    f'(x) = \frac{1}{\phi'(y)}
  \end{displaymath}
\end{theorem}

\subsection{公式表}
下面是一些常用且容易遗忘的函数的导数:
\begin{center}
  \begin{tabular}{|c|c|}
    \hline 
    $f(x)$ & $f'(x)$ \\ 
    \hline 
    $a^x$ & $a^x\ln a$  \\ 
    \hline 
    $\log_a |x|$ & $\frac{1}{x\ln a}$ \\ 
    \hline 
    $\arcsin x$ & $\frac{1}{\sqrt{1-x^2}}$ \\ 
    \hline 
    $\arccos x$ & $-\frac{1}{\sqrt{1-x^2}}$ \\ 
    \hline 
    $\arctan x$ & $\frac{1}{1+x^2}$ \\ 
    \hline 
  \end{tabular} 
\end{center}

\section{高阶导数与高阶微分}

\section{微分中值定理}

\section{洛必达法则}

\section{泰勒公式}

\section{利用导数研究函数的性质}

\section{利用导数作函数的图形}


\part{线性代数}
\chapter{矩阵}

\section{矩阵及其运算}
这一小节主要介绍了矩阵的基本概念和基本运算。

矩阵的概念对学过线性代数的人来说是稀松平常的了,
所以这里主要给出了重要概念和特殊矩阵的定义和符号,
方便后面的讨论。

矩阵的基本运算涉及了线性运算、乘积、转置。
大家应该对此都很熟悉,这里就简单给出了一些性质,并不作证明。
此外,我们还由转置运算给出了对称矩阵、反对称矩阵的概念。

\subsection{矩阵的概念}
\begin{definition}[矩阵]
  由$m\times n$个数排成$m$行$n$列的矩形数表
  \begin{equation} \label{eq:mat-def}
    \mmA = \mmat{cccc}{
      a_{11} & a_{12} & \cdots & a_{1n} \\
      a_{21} & a_{22} & \cdots & a_{2n} \\
      \vdots & \vdots & \ddots & \vdots \\
      a_{m1} & a_{m2} & \cdots & a_{mn} }
  \end{equation}
  称为一个$m\times n$矩阵。
  其中数$a_{ij}$称为矩阵$\mmA$的元素。
  $i$为行标,$j$为列标,
  $a_{ij}$是$\mmA$位于第$i$行第$j$列的元素,或简称$\mmA$的$(i,j)$元素。
  式\eqref{eq:mat-def}可简记为$\mmA=(a_{ij})_{m\times n}$或$(a_{ij})$。
\end{definition}

下面是一些特殊的矩阵:
\begin{itemize}
  \item
  $m\times 1$矩阵又称为\textbf{列矩阵}或$m$\textbf{维列向量}。
  \item
  $1\times n$矩阵又称为\textbf{行矩阵}或$n$\textbf{维行向量}。
  \item
  $n\times n$矩阵又称为$n$\textbf{阶方阵}。
  对于$n$阶方阵$\mmA=(a_{ij})$,
  其对角线上的元素$a_{11},\dots,a_{nn}$又称\textbf{主对角线元素}。
  \item
  如果一个$n$阶方阵除主对角线元素外,其余元素都为0,
  那么我们称这种矩阵为\textbf{对角矩阵},记为
  \[
    \mdiag(k_1, k_2\dots,k_n)= \mmat{cccc}{
      k_1 &     &        & \\
          & k_2 &        & \\
          &     & \ddots & \\
          &     &        & k_n }
  \]
  \item
  $\mmI = \mdiag(1,1,\dots,1)$被称为\textbf{单位矩阵}。
  \item
  \textbf{零矩阵}$\mmZero$是所有元素都为0的矩阵。
  \item
  设$\mmA=(a_{ij})$是$n$阶方阵。
  若当$i>j$时,$a_{ij}=0$,则称$\mmA$为\textbf{上三角阵}。
  若当$i<j$时,$a_{ij}=0$,则称$\mmA$为\textbf{下三角阵}。
  两者统称\textbf{三角阵}。
\end{itemize}

\begin{definition}[矩阵相等]
  如果$\mmA=(a_{ij})$与$\mmB=(b_{ij})$都是$m\times n$矩阵,
  且对$i=1,2,\dots,m,\ j=1,2,\dots,n$,有$a_{ij}=b_{ij}$,
  则称$\mmA$与$\mmB$相等,记作$\mmA=\mmB$。
\end{definition}

\subsection{矩阵的线性运算}
矩阵的线性运算包含\textbf{加法}和\textbf{数乘}。定义比较简单,在此不加赘述。

根据后面线性空间的定义\ref{def:linear-space},
矩阵空间也是线性空间。

\subsection{矩阵的乘法}
\begin{definition}[矩阵的乘法]
  设$\mmA=(a_{ik})_{m\times s}, \mmB=(b_{kj})_{s\times n}$,令
  \begin{equation}
  c_{ij}=\sum_{k=1}^{s}a_{ik}b_{kj}
  \end{equation}
  则$\mmC=(c_{ij})_{m\times n}$被称为$\mmA$和$\mmB$的乘积,
  记为$\mmC = \mmA\mmB$。
\end{definition}

矩阵的乘法满足分配率和结合律,但不满足交换律。

\subsection{矩阵的转置}
\begin{definition}[矩阵的转置]
  设$\mmA=(a_{ij})_{m\times n}$,
  令$b_{ij}=a_{ji}$,
  则$\mmB=(b_{ij})_{n\times m}$被称为$\mmA$的转置矩阵,
  记为$\mmA^\mT$。
\end{definition}

\begin{theorem}[转置矩阵的性质]
  转置矩阵具有以下性质:
  \begin{enumerate}
    \item $\left(\mmA^\mT\right)^\mT = \mmA$
    \item $(\mmA+\mmB)^\mT=\mmA^\mT + \mmB^\mT$
    \item $(k\mmA)^\mT=k\mmA^\mT$
    \item $(\mmA\mmB)^\mT=\mmB^\mT\mmA^\mT$
  \end{enumerate}
\end{theorem}

\begin{definition}[对称矩阵与反对称矩阵]
  对于$n$阶方阵$\mmA=(a_{ij})$,
  如果满足$\mmA^\mT=\mmA$,则称$\mmA$为\textbf{对称矩阵}。
  如果满足$\mmA^\mT=-\mmA$,则称$\mmA$为\textbf{反对称矩阵}。
\end{definition}

\begin{remark}
  不难看出,反对称矩阵的主对角线元素都为零。
\end{remark}

\section{矩阵的分块}
矩阵的分块是化简矩阵运算简单而又重要的思想。
其中最重要的是分块矩阵的乘法。

\subsection{分块矩阵的概念}
\begin{definition}[分块矩阵]
  一般地,将一个$m\times n$矩阵$\mmA$用横线划分成$r$块,
  用竖线划分成$s$块,就能得到一个$r\times s$分块矩阵。
  \begin{equation} \label{eq:mat-partition}
  \mmA = \mmat{cccc}{
    \mmA_{11} & \mmA_{12} & \cdots & \mmA_{1s} \\
    \mmA_{21} & \mmA_{22} & \cdots & \mmA_{2s} \\
    \vdots    & \vdots    & \ddots & \vdots    \\
    \mmA_{r1} & \mmA_{r2} & \cdots & \mmA_{rs} }
    = (\mmA_{ij})_{r\times s}
  \end{equation}
  其中,$\mmA_{ij}(i=1,\dots,r,\ j=1,\dots,s)$是$m_i\times n_j$矩阵,
  $\sum_{i=1}^{r}m_i = m$,$\sum_{j=1}^{s}n_j=n$。
\end{definition}

最常用的一种矩阵分块方法是把$m\times n$划分成$m$个行向量,
或者$n$个列向量。

\subsection{分块矩阵的运算}
分块矩阵的运算中需要注意的是转置和乘法。

转置运算不仅需要把矩阵的每一行转成列,而且内部子块也需要转置。
对于式\eqref{eq:mat-partition}中的矩阵$\mmA$,它的转置是
\[
  \mmA^\mT = \mmat{cccc}{
    \mmA_{11}^\mT & \mmA_{21}^\mT & \cdots & \mmA_{r1}^\mT \\
    \mmA_{12}^\mT & \mmA_{22}^\mT & \cdots & \mmA_{r2}^\mT \\
    \vdots        & \vdots        & \ddots & \vdots        \\
    \mmA_{1s}^\mT & \mmA_{2s}^\mT & \cdots & \mmA_{sr}^\mT }
\]

对于乘法运算,
给定两个矩阵$\mmA=(a_{ij})_{m\times n}$,$\mmB=(b_{jk})_{n\times p}$,
只要使$\mmB$的行分法与$\mmA$的列分法一致,
就能把子块当作数一样按照矩阵乘法的规则进行计算。

\subsection{准对角矩阵}
\begin{definition}[准对角矩阵]
  对于$n$阶方阵$\mmA$,
  如果有一种分法,使$\mmA$的主对角线以外的子块都是零矩阵,
  且主对角线上子块都是方阵,则称$\mmA$为准对角矩阵。
\end{definition}

\begin{remark}
  当然,准对角矩阵包含对角矩阵作为特殊情况。
\end{remark}

设$\mmA$和$\mmB$都是$n$阶方阵。
如果有相同的分法,使得
\[
\mmA = \mmat{cccc}{
  \mmA_1 &        &        & \\
         & \mmA_2 &        & \\
         &        & \ddots & \\
         &        &        & \mmA_n },\quad
\mmB = \mmat{cccc}{
  \mmB_1 &        &        & \\
         & \mmB_2 &        & \\
         &        & \ddots & \\
         &        &        & \mmB_n }
\]
都是准对角矩阵,那么显然有
\[
\mmA\mmB = \mmat{cccc}{
    \mmA_1\mmB_1 &              &        & \\
                 & \mmA_2\mmB_2 &        & \\
                 &              & \ddots & \\
                 &              &        & \mmA_n\mmB_n }
\]

\section{逆矩阵与初等变换}
本小节开始讨论逆矩阵,
并试图通过初等变换的概念,从更本质的视角来看待矩阵
——矩阵就是所谓标准型乘上有限个初等阵。
也因为初等阵具有良好的性质——可逆性,
所以矩阵可逆当且仅当它的标准型就是单位矩阵。

\subsection{逆矩阵}
\begin{definition}[逆矩阵]
  设$\mmA$是$n$阶方阵。如果存在$n$阶方阵$\mmB$,使
  \begin{equation}
    AB=BA=I
  \end{equation}
  则称$\mmB$为$\mmA$的\textbf{逆矩阵},记作$A^{-1}$,
  并说$\mmA$是\textbf{非奇异矩阵}或\textbf{可逆矩阵},
  否则便称$\mmA$是\textbf{奇异矩阵}或\textbf{不可逆矩阵}。
\end{definition}

\begin{theorem}[逆矩阵的唯一性]
  可逆矩阵的逆矩阵是唯一的。
\end{theorem}

\begin{theorem}[可逆矩阵的性质]
  若$\mmA,\mmB$是可逆矩阵,
  则$\mmA^{-1}$,$k\mmA\,(k\neq 0)$,$\mmA^\mT$,$\mmA\mmB$都是可逆矩阵,且
  \begin{enumerate}
    \item $(\mmA^{-1})^{-1} = \mmA$
    \item $(k\mmA)^{-1} = \frac{1}{k}\mmA^{-1}$
    \item $(\mmA^\mT)^{-1} = (\mmA^{-1})^\mT$
    \item $(\mmA\mmB)^{-1} = \mmB^{-1}\mmA^{-1}$
  \end{enumerate}
\end{theorem}

\subsection{初等变换与初等阵}
矩阵的\textbf{初等变换}可以分为\textbf{初等行变换}和\textbf{初等列变换}。
下面主要讨论初等行变换。初等列变换与之相似。

\begin{definition}[初等行变换]
  初等行变换有三种:
  \begin{enumerate}
    \item 交换矩阵的第$i$行和第$j$行,记为$\mTfRowSwi{i}{j}$。
    \item 将矩阵的第$i$行乘非零常数$k$,记为$\mTfRowMul{k}{j}$。
    \item 把矩阵的第$i$行加上第$j$行的$k$倍,记为$\mTfRowAdd{i}{+k}{j}$。
  \end{enumerate}
\end{definition}

\begin{definition}[初等阵]
  单位矩阵经过一次初等变换得到的矩阵叫做\textbf{初等矩阵},
  或简称\textbf{初等阵}。
\end{definition}

\begin{remark}
  显然,初等阵是方阵,并且都是可逆的。
\end{remark}

使用三种初等行变换能得到三个初等阵,
我们把它们分别记作
$\mmRowSwi{i}{j}$,$\mmRowMul{k}{i}$和$\mmRowAdd{i}{+k}{j}$。
对单位矩阵实行一次初等列变换,同样能得到上述三种形式的初等阵,
所以初等阵只有以上三种。

\begin{theorem}[一般矩阵的初等变换与初等阵的联系]
  对于一个$m\times n$矩阵$\mmA$做一次初等行变换,
  就相当于对$\mmA$左乘一个$m$阶初等矩阵;
  对$\mmA$做一次初等列变换,
  就相当于对$\mmA$右乘一个$n$阶初等矩阵。
\end{theorem}

\begin{definition}[矩阵等价]
  如果矩阵$\mmA$可以通过有限次初等变换化为矩阵$\mmB$,
  那么就说$\mmA$与$\mmB$等价。
\end{definition}

\begin{remark}
  矩阵等价是一个等价关系,即满足自反性、对称性、传递性。
\end{remark}

\begin{theorem}[标准型]
  任意非零$m\times n$矩阵$\mmA$都等价于矩阵
  \[
    \mmat{cc}{\mmI_r & \mmZero \\ \mmZero & \mmZero}
    \quad (1\le r \le \min(m,n))
  \]
  它称为矩阵$\mmA$的\textbf{标准型}。
  换句话说,任意非零$m\times n$矩阵$\mmA$,
  必有初等矩阵$\mmP_1,\dots,\mmP_s$,$\mmQ_1,\dots,\mmQ_t$,
  使得
  \[
    \mmP_s\dots\mmP_1\cdot\mmA\cdot\mmQ_1\dots\mmQ_t =
      \mmat{cc}{\mmI_r & \mmZero \\ \mmZero & \mmZero}
  \]
\end{theorem}

因为初等矩阵可逆,所以有如下推论

\begin{corollary}
  对于任意非零$m\times n$矩阵$\mmA$,
  存在可逆$m\times m$矩阵$\mmP$和可逆$n\times n$矩阵$\mmQ$,
  使得
  \[
    \mmP\mmA\mmQ = \mmat{cc}{\mmI_r & \mmZero \\ \mmZero & \mmZero}
    \quad (1\le r \le \min(m,n))
  \]
  或写成
  \[
    \mmA = \mmP^{-1}\mmat{cc}{\mmI_r & \mmZero \\ \mmZero & \mmZero}\mmQ^{-1}
    \quad (1\le r \le \min(m,n))
  \]
\end{corollary}

\subsection{矩阵可逆的充要条件}

\begin{theorem}
  设$\mmA,\mmB$是$n$阶方阵,
  若$\mmA\mmB=\mmI_n$,
  则$\mmA,\mmB$都是可逆阵,且它们互为逆阵。
\end{theorem}

\begin{theorem} \label{thrm:inv-equiv-cond}
  设$\mmA$是$n$阶方阵,则下列结论等价:
  \begin{enumerate}
    \item $\mmA$是可逆矩阵。
    \item $\mmA$可以表示为有限个初等阵的乘积。
    \item $\mmA$可以经过有限次初等变换化为单位矩阵。
  \end{enumerate}
\end{theorem}

\subsection{用初等变换求逆矩阵}
\[
  \mmat{c|c}{\mmA & \mmI} \xrightarrow{\text{初等行变换}}
  \mmat{c|c}{\mmI & \mmA^{-1}}
\]

\chapter{行列式}

\section{行列式}
本小节开始讨论行列式的概念。

首先我们给出了行列式的定义。
但是,如果我们根据行列式的定义来计算行列式,除了一些特殊矩阵,
用计算科学的说法,大部分$n$阶行列式的计算复杂度是$O(n!)$,
因此我们需要探索更高效的方法来计算行列式。

接下来,我们给出了行列式的一些基本性质。
有了这些性质,我们得到了新的行列式计算方法,
而且稍加分析,就能惊喜地看出这个方法的时间复杂度是$O(n^3)$。

\subsection{$n$阶行列式}
\begin{definition}[$n$级排列]
  $n$个自然数按任意固定的顺序构成的一个排列称为$n$\textbf{级排列}。
  所有$n$级排列构成的集合记作$\pi_n$。
\end{definition}

\begin{definition}[逆序数]
  在一个$n$级排列$(i_1,i_2,\dots,i_n)$中,
  如果$i_r>i_s$,但是$r<s$,即前面的数大于后面的数,
  就称这两个数构成\textbf{逆序对}。
  一个排列中逆序对的个数叫做这个排列的\textbf{逆序数},
  记作$\tau(i_1,i_2,\dots,i_n)$。
\end{definition}

\begin{definition}[奇、偶排列]
  逆序数为奇数的排列叫做\textbf{奇排列};
  逆序数为偶数的排列叫做\textbf{偶排列}。
\end{definition}

\begin{definition}[$n$阶行列式]
  设$\mmA=(a_{ij})_{n\times n}$是$n$阶方阵,它的$n$阶行列式为
  \begin{align*}
    |\mmA| &= \sum_{(j_1,j_2,\dots,j_n)\in\pi_n}
    (-1)^{\tau(j_1,j_2,\dots,j_n)}a_{1j_1}a_{2j_2}\dots a_{nj_n} \\
    &= \sum_{(i_1,i_2,\dots,i_n)\in\pi_n}
    (-1)^{\tau(i_1,i_2,\dots,i_n)}a_{i_1 1}a_{i_2 2}\dots a_{i_n n}
  \end{align*}
\end{definition}

特殊矩阵的行列式:
\begin{enumerate}
  \item $|\mdiag(d_1,d_2,\dots,d_n)| = d_1d_2\dots d_n$
  \item 三角阵的行列式为对角线之积
\end{enumerate}

\subsection{行列式的性质}
\begin{theorem}
  若$\mmA$为方阵,那么$|\mmA^\mT| = |\mmA|$
\end{theorem}

\begin{theorem}
  \begin{align*}
    &\quad\mdet{cccc}{
      a_{11} & a_{12} & \cdots & a_{1n} \\
      \vdots &        &        &        \\
      b_{i1} + c_{i1} & b_{i2} + c_{i2}  & \cdots & b_{in} + c_{in} \\
      \vdots &        &        &        \\
      a_{n1} & a_{n2} & \cdots & a_{nn} } \\
    &= \mdet{cccc}{
      a_{11} & a_{12} & \cdots & a_{1n} \\
      \vdots &        &        &        \\
      b_{i1} & b_{i2} & \cdots & b_{in} \\
      \vdots &        &        &        \\
      a_{n1} & a_{n2} & \cdots & a_{nn} }
    + \mdet{cccc}{
      a_{11} & a_{12} & \cdots & a_{1n} \\
      \vdots &        &        &        \\
      c_{i1} & c_{i2} & \cdots & c_{in} \\
      \vdots &        &        &        \\
      a_{n1} & a_{n2} & \cdots & a_{nn} }
    \end{align*}
\end{theorem}

\begin{theorem}
  设$\mmA$为$n$阶方阵。
  若$\mmA$有两行元素相等或对应成比例,
  那么$|\mmA| = 0$。
\end{theorem}

\begin{theorem}
  \begin{displaymath}
    \mdet{cccc}{
      a_{11} & a_{12} & \cdots & a_{1n} \\
      \vdots &        &        &        \\
      \lambda a_{i1} & \lambda a_{i2} & \cdots & \lambda a_{in} \\
      \vdots &        &        &        \\
      a_{n1} & a_{n2} & \cdots & a_{nn} }
    = \lambda \mdet{cccc}{
      a_{11} & a_{12} & \cdots & a_{1n} \\
      \vdots &        &        &        \\
      a_{i1} & a_{i2} & \cdots & a_{in} \\
      \vdots &        &        &        \\
      a_{n1} & a_{n2} & \cdots & a_{nn} }
  \end{displaymath}
\end{theorem}

\begin{theorem} \label{thrm:det-rowswi}
  交换方阵$\mmA$的两行,仅改变$\mmA$行列式的符号,
  即\[ |\mmRowSwi{i}{j}\mmA| = -|\mmA| \]
\end{theorem}

\begin{theorem} \label{thrm:det-rowadd}
  对方阵$\mmA$做$\mTfRowAdd{i}{+k}{j}$变换,不改变$\mmA$的行列式,
  即\[ |\mmRowAdd{i}{+k}{j}\mmA| = |\mmA| \]
\end{theorem}

\subsection{用行列式的性质求行列式}
使用行列式的性质把行列式转化成上三角矩阵的行列式。
后者的行列式即为对角线之积。

\section{行列式按行(列)展开}
本小节讨论另一种行列式的计算方法。

\subsection{行列式按一行(列)展开}
\begin{definition}[余子式]
  在$n$阶行列式$|\mmA|=|a_{ij}|_{n\times n}$中划去第$i$行第$j$列后
  所剩下的$(n-1)^2$个元素按照原来的相对位置排成的$n-1$阶子式$M_{ij}$
  叫做元素$a_{ij}$在$\mmA$中的\textbf{余子式}。
  而\[ A_{ij} = (-1)^{i+j}M_{ij} \]
  叫做元素$a_{ij}$在$\mmA$中的\textbf{代数余子式}。
  这里$1\le i, j \le n$。
\end{definition}

\begin{theorem}[按行(列)展开] \label{thrm:det-expansion}
  $n$阶行列式$|\mmA|=|a_{ij}|_{n\times n}$
  等于任一行(列)各元素与其代数余子式的乘积之和,即
  \begin{align*}
  |\mmA| &= a_{i1}A_{i1}+a_{i2}A_{i2}+\dots+a_{in}A_{in}\quad(i=1,2,\dots,n) \\
  |\mmA| &= a_{1j}A_{1j}+a_{2j}A_{2j}+\dots+a_{nj}A_{nj}\quad(j=1,2,\dots,n)
  \end{align*}
\end{theorem}

\begin{theorem} \label{thrm:det-expansion-zero}
  $n$阶行列式$|\mmA|=|a_{ij}|_{n\times n}$
  的任一行(列)元素与另一行(列)元素的代数余子式乘积之和等于零,即
  \begin{align*}
    a_{i1}A_{j1}+a_{i2}A_{j2}+\dots+a_{in}A_{jn} &= 0\quad(i\neq j) \\
    a_{1i}A_{1j}+a_{2i}A_{2j}+\dots+a_{ni}A_{nj} &= 0\quad(i\neq j)
  \end{align*}
\end{theorem}

\begin{remark}
  定理\ref{thrm:det-expansion}和\ref{thrm:det-expansion-zero}
  可以统一表示为(仅列出行的情况)
  \begin{displaymath}
  \sum_{k=1}^{n}a_{ik}A_{jk} = \begin{cases}
    |\mmA| & (i=j) \\
    0      & (i\neq j)
  \end{cases}
  \end{displaymath}
\end{remark}

\subsection{范德蒙(Vandermonde)行列式}
\begin{displaymath}
  V_n = \mdet{ccccc}{
    1         & 1         & 1         & \cdots & 1      \\
    x_1       & x_2       & x_3       & \cdots & x_n    \\
    x_1^2     & x_2^2     & x_3^2     & \cdots & x_n^2  \\
    \vdots    & \vdots    & \vdots    & \ddots & \vdots \\
    x_1^{n-1} & x_2^{n-1} & x_3^{n-1} & \cdots & x_n^{n-1}
  } = \prod_{1\le i<j\le n}(x_j - x_i)
\end{displaymath}

\section{用行列式求逆阵\ 克莱姆法则}
本小节通过引入伴随矩阵这个中间媒介,
来探索方阵可逆与行列式之间的关系,
从而得到方阵可逆的充要条件,以及通过行列式求逆阵方法。

有了上述方法,我们就能利用行列式来解线性方程组,
并总结出了克莱姆法则。但克莱姆法则也有其局限性。
后面会讲高斯消元来解一般线性方程组。

\subsection{用行列式求逆矩阵}
\begin{theorem}[行列式乘法规则] \label{thrm:det-mul}
  对任意$n$阶方阵$\mmA,\mmB$,有
  \[ |\mmA\mmB|=|\mmA||\mmB| \]
\end{theorem}

\begin{remark}
  证明思路是,首先证明引理:对任意$n$阶方阵$\mmA$和$n$阶初等阵$\mmP$,
  有$|\mmA\mmP|=|\mmP\mmA|=|\mmP||\mmA|$。
  然后利用定理\ref{thrm:inv-equiv-cond}对$\mmB$讨论:
  如果$\mmB$可逆,就能拆成一系列初等阵之积,从而利用引理得证;
  如果$\mmB$不可逆,只要证明等式两边都等于0。
\end{remark}

\begin{theorem}[行列式数乘规则] \label{thrm:det-num-mul}
  对任意$n$阶方阵$\mmA$和$\lambda\in\mfR$,有
  \[ |\lambda\mmA| = \lambda^n|\mmA| \]
\end{theorem}

\begin{definition}[伴随矩阵]
  $n$阶方阵$\mmA$的\textbf{伴随矩阵}为
  \begin{displaymath}
    \mmA^* = \mmat{cccc}{
      A_{11} & A_{12} & \cdots & A_{1n} \\
      A_{21} & A_{22} & \cdots & A_{2n} \\
      \vdots & \vdots & \ddots & \vdots \\
      A_{m1} & A_{m2} & \cdots & A_{mn} }^\mT
  \end{displaymath}
\end{definition}

\begin{theorem}[伴随矩阵的性质] \label{thrm:adjugate-mat-prop}
  设$\mmA$是$n$阶方阵,那么有
  \[ \mmA\mmA^*=\mmA^*\mmA=|\mmA|\mmI \]
\end{theorem}

\begin{remark}
  伴随矩阵就是为了这个性质而定义的。
\end{remark}

\begin{theorem}[伴随矩阵的行列式]
  设$\mmA$是$n$阶方阵,那么有
  \[ |\mmA^*| = |\mmA|^{n-1} \]
\end{theorem}

\begin{remark}
  该式对$|\mmA|$取任意值都成立。
  当$|\mmA|\neq 0$时,
  对伴随矩阵的性质\ref{thrm:adjugate-mat-prop}
  应用行列式乘法规则\ref{thrm:det-mul}
  以及数乘规则\ref{thrm:det-num-mul}即可证明该式。
  当$|\mmA|=0$时,
  则直接通过定义来证明。
\end{remark}

\begin{theorem}[方阵可逆的充要条件]
  设$\mmA$是$n$阶方阵,那么
  \[ \mmA\ \text{可逆} \iff |\mmA| \neq 0 \]
  并且对于可逆矩阵$\mmA$有
  \begin{enumerate}
    \item $\mmA^{-1} = \mmA^*/|\mmA|$
    \item $(\mmA^*)^{-1} = \mmA/|\mmA|$
  \end{enumerate}
\end{theorem}

\subsection{线性代数方程组与克莱姆(Cramer)法则}
$n$元方程组的矩阵形式为
\begin{equation} \label{eq:linear-eq:set}
  \mmA \mvx = \mvb
\end{equation}
其中$\mmA = (a_{ij})_{n\times n}$是\textbf{系数矩阵},
$\overline{\mmA}=(\mmA, \mvb)$是\textbf{增广矩阵},
$\mvx = (x_1,\dots,x_n)^\mT$与$\mvb = (b_1,\dots,b_n)^\mT$是$n$维列向量,
分别称为\textbf{未知向量}与\textbf{常数向量}。
若$\mvx$的一组取值$\hat{\mvx}$能满足\eqref{eq:linear-eq:set},
则称$\hat{\mvx}$为\eqref{eq:linear-eq:set}的一个\textbf{解向量}。

若常数向量不为$\mvZero$,则称\eqref{eq:linear-eq:set}为\textbf{非齐次线性方程组},
否则,叫做\textbf{齐次线性方程组}。

\begin{theorem}[克莱姆(Cramer)法则]
  若$n$元线性代数方程组的系数矩阵$\mmA$的行列式$|\mmA|\neq 0$,
  则方程组有唯一解$\mvx = (x_1,\dots,x_n)$,其中
  \begin{displaymath}
    x_j = \frac{D_j}{|\mmA|} \quad (j=1,\dots,n)
  \end{displaymath}
  $D_j$是将$|\mmA|$的第$j$列换成$\mvb$所得的行列式。
\end{theorem}

\begin{corollary}[齐次线性方程组有非零解的充要条件]
  对于$n$元齐次线性方程组$\mmA\mvx=\mvZero$,
  若$|\mmA|\neq 0$,则只有零解;
  若$|\mmA| = 0$,则有无穷多非零解。
\end{corollary}

\begin{remark}
  值得注意的是,克莱姆法则只适用于方程个数等于未知量个数的方程组,
  而且系数行列式不能为零。
  此外,克莱姆法则的计算复杂度也更高,为$O(n^4)$,
  不如后面要介绍的高斯消元法$O(n^3)$的复杂度。
\end{remark}
\chapter{向量组的线性相关性与线性代数方程组}

\section{向量组的线性无关}
在一个方程组中,会有一些方程是``无用的'',
也就是说,它能由其它方程通过线性运算表示出来。
向量组的线性无关与此概念紧密联系——向量组即是一个方程的系数矩阵。
向量组的极大线性无关组也就反应了在方程组中去掉那些无用的方程后得到的方程组。
向量组的秩即是这些有用的方程的个数。

\subsection{线性相关与线性无关}
\begin{definition}[线性组合与线性表示]
  设$\alpha,\alpha_1,\alpha_2,\dots,\alpha_r$是一组$n$维向量。
  如果存在数$k_1,k_2,\dots,k_r$,使得
  \[ \alpha = k_1\alpha_1+k_2\alpha_2+\dots k_r\alpha_r \]
  则称$\alpha$是$\alpha_1,\alpha_2,\dots,\alpha_r$的\textbf{线性组合},
  或者说$\alpha$可由$\alpha_1,\alpha_2,\dots,\alpha_r$\textbf{线性表示},
  其中,$k_1,k_2,\dots,k_r$叫做\textbf{表示系数}。
\end{definition}

\begin{definition}[线性相关与线性无关]
  设$\alpha_1,\alpha_2,\dots,\alpha_r$是一组$n$维向量。
  如果存在一组\textbf{不全为零}的数$k_1,k_2,\dots,k_r$,使得
  \[ k_1\alpha_1+k_2\alpha_2+\dots k_r\alpha_r = \mvZero \]
  则称向量组$\alpha_1,\alpha_2,\dots,\alpha_r$\textbf{线性相关};
  否则,称\textbf{线性无关}。
\end{definition}

\begin{theorem}[线性相关与线性组合的联系]
    $\alpha_1,\alpha_2,\dots,\alpha_r\ (r\ge 2)$线性相关
    当且仅当至少有一个向量是其它向量的线性组合。
\end{theorem}

\begin{theorem}[线性相关与线性无关的等价条件]
  设$\alpha_1,\alpha_2,\dots,\alpha_r$是一组$n$维列向量,
  $\mvk = (k_1,k_2,\dots,k_r)^\mT$,
  矩阵$\mmA = (\alpha_1,\alpha_2,\dots,\alpha_r)$,那么
  \begin{align*}
  \alpha_1,\alpha_2,\dots,\alpha_r\ \text{线性相关}
  &\iff \mmA\mvk = \mvZero\ \text{有非零解} \\
  &\iff |\mmA|=0 
  \end{align*}
  \begin{align*}
  \alpha_1,\alpha_2,\dots,\alpha_r\ \text{线性无关}
  &\iff \mmA\mvk = \mvZero\ \text{仅有零解} \\
  &\iff |\mmA|\neq 0 
  \end{align*}
\end{theorem}

\subsection{向量组组内关系}
\begin{theorem}[接长与补短]
  设向量组$S_1: \alpha_1,\alpha_2,\dots,\alpha_r$。
  若在每个向量中添加一个分量,把它变成$n+1$维向量组$S_2$,
  那么$S_1$线性无关能推出$S_2$线性无关,
  $S_2$线性相关能推出$S_1$线性相关。
\end{theorem}

\begin{theorem}[部分与整体]
  设向量组
  \[ S_1: \alpha_1,\alpha_2,\dots,\alpha_r \]
  与向量组
  \[ S_2: \alpha_1,\alpha_2,\dots,\alpha_r,\alpha_{r+1},\dots,\alpha_s \]
  是两个$n$维向量组。
  若$S_1$线性相关,则$S_2$线性相关;
  反之,若$S_2$线性无关,则$S_1$线性无关。
\end{theorem}

\begin{remark}
  直白地说,就是部分相关可以推出整体相关,整体无关可以推出部分无关。
\end{remark}

\begin{theorem}
  任意$n+1$个$n$维向量一定线性相关。
\end{theorem}

\subsection{向量组组间关系}
\begin{definition}[向量组等价]
  设向量组
  \[ S_1: \alpha_1,\alpha_2,\dots,\alpha_r \]
  \[ S_2: \beta_1,\beta_2,\dots,\beta_s \]
  若$S_1$中每个向量都可以被向量组$S_2$线性表示,
  则称向量组$S_1$可被$S_2$\textbf{线性表出}。
  若$S_1$与$S_2$能互相线性表出,那么称$S_1$和$S_2$\textbf{等价}。
\end{definition}

\begin{remark}
  向量组的等价是等价关系。
\end{remark}

\begin{theorem} \label{thrm-vector-set-size}
  设有两个向量组$S_1$和$S_2$,分别含有$r$和$s$个向量。
  若$S_1$能由$S_2$线性表出,且$r > s$,那么$S_1$线性相关。
  反之,若$S_1$线性无关且能由$S_2$线性表出,那么$r \le s$。
\end{theorem}

\begin{corollary} \label{thrm-vector-set-equiv}
  若两个线性无关的向量组$S_1$和$S_2$等价,那么它们包含的向量个数相同。
\end{corollary}

\subsection{极大线性无关组与向量组的秩}
\begin{definition}
  若一个向量组中有部分向量$\alpha_1,\alpha_2,\dots,\alpha_s$具有下面两个性质:
  \begin{enumerate}
    \item
    $\alpha_1,\alpha_2,\dots,\alpha_s$线性无关;
    \item
    从原向量组中任选一个新向量(如果还有的话)加入到这个向量组中,
    所得的部分向量组就线性相关了。
  \end{enumerate}
  那么称$\alpha_1,\alpha_2,\dots,\alpha_s$为原向量组的\textbf{极大线性无关组}。
\end{definition}

\begin{theorem} \label{thrm-vector-set-self-equiv}
   向量组的任意一个极大线性无关组都与向量组本身等价。
\end{theorem}

\begin{remark}
  一个向量组的极大线性无关组不一定是唯一的,
  但是根据定理\ref{thrm-vector-set-self-equiv}
  和推论\ref{thrm-vector-set-equiv}可以证明,
  它们的大小一定是相同的。
\end{remark}

\begin{definition}[向量组的秩]
  向量组$S$的极大线性无关组所含向量的个数称为这个向量组的\textbf{秩},
  记作$r(S)$。
\end{definition}

有的秩的概念,定理\ref{thrm-vector-set-size}
和推论\ref{thrm-vector-set-equiv}可以做出如下推广:
\begin{theorem}
  设有向量组$S_1$和$S_2$。
  \begin{enumerate}
    \item
    若$S_1$能由$S_2$线性表出,那么$r(S_1) \le r(S_2)$。
    \item
    若$S_1$与$S_2$等价,那么$r(S_1)=r(S_2)$。
  \end{enumerate}
\end{theorem}

\section{矩阵的秩}
矩阵如果按行划分,或者按列划分,其实都能看成一个向量组。
既然向量组有秩,那么矩阵的秩也可以因此定义出来。
我们会发现,不管是按行划分,还是按列划分,
行向量组与列向量组的秩都是相同的,
这个同一的值就是矩阵的秩。

\subsection{矩阵的行秩、列秩和秩}
\begin{definition}[行秩与列秩]
  一个矩阵$\mmA=(a_{ij})_{m\times n}$的行向量组的秩叫做$\mmA$的\textbf{行秩},
  列向量组的秩叫做$\mmA$的\textbf{列秩}。
\end{definition}

\begin{theorem}[行秩与列秩的不变性]
  一个矩阵的行秩和列秩在初等变换下保持不变。
\end{theorem}

\begin{remark}
  因为任何矩阵都能通过初等变换化为标准型,标准型的行秩与列秩相同,
  所以更进一步的结论呼之欲出。
\end{remark}

\begin{theorem}[行秩与列秩相等]
  一个矩阵的行秩等于列秩。
\end{theorem}

\begin{remark}
  有了这个定理,我们就能把行秩和列秩统称为秩。
\end{remark}

\begin{theorem}[秩]
  矩阵$\mmA$的行秩或列秩称为这个矩阵的\textbf{秩},记作$r(\mmA)$。
\end{theorem}

\subsection{矩阵的秩的性质}
\begin{theorem}
  设$\mmA=(a_{ij})_{m\times s}, \mmB=(b_{jk})_{s\times p}$,则
  \[ r(\mmA\mmB) \le \min\{ r(\mmA), r(\mmB) \} \]
\end{theorem}

\begin{remark}
  根据需要也可以写成$r(\mmA\mmB) \le r(\mmA)$,$r(\mmA\mmB)\le r(\mmB)$。
\end{remark}

\begin{theorem}
  设$\mmA$是$m\times n$阶矩阵,
  $\mmP$是$m$阶可逆方阵,$\mmQ$是$n$阶可逆方阵,则
  \[ r(\mmA) = r(\mmP\mmA) = r(\mmA\mmQ) \]
\end{theorem}

\begin{theorem}
  设$\mmA$和$\mmB$都是$m\times n$阶矩阵,则
  \[ r(\mmA+\mmB) \le r(\mmA) + r(\mmB) \]
\end{theorem}

\section{线性代数方程组}
先前我们已经讨论过使用行列式与伴随矩阵的方法(克莱姆法则)来解方程组了。
本小节继上面对向量组相关性的讨论,来研究解线性代数方程组的新方法:
高斯消元法。

\subsection{高斯消元}
对于线性代数方程组$\mmA\mvx=\mvb$,它的增广矩阵$\overline{\mmA}$是
\begin{displaymath}
  \mmat{cccc|c}{
    a_{11} & a_{12} & \cdots & a_{1n} & b_1 \\
    a_{21} & a_{22} & \cdots & a_{2n} & b_2 \\
    \vdots & \vdots & \ddots & \vdots & \vdots \\
    a_{m1} & a_{m2} & \cdots & a_{mn} & b_m 
  }
\end{displaymath}
通过高斯消元,可以转化为如下形式:
\begin{equation} \label{eq-gauss-elim}
  \mmat{cccccc|c}{
    c_{11} & c_{12} & \cdots & c_{1r} & \cdots & c_{1n} & c_1    \\
           & c_{22} & \cdots & c_{2r} & \cdots & c_{2n} & c_2    \\
           &        & \ddots & \vdots &        & \vdots & \vdots \\
           &        &        & c_{rr} & \cdots & c_{rn} & c_r    \\
           &        &        &        &        &        & c_{r+1}
  }
\end{equation}
其中$c_{ii}\neq 0\ (i=1,\dots,r)$。
当$c_{r+1}=0$时,方程组有解。
当$c_{r+1}\neq 0$时,方程组无解。
因此有如下结论:

\begin{theorem}[线性代数方程组有解的充要条件]
  线性代数方程组$\mmA\mvx=\mvb$有解当且仅当
  \[ r(\mmA) = r(\overline{\mmA}) \]
  若有解:
  \begin{enumerate}
    \item 若$r(\mmA)=n$,则有唯一解。
    \item 若$r(\mmA) < n$,则有无穷多解。
  \end{enumerate}
\end{theorem}

\begin{remark}
  对于齐次方程组$\mmA\mvx=\mvZero$,
  必有$r(\mmA)=r(\overline{\mmA})$,
  因此齐次方程组一定有解。
  当$r(\mmA)=n$时,则只有零解。
  当$r(\mmA)\neq n$时,则有无穷多解。
\end{remark}

\subsection{线性代数方程组解的结构}
经过高斯消元得到的式\ref{eq-gauss-elim}如果有解(即$c_{r+1}=0$),
那么可以进一步转化为
\begin{equation} \label{eq-gauss-elim-further}
  \mmat{ccccccc|c}{
    1 &        &        &   & d_{r1}   & \cdots & d_{r1} & d_1    \\
      & 1      &        &   & d_{r2}   & \cdots & d_{r2} & d_2    \\
      &        & \ddots &   & \vdots   & \ddots & \vdots & \vdots \\
      &        &        & 1 & d_{rr+1} & \cdots & d_{rn} & d_r
  }
\end{equation}
方程的解应该是一目了然的了。
据此,我们开始讨论线性代数方程组解的结构。
首先是齐次线性方程组。

\begin{definition}[基础解系]
  齐次线性方程组$\mmA\mvx=\mvZero$的解向量组(构成\textbf{解空间})
  的一个极大线性无关组
  叫做它的一个\textbf{基础解系}。
\end{definition}

\begin{theorem}
  若齐次线性方程组的系数矩阵$\mmA$的秩小于$n$,
  则方程组必有基础解系。且基础解系所含解的个数等于$n-r$。
\end{theorem}

\begin{remark}
  把式\ref{eq-gauss-elim-further}中$d_1,\dots,d_r$设为0,
  则有如下形式的解:
  \begin{displaymath}
  \meqs{rcl}{
    x_1 &=& \xi_{11}x_{r+1}+\xi_{12}x_{r+2}+\cdots+\xi_{1n-r}x_{n} \\
    x_2 &=& \xi_{21}x_{r+1}+\xi_{22}x_{r+2}+\cdots+\xi_{2n-r}x_{n} \\
    & \vdots & \\
    x_r &=& \xi_{r1}x_{r+1}+\xi_{r2}x_{r+2}+\cdots+\xi_{rn-r}x_{n} \\
  }
  \end{displaymath}
  由此得到一个基础解系为
  \begin{displaymath}
  \meqs{rcl} {
    \xi_1 &=& (\xi_{11}, \xi_{21}, \dots, \xi_{r1}, 1, 0, \dots, 0)^\mT \\
    \xi_2 &=& (\xi_{12}, \xi_{22}, \dots, \xi_{r2}, 0, 1, \dots, 0)^\mT \\
    & \vdots & \\
    \xi_{n-r} &=& (\xi_{1n-r}, \xi_{2n-r}, \dots, \xi_{rn-r}, 0, 0, \dots, 1)^\mT
  }
  \end{displaymath}
  方程组任何解$\xi$都可以由$\xi_1,\dots,\xi_{n-r}$线性表示。
\end{remark}

接下来是非齐次线性方程组的解的结构。
设非齐次线性方程组为$\mmA\mvx=\mvb$。
它对应的齐次线性方程组为$\mmA\mvx=\mvZero$,
称为\textbf{导出组}。它们解的结构有着密切的联系:
\begin{enumerate}
  \item
  若$\eta_1,\eta_2$是$\mmA\mvx=\mvb$的解,
  那么$\eta_1-\eta_2$是$\mmA\mvx=\mvZero$的解。
  \item
  若$\eta$是$\mmA\mvx=\mvb$的解,$\xi$是$\mmA\mvx=\mvZero$的解,
  那么$\eta+\xi$是$\mmA\mvx=\mvb$的解。
\end{enumerate}

\begin{theorem}[非齐次线性方程组的通解]
  设$\eta_0$是$\mmA\mvx=\mvb$的一个解,那么方程的任意解$\eta$都能表示为
  \[ \eta = \eta_0 + k_1\xi_1 + \cdots + k_{n-r}\xi_{n-r} \]
  其中$\xi_1,\dots,\xi_{n-r}$是方程导出组$\mmA\mvx=\mvZero$的基础解系,
  $k_1,\dots,k_{n-r} \in \mfR$。
\end{theorem}

\chapter{线性空间与线性变换}

\section{线性空间}
在这一小节,我们把先前向量的概念进一步抽象,得到了线性空间的概念。
向量组有线性相关、线性无关、极大线性无关组等概念,
这在线性空间中就转变成纬度、基、坐标的概念。

线性空间和物理的联系也是很紧密的。
因为向量是物理的常用量,而线性空间就是从向量抽象来的。
回忆一下物理学的知识,我们知道一个物理量在不同参考系的表示是不一样的。
对应到线性空间上来,我们会发现一个元素在不同基底下的坐标也是不一样的。
所以我们要研究这些坐标与基底的变换关系。

最后我们会简单讨论一下子空间的概念。

\subsection{线性空间}
\begin{definition}[线性空间] \label{def-linear-space}
  设$V$是非空集合,$\mfF$是一个数域。
  若在$V$上定义两种运算:加法$\oplus$和数乘$\otimes$,满足
  \begin{description}
    \item[加法]
    \begin{enumerate}
      \item 封闭性:
      $\forall\alpha,\beta\in V,\ \alpha\oplus\beta\in V$。
      \item 交换律:
      $\forall\alpha,\beta\in V,\ \alpha\oplus\beta = \beta\oplus\alpha$
      \item 结合律:
      $\forall\alpha,\beta,\gamma\in V,\ 
        (\alpha\oplus\beta)\oplus\gamma = \alpha\oplus(\beta\oplus\gamma)$。
      \item 零元存在:
      $\exists\theta\in V,\forall\alpha\in V,\ \alpha\oplus\theta=\alpha$。
      \item 逆元存在:
      $\forall\alpha\in V,\exists\beta\in V,\ \alpha\oplus\beta=\theta$。
      我们把$\beta$记为``$-\alpha$''。
    \end{enumerate}  
    \item[数乘]
    \begin{enumerate}
      \item 封闭性:
      $\forall\lambda\in\mfF,\alpha\in V,\ \lambda\otimes\alpha\in V$。
      \item 结合律:
      $\forall\lambda_1,\lambda_2\in\mfF,\alpha\in V,\ 
        \lambda_1\otimes(\lambda_2\otimes\alpha) =
        (\lambda_1\lambda_2)\otimes\alpha$。
      \item 分配律1:
      $\forall\lambda_1,\lambda_2\in\mfF,\alpha\in V,\ 
        (\lambda_1\oplus\lambda_2)\otimes\alpha =
        (\lambda_1\otimes\alpha)\oplus(\lambda_2\otimes\alpha)$
      \item 分配律2:
      $\forall\lambda\in\mfF,\alpha,\beta\in V,\ 
        \lambda\otimes(\alpha\oplus\beta) =
        (\lambda\otimes\alpha)\oplus(\lambda\otimes\beta)$。
      \item 单位元存在:
      $\forall\alpha\in V,\ 
        1\otimes\alpha = \alpha$。
    \end{enumerate}
  \end{description}
  则称$V$是数域$\mfF$上的一个\textbf{线性空间},又称\textbf{向量空间}。
  若$\mfF$是实数域$\mfR$,则称$V$是\textbf{实线性空间}。
  若$\mfF$是复数域$\mfC$,则称$V$是\textbf{复线性空间}。
\end{definition}

\begin{remark}
  如果有点抽象代数的背景,读者不难发现上述加法满足的是阿贝尔群的性质。
\end{remark}

\begin{theorem}[线性空间的性质]
  设$V$是$\mfF$上的线性空间,那么
  \begin{enumerate}
    \item 加法零元的具有唯一性
    \item 加法逆元的具有唯一性
    \item 加法零元的求法:
    $\forall\alpha\in V,\ 0\otimes\alpha=\theta$
    \item 加法逆元的求法:
    $\forall\alpha\in V,\ (-1)\otimes\alpha=-\alpha$
    \item $\forall k\in\mfF,\ k\otimes\theta=\theta$
    \item 若$\lambda\otimes\alpha=\theta$,则$\lambda=0$或$\alpha=\theta$。
  \end{enumerate}
\end{theorem}
为了书写与阅读的方便,以后``$\oplus$''用正常的加号来表示,
``$\otimes$''可省略。参考向量的记法。

\subsection{基与坐标}
向量组的线性相关、线性无关、线性表示的概念也适用于线性空间。

\begin{definition}[线性相关与线性无关]
  设$V$是线性空间,$\alpha_1,\alpha_2,\dots,\alpha_s\in V$。
  若数域中存在一组不全为0的数$k_1,k_2,\dots,k_s$,使得
  \[ k_1\alpha_1 + k_2\alpha_2 + \dots + k_s\alpha_s = \theta \]
  则称$\alpha_1,\alpha_2,\dots,\alpha_s$\textbf{线性相关}。
  否则,则称\textbf{线性无关}。
\end{definition}

\begin{definition}[线性表示]
  设$V$是线性空间,$\alpha,\alpha_1,\alpha_2,\dots,\alpha_s\in V$。
  若数域中存在一组数$k_1,k_2,\dots,k_s$,使得
  \[ \alpha = k_1\alpha_1 + k_2\alpha_2 + \dots + k_s\alpha_s \]
  则称$\alpha$为$\alpha_1,\alpha_2,\dots,\alpha_s$的\textbf{线性组合},
  也称$\alpha$可以由$\alpha_1,\alpha_2,\dots,\alpha_s$\textbf{线性表示}。
\end{definition}

\begin{definition}[维数]
  如果一个线性空间$V$中,线性无关的元素的最大个数是$n$,
  则称该线性空间是$n$维的,记作$\mdim V = n$。
  
  如果对任意正整数$N$,总存在$N$个线性无关的元素,
  则称该线性空间是\textbf{无穷维线性空间}。
  不是无穷维的线性空间叫做有穷维线性空间。
\end{definition}

\begin{remark}
  本章中我们不讨论无穷维线性空间。
\end{remark}

\begin{definition}[基与坐标]
  若$\alpha_1,\alpha_2,\dots,\alpha_n$是$n$维线性空间$V$中一组线性无关的元素,
  且$V$中任意元素$\alpha$都可由$\alpha_1,\alpha_2,\dots,\alpha_n$线性表示,
  即\[ \alpha = k_1\alpha_1 + k_2\alpha_2 + \dots + k_n\alpha_n \]
  则称$\alpha_1,\alpha_2,\dots,\alpha_n$是$V$的一组\textbf{基底}(简称\textbf{基})。
  其中,有序元组$(k_1,k_2,\dots,k_n)$称为$\alpha$在
  基底$\alpha_1,\alpha_2,\dots,\alpha_n$下的坐标。
\end{definition}

\begin{remark}
  线性空间的基对应的是极大线性无关组的概念。
\end{remark}

\subsection{基变换与坐标变换}
\begin{theorem}[基底变换公式]
  设$\alpha_1,\alpha_2,\dots,\alpha_n$与$\beta_1,\beta_2,\dots,\beta_n$是
  线性空间$V$的两组基底,那么可以写成
  \begin{equation} \label{eq-basis-transform}
    (\beta_1,\beta_2,\dots,\beta_n) = (\alpha_1,\alpha_2,\dots,\alpha_n)
    \mmat{cccc}{
      a_{11} & a_{12} & \cdots & a_{1n} \\
      a_{21} & a_{22} & \cdots & a_{2n} \\
      \vdots & \vdots & \ddots & \vdots \\
      a_{m1} & a_{m2} & \cdots & a_{mn} }
  \end{equation}
  我们把它简记为
  \begin{displaymath}
    \mmBasis{\beta} = \mmBasis{\alpha}\cdot\mmP
  \end{displaymath}
  我们称$\mmP$是$\mmBasis{\alpha}$到$\mmBasis{\beta}$的\textbf{过渡矩阵},
  式\ref{eq-basis-transform}是\textbf{基底变换公式}。
\end{theorem}

\begin{remark}
  过渡矩阵$\mmP$是非奇异矩阵。
  因为如果$\mmP$是奇异的,就存在$\mvx\neq\mvZero$,使得$\mmP\mvx=\mvZero$。
  从而有$\mmBasis{\beta}\mvx = \mmBasis{\alpha}\mmP\mvx = \mvZero$。
  于是$\mmBasis{\beta}$是奇异矩阵,这与``$\beta_1,\dots,\beta_n$是基底''是矛盾的。
\end{remark}

\begin{theorem}[坐标变换公式]
  设$\mmBasis{\alpha}$和$\mmBasis{\beta}$是线性空间$V$的两组基底,
  $\mmP$是从$\mmBasis{\alpha}$到$\mmBasis{\beta}$的过渡矩阵。
  若$\mvx=(x_1,\dots,x_n)^T$和$\mvy=(y_1,\dots,y_n)^T$
  分别是元素$\alpha\in V$在基底$\mmBasis{\alpha}$和$\mmBasis{\beta}$下的坐标,即
  \[ \alpha = \mmBasis{\beta}\mvy = \mmBasis{\alpha}\mvx \]
  那么有
  \begin{equation} \label{eq-coordinate-transform}
    \mvy = \mmP^{-1}\mvx
  \end{equation}
  式\ref{eq-coordinate-transform}被称为\textbf{坐标变换公式}。
\end{theorem}

\subsection{子空间}
\begin{definition}
  设$V$是数域$\mfF$上的线性空间。若$W\subset V$也是数域$\mfF$上的线性空间,
  则称$W$是$V$的\textbf{子空间}。
  $W=\{\theta\}$叫做\textbf{平凡子空间}。
\end{definition}

\begin{theorem}[子空间的充要条件]
  设$V$是数域$\mfF$上线性空间。
  $W\subset V$是$V$的子空间的充要条件是:
  1. 对加法封闭;2. 对乘法封闭。
  换句话说,即是$\forall\alpha,\beta\in W,\lambda,\mu\in\mfF$,
  \[ \lambda\alpha + \mu\beta \in W \]
\end{theorem}

\begin{definition}[子集张成的子空间]
    设$V$是数域$\mfF$上的线性空间,$\alpha_1,\alpha_2,\dots,\alpha_s\in V$。
    我们定义$V$的子空间
    \begin{displaymath}
    \mspan\{ \alpha_1,\alpha_2,\dots,\alpha_s \} =
    \{ \alpha: \alpha=\sum_{i=1}^{s}k_i\alpha_i, \forall k_1,\dots,k_s\in\mfF \}
    \end{displaymath}
    称作由向量组$\alpha_1,\alpha_2,\dots,\alpha_s$张成的子空间。
\end{definition}

\begin{definition}[线性空间的和]
  设$W_1,W_2$是线性空间的两个子空间,则它们的\textbf{和}定义为
  \begin{displaymath}
    W_1+W_2 = \{ u: u=\alpha+\beta, \forall\alpha\in W_1,\beta\in W_2 \}
  \end{displaymath}
\end{definition}

\begin{theorem}
  设$V$是线性空间,$W_1,W_2$是$V$的子空间,那么有以下结论:
  \begin{enumerate}
    \item $W_1\cap W_2$是$V$的子空间。
    \item $W_1\cup W_2$不是$V$的子空间。
    \item $W_1+W_2$是$V$的子空间。
  \end{enumerate}
\end{theorem}

\begin{theorem}[维数定理]
  设$W_1,W_2$是线性空间$V$的两个有限维子空间,则有
  \begin{displaymath}
  \mdim W_1 + \mdim W_2 = \mdim(W_1+W_2) + \dim(W_1\cap W_2)
  \end{displaymath}
\end{theorem}

\begin{definition}[直接和]
  设$V_1,V_2$是线性空间$V$的两个子空间。
  若对任意$\alpha\in V$,存在唯一的$\alpha_1\in V_1, \alpha_2\in V_2$,
  使得$\alpha=\alpha_1+\alpha_2$,
  则称$V$是$V_1$和$V_2$的\textbf{直接和}或\textbf{直和},
  记作$V=V_1\oplus V_2$。
  也称$V_1$和$V_2$是$V$内的\textbf{互补空间}。
\end{definition}

\begin{theorem}[直接和的等价条件]
  设$V_1,V_2$是线性空间$V$的两个子空间。
  \begin{align*}
    V=V_1\oplus V_2
    &\iff V = V_1 + V_2\ \text{且}\ V_1\cap V_2 = \{ \theta \} \\
    &\iff \mdim (V_1\oplus V_2) = \mdim V_1 + \mdim V_2
  \end{align*}
\end{theorem}

\section{线性变换}
如果说线性空间是对向量的抽象,那么线性变换就是对矩阵的抽象。
回想一下,我们在上一小节研究了基与坐标变换的公式,
公式中就是用矩阵来刻画了``变换''这个过程。
在介绍换线性变换的基本概念之后,
我们会证明,在取定一组基下,线性变换与矩阵有着一一对应的关系。

虽然线性变换在一组基下仅有唯一的矩阵与之对应,
但换个角度来看,有不同的基就会有不同的矩阵与之对应。
这些矩阵间又有什么关系呢?这就引出了相似矩阵的概念。

\subsection{线性变换的概念}
\begin{definition}[线性变换]
  设$V,W$是数域$\mfF$上的线性空间。
  函数$T: V\mapsto W$被称为$V$到$W$的\textbf{变换}。
  若对任意$\alpha,\beta\in V, k \in\mfF$,$T$满足
  \[ T(\alpha+\beta)=T(\alpha)+T(\beta),\quad T(k\alpha) = kT(\alpha) \]
  则称$T$是$V$到$W$的\textbf{线性变换}。
\end{definition}

\begin{theorem}[线性变换的性质]
  设$V,W$是数域$\mfF$上的线性空间,$T$是$V$到$W$的线性变换,
  $\theta_1$和$\theta_2$分别是$V$和$W$上的零元。
  $T$满足以下性质:
  \begin{enumerate}
    \item
    $T(\theta_1) = \theta_2$
    \item
    $\forall\alpha,\beta\in V,\lambda,\mu\in\mfF,\ 
      T(\lambda\alpha+\mu\beta)=\lambda T(\alpha)+\mu T(\beta)$
    \item
    若$V$中的$\alpha_1,\alpha_2,\dots,\alpha_s$线性相关,
    则$T(\alpha_1),T(\alpha_2),\dots,T(\alpha_s)$也线性相关。
  \end{enumerate}
\end{theorem}

\begin{definition}[特殊的线性变换]
  设$V$是数域$\mfF$上的线性空间,$T$是$V$上的线性变换。
  \begin{enumerate}
    \item
    若对任意$\alpha\in V$,$T(\alpha)=\theta$,
    则称$T$为\textbf{零变换},常用$T_0$来表示。
    \item
    若对任意$\alpha\in V$,$T(\alpha)=\alpha$,
    则称$T$为\textbf{恒等变换},常用$E$或$I$表示。
    \item 
    若对任意$\alpha\in V$,$T(\alpha)=k\alpha$,其中$k\in\mfF$,
    则称$T$为\textbf{数乘变换},常用$T_k$来表示。
  \end{enumerate}
\end{definition}

\begin{definition}[象空间与核空间]
  设$T$是线性空间$V$到$W$的线性变换,
  则$T$的\textbf{象空间}定义为:
  \[ \mim(T) = \{ \xi: \xi=T(\alpha), \alpha\in V\} \]
  $T$的\textbf{核空间}定义为:
  \[ \mker(T) = \{ \alpha: T(\alpha)=\theta, \alpha\in V \} \]
\end{definition}

\begin{theorem}[象空间与核空间的性质]
  设$T$是线性空间$V$到$W$的线性变换,那么
  \begin{enumerate}
    \item 
    $\mim(T)$是$V$的子空间,$\mker(T)$是$W$的子空间。
    \item
    $\mdim(\mim(T)) + \mdim(\ker(T)) = \dim(V)$
  \end{enumerate}
\end{theorem}

\subsection{线性变换的运算和可逆线性变换}
设$L(V)$是线性空间$V$上所有线性变换构成的集合。
类似矩阵,给定一个数域$\mfF$,
我们也可以在$L(V)$上定义加法、数乘和乘法运算:
\begin{description}
  \item[加法]
  对任意$T_1,T_2\in L(V),\alpha\in V$,$(T_1+T_2)(\alpha)=T_1(\alpha)+T_2(\alpha)$。
  \item[数乘]
  对任意$T\in L(V),\alpha\in V$,$(kT)(\alpha)=kT(\alpha)$,
  其中$k\in\mfF$。
  \item[乘法]
  因为线性变换定义上是个函数,所以两个线性变换相乘定义为两个函数的复合。
\end{description}

\begin{theorem}
  拥有上面定义的加法、数乘规则的$L(V)$是线性空间。
\end{theorem}

\begin{definition}[可逆线性变换]
  设$T$是线性空间$V$上的一个线性变换。
  如果$V$上存在一个变换$\sigma$,使得
  \[ T\sigma = \sigma T = E \]
  其中$E$是$V$上的恒等变换,则称$\sigma$是$T$的\textbf{逆变换}。
  不难看出,如果$T$的逆变换存在,那么它必然是唯一的,
  因此,我们把$T$的逆变换记作$T^{-1}$,
  并称$T$为\textbf{可逆线性变换}。
\end{definition}

\subsection{线性变换的矩阵表示}
接下来,我们讨论在给定一组基下,线性变换与矩阵有一一对应的关系。
需要注意的是,我们这里讨论的线性变换是从一个线性空间到自身的变换,
所以这里的矩阵也就是方阵。

首先,如果知道一个$n$维线性空间$V$上的线性变换$T$,
我们就能知道它在基$\epsilon_1,\epsilon_2\dots,\epsilon_n$下对应的矩阵:
对任意$i=1,2,\dots,n$,应该有一组坐标$a_{1i},a_{2i},\dots,a_{ni}$,使得
\begin{displaymath}
  T(\epsilon_i) = a_{1i}\epsilon_1+a_{2i}\epsilon_2+\dots+a_{ni}\epsilon_n
\end{displaymath}
写成矩阵的形式,即是
\begin{displaymath}
  (T(\epsilon_1), T(\epsilon_2),\dots,T(\epsilon_n)) =
    (\epsilon_1,\epsilon_2,\dots,\epsilon_n)\mmA
\end{displaymath}
其中$\mmA=(a_{ij})_{n\times n}$。
这样,对任意$\alpha=x_1\epsilon_1+x_2\epsilon_2+\dots+x_n\epsilon_n\in V$,
我们就能通过矩阵求出它经过线性变换后元素:
\begin{equation} \label{eq-coord-after-linear-trans}
  T(\alpha) = (\epsilon_1,\epsilon_2,\dots,\epsilon_n)\mmA
    (x_1,x_2,\dots,x_n)^T
\end{equation}
因此,我们称$\mmA$是
\textbf{线性变换$T$在基$\epsilon_1,\epsilon_2\dots,\epsilon_n$下的矩阵}。

反之,在线性空间$V$的一组基$\epsilon_1,\epsilon_2\dots,\epsilon_n$下,
给定矩阵$\mmA=(a_{ij})_{n\times n}$,
存在线性变换$T$,使得
\begin{displaymath}
  (T(\epsilon_1), T(\epsilon_2),\dots,T(\epsilon_n)) =
    (\epsilon_1,\epsilon_2,\dots,\epsilon_n)\mmA
\end{displaymath}
这个$T$是这样构造的:
对任意$\alpha=x_1\epsilon_1+x_2\epsilon_2+\dots+x_n\epsilon_n\in V$,
\begin{displaymath}
  T(\alpha) = (\epsilon_1,\epsilon_2,\dots,\epsilon_n)\mmA
    (x_1,x_2,\dots,x_n)^T
\end{displaymath}

总结一下上面的论述,有如下定理:
\begin{theorem}
  在线性空间$V$的一组基$\epsilon_1,\epsilon_2\dots,\epsilon_n$下,
  线性变换$T$与$n$阶方阵$\mmA$一一对应。
  $\mmA$的第$i$列就是$T(\epsilon_i)$在
  基$\epsilon_1,\epsilon_2\dots,\epsilon_n$下的坐标。
\end{theorem}

\subsection{相似矩阵}
\begin{theorem}
  设$n$为线性空间$V$的两组基底为$\epsilon_1,\epsilon_2\dots,\epsilon_n$以及
  $\eta_1,\eta_2,\dots,\eta_n$。由$\epsilon_1,\epsilon_2\dots,\epsilon_n$到
  $\eta_1,\eta_2,\dots,\eta_n$的过渡矩阵为$\mmP$。
  $V$上线性变换$T$在这两组基下的矩阵分别为$\mmA,\mmB$,那么有
  \[ \mmB = \mmP^{-1}\mmA\mmP \]  
\end{theorem}

\begin{remark}
  直观的理解就是,
  \begin{displaymath}
    (\epsilon_1,\epsilon_2\dots,\epsilon_n)\xrightarrow{\mmP}
    (\eta_1,\eta_2,\dots,\eta_n)\xrightarrow{\mmB}
    T(\eta_1,\eta_2,\dots,\eta_n)
  \end{displaymath}
  等效于
  \begin{displaymath}
  (\epsilon_1,\epsilon_2\dots,\epsilon_n)\xrightarrow{\mmA}
  T(\epsilon_1,\epsilon_2\dots,\epsilon_n)\xrightarrow{\mmP}
  T(\eta_1,\eta_2,\dots,\eta_n)
  \end{displaymath}
  所以有$\mmP\mmB=\mmA\mmP$。变形即是上面定理的结果。
  这也是这个定理的证明思路。
\end{remark}

\begin{definition}[相似矩阵]
  设$\mmA,\mmB$是两个同型矩阵。
  若存在满秩矩阵$P$,使得$\mmB=\mmP^{-1}\mmA\mmP$,
  则称$\mmB$是$\mmA$的\textbf{相似矩阵},记作$\mmA\sim\mmB$。
\end{definition}

\begin{remark}
  矩阵的相似是等价关系。
\end{remark}

\begin{theorem}[矩阵相似的等价条件]
  两个$n$阶方阵$\mmA,\mmB$相似当且仅当
  它们是$n$维线性空间$V$上的某一线性变换$T$在不同基下的矩阵。
\end{theorem}

\section{特征值与特征向量}
TODO
\subsection{特征值与特征向量}
对于$n$维线性空间$V$内的一个线性变换$T$,
我们总希望找一组基$\xi_1,\xi_2,\dots,\xi_n$,
使得$T$在这组基下的矩阵具有最简单的形式,即有
\begin{displaymath}
  (T(\xi_1), T(\xi_2),\dots,T(\xi_n)) = (\xi_1,\xi_2,\dots,\xi_n)
    \cdot \mdiag(\lambda_1,\lambda_2,\dots,\lambda_n)
\end{displaymath}
这样,根据式\ref{eq-coord-after-linear-trans},
任意$\alpha\in V$经过$T$变换后的坐标就是把它在
基$\xi_1,\xi_2,\dots,\xi_n$下的坐标放大常数倍。

根据上面的分析,我们需要找这样一组$\xi$和$\lambda$,使得$T(\xi) = \lambda\xi$。
这就引出了线性变换的特征值和特征向量的定义(注意,还不是矩阵的特征值与特征向量)。

\begin{definition}[线性变换的特征值和特征向量]
  设$V$是数域$\mfF$上的一个线性空间,$T$是$V$上的一个线性变换。
  如果对$\lambda\in\mfF$,存在非零向量$\xi\in V$,使得
  \[ T(\xi) = \lambda\xi \]
  则称$\lambda$是$T$的一个\textbf{特征值},
  而称$\xi$是$T$对应于$\lambda$的一个\textbf{特征向量}。
\end{definition}

\begin{theorem}[特征子空间]
  设$V$是数域$\mfF$上的线性空间,$T$是$V$上的线性变换,
  则给定一个特征值$\lambda\in\mfF$,
  \begin{displaymath}
    V_\lambda = \{ \xi\in V: T(\xi)=\lambda\xi \}
  \end{displaymath}
  构成了$V$的一个子空间。
  我们把它称为$T$对应于$\lambda$的特征子空间。
\end{theorem}

下面,我们要考虑如何求解线性变换的特征值与特征向量。
因为线性变换在给定一组基下与矩阵有着一一对应的关系,
所以我们尝试从矩阵开始入手。

我们设$\epsilon_1,\epsilon_2,\dots,\epsilon_n$是线性空间$V$的一组基,
$T$在这组基下的矩阵是$\mmA$。
设$\xi$是线性变换$T$的一个特征向量,它对应的特征值是$\lambda$,
它在基$\epsilon_1,\epsilon_2,\dots,\epsilon_n$下的坐标是
$\mvx=(x_1,\dots,x_n)^T$。
那么,根据式\ref{eq-coord-after-linear-trans},我们有
\[ T(\xi) = (\epsilon_1,\epsilon_2,\dots,\epsilon_n)\mmA\mvx \]
又因为
\[ T(\xi) = \lambda\xi = \lambda(\epsilon_1,\epsilon_2,\dots,\epsilon_n)\mvx \]
所以联立以上两个等式,我们就有
\begin{displaymath}
  (\epsilon_1,\epsilon_2,\dots,\epsilon_n)(\mmA\mvx - \lambda\mvx)=0
\end{displaymath}
因为$\epsilon_1,\epsilon_2,\dots,\epsilon_n$线性无关,
所以只可能是$\mmA\mvx - \lambda\mvx = \mvZero$,即
\begin{equation} \label{eq-eigen}
  (\mmA - \lambda\mmI)\mvx = \mvZero
\end{equation}

接下来我们需要解这个方程来得到特征值$\lambda$
和特征向量对应的坐标$\mvx$。
想要$\mvx$有非零解,充要条件是$|\mmA - \lambda\mmI| = 0$。
我们就能根据这一条件解出$\lambda$,
然后把$\lambda$代入上面的方程\ref{eq-eigen}中,
最后按照解方程的一般步骤,就能获得$\mvx$的基础解系。

我们还能发现,对式\ref{eq-eigen}变形,能得到类似于$T(\xi)=\lambda\xi$形式的等式
\begin{displaymath}
\mmA\mvx=\lambda\mvx
\end{displaymath}
而且上面一大堆分析都是针对矩阵$\mmA$的,和线性变换$T$基本上没关系。
所以,我们不如也定义矩阵的特征值与特征向量,先针对更加具体的矩阵进行研究,
再与$T$的特征值与特征向量联系起来。

\begin{definition}[矩阵的特征值和特征向量]
  对于$n$阶矩阵$\mmA$,记
  \[P(\lambda)=|\mmA-\lambda\mmI|\]
  $P(\lambda)$被称为$\mmA$的\textbf{特征多项式}。
  $P(\lambda)=0$的根称为$\mmA$的\textbf{特征值}或\textbf{特征根}。
  如果$\lambda$是$P(\lambda)=0$的$k$重根,
  那么又称$\lambda$是\textbf{$k$重特征值},$k$叫做\textbf{代数重数}。
  
  对于一个特征值$\lambda_0$,
  我们称方程$(\mmA-\lambda_0\mmI)\mvx=0$的非零解为
  $\mmA$对应于$\lambda_0$的\textbf{特征向量}。
  解空间线性无关的特征向量的个数叫做\textbf{几何重数}。
\end{definition}

\begin{remark}
  线性变换和矩阵的特征值还是有区别的。
  线性变换的特征值属于线性空间的数域$\mfF$,
  而矩阵的特征值属于复数域。
\end{remark}

\begin{definition}[矩阵的谱]
  $n$阶矩阵$\mmA$的所有特征值$\lambda_1,\dots,\lambda_n$
  叫做矩阵$\mmA$的\textbf{谱}。
  $\max\{|\lambda_1|,\dots,|\lambda_n|\}$被称为\textbf{谱半径}。
\end{definition}

\begin{definition}[矩阵的迹]
  设矩阵$\mmA=(a_{ij})_{n\times n}$。
  定义$\sum_{i=1}^{n}a_{ii}$叫做矩阵$\mmA$的迹,记为$\mtr\mmA$。
\end{definition}

\begin{theorem}[特征值的性质]
  设$n$阶矩阵$\mmA$的所有特征值为$\lambda_1,\dots,\lambda_n$,则
  \begin{enumerate}
    \item $|\mmA| = \prod_{i=1}^{n}\lambda_i$
    \item $\mtr\mmA = \sum_{i=1}^{n}\lambda_i$
  \end{enumerate}
\end{theorem}

\subsection{矩阵之间特征值的关联}
\begin{theorem}[矩阵运算对特征值和特征向量的影响]
  设$\mmA$是$n$阶方阵,$\lambda_0$是$\mmA$的一个特征值,
  $\alpha_0$是对应$\lambda_0$的一个特征向量。
  \begin{enumerate}
    \item
    对于逆矩阵,有$\mmA^{-1}\alpha_0 = \frac{1}{\lambda_0}\alpha_0$
    \item
    对于伴随矩阵,有$\mmA^*\alpha_0 = \frac{|\mmA|}{\lambda_0}\alpha_0$
    \item
    对于转置矩阵,它的特征多项式与$\mmA$的特征多项式相同,
    所以它们有相同的特征值。但是特征向量未必相同。
    \item
    对于矩阵多项式$f(\mmA)=\sum_{i=1}^{k} c_i\mmA^i$,
    有$f(\mmA)\alpha_0=\left(\sum_{i=1}^{k} c_i\lambda_0^i\right)\alpha_0$
  \end{enumerate}
\end{theorem}

\begin{remark}
  如果知道$\mmA$的特征值和特征向量,就可以利用上述定理
  快速求出这些相关的矩阵的特征值与特征向量。
\end{remark}

\begin{theorem}[相似矩阵之间特征值的联系]
  相似矩阵有相同的特征多项式,从而也有相同的谱。
\end{theorem}

\begin{remark}
  这符合直观:相似矩阵是同一个线性变换在不同基下的矩阵,
  因此求出的特征值都应该是相同的。
\end{remark}

\subsection{矩阵的对角化}
TODO


\chapter{欧几里得空间}

\section{欧几里得空间}
我们知道线性空间是对向量的抽象,而向量还有内积的概念没有出现在线性空间中。
所以,本小节在实线性空间的基础上引入内积操作,
从而建立了长度、交角、正交性的概念。
TODO


\part{概率论}
TODO

\part{数理统计}
TODO

% TODO: \printindex

\end{document}