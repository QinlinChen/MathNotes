\documentclass[hyperref,a4paper,UTF8]{ctexbook}

\usepackage{amsmath}
\usepackage{amsfonts}
\usepackage{amssymb}
\usepackage{amsthm}
\usepackage{textcomp}

\theoremstyle{definition} \newtheorem{definition}{定义}[chapter]
\theoremstyle{definition} \newtheorem{theorem}[definition]{定理}
\theoremstyle{definition} \newtheorem{lemma}[definition]{引理}
\theoremstyle{definition} \newtheorem{corollary}[definition]{推论}
\theoremstyle{remark} \newtheorem*{remark}{Remark}

% math field begin
\newcommand{\mf}[1]{\mathbb{#1}}
\newcommand{\mfN}{\mf{N}}
\newcommand{\mfZ}{\mf{Z}}
\newcommand{\mfR}{\mf{R}}
\newcommand{\mfQ}{\mf{Q}}
% math field end

% math matrix begin
\newcommand{\mm}[1]{\mathbf{#1}}
\newcommand{\mmA}{\mm{A}}
\newcommand{\mmB}{\mm{B}}
\newcommand{\mmC}{\mm{C}}
\newcommand{\mmE}{\mm{E}}
\newcommand{\mmI}{\mm{I}}
\newcommand{\mmP}{\mm{P}}
\newcommand{\mmQ}{\mm{Q}}
\newcommand{\mmZero}{\mm{0}}
\newcommand{\mmat}[2]{\left(\begin{array}{#1} #2 \end{array}\right)}
% three elementry row transformation and matrix
\newcommand{\mTfRowSwi}[2]{#1 \leftrightarrow #2}
\newcommand{\mTfRowMul}[2]{#1(#2)}
\newcommand{\mTfRowAdd}[3]{#1 #2 (#3)}
\newcommand{\mmRowSwi}[2]{P(\mTfRowSwi{#1}{#2})}
\newcommand{\mmRowMul}[2]{P(\mTfRowMul{#1}{#2})}
\newcommand{\mmRowAdd}[3]{P(\mTfRowAdd{#1}{#2}{#3})}
% math matrix end

% math vector begin
\newcommand{\mv}[1]{\vec{#1}}
\newcommand{\mvx}{\mv{x}}
% math vector end

% math function name begin
\newcommand{\mdiag}{\textrm{diag}}
% math function name end

\begin{document}
\author{陈钦霖}
\title{数学笔记}
\date{\today}

\maketitle
\tableofcontents

\part{微积分}
\chapter{分析基础}

\section{基础概念}
\subsection{邻域}
对于实数$a\in\mfR$,定义它的邻域为
$N(a,\delta) = \{ x: |x-a| < \delta \}$,
其中,实数$\delta > 0$。

同样,可以定义$a$的左邻域为
$N(a,\delta)_+ = \{ x: 0 \le x - a < \delta \}$,
右邻域为
$N(a,\delta)_- = \{ x: -\delta < x - a \le 0  \}$。

\subsection{常用等式与不等式}
\begin{enumerate}
  \item 
  $1^2+2^2+\dots +n^2 = n(n+1)(2n+1)/6$
  \item
  $(\sum_{i}a_i b_i)^2 \le (\sum_{i}a_i^2)(\sum_{i}b_i)^2$
  \item
  $\sin x < x < \tan x \quad (0 < x < \pi/2)$
  \item
  若$x_1,\dots,x_n$符号相同且都大于$-1$,那么
  \[(1+x_1)(1+x_2)\dots(1+x_n)\ge 1+x_1+x_2+\dots+x_n\]
\end{enumerate}

\section{极限的概念}
\chapter{一元函数的微分学}

\section{导数与微分的概念}
本小节给出导数与微分的定义。
并且介绍导数与连续的关系、导数与微分的关系。

\subsection{导数的概念}
\begin{definition}[导数]
  设函数$f$在点$a$的某一邻域内有定义。
  若极限
  \begin{displaymath}
    \lim_{\Delta x\to 0}\frac{f(a+\Delta x)-f(a)}{\Delta x}
  \end{displaymath}
  存在有限,则称函数$f$在点$a$\textbf{可导},
  并称此极限值为$f$在点$a$的\textbf{导数},记为
  \begin{displaymath}
    f'(a)\quad\textrm{或}\quad \left.\frac{\md f}{\md x}\right|_a
  \end{displaymath}
\end{definition}

\begin{definition}[左导数和右导数]
  极限
  \begin{displaymath}
    \lim_{\Delta x\to 0-}\frac{f(a+\Delta x)-f(a)}{\Delta x}
    \quad\textrm{以及}\quad
    \lim_{\Delta x\to 0+}\frac{f(a+\Delta x)-f(a)}{\Delta x}
  \end{displaymath}
  被称为函数$f$在点$a$的\textbf{左导数}和\textbf{右导数}
\end{definition}
\begin{remark}
  $f$在点$a$可导$\iff$ $f$在点$a$的左导数和右导数存在且相等。
\end{remark}

\begin{theorem}[可导与连续的关系]
  若函数$f$在点$a$可导,则$f$在点$a$连续。
\end{theorem}
\begin{remark}
  反之不成立。即在点$a$连续的函数未必在点$a$可导。
\end{remark}

\subsection{微分的概念}
\begin{definition}[微分]
  设函数$f$在区间$I$上有定义。
  对点$a\in I$,当自变量有增量$\Delta x$时,
  相应地,因变量$y$有增量$\Delta y=f(a+\Delta x)-f(a)$。
  若当$\Delta x\to 0$时,有
  \begin{displaymath}
    \Delta y = A\Delta x +o(\Delta x)
  \end{displaymath}
  其中$A$与$\Delta x$无关(一般与$a$有关),
  则称函数$f$在点$a$\textbf{可微},
  且称$A\Delta x$为$f$在点$a$的\textbf{微分},记作
  \begin{displaymath}
    \md f(a)\quad\textrm{或}\quad \left.\md y\right|_a
  \end{displaymath}
\end{definition}

\begin{remark}
  当$\Delta x\to 0$时,无穷小$\md y$是$\Delta y$的主部。
  又因为$\md y$关于$\Delta x$是一次的,
  故称$\md y$是$\Delta y$的\textbf{线性主部}。
\end{remark}

\begin{theorem}[可导与可微等价]
  函数$f$在点$a$可微的充要条件是
  $f$在点$a$存在有限导数$f'(a)$,
  于是就有
  \begin{displaymath}
    \md y = f'(a)\Delta x
  \end{displaymath}
\end{theorem}

\begin{remark}
  我们约定自变量$x$的增量$\Delta x$为自变量的微分,
  即$\md x = \Delta x$,
  于是上面的微分可以写成
  \begin{displaymath}
    \md y = f'(a)\md x
  \end{displaymath}
  需要注意的是,
  $\frac{\md y}{\md x} = f'(x)$中的左半部分应当视为一个整体,
  而不是一个分子除以分母的形式。
  只是因为上述定理的存在,
  才使得看起来只要把等式左边的分母移到右边就能得到$f$的微分形式。
  事实上,\textbf{该定理只对一元函数成立}。
\end{remark}

\section{导数与微分的计算}
本小节首先说明了导数和微分的四则运算法则。
然后讨论了复合函数的求导法则以及一阶微分形式的不变性。
接着讨论了反函数导数的求法。
最后,我们列出一些常用且容易遗忘的函数的导数。

\subsection{四则运算}
\begin{theorem}[导数的四则运算]
  设函数$u,v$在点$x$可导,那么
  \begin{enumerate}
    \item 
    $y=u\pm v$在点$x$可导,且有
    \begin{displaymath}
      (u\pm v)'=u'\pm v'
    \end{displaymath}
    \item 
    $y=uv$在点$x$可导,且有
    \begin{displaymath}
      (uv)' = u'v+uv'
    \end{displaymath}
    \item 
    当$v(x)\neq 0$时,$y=u/v$在点$x$可导,且有
    \begin{displaymath}
      \left(\frac{u}{v}\right)'=\frac{u'v-uv'}{v^2}
    \end{displaymath}
  \end{enumerate}
\end{theorem}

\begin{theorem}[微分的四则运算]
  设函数$u,v$在点$x$可微,那么
  \begin{enumerate}
    \item 
    $\md (u\pm v) = \md u \pm \md v$
    \item 
    $\md (uv) = v\md u + u \md v$
    \item 
    $\md (\frac{u}{v}) = \frac{v\md u - u\md v}{v^2}$
  \end{enumerate}
\end{theorem}

\subsection{复合函数的导数}
\begin{theorem}[复合函数的导数]
  设函数$y=f(u)$与$u=u(x)$在$x_0$的邻域能构成复合函数,
  且$u=u(x)$在点$x_0$可导,$y=f(x)$在点$u_0=u(x_0)$可导。
  那么,复合函数$f\circ u$在点$x_0$可导,且有
  \begin{displaymath}
    (f\circ u)'(x_0) = f'(u_0)u'(x_0)
  \end{displaymath}
\end{theorem}

\begin{corollary}[一阶微分形式不变性]
  设$y=f(u)$,则无论$u$是自变量还是中间变量$u=u(x)$,
  其微分形式不变,都是
  \begin{displaymath}
    \md y = f'(u)\md u
  \end{displaymath}
\end{corollary}

\begin{remark}
  该定理对高阶微分不成立:若$y=f(x)$,但$x$是中间变量,
  即$x=u(t)$,那么有
  \begin{align*}
    \md^2 y
    &= \md (f'(x)\md x) \\
    &= f''(x)\md x^2 + f'(x)\md^2 x
      & (\text{不是$f''(x)\md x^2$}) \\
    &= f''(x)\left(u'(t)\md t\right)^2 + f'(x)(u''(t)\md t^2)\\
    &= \left(f''(x)u'(t)^2 + f'(x)u''(t)\right)\md t^2
  \end{align*}
\end{remark}

\subsection{反函数的导数}
\begin{theorem}[反函数的导数]
  设函数$x=\phi(y)$在某一区间$I$内严格单调,
  有在区间$I$内一点$y$出导数$\phi(y)$存在且不为零,
  则反函数$y=f(x)$在对应点$x$出具有导数$f'(x)$,且
  \begin{displaymath}
    f'(x) = \frac{1}{\phi'(y)}
  \end{displaymath}
\end{theorem}

\subsection{公式表}
下面是一些常用且容易遗忘的函数的导数:
\begin{center}
  \begin{tabular}{|c|c|}
    \hline 
    $f(x)$ & $f'(x)$ \\ 
    \hline 
    $a^x$ & $a^x\ln a$  \\ 
    \hline 
    $\log_a |x|$ & $\frac{1}{x\ln a}$ \\ 
    \hline 
    $\arcsin x$ & $\frac{1}{\sqrt{1-x^2}}$ \\ 
    \hline 
    $\arccos x$ & $-\frac{1}{\sqrt{1-x^2}}$ \\ 
    \hline 
    $\arctan x$ & $\frac{1}{1+x^2}$ \\ 
    \hline 
  \end{tabular} 
\end{center}

\section{高阶导数与高阶微分}

\section{微分中值定理}

\section{洛必达法则}

\section{泰勒公式}

\section{利用导数研究函数的性质}

\section{利用导数作函数的图形}


\part{线性代数}
\chapter{矩阵\ 行列式\ 线性方程组}

\section{矩阵及其运算}

\subsection{矩阵的概念}
\begin{definition}[矩阵]
  由$m\times n$个数排成$m$行$n$列的矩形数表
  \begin{equation} \label{eq:mat-def}
    \mmA = \mmat{cccc}{
      a_{11} & a_{12} & \cdots & a_{1n} \\
      a_{21} & a_{22} & \cdots & a_{2n} \\
      \vdots & \vdots & \ddots & \vdots \\
      a_{m1} & a_{m2} & \cdots & a_{mn} }
  \end{equation}
  称为一个$m\times n$矩阵。
  其中数$a_{ij}$称为矩阵$\mmA$的元素。
  $i$为行标,$j$为列标,
  $a_{ij}$是$\mmA$位于第$i$行第$j$列的元素,或简称$\mmA$的$(i,j)$元素。
  式\eqref{eq:mat-def}可简记为$\mmA=(a_{ij})_{m\times n}$或$(a_{ij})$。
\end{definition}

下面是一些特殊的矩阵:
\begin{itemize}
  \item
  $m\times 1$矩阵又称为\textbf{列矩阵}或$m$\textbf{维列向量},
  下文主要会用$\beta$来表示。
  \item
  $1\times n$矩阵又称为\textbf{行矩阵}或$n$\textbf{维行向量},
  下文主要会用$\alpha$来表示。
  \item
  $n\times n$矩阵又称为$n$\textbf{阶方阵}。
  对于$n$阶方阵$\mmA=(a_{ij})$,
  其对角线上的元素$a_{11},\dots,a_{nn}$又称\textbf{主对角线元素}。
  \item
  如果一个$n$阶方阵除主对角线元素外,其余元素都为0,
  那么我们称这种矩阵为\textbf{对角矩阵},记为
  \[ \mdiag(k_1, k_2\dots,k_n)= \mmat{cccc}{
    k_1 &     &        & \\
        & k_2 &        & \\
        &     & \ddots & \\
        &     &        & k_n } \]
  \item
  $\mmI = \mdiag(1,1,\dots,1)$被称为\textbf{单位矩阵}。
\end{itemize}

\begin{definition}[矩阵相等]
    如果$\mmA=(a_{ij})$与$\mmB=(b_{ij})$都是$m\times n$矩阵,
    且对$i=1,2,\dots,m,\ j=1,2,\dots,n$,有$a_{ij}=b_{ij}$,
    则称$\mmA$与$\mmB$相等,记作$\mmA=\mmB$。
\end{definition}

\subsection{矩阵的线性运算}
矩阵的线性运算包含\textbf{加法}和\textbf{数乘}。定义比较简单,在此不加赘述。

它们满足的性质与之后要定义的线性空间一致。

\subsection{矩阵的乘法}
\begin{definition}[矩阵的乘法]
  设$\mmA=(a_{ik})_{m\times s}, \mmB=(b_{kj})_{s\times n}$,令
  \begin{equation}
  c_{ij}=\sum_{k=1}^{s}a_{ik}b_{kj}
  \end{equation}
  则$\mmC=(c_{ij})_{m\times n}$被称为$\mmA$和$\mmB$的乘积,
  记为$\mmC = \mmA\mmB$。
\end{definition}

矩阵的乘法满足分配率和结合律,但不满足交换律。

\subsection{矩阵的转置}
\begin{definition}[矩阵的转置]
    设$\mmA=(a_{ij})_{m\times n}$,
    令$b_{ij}=a_{ji}$,
    则$\mmB=(b_{ij})_{n\times m}$被称为$\mmA$的转置矩阵,
    记为$\mmA^T$。
\end{definition}

转置矩阵具有以下性质:
\begin{enumerate}
  \item $\left(\mmA^T\right)^T = \mmA$
  \item $(\mmA+\mmB)^T=\mmA^T + \mmB^T$
  \item $(k\mmA)^T=k\mmA^T$
  \item $(\mmA\mmB)^T=\mmB^T\mmA^T$
\end{enumerate}

\begin{definition}[对称矩阵与反对称矩阵]
  对于$n$阶方阵$\mmA=(a_{ij})$,
  如果满足$\mmA^T=\mmA$,则称$\mmA$为\textbf{对称矩阵}。
  如果满足$\mmA^T=-\mmA$,则称$\mmA$为\textbf{反对称矩阵}。
\end{definition}

\begin{remark}
  不难看出,反对称矩阵的主对角线元素都为零。
\end{remark}

\section{矩阵的分块}
TODO
\chapter{线性空间\ 线性变换\ 欧几里得空间}

\section{线性空间}
在这一小节,我们把先前向量的概念进一步抽象,得到了线性空间的概念。
向量组有线性相关、线性无关、极大线性无关组等概念,
这在线性空间中就转变成纬度、基、坐标的概念。

线性空间和物理的联系也是很紧密的。
因为向量是物理的常用量,而线性空间就是从向量抽象来的。
回忆一下物理学的知识,我们知道一个物理量在不同参考系的表示是不一样的。
对应到线性空间上来,我们会发现一个元素在不同基底下的坐标也是不一样的。
所以我们要研究这些坐标与基底的变换关系。

最后我们会简单讨论一下子空间的概念。

\subsection{线性空间}
\begin{definition}[线性空间] \label{def-linear-space}
  设$V$是非空集合,$\mfF$是一个数域。
  若在$V$上定义两种运算:加法$\oplus$和数乘$\otimes$,满足
  \begin{description}
    \item[加法]
    \begin{enumerate}
      \item 封闭性:
      $\forall\alpha,\beta\in V,\ \alpha\oplus\beta\in V$。
      \item 交换律:
      $\forall\alpha,\beta\in V,\ \alpha\oplus\beta = \beta\oplus\alpha$
      \item 结合律:
      $\forall\alpha,\beta,\gamma\in V,\ 
        (\alpha\oplus\beta)\oplus\gamma = \alpha\oplus(\beta\oplus\gamma)$。
      \item 零元存在:
      $\exists\theta\in V,\forall\alpha\in V,\ \alpha\oplus\theta=\alpha$。
      \item 逆元存在:
      $\forall\alpha\in V,\exists\beta\in V,\ \alpha\oplus\beta=\theta$。
      我们把$\beta$记为``$-\alpha$''。
    \end{enumerate}  
    \item[数乘]
    \begin{enumerate}
      \item 封闭性:
      $\forall\lambda\in\mfF,\alpha\in V,\ \lambda\otimes\alpha\in V$。
      \item 结合律:
      $\forall\lambda_1,\lambda_2\in\mfF,\alpha\in V,\ 
        \lambda_1\otimes(\lambda_2\otimes\alpha) =
        (\lambda_1\lambda_2)\otimes\alpha$。
      \item 分配律1:
      $\forall\lambda_1,\lambda_2\in\mfF,\alpha\in V,\ 
        (\lambda_1\oplus\lambda_2)\otimes\alpha =
        (\lambda_1\otimes\alpha)\oplus(\lambda_2\otimes\alpha)$
      \item 分配律2:
      $\forall\lambda\in\mfF,\alpha,\beta\in V,\ 
        \lambda\otimes(\alpha\oplus\beta) =
        (\lambda\otimes\alpha)\oplus(\lambda\otimes\beta)$。
      \item 单位元存在:
      $\forall\alpha\in V,\ 
        1\otimes\alpha = \alpha$。
    \end{enumerate}
  \end{description}
  则称$V$是数域$\mfF$上的一个线性空间。
  若$\mfF$是实数域$\mfR$,则称$V$是\textbf{实线性空间}。
  若$\mfF$是复数域$\mfC$,则称$V$是\textbf{复线性空间}。
\end{definition}

\begin{remark}
  如果有点抽象代数的背景,读者不难发现上述加法满足的是阿贝尔群的性质。
\end{remark}

\begin{theorem}[线性空间的性质]
  设$V$是$\mfF$上的线性空间,那么
  \begin{enumerate}
    \item 加法零元的具有唯一性
    \item 加法逆元的具有唯一性
    \item 加法零元的求法:
    $\forall\alpha\in V,\ 0\otimes\alpha=\theta$
    \item 加法逆元的求法:
    $\forall\alpha\in V,\ (-1)\otimes\alpha=-\alpha$
    \item $\forall k\in\mfF,\ k\otimes\theta=\theta$
    \item 若$\lambda\otimes\alpha=\theta$,则$\lambda=0$或$\alpha=\theta$。
  \end{enumerate}
\end{theorem}
为了书写与阅读的方便,以后``$\oplus$''用正常的加号来表示,
``$\otimes$''可省略。参考向量的记法。

\subsection{基与坐标}
向量组的线性相关、线性无关、线性表示的概念也适用于线性空间。

\begin{definition}[线性相关与线性无关]
  设$V$是线性空间,$\alpha_1,\alpha_2,\dots,\alpha_s\in V$。
  若数域中存在一组不全为0的数$k_1,k_2,\dots,k_s$,使得
  \[ k_1\alpha_1 + k_2\alpha_2 + \dots + k_s\alpha_s = \theta \]
  则称$\alpha_1,\alpha_2,\dots,\alpha_s$\textbf{线性相关}。
  否则,则称\textbf{线性无关}。
\end{definition}

\begin{definition}[线性表示]
  设$V$是线性空间,$\alpha,\alpha_1,\alpha_2,\dots,\alpha_s\in V$。
  若数域中存在一组数$k_1,k_2,\dots,k_s$,使得
  \[ \alpha = k_1\alpha_1 + k_2\alpha_2 + \dots + k_s\alpha_s \]
  则称$\alpha$为$\alpha_1,\alpha_2,\dots,\alpha_s$的\textbf{线性组合},
  也称$\alpha$可以由$\alpha_1,\alpha_2,\dots,\alpha_s$\textbf{线性表示}。
\end{definition}

\begin{definition}[维数]
  如果一个线性空间$V$中,线性无关的元素的最大个数是$n$,
  则称该线性空间是$n$维的,记作$\mdim V = n$。
  
  如果对任意正整数$N$,总存在$N$个线性无关的元素,
  则称该线性空间是\textbf{无穷维线性空间}。
  不是无穷维的线性空间叫做有穷维线性空间。
\end{definition}

\begin{remark}
  本章中我们不讨论无穷维线性空间。
\end{remark}

\begin{definition}[基与坐标]
  若$\alpha_1,\alpha_2,\dots,\alpha_n$是$n$维线性空间$V$中一组线性无关的元素,
  且$V$中任意元素$\alpha$都可由$\alpha_1,\alpha_2,\dots,\alpha_n$线性表示,
  即\[ \alpha = k_1\alpha_1 + k_2\alpha_2 + \dots + k_n\alpha_n \]
  则称$\alpha_1,\alpha_2,\dots,\alpha_n$是$V$的一组\textbf{基底}(简称\textbf{基})。
  其中,有序元组$(k_1,k_2,\dots,k_n)$称为$\alpha$在
  基底$\alpha_1,\alpha_2,\dots,\alpha_n$下的坐标。
\end{definition}

\begin{remark}
  线性空间的基对应的是极大线性无关组的概念。
\end{remark}

\subsection{基变换与坐标变换}
\begin{theorem}[基底变换公式]
  设$\alpha_1,\alpha_2,\dots,\alpha_n$与$\beta_1,\beta_2,\dots,\beta_n$是
  线性空间$V$的两组基底,那么可以写成
  \begin{equation} \label{eq-basis-transform}
    (\beta_1,\beta_2,\dots,\beta_n) = (\alpha_1,\alpha_2,\dots,\alpha_n)
    \mmat{cccc}{
      a_{11} & a_{12} & \cdots & a_{1n} \\
      a_{21} & a_{22} & \cdots & a_{2n} \\
      \vdots & \vdots & \ddots & \vdots \\
      a_{m1} & a_{m2} & \cdots & a_{mn} }
  \end{equation}
  我们把它简记为
  \begin{displaymath}
    \mmBasis{\beta} = \mmBasis{\alpha}\cdot\mmP
  \end{displaymath}
  我们称$\mmP$是$\mmBasis{\alpha}$到$\mmBasis{\beta}$的\textbf{过渡矩阵},
  式\ref{eq-basis-transform}是\textbf{基底变换公式}。
\end{theorem}

\begin{remark}
  过渡矩阵$\mmP$是非奇异矩阵。
  因为如果$\mmP$是奇异的,就存在$\mvx\neq\mvZero$,使得$\mmP\mvx=\mvZero$。
  从而有$\mmBasis{\beta}\mvx = \mmBasis{\alpha}\mmP\mvx = \mvZero$。
  于是$\mmBasis{\beta}$是奇异矩阵,这与``$\beta_1,\dots,\beta_n$是基底''是矛盾的。
\end{remark}

\begin{theorem}[坐标变换公式]
  设$\mmBasis{\alpha}$和$\mmBasis{\beta}$是线性空间$V$的两组基底,
  $\mmP$是从$\mmBasis{\alpha}$到$\mmBasis{\beta}$的过渡矩阵。
  若$\mvx=(x_1,\dots,x_n)^T$和$\mvy=(y_1,\dots,y_n)^T$
  分别是元素$\alpha\in V$在基底$\mmBasis{\alpha}$和$\mmBasis{\beta}$下的坐标,即
  \[ \alpha = \mmBasis{\beta}\mvy = \mmBasis{\alpha}\mvx \]
  那么有
  \begin{equation} \label{eq-coordinate-transform}
    \mvy = \mmP^{-1}\mvx
  \end{equation}
  式\ref{eq-coordinate-transform}被称为\textbf{坐标变换公式}。
\end{theorem}

\subsection{子空间}
\begin{definition}
  设$V$是数域$\mfF$上的线性空间。若$W\subset V$也是数域$\mfF$上的线性空间,
  则称$W$是$V$的\textbf{子空间}。
  $W=\{\theta\}$叫做\textbf{平凡子空间}。
\end{definition}

\begin{theorem}[子空间的充要条件]
  设$V$是数域$\mfF$上线性空间。
  $W\subset V$是$V$的子空间的充要条件是:
  1. 对加法封闭;2. 对乘法封闭。
  换句话说,即是$\forall\alpha,\beta\in W,\lambda,\mu\in\mfF$,
  \[ \lambda\alpha + \mu\beta \in W \]
\end{theorem}

\begin{definition}[子集张成的子空间]
    设$V$是数域$\mfF$上的线性空间,$\alpha_1,\alpha_2,\dots,\alpha_s\in V$。
    我们定义$V$的子空间
    \begin{displaymath}
    \mspan\{ \alpha_1,\alpha_2,\dots,\alpha_s \} =
    \{ \alpha: \alpha=\sum_{i=1}^{s}k_i\alpha_i, \forall k_1,\dots,k_s\in\mfF \}
    \end{displaymath}
    称作由向量组$\alpha_1,\alpha_2,\dots,\alpha_s$张成的子空间。
\end{definition}

\begin{definition}[线性空间的和]
  设$W_1,W_2$是线性空间的两个子空间,则它们的\textbf{和}定义为
  \begin{displaymath}
    W_1+W_2 = \{ u: u=\alpha+\beta, \forall\alpha\in W_1,\beta\in W_2 \}
  \end{displaymath}
\end{definition}

\begin{theorem}
  设$V$是线性空间,$W_1,W_2$是$V$的子空间,那么有以下结论:
  \begin{enumerate}
    \item $W_1\cap W_2$是$V$的子空间。
    \item $W_1\cup W_2$不是$V$的子空间。
    \item $W_1+W_2$是$V$的子空间。
  \end{enumerate}
\end{theorem}

\begin{theorem}[维数定理]
  设$W_1,W_2$是线性空间$V$的两个有限维子空间,则有
  \begin{displaymath}
  \mdim W_1 + \mdim W_2 = \mdim(W_1+W_2) + \dim(W_1\cap W_2)
  \end{displaymath}
\end{theorem}

\begin{definition}[直接和]
  设$V_1,V_2$是线性空间$V$的两个子空间。
  若对任意$\alpha\in V$,存在唯一的$\alpha_1\in V_1, \alpha_2\in V_2$,
  使得$\alpha=\alpha_1+\alpha_2$,
  则称$V$是$V_1$和$V_2$的\textbf{直接和}或\textbf{直和},
  记作$V=V_1\oplus V_2$。
  也称$V_1$和$V_2$是$V$内的\textbf{互补空间}。
\end{definition}

\begin{theorem}[直接和的等价条件]
  设$V_1,V_2$是线性空间$V$的两个子空间。
  \begin{align*}
    V=V_1\oplus V_2
    &\iff V = V_1 + V_2\ \text{且}\ V_1\cap V_2 = \{ \theta \} \\
    &\iff \mdim (V_1\oplus V_2) = \mdim V_1 + \mdim V_2
  \end{align*}
\end{theorem}

\section{线性变换}
如果说线性空间是对向量的抽象,那么线性变换就是对矩阵的抽象。
回想一下,我们在上一小节研究了基与坐标变换的公式,
公式中就是用矩阵来刻画了``变换''这个过程。
在介绍换线性变换的基本概念之后,
我们会证明,在取定一组基下,线性变换与矩阵有着一一对应的关系。

虽然线性变换在一组基下仅有唯一的矩阵与之对应,
但换个角度来看,有不同的基就会有不同的矩阵与之对应。
这些矩阵间又有什么关系呢?这就引出了相似矩阵的概念。

\subsection{线性变换的概念}
\begin{definition}[线性变换]
  设$V,W$是数域$\mfF$上的线性空间。
  函数$T: V\mapsto W$被称为$V$到$W$的\textbf{变换}。
  若对任意$\alpha,\beta\in V, k \in\mfF$,$T$满足
  \[ T(\alpha+\beta)=T(\alpha)+T(\beta),\quad T(k\alpha) = kT(\alpha) \]
  则称$T$是$V$到$W$的\textbf{线性变换}。
\end{definition}

\begin{theorem}[线性变换的性质]
  设$V,W$是数域$\mfF$上的线性空间,$T$是$V$到$W$的线性变换,
  $\theta_1$和$\theta_2$分别是$V$和$W$上的零元。
  $T$满足以下性质:
  \begin{enumerate}
    \item
    $T(\theta_1) = \theta_2$
    \item
    $\forall\alpha,\beta\in V,\lambda,\mu\in\mfF,\ 
      T(\lambda\alpha+\mu\beta)=\lambda T(\alpha)+\mu T(\beta)$
    \item
    若$V$中的$\alpha_1,\alpha_2,\dots,\alpha_s$线性相关,
    则$T(\alpha_1),T(\alpha_2),\dots,T(\alpha_s)$也线性相关。
  \end{enumerate}
\end{theorem}

\begin{definition}[特殊的线性变换]
  设$V$是数域$\mfF$上的线性空间,$T$是$V$上的线性变换。
  \begin{enumerate}
    \item
    若对任意$\alpha\in V$,$T(\alpha)=\theta$,
    则称$T$为\textbf{零变换},常用$T_0$来表示。
    \item
    若对任意$\alpha\in V$,$T(\alpha)=\alpha$,
    则称$T$为\textbf{恒等变换},常用$E$或$I$表示。
    \item 
    若对任意$\alpha\in V$,$T(\alpha)=k\alpha$,其中$k\in\mfF$,
    则称$T$为\textbf{数乘变换},常用$T_k$来表示。
  \end{enumerate}
\end{definition}

\begin{definition}[象空间与核空间]
  设$T$是线性空间$V$到$W$的线性变换,
  则$T$的\textbf{象空间}定义为:
  \[ \mim(T) = \{ \xi: \xi=T(\alpha), \alpha\in V\} \]
  $T$的\textbf{核空间}定义为:
  \[ \mker(T) = \{ \alpha: T(\alpha)=\theta, \alpha\in V \} \]
\end{definition}

\begin{theorem}[象空间与核空间的性质]
  设$T$是线性空间$V$到$W$的线性变换,那么
  \begin{enumerate}
    \item 
    $\mim(T)$是$V$的子空间,$\mker(T)$是$W$的子空间。
    \item
    $\mdim(\mim(T)) + \mdim(\ker(T)) = \dim(V)$
  \end{enumerate}
\end{theorem}

\subsection{线性变换的运算和可逆线性变换}
设$L(V)$是线性空间$V$上所有线性变换构成的集合。
类似矩阵,给定一个数域$\mfF$,
我们也可以在$L(V)$上定义加法、数乘和乘法运算:
\begin{description}
  \item[加法]
  对任意$T_1,T_2\in L(V),\alpha\in V$,$(T_1+T_2)(\alpha)=T_1(\alpha)+T_2(\alpha)$。
  \item[数乘]
  对任意$T\in L(V),\alpha\in V$,$(kT)(\alpha)=kT(\alpha)$,
  其中$k\in\mfF$。
  \item[乘法]
  因为线性变换定义上是个函数,所以两个线性变换相乘定义为两个函数的复合。
\end{description}

\begin{theorem}
  拥有上面定义的加法、数乘规则的$L(V)$是线性空间。
\end{theorem}

\begin{definition}[可逆线性变换]
  设$T$是线性空间$V$上的一个线性变换。
  如果$V$上存在一个变换$\sigma$,使得
  \[ T\sigma = \sigma T = E \]
  其中$E$是$V$上的恒等变换,则称$\sigma$是$T$的\textbf{逆变换}。
  不难看出,如果$T$的逆变换存在,那么它必然是唯一的,
  因此,我们把$T$的逆变换记作$T^{-1}$,
  并称$T$为\textbf{可逆线性变换}。
\end{definition}

\subsection{线性变换的矩阵表示}
接下来,我们讨论在给定一组基下,线性变换与矩阵有一一对应的关系。
需要注意的是,我们这里讨论的线性变换是从一个线性空间到自身的变换,
所以这里的矩阵也就是方阵。

首先,如果知道一个$n$维线性空间$V$上的线性变换$T$,
我们就能知道它在基$\epsilon_1,\epsilon_2\dots,\epsilon_n$下对应的矩阵:
对任意$i=1,2,\dots,n$,应该有一组坐标$a_{1i},a_{2i},\dots,a_{ni}$,使得
\[ T(\epsilon_i) = a_{1i}\epsilon_1+a_{2i}\epsilon_2+\dots+a_{ni}\epsilon_n \]
写成矩阵的形式,即是
\[ (T(\epsilon_1), T(\epsilon_2),\dots,T(\epsilon_n)) =
  (\epsilon_1,\epsilon_2,\dots,\epsilon_n)\mmA \]
其中$\mmA=(a_{ij})_{n\times n}$。
这样,对任意$\alpha=x_1\epsilon_1+x_2\epsilon_2+\dots+x_n\epsilon_n\in V$,
我们就能通过矩阵求出它经过线性变换后元素:
\[ T(\alpha)=(\epsilon_1,\epsilon_2,\dots,\epsilon_n)\mmA
  (x_1,x_2,\dots,x_n)^T \]
因此,我们称$\mmA$是
\textbf{线性变换$T$在基$\epsilon_1,\epsilon_2\dots,\epsilon_n$下的矩阵}。

反之,在线性空间$V$的一组基$\epsilon_1,\epsilon_2\dots,\epsilon_n$下,
给定矩阵$\mmA=(a_{ij})_{n\times n}$,
存在线性变换$T$,使得
\[ (T(\epsilon_1), T(\epsilon_2),\dots,T(\epsilon_n)) =
  (\epsilon_1,\epsilon_2,\dots,\epsilon_n)\mmA \]
这个$T$是这样构造的:
对任意$\alpha=x_1\epsilon_1+x_2\epsilon_2+\dots+x_n\epsilon_n\in V$,
\[ T(\alpha) = (\epsilon_1,\epsilon_2,\dots,\epsilon_n)\mmA
  (x_1,x_2,\dots,x_n)^T \]

总结一下上面的论述,有如下定理:
\begin{theorem}
  在线性空间$V$的一组基$\epsilon_1,\epsilon_2\dots,\epsilon_n$下,
  线性变换$T$与$n$阶方阵$\mmA$一一对应。
  $\mmA$的第$i$列就是$T(\epsilon_i)$在
  基$\epsilon_1,\epsilon_2\dots,\epsilon_n$下的坐标。
\end{theorem}

\subsection{相似矩阵}
\begin{theorem}
  设$n$为线性空间$V$的两组基底为$\epsilon_1,\epsilon_2\dots,\epsilon_n$以及
  $\eta_1,\eta_2,\dots,\eta_n$。由$\epsilon_1,\epsilon_2\dots,\epsilon_n$到
  $\eta_1,\eta_2,\dots,\eta_n$的过渡矩阵为$\mmP$。
  $V$上线性变换$T$在这两组基下的矩阵分别为$\mmA,\mmB$,那么有
  \[ \mmB = \mmP^{-1}\mmA\mmP \]  
\end{theorem}

\begin{remark}
  直观的理解就是,
  \begin{displaymath}
    (\epsilon_1,\epsilon_2\dots,\epsilon_n)\xrightarrow{\mmP}
    (\eta_1,\eta_2,\dots,\eta_n)\xrightarrow{\mmB}
    T(\eta_1,\eta_2,\dots,\eta_n)
  \end{displaymath}
  等效于
  \begin{displaymath}
  (\epsilon_1,\epsilon_2\dots,\epsilon_n)\xrightarrow{\mmA}
  T(\epsilon_1,\epsilon_2\dots,\epsilon_n)\xrightarrow{\mmP}
  T(\eta_1,\eta_2,\dots,\eta_n)
  \end{displaymath}
  所以有$\mmP\mmB=\mmA\mmP$。变形即是上面定理的结果。
  这也是这个定理的证明思路。
\end{remark}

\begin{definition}[相似矩阵]
  设$\mmA,\mmB$是两个同型矩阵。
  若存在满秩矩阵$P$,使得$\mmB=\mmP^{-1}\mmA\mmP$,
  则称$\mmB$是$\mmA$的\textbf{相似矩阵},记作$\mmA\sim\mmB$。
\end{definition}

\begin{remark}
  矩阵的相似是等价关系。
\end{remark}

\begin{theorem}[矩阵相似的等价条件]
  两个$n$阶方阵$\mmA,\mmB$相似当且仅当
  它们是$n$维线性空间$V$上的某一线性变换$T$在不同基下的矩阵。
\end{theorem}

\section{特征值与特征向量}
TODO


\part{概率论}
TODO

\part{数理统计}
TODO

\end{document}