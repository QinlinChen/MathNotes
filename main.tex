\documentclass[hyperref,a4paper,UTF8]{ctexbook}

\usepackage{amsmath}
\usepackage{amsfonts}
\usepackage{amssymb}
\usepackage{amsthm}
\usepackage{textcomp}

\theoremstyle{definition} \newtheorem{definition}{定义}[chapter]
\theoremstyle{definition} \newtheorem{lemma}{引理}[chapter]
\theoremstyle{definition} \newtheorem{theorem}{定理}[chapter]
\theoremstyle{remark} \newtheorem*{remark}{Remark}

% math field
\newcommand{\mf}[1]{\mathbb{#1}}
\newcommand{\mfN}{\mf{N}}
\newcommand{\mfZ}{\mf{Z}}
\newcommand{\mfR}{\mf{R}}
\newcommand{\mfQ}{\mf{Q}}
% math matrix
\newcommand{\mm}[1]{\mathbf{#1}}
\newcommand{\mmA}{\mm{A}}
\newcommand{\mmB}{\mm{B}}
\newcommand{\mmC}{\mm{C}}
\newcommand{\mmE}{\mm{E}}
\newcommand{\mmI}{\mm{I}}
\newcommand{\mmat}[2]{\left(\begin{array}{#1} #2 \end{array}\right)}
% math vector
\newcommand{\mv}[1]{\vec{#1}}
\newcommand{\mvx}{\mv{x}}
% math function
\newcommand{\mdiag}{\textrm{diag}}

\begin{document}

\author{陈钦霖}
\title{数学笔记}
\date{\today}

\maketitle
\tableofcontents

\part{微积分}
\chapter{分析基础}

\section{基础概念}
这一节主要介绍了邻域的概念和常用的不等式。
关于函数的基础知识,这里就不再赘述了。

\subsection{邻域}
对于实数$a\in\mfR$,定义它的邻域为
$N(a,\delta) = \{ x: |x-a| < \delta \}$,
其中,实数$\delta > 0$。

同样,可以定义$a$的左邻域为
$N(a,\delta)_+ = \{ x: 0 \le x - a < \delta \}$,
右邻域为
$N(a,\delta)_- = \{ x: -\delta < x - a \le 0  \}$。

\subsection{常用等式与不等式}
\begin{enumerate}
  \item
  $1^2+2^2+\dots +n^2 = n(n+1)(2n+1)/6$
  \item
  $(\sum_{i}a_i b_i)^2 \le (\sum_{i}a_i^2)(\sum_{i}b_i)^2$
  \item
  $\sin x < x < \tan x \quad (0 < x < \pi/2)$
  \item
  若$x_1,\dots,x_n$符号相同且都大于$-1$,那么
  \[(1+x_1)(1+x_2)\dots(1+x_n)\ge 1+x_1+x_2+\dots+x_n\]
\end{enumerate}

\section{极限的概念}
这一节首先介绍数列极限和函数极限的定义。微积分的大厦自此开始建立。
然后我们介绍比较重要的函数左极限与右极限的概念,
这是因为函数极限存在与左极限、右极限之间有一定关系。
最后,我们介绍函数极限和数列极限的关系,
从而获得另一个判定函数极限存在的方法。

\subsection{数列的极限}
数列的极限主要讨论自变量$n$无限增大时(记为$n\to+\infty$或$n\to\infty$),
因变量$a_n$的变化趋向。
当$n\to\infty$时,如果$a_n$无限接近于某一定值,
则称$\{a_n\}$为\textbf{收敛数列},
否则称$\{a_n\}$为\textbf{发散数列}。

对于收敛数列,我们用下面的$\epsilon$-$N$语言来定义它的极限。
\begin{definition}[数列极限的$\epsilon$-$N$定义]
  \begin{displaymath}
    \lim_{n\to\infty}a_n=A
    \iff \forall\epsilon>0,\exists N,\forall n>N,|a_n - A|<\epsilon
  \end{displaymath}
\end{definition}
\begin{remark}
  当要描述数列不收敛到某一个值时,只要把上面的定义取否命题即可。
\end{remark}

对于发散数列,它有四种情况:发散到正无穷,发散到负无穷,绝对值发散到正无穷,以及振荡。
我们用$G$-$N$语言来描述前三种情况。
\begin{definition}[数列极限的$G$-$N$的定义]
  \begin{align*}
    &\lim_{n\to\infty}a_n=+\infty
    \iff \forall G>0,\exists N,\forall n>N, a_n > G \\
    &\lim_{n\to\infty}a_n=-\infty
    \iff \forall G>0,\exists N,\forall n>N, a_n < -G \\
    &\lim_{n\to\infty}a_n=\infty
    \iff \forall G>0,\exists N,\forall n>N, |a_n| > G
  \end{align*}
\end{definition}

\subsection{函数的极限}
函数的极限研究函数在自变量的某变化过程中相应的因变量的变化趋向。
与数列极限不同的是,数列极限的自变量只有一种变化趋向,即$n\to\infty$,
而函数极限的自变量有六种变化趋向。

对于自变量$x$变化的趋向,我们定义下面两类共6种模板:
\begin{itemize}
  \item $\delta$模板:
  \begin{align*}
    x\to a
    &\iff \dots, \exists \delta > 0,\forall x, 0<|x-a|<\delta, \dots \\
    x\to a+0
    &\iff \dots, \exists \delta > 0,\forall x, 0<x-a<\delta, \dots \\
    x\to a-0
    &\iff \dots, \exists \delta > 0,\forall x, -\delta<x-a<0, \dots
  \end{align*}
    \item $H$模板:
  \begin{align*}
    x\to+\infty
    &\iff \dots, \exists H > 0, \forall x > H, \dots \\
    x\to-\infty
    &\iff \dots, \exists H > 0, \forall x < -H, \dots \\
    x\to\infty
    &\iff \dots, \exists H > 0, \forall |x| > H, \dots
  \end{align*}
\end{itemize}

对于因变量$f(x)$,我们定义下面两类共4种模板:
\begin{itemize}
  \item $\epsilon$模板
  \begin{displaymath}
    f(x)\to A
    \iff \forall\epsilon>0, \dots, |f(x)-A|<\epsilon
  \end{displaymath}
  \item $G$模板
  \begin{align*}
    f(x)\to+\infty
    &\iff \forall G>0,\dots, f(x) > G \\
    f(x)\to-\infty
    &\iff \forall G>0,\dots, f(x) < -G \\
    f(x)\to\infty
    &\iff \forall G>0,\dots, |f(x)| > G
  \end{align*}
\end{itemize}

以上模板可以组合成函数极限的
``$\epsilon$-$\delta$''、``$\epsilon$-$H$''、``$G$-$\delta$''、``$G$-$H$''定义。
以下面将要说明的左极限为例,我们可以组合出左极限的$\epsilon$-$\delta$定义:
\begin{displaymath}
  \lim_{x\to a-0}f(x)=A
  \iff \forall\epsilon>0,\exists\delta>0,\forall x, -\delta<x-a<0, |f(x)-A|<\epsilon
\end{displaymath}

\subsection{左极限\ 右极限}
函数$f$在$x\to a-0$和$x\to a+0$存在有限极限$A$的情况特别重要,
分别称为函数$f$在点$a$的\textbf{左}、\textbf{右极限},
统称为\textbf{单侧极限},记为
\begin{displaymath}
  \lim_{x\to a-0}f(x)=A\quad\text{或}\quad f(a-0)=A\quad\text{或}\quad f(a-)=A
\end{displaymath}
\begin{displaymath}
\lim_{x\to a+0}f(x)=A\quad\text{或}\quad f(a+0)=A\quad\text{或}\quad f(a+)=A
\end{displaymath}

\begin{theorem}[函数极限与单侧极限的关系]
  \begin{displaymath}
    \lim_{x\to a}f(x)=A \iff f(a-0)=f(a+0)=A
  \end{displaymath}
\end{theorem}

\begin{corollary}
  若函数$f$在点$a$存在左、右极限但不相等,
  或$f$在点$a$的左、右极限至少有一个不存在,
  则$f$在点$a$不存在极限。
\end{corollary}
\begin{remark}
  该推论可以用来证明函数极限不存在。
\end{remark}

\subsection{数列极限与函数极限的关系}
\begin{theorem}[数列极限与函数极限的关系]
  \begin{displaymath}
    \lim_{x\to a} f(x) = A
    \iff \forall \{x_n\}: x_n\neq a, x_n\to a(n\to\infty),
    \text{有}\lim_{n\to\infty} f(x_n) = A
  \end{displaymath}
\end{theorem}
\begin{remark}
  数列极限可以看成函数极限的特例,
  函数极限又可以转化为数列极限来讨论。
\end{remark}

\begin{corollary}
  设$\{x_n\}$和$\{y_n\}$是两个包含在$a$的去心邻域中且极限收敛到$a$的数列。
  若$\lim_{n\to\infty}f(x_n)$和$\lim_{n\to\infty}f(y_n)$极限存在但不相等,
  或者其中至少有一个不存在,那么$\lim_{x\to a}f(x)$不存在。
\end{corollary}
\begin{remark}
  该推论为证明函数极限不存在提供了简单的方法。
\end{remark}

\section{极限的性质与运算法则} \label{sec:limit-property}
当函数或数列存在有限极限时,极限具有唯一性、有界性、保序性和夹逼性。
同时,极限还可以与四则运算交换顺序。
必须指出的是,上述性质及运算法则对函数极限在其自变量$x$的六种极限过程及
数列极限在$n\to\infty$的极限过程都是同样成立的。
为了节省篇幅,我们仅就极限过程$x\to a$和$n\to\infty$叙述。

\subsection{极限的性质}
以下为统一写法,用$X,Y$代表因变量$x_n$或$f(x)$,而$\lim$代表某同一过程的极限。

\begin{theorem}[唯一性]
  设$\lim X=A,\lim Y=B$,$A,B$为有限数,则$A=B$。
\end{theorem}

\begin{theorem}[收敛函数的局部有界性]
  若$\lim_{x\to a}f(x)=A$(有限数),则函数$f$在$a$点附近有界。
\end{theorem}

\begin{theorem}[收敛数列的整体有界性]
  若$\lim_{n\to \infty}x_n=A$(有限数),则$\{x_n\}$是有界数列。
\end{theorem}

\begin{theorem}[函数极限的保序性]
  设$\lim_{x\to a} f(x)=A,\lim_{x\to a} g(x)=B$,$A,B$为有限数,则
  \begin{enumerate}
    \item
    若$A<B$,则存在$a$的某个去心邻域$N(a,\delta)-\{a\}$,
    在该邻域内,有$f(x) < g(x)$。
    \item
    若在$a$的某个去心邻域内有$f(x) \le g(x)$,
    则$A\le B$。
  \end{enumerate}
\end{theorem}

\begin{corollary}[函数极限的保号性质] \label{thrm:limit-sign-preserve}
  设$\lim_{x\to a} f(x)=A$(有限数),则
  \begin{enumerate}
    \item
    若$A>C$($A<C$),则存在$a$的某个去心邻域,
    在该邻域内,有$f(x)>C$($f(x)<C$)。
    \item
    若在$a$的某个去心邻域内有$f(x)\le C$($f(x)\ge C$),
    则$A\le C$($A\ge C$)。
  \end{enumerate}
\end{corollary}

\begin{remark}
  数列极限的保序性和保号性质的推论与函数极限类似。这里不再赘述了。
\end{remark}

根据保号性,就能得到夹逼定理:
\begin{theorem}[夹逼定理] \label{thrm:squeeze}
  设$X\le Z\le Y$。若$\lim X = \lim Y = A$,则$\lim Z = A$。
\end{theorem}

\subsection{运算法则}
设$\lim X=A,\lim Y=B$,$A,B$为有限数,则
\begin{itemize}
  \item
  $\lim (X\pm Y) = A \pm B$
  \item
  $\lim (cX) = cA$,$c$是常数。
  \item
  $\lim (XY) = AB$
  \item
  $\lim (X/Y) = A/B$,$B\neq 0$
\end{itemize}

\subsection{复合函数的极限}
\begin{theorem}[复合函数的极限]
  若函数$y=g(u)$与$u=\phi(x)$在点$a$的去心邻域上能构成复合函数,且满足
  \begin{enumerate}
    \item $\lim_{x\to a}\phi(x)=b$,且$x\neq a$时,$\phi(x)\neq b$,
    \item $\lim_{u\to b}g(u)=A$(有限),
  \end{enumerate}
  则$\lim_{x\to a}g(\phi(x))=A$。
\end{theorem}

\section{极限的存在准则}
本小节的主要内容是极限存在的两个准则和两个重要极限。
此外,我们还简单介绍了一下刻画实数系的连续性的方法。

\subsection{实数系的连续性}
实数系的连续性是使求极限这一分析运算能封闭的必备条件。
下面列举了刻画实数系的连续性的7个等价定理:
\begin{enumerate}
  \item
  确界存在定理:非空有界数集必有确界。
  \item
  单调有界定理:单调有界的数列必存在有限极限。
  \item
  区间套定理:闭区间套可以套出一个点。
  \item
  有限覆盖定理:闭区间上的任意开覆盖必有有限子覆盖。
  \item
  聚点定理:有界无限点列必有聚点。
  \item
  致密性定理:有界数列必有收敛子列。
  \item
  柯西收敛准则:若对任意$\epsilon > 0$,存在$N$,使得
  对任意$m,n\ge N$,有$|a_m-a_n|\le \epsilon$,
  则$\{a_n\}$有极限。
\end{enumerate}
以上内容只是做个小拓展。我也理解得不够深刻。

\subsection{极限的存在准则}
极限存在的两个准则分别是:夹逼定理(见定理\ref{thrm:squeeze})和单调有界定理。

\begin{theorem}[数列的单调有界定理]
  单调有界的数列必存在有限极限。
\end{theorem}

\begin{theorem}[函数的单调有界定理]
  设$f$是定义在区间$I$上的单调有界函数,
  则$f$在$I$的任意一点上都存在有限的单侧极限。
\end{theorem}

\subsection{两个重要极限}
\begin{displaymath}
  \lim_{x\to 0}\frac{\sin x}{x}=1
\end{displaymath}
\begin{displaymath}
  \lim_{x\to\infty}\left(1+x\right)^{\frac{1}{x}} = e
  \quad\text{或}\quad
  \lim_{x\to 0}\left(1+\frac{1}{x}\right)^{x} = e
\end{displaymath}

\section{连续}
本小节首先介绍函数连续性的概念(含左连续与右连续)与间断点的类型。
然后讨论连续函数的性质和初等函数的连续性。
最后,我们给出闭区间上连续函数的性质,
包括有界性定理、最值定理、介值定理和零点定理。

\subsection{连续与间断}
\begin{definition}[函数在点$a$连续与间断]
  设函数在点$a$的某个邻域上有定义。
  如果$\lim_{x\to a}f(x)=f(a)$,
  则称$f$在点$a$\textbf{连续},
  否则称$f$在点$a$\textbf{间断}。
  若$f(a-0)=f(a)$,则称$f$在点$a$\textbf{左连续}。
  若$f(a+0)=f(a)$,则称$f$在点$a$\textbf{右连续}。
\end{definition}
\begin{remark}
  显然,$f$在点$a$连续的充要条件有
  \begin{displaymath}
    f(a+0)=f(a-0)=f(a)
  \end{displaymath}
\end{remark}

函数$f$在点$a$连续有以下等价的说法:
\begin{align*}
  \text{$f$在点$a$连续}
  &\iff \text{$f$在$a$的某邻域有定义且$\lim_{x\to a}f(x)=a$} \\
  &\iff \lim_{\Delta x\to 0}f(a+\Delta x)=f(a) \\
  &\iff \lim_{\Delta x\to 0}\Delta f = 0
\end{align*}

如果函数$f$在点$a$间断,那么可以把间断点分成以下两类:
\begin{center}
  \begin{tabular}{|c|l|l|}
    \hline
    间断点类型 & \multicolumn{1}{c|}{$f(a+0)$与$f(a-0)$}
      & \multicolumn{1}{c|}{说明} \\
    \hline
    \multirow{2}{*}{第I类间断点}
      & \begin{tabular}[c]{@{}l@{}}存在且相等,\\ 极限值为$A$\end{tabular}
      & \begin{tabular}[c]{@{}l@{}}也称\textbf{可去间断点},因$f$在点$a$\\
      无定义或$f(a)\neq A$引起间断\end{tabular} \\
    \cline{2-3}
    & 存在但不相等 & 也称\textbf{跳跃型间断点} \\
    \hline
    \multirow{2}{*}{第II间断点} & 至少有一个是无穷大 & 也称\textbf{无穷型间断点} \\
    \cline{2-3}
    & 至少有一个不存在 &  \\
    \hline
  \end{tabular}
\end{center}

\subsection{连续函数的局部性质与运算法则}
若函数$f$在点$a$连续,根据\ref{sec:limit-property}节描述的
存在有限极限的函数的局部性质和运算法则,
能得到以下定理:

\begin{theorem}[局部有界性]
  若函数$f$在点$a$连续,则$f$在点$a$的某个邻域里有界。
\end{theorem}

\begin{theorem}[局部保号性]
  若函数$f$在点$a$连续,且$f(a)\neq 0$,
  则函数$f$在点$a$的某邻域内与$f(a)$同号。

  又若$f(a)>p(<q)$,则存在$a$的某个邻域$U$,在$U$内有
  $f(x)>p(<q)$。
\end{theorem}

\begin{theorem}[四则运算]
  若函数$f,g$都在点$a$连续,则它们的和、差、积、商
  \begin{displaymath}
    f\pm g,\ f\cdot g,\ \frac{f}{g}(g(a)\neq 0)
  \end{displaymath}
  也都在点$a$连续。
\end{theorem}

\begin{theorem}[复合函数的连续性]
   若函数$y=g(u)$与$u=\phi(x)$在点$a$的邻域上能构成复合函数,
   且函数$\phi$在点$a$连续,函数$g$在点$b=\phi(a)$连续,
   则复合函数$g\circ\phi$在点$a$连续。
\end{theorem}

\subsection{初等函数的连续性}
\begin{theorem}[初等函数的连续性]
  初等函数在其自然定义域内都是连续的。
\end{theorem}

\subsection{闭区间上连续函数的性质}
\begin{theorem}[有界性定理] \label{thrm:continuous-func-bounded}
  若$f$在闭区间$[a,b]$上连续,那么$f$在闭区间$[a,b]$上有界,即
  存在$m,M$,使得对任意$x\in[a,b]$,有$m\le f(x)\le M$。
\end{theorem}

\begin{theorem}[最值定理]
  若$f$在闭区间$[a,b]$上连续,那么$f$在闭区间$[a,b]$上有最大值和最小值,即
  存在$x_1,x_2\in[a,b]$,使得对任意$x\in[a,b]$,有
  $f(x_1)\le f(x)\le f(x_2)$。
\end{theorem}

\begin{theorem}[介值定理]
  若$f$在闭区间$[a,b]$上连续,
  记$m=\min_{x\in[a,b]}f(x)$,$M=\max_{x\in[a,b]}f(x)$,
  则$f$取到介于$m,M$之间的所有值,即对任意$m<\mu<M$,
  必然存在$\xi\in(a,b)$,使得$f(\xi)=\mu$。
\end{theorem}
\begin{remark}
  介值定理可以推广到在任意区间$I$上也成立。只要把$m$和$M$的重新记为
  $\inf_{x\in I}f(x)$和$\sup_{x\in I}f(x)$。
\end{remark}

\begin{theorem}[零点定理]
  若$f$在闭区间$[a,b]$上连续,且$f(a)\cdot f(b) < 0$,
  则必存在一点$\xi\in(a,b)$,使得$f(\xi)=0$。
\end{theorem}

\section{无穷小量}
我们首先给出无穷小量的定义及其性质。
然后介绍无穷小量的比较方法。
最后讨论不定型的极限。
以下介绍的内容都适用于无穷大量。

\subsection{无穷小量及其性质}
\begin{definition}[无穷小量]
  在某一极限过程中,以零为极限的变量$\alpha$(数列或函数)
  称为该极限过程中的\textbf{无穷小量},常称为无穷小,
  记为$\alpha=o(1)$。
\end{definition}
\begin{remark}
  无穷大量的定义类似。
\end{remark}

\begin{theorem}[无穷小量与极限]
  $\lim_{x\to a}f(x)=A$(有限)$\iff f(x)=A+\alpha(x)$,
  其中$\alpha(x)\to 0\ (x\to a)$是一个无穷小量。
\end{theorem}

\begin{theorem}[无穷小量的性质]
  设$\alpha,\beta$是某个过程中的两个无穷小量,那么
  \begin{enumerate}
    \item
    $\alpha\pm\beta$是无穷小量。
    \item
    $\alpha\cdot\beta$是无穷小量。
    \item
    若$\gamma$有界,则$\alpha\cdot\gamma$是无穷小量。
  \end{enumerate}
\end{theorem}

\subsection{无穷小量的比较}
\begin{definition}[无穷小量的比较]
  设$\alpha,\beta$是同一极限过程中的两个无穷小量。
  \begin{enumerate}
    \item
    若$\lim \frac{\alpha}{\beta}=0$,
    则称$\alpha$是关于$\beta$的\textbf{高阶无穷小量},
    记为$\alpha = o(\beta)$。

    \item
    若$\lim \frac{\alpha}{\beta}=A$(非零有限),
    则称$\alpha$与$\beta$是\textbf{同阶无穷小量},
    记为$\alpha = \Theta(\beta)$。

    \item
    若$\lim \frac{\alpha}{\beta}=1$,
    则称$\alpha$与$\beta$是\textbf{等价无穷小量},
    记为$\alpha \sim \beta$。

    以上三种情况,$\alpha$的阶都不低于$\beta$的阶,
    可以记为$\frac{\alpha}{\beta}=O(1)$或$\alpha=O(\beta)$。

    \item
    若$\lim \frac{\alpha}{\beta^s}=A$(非零有限),
    其中$\alpha$是大于0的常数,
    则称$\alpha$是关于$\beta$的\textbf{$s$阶无穷小量}。
  \end{enumerate}
\end{definition}

\begin{remark}
  关于无穷小量的比较,有下面一些注意点:
  \begin{enumerate}
    \item
    在使用$o,\Theta,O,\sim$记号的时候,必须指出自变量趋于什么,例如:
    \begin{displaymath}
      y_n=o(x_n), n\to\infty,\text{是指}\lim_{n\to\infty}\frac{y_n}{x_n}=0
    \end{displaymath}
    \item
    $\alpha=o(\beta)$,$\alpha=\Theta(\beta)$,$\alpha=O(\beta)$
    中的等号有着特殊的含义。
    实际上,$o(\beta)$,$\Theta(\beta)$,$O(\beta)$表示函数类,
    所以等号的含义是``属于''。
    \item
    $o,\Theta,O$记号在算法复杂度分析中也用到了呢。
  \end{enumerate}
\end{remark}

\begin{theorem}[替换定理]
  设$X_1,X_2,Y_1,Y_2$是同一极限过程中的无穷小量。
  若$X_1\sim X_2$,$Y_1\sim Y_2$,且$\lim Y_1/X_1$存在,则
  \begin{displaymath}
    \lim \frac{Y_2}{X_2}=\lim \frac{Y_1}{X_1}
  \end{displaymath}
\end{theorem}

\begin{theorem}[常用的等价无穷小]
  当$x\to 0$时,
  \begin{align*}
    & \sin x \sim x;\quad \tan x \sim x;\quad 1-\cos x \sim \frac{1}{2}x^2;  \\
    & \arcsin x \sim x;\quad \arctan x \sim x;\quad \ln(x+1)\sim x; \\
    & e^x-1\sim x;\quad (1+x)^\mu - 1 \sim \mu x
  \end{align*}
\end{theorem}

\subsection{不定型}
不定型主要包括以下7类:
\begin{displaymath}
  \frac{0}{0};\ \frac{\infty}{\infty};\ \infty-\infty;\ 0\cdot\infty;
  \ 1^\infty;\ 0^0;\ \infty^0.
\end{displaymath}
\chapter{一元函数的微分学}

\section{导数与微分的概念}
本小节给出导数与微分的定义。
并且介绍导数与连续的关系、导数与微分的关系。

\subsection{导数的概念}
\begin{definition}[导数]
  设函数$f$在点$a$的某一邻域内有定义。
  若极限
  \begin{displaymath}
    \lim_{\Delta x\to 0}\frac{f(a+\Delta x)-f(a)}{\Delta x}
  \end{displaymath}
  存在有限,则称函数$f$在点$a$\textbf{可导},
  并称此极限值为$f$在点$a$的\textbf{导数},记为
  \begin{displaymath}
    f'(a)\quad\textrm{或}\quad \left.\frac{\md f}{\md x}\right|_a
  \end{displaymath}
\end{definition}

\begin{definition}[左导数和右导数]
  极限
  \begin{displaymath}
    \lim_{\Delta x\to 0-}\frac{f(a+\Delta x)-f(a)}{\Delta x}
    \quad\textrm{以及}\quad
    \lim_{\Delta x\to 0+}\frac{f(a+\Delta x)-f(a)}{\Delta x}
  \end{displaymath}
  被称为函数$f$在点$a$的\textbf{左导数}和\textbf{右导数}
\end{definition}
\begin{remark}
  $f$在点$a$可导$\iff$ $f$在点$a$的左导数和右导数存在且相等。
\end{remark}

\begin{theorem}[可导与连续的关系]
  若函数$f$在点$a$可导,则$f$在点$a$连续。
\end{theorem}
\begin{remark}
  反之不成立。即在点$a$连续的函数未必在点$a$可导。
\end{remark}

\subsection{微分的概念}
\begin{definition}[微分]
  设函数$f$在区间$I$上有定义。
  对点$a\in I$,当自变量有增量$\Delta x$时,
  相应地,因变量$y$有增量$\Delta y=f(a+\Delta x)-f(a)$。
  若当$\Delta x\to 0$时,有
  \begin{displaymath}
    \Delta y = A\Delta x +o(\Delta x)
  \end{displaymath}
  其中$A$与$\Delta x$无关(一般与$a$有关),
  则称函数$f$在点$a$\textbf{可微},
  且称$A\Delta x$为$f$在点$a$的\textbf{微分},记作
  \begin{displaymath}
    \md f(a)\quad\textrm{或}\quad \left.\md y\right|_a
  \end{displaymath}
\end{definition}

\begin{remark}
  当$\Delta x\to 0$时,无穷小$\md y$是$\Delta y$的主部。
  又因为$\md y$关于$\Delta x$是一次的,
  故称$\md y$是$\Delta y$的\textbf{线性主部}。
\end{remark}

\begin{theorem}[可导与可微等价]
  函数$f$在点$a$可微的充要条件是
  $f$在点$a$存在有限导数$f'(a)$,
  于是就有
  \begin{displaymath}
    \md y = f'(a)\Delta x
  \end{displaymath}
\end{theorem}

\begin{remark}
  我们约定自变量$x$的增量$\Delta x$为自变量的微分,
  即$\md x = \Delta x$,
  于是上面的微分可以写成
  \begin{displaymath}
    \md y = f'(a)\md x
  \end{displaymath}
  需要注意的是,
  $\frac{\md y}{\md x} = f'(x)$中的左半部分应当视为一个整体,
  而不是一个分子除以分母的形式。
  只是因为上述定理的存在,
  才使得看起来只要把等式左边的分母移到右边就能得到$f$的微分形式。
  事实上,\textbf{该定理只对一元函数成立}。
\end{remark}

\section{导数与微分的计算}
本小节首先说明了导数和微分的四则运算法则。
然后讨论了复合函数的求导法则以及一阶微分形式的不变性。
接着讨论了反函数导数的求法。
最后,我们列出一些常用且容易遗忘的函数的导数。

\subsection{四则运算}
\begin{theorem}[导数的四则运算]
  设函数$u,v$在点$x$可导,那么
  \begin{enumerate}
    \item 
    $y=u\pm v$在点$x$可导,且有
    \begin{displaymath}
      (u\pm v)'=u'\pm v'
    \end{displaymath}
    \item 
    $y=uv$在点$x$可导,且有
    \begin{displaymath}
      (uv)' = u'v+uv'
    \end{displaymath}
    \item 
    当$v(x)\neq 0$时,$y=u/v$在点$x$可导,且有
    \begin{displaymath}
      \left(\frac{u}{v}\right)'=\frac{u'v-uv'}{v^2}
    \end{displaymath}
  \end{enumerate}
\end{theorem}

\begin{theorem}[微分的四则运算]
  设函数$u,v$在点$x$可微,那么
  \begin{enumerate}
    \item 
    $\md (u\pm v) = \md u \pm \md v$
    \item 
    $\md (uv) = v\md u + u \md v$
    \item 
    $\md (\frac{u}{v}) = \frac{v\md u - u\md v}{v^2}$
  \end{enumerate}
\end{theorem}

\subsection{复合函数的导数}
\begin{theorem}[复合函数的导数]
  设函数$y=f(u)$与$u=u(x)$在$x_0$的邻域能构成复合函数,
  且$u=u(x)$在点$x_0$可导,$y=f(x)$在点$u_0=u(x_0)$可导。
  那么,复合函数$f\circ u$在点$x_0$可导,且有
  \begin{displaymath}
    (f\circ u)'(x_0) = f'(u_0)u'(x_0)
  \end{displaymath}
\end{theorem}

\begin{corollary}[一阶微分形式不变性]
  设$y=f(u)$,则无论$u$是自变量还是中间变量$u=u(x)$,
  其微分形式不变,都是
  \begin{displaymath}
    \md y = f'(u)\md u
  \end{displaymath}
\end{corollary}

\begin{remark}
  该定理对高阶微分不成立:若$y=f(x)$,但$x$是中间变量,
  即$x=u(t)$,那么有
  \begin{align*}
    \md^2 y
    &= \md (f'(x)\md x) \\
    &= f''(x)\md x^2 + f'(x)\md^2 x
      & (\text{不是$f''(x)\md x^2$}) \\
    &= f''(x)\left(u'(t)\md t\right)^2 + f'(x)(u''(t)\md t^2)\\
    &= \left(f''(x)u'(t)^2 + f'(x)u''(t)\right)\md t^2
  \end{align*}
\end{remark}

\subsection{反函数的导数}
\begin{theorem}[反函数的导数]
  设函数$x=\phi(y)$在某一区间$I$内严格单调,
  有在区间$I$内一点$y$出导数$\phi(y)$存在且不为零,
  则反函数$y=f(x)$在对应点$x$出具有导数$f'(x)$,且
  \begin{displaymath}
    f'(x) = \frac{1}{\phi'(y)}
  \end{displaymath}
\end{theorem}

\subsection{公式表}
下面是一些常用且容易遗忘的函数的导数:
\begin{center}
  \begin{tabular}{|c|c|}
    \hline 
    $f(x)$ & $f'(x)$ \\ 
    \hline 
    $a^x$ & $a^x\ln a$  \\ 
    \hline 
    $\log_a |x|$ & $\frac{1}{x\ln a}$ \\ 
    \hline 
    $\arcsin x$ & $\frac{1}{\sqrt{1-x^2}}$ \\ 
    \hline 
    $\arccos x$ & $-\frac{1}{\sqrt{1-x^2}}$ \\ 
    \hline 
    $\arctan x$ & $\frac{1}{1+x^2}$ \\ 
    \hline 
  \end{tabular} 
\end{center}

\section{高阶导数与高阶微分}

\section{微分中值定理}

\section{洛必达法则}

\section{泰勒公式}

\section{利用导数研究函数的性质}

\section{利用导数作函数的图形}


\part{线性代数}
\chapter{矩阵\ 行列式\ 线性方程组}

\section{矩阵及其运算}
这一小节主要介绍了矩阵的基本概念和基本运算。

矩阵的概念对学过线性代数的人来说是稀松平常的了,
所以这里主要给出了重要概念和特殊矩阵的定义和符号,
方便后面的讨论。

矩阵的基本运算涉及了线性运算、乘积、转置。
大家应该对此都很熟悉,这里就简单给出了一些性质,并不作证明。
此外,我们还由转置运算给出了对称矩阵、反对称矩阵的概念。

\subsection{矩阵的概念}
\begin{definition}[矩阵]
  由$m\times n$个数排成$m$行$n$列的矩形数表
  \begin{equation} \label{eq:mat-def}
    \mmA = \mmat{cccc}{
      a_{11} & a_{12} & \cdots & a_{1n} \\
      a_{21} & a_{22} & \cdots & a_{2n} \\
      \vdots & \vdots & \ddots & \vdots \\
      a_{m1} & a_{m2} & \cdots & a_{mn} }
  \end{equation}
  称为一个$m\times n$矩阵。
  其中数$a_{ij}$称为矩阵$\mmA$的元素。
  $i$为行标,$j$为列标,
  $a_{ij}$是$\mmA$位于第$i$行第$j$列的元素,或简称$\mmA$的$(i,j)$元素。
  式\eqref{eq:mat-def}可简记为$\mmA=(a_{ij})_{m\times n}$或$(a_{ij})$。
\end{definition}

下面是一些特殊的矩阵:
\begin{itemize}
  \item
  $m\times 1$矩阵又称为\textbf{列矩阵}或$m$\textbf{维列向量},
  下文主要会用$\beta$来表示。
  \item
  $1\times n$矩阵又称为\textbf{行矩阵}或$n$\textbf{维行向量},
  下文主要会用$\alpha$来表示。
  \item
  $n\times n$矩阵又称为$n$\textbf{阶方阵}。
  对于$n$阶方阵$\mmA=(a_{ij})$,
  其对角线上的元素$a_{11},\dots,a_{nn}$又称\textbf{主对角线元素}。
  \item
  如果一个$n$阶方阵除主对角线元素外,其余元素都为0,
  那么我们称这种矩阵为\textbf{对角矩阵},记为
  \[
    \mdiag(k_1, k_2\dots,k_n)= \mmat{cccc}{
      k_1 &     &        & \\
          & k_2 &        & \\
          &     & \ddots & \\
          &     &        & k_n }
  \]
  \item
  $\mmI = \mdiag(1,1,\dots,1)$被称为\textbf{单位矩阵}。
  \item
  \textbf{零矩阵}$\mmZero$是所有元素都为0的矩阵。
  \item
  设$\mmA=(a_{ij})$是$n$阶方阵。
  若当$i>j$时,$a_{ij}=0$,则称$\mmA$为\textbf{上三角阵}。
  若当$i<j$时,$a_{ij}=0$,则称$\mmA$为\textbf{下三角阵}。
  两者统称\textbf{三角阵}。
\end{itemize}

\begin{definition}[矩阵相等]
    如果$\mmA=(a_{ij})$与$\mmB=(b_{ij})$都是$m\times n$矩阵,
    且对$i=1,2,\dots,m,\ j=1,2,\dots,n$,有$a_{ij}=b_{ij}$,
    则称$\mmA$与$\mmB$相等,记作$\mmA=\mmB$。
\end{definition}

\subsection{矩阵的线性运算}
矩阵的线性运算包含\textbf{加法}和\textbf{数乘}。定义比较简单,在此不加赘述。

它们满足的性质与之后要定义的线性空间一致。

\subsection{矩阵的乘法}
\begin{definition}[矩阵的乘法]
  设$\mmA=(a_{ik})_{m\times s}, \mmB=(b_{kj})_{s\times n}$,令
  \begin{equation}
  c_{ij}=\sum_{k=1}^{s}a_{ik}b_{kj}
  \end{equation}
  则$\mmC=(c_{ij})_{m\times n}$被称为$\mmA$和$\mmB$的乘积,
  记为$\mmC = \mmA\mmB$。
\end{definition}

矩阵的乘法满足分配率和结合律,但不满足交换律。

\subsection{矩阵的转置}
\begin{definition}[矩阵的转置]
    设$\mmA=(a_{ij})_{m\times n}$,
    令$b_{ij}=a_{ji}$,
    则$\mmB=(b_{ij})_{n\times m}$被称为$\mmA$的转置矩阵,
    记为$\mmA^T$。
\end{definition}

\begin{theorem}[转置矩阵的性质]
  转置矩阵具有以下性质:
  \begin{enumerate}
    \item $\left(\mmA^T\right)^T = \mmA$
    \item $(\mmA+\mmB)^T=\mmA^T + \mmB^T$
    \item $(k\mmA)^T=k\mmA^T$
    \item $(\mmA\mmB)^T=\mmB^T\mmA^T$
  \end{enumerate}
\end{theorem}

\begin{definition}[对称矩阵与反对称矩阵]
  对于$n$阶方阵$\mmA=(a_{ij})$,
  如果满足$\mmA^T=\mmA$,则称$\mmA$为\textbf{对称矩阵}。
  如果满足$\mmA^T=-\mmA$,则称$\mmA$为\textbf{反对称矩阵}。
\end{definition}

\begin{remark}
  不难看出,反对称矩阵的主对角线元素都为零。
\end{remark}

\section{矩阵的分块}
矩阵的分块是化简矩阵运算简单而又重要的思想。
其中最重要的是分块矩阵的乘法。

\subsection{分块矩阵的概念}
\begin{definition}[分块矩阵]
  一般地,将一个$m\times n$矩阵$\mmA$用横线划分成$r$块,
  用竖线划分成$s$块,就能得到一个$r\times s$分块矩阵。
  \begin{equation} \label{eq:mat-partition}
  \mmA = \mmat{cccc}{
    \mmA_{11} & \mmA_{12} & \cdots & \mmA_{1s} \\
    \mmA_{21} & \mmA_{22} & \cdots & \mmA_{2s} \\
    \vdots    & \vdots    & \ddots & \vdots    \\
    \mmA_{r1} & \mmA_{r2} & \cdots & \mmA_{rs} }
    = (\mmA_{ij})_{r\times s}
  \end{equation}
  其中,$\mmA_{ij}(i=1,\dots,r,\ j=1,\dots,s)$是$m_i\times n_j$矩阵,
  $\sum_{i=1}^{r}m_i = m$,$\sum_{j=1}^{s}n_j=n$。
\end{definition}

最常用的一种矩阵分块方法是把$m\times n$划分成$m$个行向量,
或者$n$个列向量。

\subsection{分块矩阵的运算}
分块矩阵的运算中需要注意的是转置和乘法。

转置运算不仅需要把矩阵的每一行转成列,而且内部子块也需要转置。
对于式\ref{eq:mat-partition}中的矩阵$\mmA$,它的转置是
\[
  \mmA^T = \mmat{cccc}{
    \mmA_{11}^T & \mmA_{21}^T & \cdots & \mmA_{r1}^T \\
    \mmA_{12}^T & \mmA_{22}^T & \cdots & \mmA_{r2}^T \\
    \vdots      & \vdots      & \ddots & \vdots      \\
    \mmA_{1s}^T & \mmA_{2s}^T & \cdots & \mmA_{sr}^T }
\]

对于乘法运算,
给定两个矩阵$\mmA=(a_{ij})_{m\times n}$,$\mmB=(b_{jk})_{n\times p}$,
只要使$\mmB$的行分法与$\mmA$的列分法一致,
就能把子块当作数一样按照矩阵乘法的规则进行计算。

\subsection{准对角矩阵}
\begin{definition}[准对角矩阵] 
  对于$n$阶方阵$\mmA$,
  如果有一种分法,使$\mmA$的主对角线以外的子块都是零矩阵,
  且主对角线上子块都是方阵,则称$\mmA$为准对角矩阵。
\end{definition}

\begin{remark}
  当然,准对角矩阵包含对角矩阵作为特殊情况。
\end{remark}


设$\mmA$和$\mmB$都是$n$阶方阵。
如果有相同的分法,使得
\[
\mmA = \mmat{cccc}{
  \mmA_1 &        &        & \\
         & \mmA_2 &        & \\
         &        & \ddots & \\
         &        &        & \mmA_n },\quad
\mmB = \mmat{cccc}{
  \mmB_1 &        &        & \\
         & \mmB_2 &        & \\
         &        & \ddots & \\
         &        &        & \mmB_n }
\]
都是准对角矩阵,那么显然有
\[
\mmA\mmB = \mmat{cccc}{
    \mmA_1\mmB_1 &              &        & \\
                 & \mmA_2\mmB_2 &        & \\
                 &              & \ddots & \\
                 &              &        & \mmA_n\mmB_n }
\]

\section{逆矩阵与初等变换}
本小节开始讨论逆矩阵,
并试图通过初等变换的概念,从更本质的视角来看待矩阵
——矩阵就是所谓标准型乘上有限个初等阵。
也因为初等阵具有良好的性质——可逆性,
所以矩阵可逆当且仅当它的标准型就是单位矩阵。

\subsection{逆矩阵}
\begin{definition}[逆矩阵]
  设$\mmA$是$n$阶方阵。如果存在$n$阶方阵$\mmB$,使
  \begin{equation}
    AB=BA=I
  \end{equation}
  则称$\mmB$为$\mmA$的\textbf{逆矩阵},记作$A^{-1}$,
  并说$\mmA$是\textbf{非奇异矩阵}或\textbf{可逆矩阵},
  否则便称$\mmA$是\textbf{奇异矩阵}或\textbf{不可逆矩阵}。
\end{definition}

\begin{theorem}[逆矩阵的唯一性]
  可逆矩阵的逆矩阵是唯一的。
\end{theorem}

\begin{theorem}[可逆矩阵的性质]
  若$\mmA,\mmB$是可逆矩阵,
  则$\mmA^{-1}$,$k\mmA\,(k\neq 0)$,$\mmA^T$,$\mmA\mmB$都是可逆矩阵,且
  \begin{enumerate}
    \item $(\mmA^{-1})^{-1} = \mmA$
    \item $(k\mmA)^{-1} = \frac{1}{k}\mmA^{-1}$
    \item $(\mmA^T)^{-1} = (\mmA^{-1})^T$
    \item $(\mmA\mmB)^{-1} = \mmB^{-1}\mmA^{-1}$
  \end{enumerate}
\end{theorem}

\subsection{初等变换与初等阵}
矩阵的\textbf{初等变换}可以分为\textbf{初等行变换}和\textbf{初等列变换}。
下面主要讨论初等行变换。初等列变换与之相似。

\begin{definition}[初等行变换]
  初等行变换有三种:
  \begin{enumerate}
    \item 交换矩阵的第$i$行和第$j$行,记为$\mTfRowSwi{i}{j}$。
    \item 将矩阵的第$i$行乘非零常数$k$,记为$\mTfRowMul{k}{j}$。
    \item 把矩阵的第$i$行加上第$j$行的$k$倍,记为$\mTfRowAdd{i}{+k}{j}$。
  \end{enumerate}
\end{definition}

\begin{definition}[初等阵]
  单位矩阵经过一次初等变换得到的矩阵叫做\textbf{初等矩阵},
  或简称\textbf{初等阵}。
\end{definition}

\begin{remark}
  显然,初等阵是方阵,并且都是可逆的。
\end{remark}

使用三种初等行变换能得到三个初等阵,
我们把它们分别记作
$\mmRowSwi{i}{j}$,$\mmRowMul{k}{i}$和$\mmRowAdd{i}{+k}{j}$。
对单位矩阵实行一次初等列变换,同样能得到上述三种形式的初等阵,
所以初等阵只有以上三种。

\begin{theorem}[一般矩阵的初等变换与初等阵的联系]
  对于一个$m\times n$矩阵$\mmA$做一次初等行变换,
  就相当于对$\mmA$左乘一个$m$阶初等矩阵;
  对$\mmA$做一次初等列变换,
  就相当于对$\mmA$右乘一个$n$阶初等矩阵。
\end{theorem}

\begin{definition}[矩阵等价]
  如果矩阵$\mmA$可以通过有限次初等变换化为矩阵$\mmB$,
  那么就说$\mmA$与$\mmB$等价。
\end{definition}

\begin{remark}
  矩阵等价是一个等价关系,即满足自反性、对称性、传递性。
\end{remark}

\begin{theorem}[标准型]
  任意非零$m\times n$矩阵$\mmA$都等价于矩阵
  \[
    \mmat{cc}{\mmI_r & \mmZero \\ \mmZero & \mmZero}
    \quad (1\le r \le \min(m,n))
  \]
  它称为矩阵$\mmA$的\textbf{标准型}。
  换句话说,任意非零$m\times n$矩阵$\mmA$,
  必有初等矩阵$\mmP_1,\dots,\mmP_s$,$\mmQ_1,\dots,\mmQ_t$,
  使得
  \[
    \mmP_s\dots\mmP_1\cdot\mmA\cdot\mmQ_1\dots\mmQ_t = 
      \mmat{cc}{\mmI_r & \mmZero \\ \mmZero & \mmZero}
  \]
\end{theorem}

因为初等矩阵可逆,所以有如下推论

\begin{corollary}
  对于任意非零$m\times n$矩阵$\mmA$,
  存在可逆$m\times m$矩阵$\mmP$和可逆$n\times n$矩阵$\mmQ$,
  使得
  \[
    \mmP\mmA\mmQ = \mmat{cc}{\mmI_r & \mmZero \\ \mmZero & \mmZero}
    \quad (1\le r \le \min(m,n))
  \]
  或写成
  \[
    \mmA = \mmP^{-1}\mmat{cc}{\mmI_r & \mmZero \\ \mmZero & \mmZero}\mmQ^{-1}
    \quad (1\le r \le \min(m,n))
  \]
\end{corollary}

\subsection{矩阵可逆的充要条件}

\begin{theorem}
  设$\mmA,\mmB$是$n$阶方阵,
  若$\mmA\mmB=\mmI_n$,
  则$\mmA,\mmB$都是可逆阵,且它们互为逆阵。
\end{theorem}

\begin{theorem}
  设$\mmA$是$n$阶方阵,则下列结论等价:
  \begin{enumerate}
    \item $\mmA$是可逆矩阵。
    \item $\mmA$可以表示为有限个初等阵的乘积。
    \item $\mmA$可以经过有限次初等变换化为单位矩阵。
  \end{enumerate}
\end{theorem}

\subsection{用初等变换求逆矩阵}
\[
  \mmat{c|c}{\mmA & \mmI} \xrightarrow{\text{初等行变换}}
  \mmat{c|c}{\mmI & \mmA^{-1}}
\]

\section{行列式}
本小节开始讨论行列式的概念。

首先我们给出了行列式的定义。
但是,如果我们根据行列式的定义来计算行列式,除了一些特殊矩阵,
用计算科学的说法,大部分$n$阶行列式的计算复杂度是$O(n!)$,
因此我们需要探索更高效的方法来计算行列式。

接下来,我们给出了行列式的一些基本性质。
有了这些性质,我们得到了新的行列式计算方法,
而且稍加分析,就能惊喜地看出这个方法的时间复杂度是$O(n^3)$。

\subsection{$n$阶行列式}
\begin{definition}[$n$级排列]
  $n$个自然数按任意固定的顺序构成的一个排列称为$n$\textbf{级排列}。
  所有$n$级排列构成的集合记作$\pi_n$。
\end{definition}

\begin{definition}[逆序数]
  在一个$n$级排列$(i_1,i_2,\dots,i_n)$中,
  如果$i_r>i_s$,但是$r<s$,即前面的数大于后面的数,
  就称这两个数构成\textbf{逆序对}。
  一个排列中逆序对的个数叫做这个排列的\textbf{逆序数},
  记作$\tau(i_1,i_2,\dots,i_n)$。
\end{definition}

\begin{definition}[奇、偶排列]
  逆序数为奇数的排列叫做\textbf{奇排列};
  逆序数为偶数的排列叫做\textbf{偶排列}。
\end{definition}

\begin{definition}[$n$阶行列式]
  设$\mmA=(a_{ij})_{n\times n}$是$n$阶方阵,它的$n$阶行列式为
  \begin{align*}
    |\mmA| &= \sum_{(j_1,j_2,\dots,j_n)\in\pi_n}
    (-1)^{\tau(j_1,j_2,\dots,j_n)}a_{1j_1}a_{2j_2}\dots a_{nj_n} \\
    &= \sum_{(i_1,i_2,\dots,i_n)\in\pi_n}
    (-1)^{\tau(i_1,i_2,\dots,i_n)}a_{i_1 1}a_{i_2 2}\dots a_{i_n n} 
  \end{align*}
\end{definition}

特殊矩阵的行列式:
\begin{enumerate}
  \item $|\mdiag(d_1,d_2,\dots,d_n)| = d_1d_2\dots d_n$
  \item 三角阵的行列式为对角线之积
\end{enumerate}

\subsection{行列式的性质}
\begin{theorem}
  若$\mmA$为方阵,那么$|\mmA^T| = |\mmA|$
\end{theorem}

\begin{theorem}
  \begin{align*}
    &\quad\mdet{cccc}{
      a_{11} & a_{12} & \cdots & a_{1n} \\
      \vdots &        &        &        \\
      b_{i1} + c_{i1} & b_{i2} + c_{i2}  & \cdots & b_{in} + c_{in} \\
      \vdots &        &        &        \\
      a_{n1} & a_{n2} & \cdots & a_{nn} } \\
    &= \mdet{cccc}{
      a_{11} & a_{12} & \cdots & a_{1n} \\
      \vdots &        &        &        \\
      b_{i1} & b_{i2} & \cdots & b_{in} \\
      \vdots &        &        &        \\
      a_{n1} & a_{n2} & \cdots & a_{nn} }
    + \mdet{cccc}{
      a_{11} & a_{12} & \cdots & a_{1n} \\
      \vdots &        &        &        \\
      c_{i1} & c_{i2} & \cdots & c_{in} \\
      \vdots &        &        &        \\
      a_{n1} & a_{n2} & \cdots & a_{nn} }
    \end{align*}
\end{theorem}

\begin{theorem}
  设$\mmA$为$n$阶方阵。
  若$\mmA$有两行元素相等或对应成比例,
  那么$|\mmA| = 0$。
\end{theorem}

\begin{theorem}
  \begin{displaymath}
    \mdet{cccc}{
      a_{11} & a_{12} & \cdots & a_{1n} \\
      \vdots &        &        &        \\
      \lambda a_{i1} & \lambda a_{i2} & \cdots & \lambda a_{in} \\
      \vdots &        &        &        \\
      a_{n1} & a_{n2} & \cdots & a_{nn} }
    = \lambda \mdet{cccc}{
      a_{11} & a_{12} & \cdots & a_{1n} \\
      \vdots &        &        &        \\
      a_{i1} & a_{i2} & \cdots & a_{in} \\
      \vdots &        &        &        \\
      a_{n1} & a_{n2} & \cdots & a_{nn} }
  \end{displaymath}
\end{theorem}

\begin{theorem} \label{thrm-det-rowswi}
  交换方阵$\mmA$的两行,仅改变$\mmA$行列式的符号,
  即\[ |\mmRowSwi{i}{j}\mmA| = -|\mmA| \]
\end{theorem}

\begin{theorem} \label{thrm-det-rowadd}
  对方阵$\mmA$做$\mTfRowAdd{i}{+k}{j}$变换,不改变$\mmA$的行列式,
  即\[ |\mmRowAdd{i}{+k}{j}\mmA| = |\mmA| \]
\end{theorem}

\begin{remark}
  由行列式的定义不难看出,
  $|\mmRowSwi{i}{j}|=-1$,$|\mmRowAdd{i}{+k}{j}|=1$。
  所以定理\ref{thrm-det-rowswi}和\ref{thrm-det-rowadd}
  也可以由下面给出的定理\ref{thrm-det-mul}推导。
\end{remark}

\subsection{用行列式的性质求行列式}
使用行列式的性质把行列式转化成上三角矩阵的行列式。
后者的行列式即为对角线之积。

\section{行列式按行展开}
TODO

\section{用行列式求逆阵\ 克莱姆法则}
\begin{theorem} \label{thrm-det-mul}
  $|\mmA\mmB|=|\mmA||\mmB|$
\end{theorem}
\chapter{线性空间\ 线性变换\ 欧几里得空间}

\section{线性空间}
在这一小节,我们把先前向量的概念进一步抽象,得到了线性空间的概念。
向量组有线性相关、线性无关、极大线性无关组等概念,
这在线性空间中就转变成纬度、基、坐标的概念。

线性空间和物理的联系也是很紧密的。
因为向量是物理的常用量,而线性空间就是从向量抽象来的。
回忆一下物理学的知识,我们知道一个物理量在不同参考系的表示是不一样的。
对应到线性空间上来,我们会发现一个元素在不同基底下的坐标也是不一样的。
所以我们要研究这些坐标与基底的变换关系。

最后我们会简单讨论一下子空间的概念。

\subsection{线性空间}
\begin{definition}[线性空间] \label{def-linear-space}
  设$V$是非空集合,$\mfF$是一个数域。
  若在$V$上定义两种运算:加法$\oplus$和数乘$\otimes$,满足
  \begin{description}
    \item[加法]
    \begin{enumerate}
      \item 封闭性:
      $\forall\alpha,\beta\in V,\ \alpha\oplus\beta\in V$。
      \item 交换律:
      $\forall\alpha,\beta\in V,\ \alpha\oplus\beta = \beta\oplus\alpha$
      \item 结合律:
      $\forall\alpha,\beta,\gamma\in V,\ 
        (\alpha\oplus\beta)\oplus\gamma = \alpha\oplus(\beta\oplus\gamma)$。
      \item 零元存在:
      $\exists\theta\in V,\forall\alpha\in V,\ \alpha\oplus\theta=\alpha$。
      \item 逆元存在:
      $\forall\alpha\in V,\exists\beta\in V,\ \alpha\oplus\beta=\theta$。
      我们把$\beta$记为``$-\alpha$''。
    \end{enumerate}  
    \item[数乘]
    \begin{enumerate}
      \item 封闭性:
      $\forall\lambda\in\mfF,\alpha\in V,\ \lambda\otimes\alpha\in V$。
      \item 结合律:
      $\forall\lambda_1,\lambda_2\in\mfF,\alpha\in V,\ 
        \lambda_1\otimes(\lambda_2\otimes\alpha) =
        (\lambda_1\lambda_2)\otimes\alpha$。
      \item 分配律1:
      $\forall\lambda_1,\lambda_2\in\mfF,\alpha\in V,\ 
        (\lambda_1\oplus\lambda_2)\otimes\alpha =
        (\lambda_1\otimes\alpha)\oplus(\lambda_2\otimes\alpha)$
      \item 分配律2:
      $\forall\lambda\in\mfF,\alpha,\beta\in V,\ 
        \lambda\otimes(\alpha\oplus\beta) =
        (\lambda\otimes\alpha)\oplus(\lambda\otimes\beta)$。
      \item 单位元存在:
      $\forall\alpha\in V,\ 
        1\otimes\alpha = \alpha$。
    \end{enumerate}
  \end{description}
  则称$V$是数域$\mfF$上的一个线性空间。
  若$\mfF$是实数域$\mfR$,则称$V$是\textbf{实线性空间}。
  若$\mfF$是复数域$\mfC$,则称$V$是\textbf{复线性空间}。
\end{definition}

\begin{remark}
  如果有点抽象代数的背景,读者不难发现上述加法满足的是阿贝尔群的性质。
\end{remark}

\begin{theorem}[线性空间的性质]
  设$V$是$\mfF$上的线性空间,那么
  \begin{enumerate}
    \item 加法零元的具有唯一性
    \item 加法逆元的具有唯一性
    \item 加法零元的求法:
    $\forall\alpha\in V,\ 0\otimes\alpha=\theta$
    \item 加法逆元的求法:
    $\forall\alpha\in V,\ (-1)\otimes\alpha=-\alpha$
    \item $\forall k\in\mfF,\ k\otimes\theta=\theta$
    \item 若$\lambda\otimes\alpha=\theta$,则$\lambda=0$或$\alpha=\theta$。
  \end{enumerate}
\end{theorem}
为了书写与阅读的方便,以后``$\oplus$''用正常的加号来表示,
``$\otimes$''可省略。参考向量的记法。

\subsection{基与坐标}
向量组的线性相关、线性无关、线性表示的概念也适用于线性空间。

\begin{definition}[线性相关与线性无关]
  设$V$是线性空间,$\alpha_1,\alpha_2,\dots,\alpha_s\in V$。
  若数域中存在一组不全为0的数$k_1,k_2,\dots,k_s$,使得
  \[ k_1\alpha_1 + k_2\alpha_2 + \dots + k_s\alpha_s = \theta \]
  则称$\alpha_1,\alpha_2,\dots,\alpha_s$\textbf{线性相关}。
  否则,则称\textbf{线性无关}。
\end{definition}

\begin{definition}[线性表示]
  设$V$是线性空间,$\alpha,\alpha_1,\alpha_2,\dots,\alpha_s\in V$。
  若数域中存在一组数$k_1,k_2,\dots,k_s$,使得
  \[ \alpha = k_1\alpha_1 + k_2\alpha_2 + \dots + k_s\alpha_s \]
  则称$\alpha$为$\alpha_1,\alpha_2,\dots,\alpha_s$的\textbf{线性组合},
  也称$\alpha$可以由$\alpha_1,\alpha_2,\dots,\alpha_s$\textbf{线性表示}。
\end{definition}

\begin{definition}[维数]
  如果一个线性空间$V$中,线性无关的元素的最大个数是$n$,
  则称该线性空间是$n$维的,记作$\mdim V = n$。
  
  如果对任意正整数$N$,总存在$N$个线性无关的元素,
  则称该线性空间是\textbf{无穷维线性空间}。
  不是无穷维的线性空间叫做有穷维线性空间。
\end{definition}

\begin{remark}
  本章中我们不讨论无穷维线性空间。
\end{remark}

\begin{definition}[基与坐标]
  若$\alpha_1,\alpha_2,\dots,\alpha_n$是$n$维线性空间$V$中一组线性无关的元素,
  且$V$中任意元素$\alpha$都可由$\alpha_1,\alpha_2,\dots,\alpha_n$线性表示,
  即\[ \alpha = k_1\alpha_1 + k_2\alpha_2 + \dots + k_n\alpha_n \]
  则称$\alpha_1,\alpha_2,\dots,\alpha_n$是$V$的一组\textbf{基底}(简称\textbf{基})。
  其中,有序元组$(k_1,k_2,\dots,k_n)$称为$\alpha$在
  基底$\alpha_1,\alpha_2,\dots,\alpha_n$下的坐标。
\end{definition}

\begin{remark}
  线性空间的基对应的是极大线性无关组的概念。
\end{remark}

\subsection{基变换与坐标变换}
\begin{theorem}[基底变换公式]
  设$\alpha_1,\alpha_2,\dots,\alpha_n$与$\beta_1,\beta_2,\dots,\beta_n$是
  线性空间$V$的两组基底,那么可以写成
  \begin{equation} \label{eq-basis-transform}
    (\beta_1,\beta_2,\dots,\beta_n) = (\alpha_1,\alpha_2,\dots,\alpha_n)
    \mmat{cccc}{
      a_{11} & a_{12} & \cdots & a_{1n} \\
      a_{21} & a_{22} & \cdots & a_{2n} \\
      \vdots & \vdots & \ddots & \vdots \\
      a_{m1} & a_{m2} & \cdots & a_{mn} }
  \end{equation}
  我们把它简记为
  \begin{displaymath}
    \mmBasis{\beta} = \mmBasis{\alpha}\cdot\mmP
  \end{displaymath}
  我们称$\mmP$是$\mmBasis{\alpha}$到$\mmBasis{\beta}$的\textbf{过渡矩阵},
  式\ref{eq-basis-transform}是\textbf{基底变换公式}。
\end{theorem}

\begin{remark}
  过渡矩阵$\mmP$是非奇异矩阵。
  因为如果$\mmP$是奇异的,就存在$\mvx\neq\mvZero$,使得$\mmP\mvx=\mvZero$。
  从而有$\mmBasis{\beta}\mvx = \mmBasis{\alpha}\mmP\mvx = \mvZero$。
  于是$\mmBasis{\beta}$是奇异矩阵,这与``$\beta_1,\dots,\beta_n$是基底''是矛盾的。
\end{remark}

\begin{theorem}[坐标变换公式]
  设$\mmBasis{\alpha}$和$\mmBasis{\beta}$是线性空间$V$的两组基底,
  $\mmP$是从$\mmBasis{\alpha}$到$\mmBasis{\beta}$的过渡矩阵。
  若$\mvx=(x_1,\dots,x_n)^T$和$\mvy=(y_1,\dots,y_n)^T$
  分别是元素$\alpha\in V$在基底$\mmBasis{\alpha}$和$\mmBasis{\beta}$下的坐标,即
  \[ \alpha = \mmBasis{\beta}\mvy = \mmBasis{\alpha}\mvx \]
  那么有
  \begin{equation} \label{eq-coordinate-transform}
    \mvy = \mmP^{-1}\mvx
  \end{equation}
  式\ref{eq-coordinate-transform}被称为\textbf{坐标变换公式}。
\end{theorem}

\subsection{子空间}
\begin{definition}
  设$V$是数域$\mfF$上的线性空间。若$W\subset V$也是数域$\mfF$上的线性空间,
  则称$W$是$V$的\textbf{子空间}。
  $W=\{\theta\}$叫做\textbf{平凡子空间}。
\end{definition}

\begin{theorem}[子空间的充要条件]
  设$V$是数域$\mfF$上线性空间。
  $W\subset V$是$V$的子空间的充要条件是:
  1. 对加法封闭;2. 对乘法封闭。
  换句话说,即是$\forall\alpha,\beta\in W,\lambda,\mu\in\mfF$,
  \[ \lambda\alpha + \mu\beta \in W \]
\end{theorem}

\begin{definition}[子集张成的子空间]
    设$V$是数域$\mfF$上的线性空间,$\alpha_1,\alpha_2,\dots,\alpha_s\in V$。
    我们定义$V$的子空间
    \begin{displaymath}
    \mspan\{ \alpha_1,\alpha_2,\dots,\alpha_s \} =
    \{ \alpha: \alpha=\sum_{i=1}^{s}k_i\alpha_i, \forall k_1,\dots,k_s\in\mfF \}
    \end{displaymath}
    称作由向量组$\alpha_1,\alpha_2,\dots,\alpha_s$张成的子空间。
\end{definition}

\begin{definition}[线性空间的和]
  设$W_1,W_2$是线性空间的两个子空间,则它们的\textbf{和}定义为
  \begin{displaymath}
    W_1+W_2 = \{ u: u=\alpha+\beta, \forall\alpha\in W_1,\beta\in W_2 \}
  \end{displaymath}
\end{definition}

\begin{theorem}
  设$V$是线性空间,$W_1,W_2$是$V$的子空间,那么有以下结论:
  \begin{enumerate}
    \item $W_1\cap W_2$是$V$的子空间。
    \item $W_1\cup W_2$不是$V$的子空间。
    \item $W_1+W_2$是$V$的子空间。
  \end{enumerate}
\end{theorem}

\begin{theorem}[维数定理]
  设$W_1,W_2$是线性空间$V$的两个有限维子空间,则有
  \begin{displaymath}
  \mdim W_1 + \mdim W_2 = \mdim(W_1+W_2) + \dim(W_1\cap W_2)
  \end{displaymath}
\end{theorem}

\begin{definition}[直接和]
  设$V_1,V_2$是线性空间$V$的两个子空间。
  若对任意$\alpha\in V$,存在唯一的$\alpha_1\in V_1, \alpha_2\in V_2$,
  使得$\alpha=\alpha_1+\alpha_2$,
  则称$V$是$V_1$和$V_2$的\textbf{直接和}或\textbf{直和},
  记作$V=V_1\oplus V_2$。
  也称$V_1$和$V_2$是$V$内的\textbf{互补空间}。
\end{definition}

\begin{theorem}[直接和的等价条件]
  设$V_1,V_2$是线性空间$V$的两个子空间。
  \begin{align*}
    V=V_1\oplus V_2
    &\iff V = V_1 + V_2\ \text{且}\ V_1\cap V_2 = \{ \theta \} \\
    &\iff \mdim (V_1\oplus V_2) = \mdim V_1 + \mdim V_2
  \end{align*}
\end{theorem}

\section{线性变换}
TODO

\part{概率论}
TODO

\part{数理统计}
TODO

\end{document}