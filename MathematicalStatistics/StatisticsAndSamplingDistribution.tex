\chapter{统计量与抽样分布}

\section{统计的基本概念}
本小节主要介绍了总体,个体,样本,统计量与抽样分布的基本概念。

\subsection{总体与个体}
\begin{definition}[总体]
  研究对象的某项数量指标的值的全体被称为\textbf{总体}。
\end{definition}

\begin{definition}[个体]
  总体中的每个元素被称为\textbf{个体}。
\end{definition}

\begin{definition}[总体分布]
  研究对象的数量指标$X$的取值在客观上有一定的分布,
  因此,可将其看做随机变量,它的分布被称为\textbf{总体分布}。
\end{definition}

\subsection{样本}
\begin{definition}[样本]
  从总体中随机抽取的一些个体被称为\textbf{样本}。
\end{definition}

\begin{definition}[抽样]
  抽得样本的过程被称为\textbf{抽样}。
\end{definition}

\begin{definition}[样本容量]
  样本中个体的数量被称为\textbf{样本容量}。
\end{definition}

\begin{definition}[样本值]
  对样本观察得到的数值被称为\textbf{样本值}。
\end{definition}

就一次具体观察而言,样本值是确定的数。
但在不同的抽样下,样本值会发生变化,因此可将样本看做是随机变量。

\begin{definition}[简单随机样本]
  设随机变量$X$的分布函数是$F(x)$。
  若相互独立的随机变量$X_1,X_2,\dots,X_n$具有同一分布函数$F$,
  则称$X_1,X_2,\dots,X_n$为从总体$X$中得到的容量为$n$的\textbf{简单随机样本},
  简称为样本,其观察值$x_1,x_2,\dots,x_n$称为样本值。
\end{definition}

样本的两个特性(对抽样的要求):
\begin{enumerate}
  \item 代表性:
  样本的每个分量$X_i$与总体$X$具有相同的分布。
  \item 独立性:
  $X_1,X_2,\dots,X_n$相互独立。
\end{enumerate}

\begin{definition}[样本联合分布/密度]
  若$X_1,X_2,\dots,X_n$为$X$的一个样本,
  $F(x)$和$p(x)$为$X$的分布函数与密度函数,
  则$X_1,X_2,\dots,X_n$的\textbf{联合分布函数}为
  \begin{displaymath}
    F^*(x_1,x_2,\dots,x_n)=\prod_{i=1}^n F(x_i)
  \end{displaymath}
  $X_1,X_2,\dots,X_n$的\textbf{联合概率密度}为
  \begin{displaymath}
    p^*(x_1,x_2,\dots,x_n)=\prod_{i=1}^n p(x_i)
  \end{displaymath}
\end{definition}

\subsection{统计量}
\begin{definition}[统计量]
  设$X_1,X_2,\dots,X_n$为来自总体$X$的一个样本。
  若$g(x_1,x_2,\dots,x_n)$是连续函数,且$g$中不含任何未知参数,则称
  \begin{displaymath}
    g(X_1,X_2,\dots,X_n)
  \end{displaymath}
  是一个\textbf{统计量}。
  设$x_1,x_2,\dots,x_n$是$X_1,X_2,\dots,X_n$的样本值,则称
  \begin{displaymath}
    g(x_1,x_2,\dots,x_n)
  \end{displaymath}
  是$g(X_1,X_2,\dots,X_n)$的观察值。
\end{definition}

\begin{remark}
  统计量也是随机变量。
\end{remark}

\begin{definition}[常用统计量]
  设$X_1,X_2,\dots,X_n$为来自总体$X$的一个样本,
  下面是一些常用统计量的定义:
  \begin{enumerate}
    \item 样本均值:
    \begin{displaymath}
      \mbar{X}=\frac{1}{n}\sum_{i=1}^{n}X_i
    \end{displaymath}
    \item 样本方差:
    \begin{displaymath}
      S_n^2 = \frac{1}{n}\sum_{i=1}^{n}(X_i-\mbar{X})^2
      = \frac{1}{n}\sum_{i=1}^{n}X_i^2- \mbar{X}^2
    \end{displaymath}
    \item 修正样本方差:
    \begin{displaymath}
      S_{n-1}^2 = \frac{1}{n-1}\sum_{i=1}^{n}(X_i-\mbar{X})^2
    \end{displaymath}
    \item 样本标准差:
    \begin{displaymath}
      S_n = \sqrt{S_n^2}
    \end{displaymath}
    \item 样本$k$阶(原点)矩:
    \begin{displaymath}
      A_k = \frac{1}{n}\sum_{i=1}^{n}X_i^k
    \end{displaymath}
    \item 样本$k$阶中心矩:
    \begin{displaymath}
      B_k = \frac{1}{n}\sum_{i=1}^{n}(X_i-\mbar{X})^k
    \end{displaymath}
  \end{enumerate}
\end{definition}

\subsection{抽样分布}
\begin{definition}[抽样分布]
  统计量的分布称为\textbf{抽样分布}。
\end{definition}

\section{正态总体的抽样分布}
本小节首先介绍正态总体的四种抽样分布以及相关的上$\alpha$分为点的概念,
它们分别是:正态总体样本的线性函数的分布,$\chi^2$分布,$t$分布和$F$分布。
然后,我们给出四个定理,它们描述了正态总体的统计量的分布。

\subsection{正态总体样本的线性函数的分布}
\begin{theorem}
  设$X_1,X_2,\dots,X_n$为来自正态总体$X\sim N(\mu,\sigma^2)$的样本,
  则统计量$U=\sum_{i=1}^{n}a_iX_i$服从正态分布
  \begin{displaymath}
    U\sim N\left(\mu\sum_{i=1}^{n}a_i,\sigma^2\sum_{i=1}^{n}a_i^2\right)
  \end{displaymath}
  若取$a_i=1/n\ (i=1,2,\dots,n)$,则
  \begin{displaymath}
    U = \mbar{X} \sim N(\mu, \frac{\sigma^2}{n})
  \end{displaymath}
\end{theorem}

\begin{definition}[标准正态分布的上$\alpha$分位点]
  设$X\sim N(0,1)$,$0<\alpha<1$,称满足
  \begin{displaymath}
    P(X>u_\alpha) = \alpha
  \end{displaymath}
  的$u_\alpha$为$N(0,1)$分布的上$\alpha$分位点。
\end{definition}

\begin{theorem}
  设$u_\alpha$是标准正态分布的上$\alpha$分位点,
  则$\Phi(u_\alpha)=1-\alpha$。
\end{theorem}

\subsection{$\chi^2$分布}
\begin{definition}[$\chi^2$分布]
  设$X_1,X_n,\dots,X_n$独立同分布于$N(0,1)$,则称随机变量
  \begin{displaymath}
    \chi^2 = X_1^2+X_2^2+\dots+X_n^2
  \end{displaymath}
  所服从的分布为自由度为$n$的$\chi^2$分布,
  记为$\chi^2\sim\chi^2(n)$。
\end{definition}

\begin{theorem}[$\chi^2$分布的可加性]
  设$X_1\sim\chi^2(n_1), X_2\sim\chi^2(n_2)$,
  且$X_1,X_2$相互独立,则
  \begin{displaymath}
    X_1+X_2\sim\chi^2(n_1+n_2)
  \end{displaymath}
\end{theorem}

\begin{theorem}[$\chi^2$分布的数字特征]
  若$X\sim\chi^2(n)$,则
  \begin{displaymath}
    \mexpect[X] = n,\quad \mvar(X) = 2n
  \end{displaymath}
\end{theorem}

\begin{definition}[$\chi^2$分布的上$\alpha$分位点]
  设$X\sim\chi^2(n)$,$0<\alpha<1$,称满足
  \begin{displaymath}
  P(X>\chi_\alpha^2(n)) = \alpha
  \end{displaymath}
  的点$\chi_\alpha^2(n)$为$\chi^2(n)$分布的上$\alpha$分位点。
\end{definition}

\subsection{$t$分布}
\begin{definition}[$t$分布]
  设$X\sim N(0,1),Y\sim \chi^2(n)$,且$X,Y$相互独立,
  则称随机变量
  \begin{displaymath}
    T=\frac{X}{\sqrt{Y/N}}
  \end{displaymath}
  服从自由度为$n$的$t$分布,记为$T\sim t(n)$。
\end{definition}

\begin{theorem}[$t$分布的性质]
  $t$分布具有下面一些性质:
  \begin{enumerate}
    \item
    $t$分布的密度函数关于$t=0$对称。
    \item
    当$n$充分大时,$t$分布的密度函数$p(t)$近似于$N(0,1)$的密度函数$\Phi(t)$。
  \end{enumerate}
\end{theorem}

\begin{definition}[$t$分布的上$\alpha$分位点]
  设$X\sim t(n)$,$0<\alpha<1$,称满足
  \begin{displaymath}
    P(X>t_\alpha(n)) = \alpha
  \end{displaymath}
  的点$t_\alpha(n)$为$t(n)$分布的上$\alpha$分位点,
\end{definition}

\begin{theorem}[$t$分布的上$\alpha$分位点的性质]
  $t_{1-\alpha}(n) = -t_\alpha(n)$。
\end{theorem}

\subsection{$F$分布}
\begin{definition}[$F$分布]
  设$U\sim\chi^2(n_1), V\sim\chi^2(n_2)$,且$U,V$独立,
  则称随机变量
  \begin{displaymath}
    F = \frac{U/n_1}{V/n_2}
  \end{displaymath}
  服从自由度为$(n_1,n_2)$的$F$分布,
  记为$F\sim F(n_1,n_2)$。
\end{definition}

\begin{theorem}[$F$分布的性质]
  若$F\sim F(n_1,n_2)$,则
  \begin{displaymath}
    \frac{1}{F} \sim F(n_2,n_2)
  \end{displaymath}
\end{theorem}

\begin{definition}[$F$分布的上$\alpha$分位点]
  设$X\sim F(n_1,n_2)$,$0<\alpha<1$,称满足
  \begin{displaymath}
    P(X>F_\alpha(n_1,n_2)) = \alpha
  \end{displaymath}
  的点$F_\alpha(n_1,n_2)$为$F(n_1,n_2)$分布的上$\alpha$分位点,
\end{definition}

\begin{theorem}[$F$分布的上$\alpha$分位点的性质]
  $F_{1-\alpha}(n_1,n_2) = 1/F_\alpha(n_2,n_1)$。
\end{theorem}

\subsection{正态总体的抽样分布的四个定理}
\begin{theorem}[样本均值的分布]
  设$X_1,X_2,\dots,X_n$是来自正态总体$X\sim N(\mu,\sigma^2)$的样本,则
  \begin{displaymath}
    \frac{\mbar{X}-\mu}{\sigma/\sqrt{n}}\sim N(0,1)
  \end{displaymath}
\end{theorem}

\begin{theorem}[样本方差的分布]
  设$X_1,X_2,\dots,X_n$是来自正态总体$X\sim N(\mu,\sigma^2)$的样本,则
  \begin{displaymath}
    \frac{nS_n^2}{\sigma^2} =
    \frac{\sum_{i=1}^n(X_i-\mbar{X})^2}{\sigma^2} \sim \chi^2(n-1)
  \end{displaymath}
  且$\mbar{X}$与$S_n^2$相互独立。
\end{theorem}

\begin{remark}
  作为对比,我们还能记得,根据$\chi^2$分布的定义有
  \begin{displaymath}
    \frac{\sum_{i=1}^n(X_i-\mu)^2}{\sigma^2} \sim \chi^2(n)
  \end{displaymath}
  把$\mu$换成$\mbar{X}$就会导致自由度的变化。
\end{remark}

\begin{theorem}[样本均值和样本方差的分布]
  设$X_1,X_2,\dots,X_n$是来自正态总体$X\sim N(\mu,\sigma^2)$的样本,则
  \begin{displaymath}
    \frac{\mbar{X}-\mu}{S_n/\sqrt{n-1}}\sim t(n-1)
  \end{displaymath}
\end{theorem}

\begin{remark}
  这是由样本均值的分布和样本方差的分布根据$t$分布的定义相除得到的。
\end{remark}

\begin{theorem}[两总体样本方差比、样本均值差的分布]
  设$X_1,X_2,\dots,X_{n_1}$是来自正态总体$X\sim N(\mu_1,\sigma_1^2)$的样本,
  $Y_1,Y_2,\dots,Y_{n_2}$是来自正态总体$Y\sim N(\mu_2,\sigma_2^2)$的样本,
  $S_1^2=\frac{1}{n_1-1}\sum_{i=1}^{n_1}(X_i-\mbar{X})^2$和
  $S_2^2=\frac{1}{n_2-1}\sum_{i=1}^{n_2}(Y_i-\mbar{Y})^2$
  分别是$X$和$Y$的修正样本方差,则
  \begin{enumerate}
    \item 样本方差比:
    \begin{displaymath}
      \frac{S_1^2/S_2^2}{\sigma_1^2/\sigma_2^2}\sim F(n_1-1,n_2-1)
    \end{displaymath}
    \item 样本均值差:
    \begin{displaymath}
      \frac{(\mbar{X}-\mbar{Y})-(\mu_1-\mu_2)}
      {\sqrt{\frac{\sigma_1^2}{n_1}+\frac{\sigma_2^2}{n_2}}}
      \sim N(0,1)
    \end{displaymath}
    \item 若$\sigma_1^2=\sigma_2^2=\sigma^2$,则
    \begin{displaymath}
      \frac{1}{\sigma^2}\left[(n_1-1)S_1^2+(n_2-1)S_2^1\right]
      \sim \chi^2(n_1+n_2-2)
    \end{displaymath}
    从而有
    \begin{displaymath}
      \frac{(\mbar{X}-\mbar{Y})-(\mu_1-\mu_2)}
      {S_w\sqrt{\frac{1}{n_1}+\frac{1}{n_2}}}
      \sim t(n_1+n_2-2)
    \end{displaymath}
    其中,$S_w^2=\frac{(n_1-1)S_1^2+(n_2-1)S_2^2}{n_1+n_2-2}$
    为$S_1^2$和$S_2^2$的加权平均。
  \end{enumerate}
\end{theorem}
