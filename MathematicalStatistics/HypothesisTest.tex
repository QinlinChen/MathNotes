\chapter{假设检验}

\section{假设检验的基本概念}
本小节介绍假设检验的基本概念,
包括显著水平,假设检验的一般步骤和假设检验存在的两类错误。

\subsection{显著水平}
在假设检验中,我们需要对小概率的说法给出统一界定,通常给出一个上限$\alpha$。
当一个事件发生的概率小于$\alpha$是,我们认为这是小概率事件,
而称$\alpha$为\textbf{显著水平}。

\subsection{假设检验的步骤}
\begin{enumerate}
  \item
  根据实际问题,提出原假设$H_0$和备择假设$H_1$;
  \item
  确定检验统计量;
  \item
  根据显著水平$\alpha$确定拒绝域;
  \item
  由样本计算统计值;
  \item
  做出判断是否接受$H_0$。
\end{enumerate}

\subsection{两类错误}
假设检验可能会出现两类错误:
\begin{enumerate}
  \item
  第一类错误:当$H_0$为真时,我们仍有可能拒绝$H_0$,此时犯了``弃真''的错误。
  第一类错误的概率就是显著水平$\alpha$。
  \item
  第二类错误:当$H_0$为假时,我们仍有可能接受$H_0$,此时犯了``存伪''的错误。
  第二类错误的概率用$\beta$表示。
\end{enumerate}

Neyman-Pearson原则:
在控制第一类错误概率的前提下,尽量减小犯第二类错误的概率。

\section{正态总体的假设检验}
本小节给出单正态总体的均值和方差、双正态总体的均值差和方差的检验方法。

\subsection{单正态总体的均值和方差的检验}
设$X_1,X_2,\dots,X_n$为正态总体$X\sim N(\mu,\sigma^2)$的一个样本。
\begin{enumerate}
  \item
  已知方差$\sigma^2$,检验均值$\mu$:
  \begin{displaymath}
    H_0:\mu=\mu_0 \qquad
    H_1: \begin{cases}
      \mu\neq\mu_0 \\
      \mu < \mu_0 \\
      \mu > \mu_0
    \end{cases} 
  \end{displaymath}
  \begin{displaymath}
    \text{检验统计量}: U=\frac{\mbar{X}-\mu}{\sigma/\sqrt{n}}\sim N(0,1)
  \end{displaymath}
  \begin{displaymath}
    \text{拒绝域}: \begin{cases}
      |\widetilde{U}| \ge u_{\alpha/2} & (\mu\neq\mu_0) \\
      \widetilde{U}   \le -u_\alpha    & (\mu < \mu_0) \\
      \widetilde{U}   \ge u_\alpha     & (\mu < \mu_0)
    \end{cases}
  \end{displaymath}
  \item
  未知方差,检验均值$\mu$:
  \begin{displaymath}
    H_0:\mu=\mu_0 \qquad
    H_1: \begin{cases}
      \mu\neq\mu_0 \\
      \mu < \mu_0 \\
      \mu > \mu_0
    \end{cases}
  \end{displaymath}
  \begin{displaymath}
    \text{检验统计量}: T=\frac{\mbar{X}-\mu}{S_n/\sqrt{n-1}}\sim t(n-1)
  \end{displaymath}
  \begin{displaymath}
    \text{拒绝域}: \begin{cases}
      |\widetilde{T}| \ge t_{\alpha/2} & (\mu\neq\mu_0) \\
      \widetilde{T}   \le -t_\alpha    & (\mu < \mu_0) \\
      \widetilde{T}   \ge t_\alpha     & (\mu < \mu_0)
    \end{cases}
  \end{displaymath}
  \item 
  检验方差$\sigma^2$:
  \begin{displaymath}
    H_0:\sigma^2=\sigma_0^2 \qquad
    H_1: \begin{cases}
      \sigma^2\neq\sigma_0^2 \\
      \sigma^2 < \sigma_0^2 \\
      \sigma^2 > \sigma_0^2
    \end{cases}
  \end{displaymath}
  \begin{displaymath}
    \text{检验统计量}: \chi^2=\frac{nS_n^2}{\sigma^2}\sim \chi^2(n-1)
  \end{displaymath}
  \begin{displaymath}
    \text{拒绝域}: \begin{cases}
      \widetilde{\chi^2} \ge \chi^2_{\alpha/2}(n-1)
        \ \text{或}\ \widetilde{\chi^2} \le \chi^2_{1-\alpha/2}(n-1)
        & (\sigma^2\neq\sigma_0^2) \\
      \widetilde{\chi^2} \le \chi^2_{1-\alpha/2}(n-1)
        & (\sigma^2 < \sigma_0^2) \\
      \widetilde{\chi^2} \ge \chi^2_{\alpha/2}(n-1)
        & (\sigma^2 > \sigma_0^2)
    \end{cases}
  \end{displaymath}
\end{enumerate}

\subsection{双正态总体的均值差和方差的检验}
设$X_1,X_2,\dots,X_{n_1}$为正态总体$X\sim N(\mu_1,\sigma_1^2)$的样本,
$Y_1,Y_2,\dots,Y_{n_2}$为正态总体$Y\sim N(\mu_2,\sigma_2^2)$的样本。
$\mbar{X},S_1^2,\mbar{Y},S_2^2$分别表示$X,Y$的样本均值与修正样本方差。
设$X,T$独立。
\begin{enumerate}
  \item 
  已知$\sigma_1^2,\sigma_2^2$,检验均值$\mu_1=\mu_2$:
  \begin{displaymath}
    H_0:\mu_1=\mu_2 \qquad
    H_1: \begin{cases}
      \mu_1\neq\mu_2 \\
      \mu_1 < \mu_2 \\
      \mu_2 > \mu_2
    \end{cases} 
  \end{displaymath}
  \begin{displaymath}
    \text{检验统计量}: U=\frac{(\mbar{X}-\mbar{Y})}
      {\sqrt{\frac{\sigma_1^2}{n_1}+\frac{\sigma_2^2}{n_2}}}\sim N(0,1)
  \end{displaymath}
  \begin{displaymath}
    \text{拒绝域}: \begin{cases}
      |\widetilde{U}| \ge u_{\alpha/2} & (\mu_1\neq\mu_2) \\
      \widetilde{U}   \le -u_\alpha    & (\mu_1 < \mu_2) \\
      \widetilde{U}   \ge u_\alpha     & (\mu_2 > \mu_2)
    \end{cases}
  \end{displaymath}
  \item
  检验方差$\sigma_1^2=\sigma_2^2$:
  \begin{displaymath}
    H_0:\sigma_1^2=\sigma_2^2 \qquad
    H_1: \begin{cases}
      \sigma_1^2\neq\sigma_2^2 \\
      \sigma_1^2 < \sigma_2^2 \\
      \sigma_1^2 > \sigma_2^2
    \end{cases} 
  \end{displaymath}
  \begin{displaymath}
    \text{检验统计量}: F=\frac{S_1^2}{S_2^2} \sim F(n_1-1,n_2-1)
  \end{displaymath}
  \begin{displaymath}
    \text{拒绝域}: \begin{cases}
      \widetilde{F} \ge F_{\alpha/2}(n_1-1,n_2-1)\ \text{或}& \\
      \qquad\widetilde{F} \le F_{1-\alpha/2}(n_1-1,n_2-1)
        & (\sigma_1^2\neq\sigma_2^2) \\
      \widetilde{F} \le F_{1-\alpha/2}(n_1-1,n_2-1)
        & (\sigma_1^2 < \sigma_2^2) \\
      \widetilde{F} \ge F_{\alpha/2}(n_1-1,n_2-1)
        & (\sigma_1^2 > \sigma_2^2)
    \end{cases}
  \end{displaymath}
\end{enumerate}
