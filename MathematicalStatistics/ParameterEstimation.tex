\chapter{参数估计}

\section{点估计}
本小节首先讨论两种点估计的方法:矩估计和极大似然估计。
然后讨论估计量的评选标准。

\subsection{点估计的概念}
参数的\textbf{点估计}就是对总体分布中的未知参数$\theta$,
以样本$X_1,X_2,\dots,X_n$构造统计量
$\hat{\theta}(X_1,X_2,\dots,X_n)$作为参数$\theta$的估计,
称$\hat{\theta}(X_1,X_2,\dots,X_n)$为参数$\theta$\textbf{估计量}。

当测得样本值$(x_1,x_2,\dots,x_n)$时,
代入$\hat{\theta}(X_1,X_2,\dots,X_n)$,
即可得到参数$\theta$\textbf{估计值}:
$\hat{\theta}(x_1,x_2,\dots,x_n)$。

\subsection{矩估计}
矩估计的思想是:以样本矩作为总体矩的估计,从而得到参数的估计量。
具体方法如下。

设$X_1,X_2,\dots,X_n$为来自总体$X$的样本,总体$X$的分布函数为
\begin{displaymath}
  F(x;\theta_1,\theta_2,\dots,\theta_k)
\end{displaymath}
其中$\theta_1,\theta_2,\dots,\theta_k$为未知参数。
记$\mu_m(\theta_1,\theta_2,\dots,\theta_k)=\mexpect[X^m]$
为总体的$m$阶矩,
$A_m=\frac{1}{n}\sum_{i=1}^{n}X_i^m$为样本的$m$阶矩。

若$\mu_m\ (m=1,2,\dots,k)$都存在,那么我们联立方程
\begin{displaymath}
  \meqs{c}{
    \mu_1(\theta_1,\theta_2,\dots,\theta_k) = A_1 \\
    \vdots \\
    \mu_k(\theta_1,\theta_2,\dots,\theta_k) = A_k
  }
\end{displaymath}
从中解出方程组的解$\hat{\theta}_1,\hat{\theta}_2,\dots,\hat{\theta}_k$。

我们把$\hat{\theta}_1,\hat{\theta}_2,\dots,\hat{\theta}_k$
作为$\theta_1,\theta_2,\dots,\theta_k$的估计量,称为\textbf{矩估计量}。
矩估计量的观察值叫做\textbf{矩估计值}。

\begin{theorem}
  无论总体$X$服从何种分布,
  若总体均值$\mu$和总体方差$\sigma^2$为未知参数,
  那么其矩估计量一定是样本均值和样本方差,即:
  \begin{displaymath}
    \hat{\mu}=\mbar{X},\quad \hat{\sigma^2}=S_n^2
  \end{displaymath}
\end{theorem}

\begin{remark}
  注意,对于矩估计来说,$\hat{\sigma^2}$和$\hat{\sigma}^2$不一定是相同的。
  前者是方差的矩估计量,后者是标准差矩估计量的平方。
  但对于后面介绍的极大似然估计来说,它们是相同的。
\end{remark}

\subsection{极大似然估计}
极大似然估计的思想是:
以样本$X_1,X_2,\dots,X_n$的观测值$x_1,x_2,\dots,x_n$来
估计参数$\theta_1,\theta_2,\dots,\theta_k$。
若选取$\hat{\theta}_1,\hat{\theta}_2,\dots,\hat{\theta}_k$
使观测值出现的概率最大,那么把
$\hat{\theta}_1,\hat{\theta}_2,\dots,\hat{\theta}_k$
作为参数$\theta_1,\theta_2,\dots,\theta_k$的估计量。
下面分离散和连续两种情形来讨论具体方法。

\begin{enumerate}
  \item
  若总体$X$为离散型,其分布律的形式已知为
  \begin{displaymath}
    P(X=x) = f(x; \theta)
  \end{displaymath}
  其中$\theta$为待估参数(这里仅讨论一个参数的情况)。
  又设$x_1,x_2,\dots,x_n$是样本$X_1,X_2,\dots,X_n$的一组样本值。
  那么事件$X_1=x_1$,$X_2=x_2$,$\dots$,$X_n=x_n$同时发生的概率为
  \begin{displaymath}
    L(\theta) = P(X_1=x_1,X_2=x_2,\dots,X_n=x_n)
    = \prod_{i=1}^n f(x_i;\theta)
  \end{displaymath}
  我们把$L(\theta)$称为样本的\textbf{极大似然函数}。取
  \begin{displaymath}
    \hat{\theta}(x_1,x_2,\dots,x_n)= \argmax_\theta L(\theta)
  \end{displaymath}
  为参数$\theta$的\textbf{极大似然估计值}。
  $\hat{\theta}(X_1,X_2,\dots,X_n)$称为$\theta$的\textbf{极大似然估计量}。
  \item
  若总体$X$为连续型,其概率密度的形式已知为
  \begin{displaymath}
    p(x;\theta)
  \end{displaymath}
  其中$\theta$为待估参数(这里仅讨论一个参数的情况)。
  设$x_1,x_2,\dots,x_n$是样本$X_1,X_2,\dots,X_n$的一组样本值,
  那么连续型随机变量的极大似然函数为
  \begin{displaymath}
    L(\theta) = \prod_{i=1}^n p(x_i;\theta)
  \end{displaymath}
  其极大似然估计量和估计值的定义和离散型的定义相同。
\end{enumerate}

现在我们讨论求解$\hat{\theta}=\argmax_\theta L(\theta)$的方法。
若极大似然函数$L$只有一个参数,即$L=L(\theta)$,
那么通过下列方程来解$\theta$:
\begin{displaymath}
  \mdiffs{\theta}{L}=0
  \quad\text{或}\quad
  \mdiffs{\theta}{\ln L}=0
\end{displaymath}
若极大似然函数$L$有多个参数,即$L=L(\theta_1,\theta_2,\dots,\theta_k)$,
那么通过下列方程组来解$\theta_1,\theta_2,\dots,\theta_k$:
\begin{displaymath}
  \meqs{c}{
    \mpartials{\theta_1}{L} = 0\\
    \vdots \\
    \mpartials{\theta_k}{L} = 0
  }\quad\text{或}\quad
  \meqs{c}{
    \mpartials{\theta_1}{\ln L} = 0\\
    \vdots \\
    \mpartials{\theta_k}{\ln L} = 0
  }
\end{displaymath}

\begin{theorem}[极大似然估计的不变性]
  设$\hat{\theta}$是$\theta$的极大似然估计量,
  $u=u(\theta)$是$\theta$的函数,且有单值反函数,
  则$\hat{u}=u(\hat{\theta})$是$u(\theta)$的极大似然估计量。
\end{theorem}

\subsection{估计量的评选标准}

\begin{definition}[无偏性]
  设$\theta$的估计量为$\hat{\theta}$。若
  $\mexpect[\hat{\theta}]=\theta$,
  则称$\hat{\theta}$是$\theta$的\textbf{无偏估计量}。
\end{definition}

\begin{remark}
  $S_n^2$不是$\sigma^2$的无偏估计量,
  而$S_{n-1}^2$才是$\sigma_2$的无偏估计量,
  但$S_{n-1}$不是$\sigma$的无偏估计量。
  由此也能看出,$\hat{\theta}$是$\theta$的无偏估计量
  不一定能推出$g(\hat{\theta})$是$g(\theta)$的无偏估计量。
\end{remark}

\begin{definition}[有效性]
  设$\hat{\theta}_1,\hat{\theta}_2$是$\theta$的无偏估计量。
  若$\mvar(\hat{\theta}_2)\le \mvar(\hat{\theta}_1)$,
  则称$\hat{\theta}_2$比$\hat{\theta}_1$\textbf{有效}。
\end{definition}

\begin{definition}[一致性]
  设$\hat{\theta}_n=\hat{\theta}_n(X_1,X_2,\dots,X_n)$
  是$\theta$的估计量。
  若$\hat{\theta}_n \mprto \theta$,
  则称$\hat{\theta}_n$是$\theta$的\textbf{一致估计量}。
\end{definition}

下面是两个常用的结论:

\begin{theorem}
  样本的$k$阶矩是总体$k$阶矩的一致性估计量。
\end{theorem}

\begin{theorem}
  设$\hat{\theta}$是$\theta$的无偏估计量。若
  \begin{displaymath}
    \lim_{n\to\infty}\mvar(\hat{\theta})=0
  \end{displaymath}
  则$\hat{\theta}$是$\theta$的一致估计量。
\end{theorem}

\section{区间估计}
% TODO

\subsection{区间估计的概念}
参数的\textbf{区间估计}是对总体分布中的未知参数$\theta$,
以样本$X_1,X_2,\dots,X_n$构造2个统计量
$\hat{\theta_1}(X_1,X_2,\dots,X_n)$和
$\hat{\theta_2}(X_1,X_2,\dots,X_n)$,
以区间$(\theta_1,\theta_2)$作为参数$\theta$的估计,
使得对给定的概率$1-\alpha$,满足:
\begin{displaymath}
  P\left(\hat{\theta_1}(X_1,X_2,\dots,X_n) < \theta <
    \hat{\theta_2}(X_1,X_2,\dots,X_n)\right) = 1-\alpha 
\end{displaymath}  

\subsection{单个正态总体的均值和方差的区间估计}

\subsection{两个正态总体的均值差和方差比的区间估计}