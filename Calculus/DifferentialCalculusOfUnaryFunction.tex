\chapter{一元函数的微分学}

\section{导数与微分的概念}
本小节给出导数与微分的定义。
并且介绍导数与连续的关系、导数与微分的关系。

\subsection{导数的概念}
\begin{definition}[导数]
  设函数$f$在点$a$的某一邻域内有定义。
  若极限
  \begin{displaymath}
    \lim_{\Delta x\to 0}\frac{f(a+\Delta x)-f(a)}{\Delta x}
  \end{displaymath}
  存在有限,则称函数$f$在点$a$\textbf{可导},
  并称此极限值为$f$在点$a$的\textbf{导数},记为
  \begin{displaymath}
    f'(a)\quad\textrm{或}\quad \left.\frac{\md f}{\md x}\right|_a
  \end{displaymath}
\end{definition}

\begin{definition}[左导数和右导数]
  极限
  \begin{displaymath}
    \lim_{\Delta x\to 0-}\frac{f(a+\Delta x)-f(a)}{\Delta x}
    \quad\textrm{以及}\quad
    \lim_{\Delta x\to 0+}\frac{f(a+\Delta x)-f(a)}{\Delta x}
  \end{displaymath}
  被称为函数$f$在点$a$的\textbf{左导数}和\textbf{右导数}
\end{definition}
\begin{remark}
  $f$在点$a$可导$\iff$ $f$在点$a$的左导数和右导数存在且相等。
\end{remark}

\begin{theorem}[可导与连续的关系]
  若函数$f$在点$a$可导,则$f$在点$a$连续。
\end{theorem}
\begin{remark}
  反之不成立。即在点$a$连续的函数未必在点$a$可导。
\end{remark}

\subsection{微分的概念}
\begin{definition}[微分]
  设函数$f$在区间$I$上有定义。
  对点$a\in I$,当自变量有增量$\Delta x$时,
  相应地,因变量$y$有增量$\Delta y=f(a+\Delta x)-f(a)$。
  若当$\Delta x\to 0$时,有
  \begin{displaymath}
    \Delta y = A\Delta x +o(\Delta x)
  \end{displaymath}
  其中$A$与$\Delta x$无关(一般与$a$有关),
  则称函数$f$在点$a$\textbf{可微},
  且称$A\Delta x$为$f$在点$a$的\textbf{微分},记作
  \begin{displaymath}
    \md f(a)\quad\textrm{或}\quad \left.\md y\right|_a
  \end{displaymath}
\end{definition}

\begin{remark}
  当$\Delta x\to 0$时,无穷小$\md y$是$\Delta y$的主部。
  又因为$\md y$关于$\Delta x$是一次的,
  故称$\md y$是$\Delta y$的\textbf{线性主部}。
\end{remark}

\begin{theorem}[可导与可微等价]
  函数$f$在点$a$可微的充要条件是
  $f$在点$a$存在有限导数$f'(a)$,
  于是就有
  \begin{displaymath}
    \md y = f'(a)\Delta x
  \end{displaymath}
\end{theorem}

\begin{remark}
  我们约定自变量$x$的增量$\Delta x$为自变量的微分,
  即$\md x = \Delta x$,
  于是上面的微分可以写成
  \begin{displaymath}
    \md y = f'(a)\md x
  \end{displaymath}
  需要注意的是,
  $\frac{\md y}{\md x} = f'(x)$中的左半部分应当视为一个整体,
  而不是一个分子除以分母的形式。
  只是因为上述定理的存在,
  才使得看起来只要把等式左边的分母移到右边就能得到$f$的微分形式。
  事实上,\textbf{该定理只对一元函数成立}。
\end{remark}

\section{导数与微分的计算}
本小节首先说明了导数和微分的四则运算法则。
然后讨论了复合函数的求导法则以及一阶微分形式的不变性。
接着讨论了反函数导数的求法。
最后,我们列出一些常用且容易遗忘的函数的导数。

\subsection{四则运算}
\begin{theorem}[导数的四则运算]
  设函数$u,v$在点$x$可导,那么
  \begin{enumerate}
    \item 
    $y=u\pm v$在点$x$可导,且有
    \begin{displaymath}
      (u\pm v)'=u'\pm v'
    \end{displaymath}
    \item 
    $y=uv$在点$x$可导,且有
    \begin{displaymath}
      (uv)' = u'v+uv'
    \end{displaymath}
    \item 
    当$v(x)\neq 0$时,$y=u/v$在点$x$可导,且有
    \begin{displaymath}
      \left(\frac{u}{v}\right)'=\frac{u'v-uv'}{v^2}
    \end{displaymath}
  \end{enumerate}
\end{theorem}

\begin{theorem}[微分的四则运算]
  设函数$u,v$在点$x$可微,那么
  \begin{enumerate}
    \item 
    $\md (u\pm v) = \md u \pm \md v$
    \item 
    $\md (uv) = v\md u + u \md v$
    \item 
    $\md (\frac{u}{v}) = \frac{v\md u - u\md v}{v^2}$
  \end{enumerate}
\end{theorem}

\subsection{复合函数的导数}
\begin{theorem}[复合函数的导数]
  设函数$y=f(u)$与$u=u(x)$在$x_0$的邻域能构成复合函数,
  且$u=u(x)$在点$x_0$可导,$y=f(x)$在点$u_0=u(x_0)$可导。
  那么,复合函数$f\circ u$在点$x_0$可导,且有
  \begin{displaymath}
    (f\circ u)'(x_0) = f'(u_0)u'(x_0)
  \end{displaymath}
\end{theorem}

\begin{corollary}[一阶微分形式不变性]
  设$y=f(u)$,则无论$u$是自变量还是中间变量$u=u(x)$,
  其微分形式不变,都是
  \begin{displaymath}
    \md y = f'(u)\md u
  \end{displaymath}
\end{corollary}

\begin{remark}
  该定理对高阶微分不成立:若$y=f(x)$,但$x$是中间变量,
  即$x=u(t)$,那么有
  \begin{align*}
    \md^2 y
    &= \md (f'(x)\md x) \\
    &= f''(x)\md x^2 + f'(x)\md^2 x
      & (\text{不是$f''(x)\md x^2$}) \\
    &= f''(x)\left(u'(t)\md t\right)^2 + f'(x)(u''(t)\md t^2)\\
    &= \left(f''(x)u'(t)^2 + f'(x)u''(t)\right)\md t^2
  \end{align*}
\end{remark}

\subsection{反函数的导数}
\begin{theorem}[反函数的导数]
  设函数$x=\phi(y)$在某一区间$I$内严格单调,
  有在区间$I$内一点$y$出导数$\phi(y)$存在且不为零,
  则反函数$y=f(x)$在对应点$x$出具有导数$f'(x)$,且
  \begin{displaymath}
    f'(x) = \frac{1}{\phi'(y)}
  \end{displaymath}
\end{theorem}

\subsection{公式表}
下面是一些常用且容易遗忘的函数的导数:
\begin{center}
  \begin{tabular}{|c|c|}
    \hline 
    $f(x)$ & $f'(x)$ \\ 
    \hline 
    $a^x$ & $a^x\ln a$  \\ 
    \hline 
    $\log_a |x|$ & $\frac{1}{x\ln a}$ \\ 
    \hline 
    $\arcsin x$ & $\frac{1}{\sqrt{1-x^2}}$ \\ 
    \hline 
    $\arccos x$ & $-\frac{1}{\sqrt{1-x^2}}$ \\ 
    \hline 
    $\arctan x$ & $\frac{1}{1+x^2}$ \\ 
    \hline 
  \end{tabular} 
\end{center}

\section{高阶导数与高阶微分}
本小节我们讨论高阶导数和高阶微分。
需要注意高阶微分与一阶微分性质的差异。

\subsection{高阶导数}
对于函数$f$,如果它的一阶导数可导,那么对之求导能得到二阶导数。
如果它的二阶导数可导,那么对之求导能得到三阶导数。
以此类推,我们可以得到$n$阶导数,并称它$n$阶可导。

对于$n$阶可导的函数$f$,我们把它的$n$阶导数记为
\begin{displaymath}
  f^{(n)}\quad\text{或}\quad \frac{\md^n f}{\md x^n}
\end{displaymath}

下面是一些常见导数的高阶导数:
\begin{center}
  \begin{tabular}{|c|c|}
    \hline 
    $f(x)$ & $f^{(n)}(x)$ \\ 
    \hline 
    $a^x$ & $a^x\ln^n a$  \\ 
    \hline 
    $x^\mu$ & $\mu(\mu-1)(\mu-2)\dots(\mu-n+1)x^{\mu-n }$ \\ 
    \hline
    $x^{-1}$ & $(-1)^n\frac{n!}{x^{n+1}}$ \\
    \hline 
    $\ln(x+1)$ & $\left[(1+x)^{-1}\right]^{(n-1)}=(-1)^{n-1}\frac{(n-1)!}{(1+x)^n}$ \\ 
    \hline 
    $\sin x$ & $\sin(x + \frac{n\pi}{2})$ \\ 
    \hline 
  \end{tabular} 
\end{center}

\begin{theorem}[高阶导数的运算公式]
  设函数$u,v$都$n$阶可导,则
  \begin{enumerate}
    \item 
    $(u\pm v)^{(n)} = u^{(n)} \pm v^{(n)}$
    \item 
    $(uv)^{(n)}=\sum_{i=0}^{n}\binom{n}{i}u^{(n-i)}v^{(i)}$
  \end{enumerate}
\end{theorem}

\subsection{高阶微分}
和高阶导数一样,对一阶微分$\md y$再求微分得到二阶微分,
记为$\md^2 y$,即
\begin{displaymath}
  \md^2 y = \md(\md y)
\end{displaymath}
类似可定义各阶微分。

对$y=f(x)$,当$x$是自变量时,$\md x = \Delta x$可任意变化,
与$x$无关,因此,对$x$求导时,$\md x$可看作常数,故有
\begin{align*}
  \md^2 y
  &= \md(f'(x)\md x) \\
  &= \md(f'(x))\md x \\
  &= f''(x)(\md x)^2
\end{align*}
为书写方便,$(\md x)^2$习惯上记作$dx^2$,于是
\begin{displaymath}
  \md^2 y = f''(x)\md x^2
\end{displaymath}
以此类推,有
\begin{displaymath}
  \md^n y = f^{(n)}(x) \md x^n
\end{displaymath}
这样,前面引入的高阶导数记号$\frac{\md^n y}{\md x^n}$,
现在可以看成是函数$y$的$n$阶微分$\md^n y$
与自变量的一阶微分$\md x$的$n$次幂之商了。

需要注意的是,当$x$不是自变量而是中间变量时,即$y=f(x)$,$x=x(t)$,
$\md x$不可视作任意变换的常数$\Delta x$了,
而是有$\md x = x'(t)\md t$,因此
\begin{align*}
  \md^2 y
  &= \md (f'(x)\md x) \\
  &= f''(x)\md x^2 + f'(x)\md^2 x \\
  &= f''(x)\left(u'(t)\md t\right)^2 + f'(x)(u''(t)\md t^2)\\
  &= \left(f''(x)u'(t)^2 + f'(x)u''(t)\right)\md t^2
\end{align*}
从上面第二行等式可以看出,高阶微分不满足形式不变性。

\section{微分中值定理}

\section{洛必达法则}

\section{泰勒公式}

\section{利用导数研究函数的性质}

\section{利用导数作函数的图形}
