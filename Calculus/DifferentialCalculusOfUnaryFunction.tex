\chapter{一元函数的微分学}

\section{导数与微分的概念}
本小节给出导数与微分的定义。
并且介绍导数与连续的关系、导数与微分的关系。

\subsection{导数的概念}
\begin{definition}[导数]
  设函数$f$在点$a$的某一邻域内有定义。
  若极限
  \begin{displaymath}
    \lim_{\Delta x\to 0}\frac{f(a+\Delta x)-f(a)}{\Delta x}
  \end{displaymath}
  存在有限,则称函数$f$在点$a$\textbf{可导},
  并称此极限值为$f$在点$a$的\textbf{导数},记为
  \begin{displaymath}
    f'(a)\quad\textrm{或}\quad \left.\frac{\md f}{\md x}\right|_a
  \end{displaymath}
\end{definition}

\begin{definition}[左导数和右导数]
  极限
  \begin{displaymath}
    \lim_{\Delta x\to 0-}\frac{f(a+\Delta x)-f(a)}{\Delta x}
    \quad\textrm{以及}\quad
    \lim_{\Delta x\to 0+}\frac{f(a+\Delta x)-f(a)}{\Delta x}
  \end{displaymath}
  被称为函数$f$在点$a$的\textbf{左导数}和\textbf{右导数}
\end{definition}
\begin{remark}
  $f$在点$a$可导$\iff$ $f$在点$a$的左导数和右导数存在且相等。
\end{remark}

\begin{theorem}[可导与连续的关系]
  若函数$f$在点$a$可导,则$f$在点$a$连续。
\end{theorem}
\begin{remark}
  反之不成立。即在点$a$连续的函数未必在点$a$可导。
\end{remark}

\subsection{微分的概念}
\begin{definition}[微分]
  设函数$f$在区间$I$上有定义。
  对点$a\in I$,当自变量有增量$\Delta x$时,
  相应地,因变量$y$有增量$\Delta y=f(a+\Delta x)-f(a)$。
  若当$\Delta x\to 0$时,有
  \begin{displaymath}
    \Delta y = A\Delta x +o(\Delta x)
  \end{displaymath}
  其中$A$与$\Delta x$无关(一般与$a$有关),
  则称函数$f$在点$a$\textbf{可微},
  且称$A\Delta x$为$f$在点$a$的\textbf{微分},记作
  \begin{displaymath}
    \md f(a)\quad\textrm{或}\quad \left.\md y\right|_a
  \end{displaymath}
\end{definition}

\begin{remark}
  当$\Delta x\to 0$时,无穷小$\md y$是$\Delta y$的主部。
  又因为$\md y$关于$\Delta x$是一次的,
  故称$\md y$是$\Delta y$的\textbf{线性主部}。
\end{remark}

\begin{theorem}[可导与可微等价]
  函数$f$在点$a$可微的充要条件是
  $f$在点$a$存在有限导数$f'(a)$,
  于是就有
  \begin{displaymath}
    \md y = f'(a)\Delta x
  \end{displaymath}
\end{theorem}

\begin{remark}
  我们约定自变量$x$的增量$\Delta x$为自变量的微分,
  即$\md x = \Delta x$,
  于是上面的微分可以写成
  \begin{displaymath}
    \md y = f'(a)\md x
  \end{displaymath}
  需要注意的是,
  $\frac{\md y}{\md x} = f'(x)$中的左半部分应当视为一个整体,
  而不是一个分子除以分母的形式。
  只是因为上述定理的存在,
  才使得看起来只要把等式左边的分母移到右边就能得到$f$的微分形式。
  事实上,\textbf{该定理只对一元函数成立}。
\end{remark}

\section{导数与微分的计算}
本小节首先说明了导数和微分的四则运算法则。
然后讨论了复合函数的求导法则以及一阶微分形式的不变性。
接着讨论了反函数导数的求法。
最后,我们列出一些常用且容易遗忘的函数的导数。

\subsection{四则运算}
\begin{theorem}[导数的四则运算]
  设函数$u,v$在点$x$可导,那么
  \begin{enumerate}
    \item
    $y=u\pm v$在点$x$可导,且有
    \begin{displaymath}
      (u\pm v)'=u'\pm v'
    \end{displaymath}
    \item
    $y=uv$在点$x$可导,且有
    \begin{displaymath}
      (uv)' = u'v+uv'
    \end{displaymath}
    \item
    当$v(x)\neq 0$时,$y=u/v$在点$x$可导,且有
    \begin{displaymath}
      \left(\frac{u}{v}\right)'=\frac{u'v-uv'}{v^2}
    \end{displaymath}
  \end{enumerate}
\end{theorem}

\begin{theorem}[微分的四则运算]
  设函数$u,v$在点$x$可微,那么
  \begin{enumerate}
    \item
    $\md (u\pm v) = \md u \pm \md v$
    \item
    $\md (uv) = v\md u + u \md v$
    \item
    $\md (\frac{u}{v}) = \frac{v\md u - u\md v}{v^2}$
  \end{enumerate}
\end{theorem}

\subsection{复合函数的导数}
\begin{theorem}[复合函数的导数]
  设函数$y=f(u)$与$u=u(x)$在$x_0$的邻域能构成复合函数,
  且$u=u(x)$在点$x_0$可导,$y=f(x)$在点$u_0=u(x_0)$可导。
  那么,复合函数$f\circ u$在点$x_0$可导,且有
  \begin{displaymath}
    (f\circ u)'(x_0) = f'(u_0)u'(x_0)
  \end{displaymath}
\end{theorem}

\begin{corollary}[一阶微分形式不变性]
  设$y=f(u)$,则无论$u$是自变量还是中间变量$u=u(x)$,
  其微分形式不变,都是
  \begin{displaymath}
    \md y = f'(u)\md u
  \end{displaymath}
\end{corollary}

\begin{remark}
  该定理对高阶微分不成立:若$y=f(x)$,但$x$是中间变量,
  即$x=u(t)$,那么有
  \begin{align*}
    \md^2 y
    &= \md (f'(x)\md x) \\
    &= f''(x)\md x^2 + f'(x)\md^2 x
      & (\text{不是$f''(x)\md x^2$}) \\
    &= f''(x)\left(u'(t)\md t\right)^2 + f'(x)(u''(t)\md t^2)\\
    &= \left(f''(x)u'(t)^2 + f'(x)u''(t)\right)\md t^2
  \end{align*}
\end{remark}

\subsection{反函数的导数}
\begin{theorem}[反函数的导数]
  设函数$x=\phi(y)$在某一区间$I$内严格单调,
  有在区间$I$内一点$y$出导数$\phi(y)$存在且不为零,
  则反函数$y=f(x)$在对应点$x$出具有导数$f'(x)$,且
  \begin{displaymath}
    f'(x) = \frac{1}{\phi'(y)}
  \end{displaymath}
\end{theorem}

\subsection{公式表}
下面是一些常用且容易遗忘的函数的导数:
\begin{center}
  \begin{tabular}{|c|c|}
    \hline
    $f(x)$ & $f'(x)$ \\
    \hline
    $a^x$ & $a^x\ln a$  \\
    \hline
    $\log_a |x|$ & $\frac{1}{x\ln a}$ \\
    \hline
    $\arcsin x$ & $\frac{1}{\sqrt{1-x^2}}$ \\
    \hline
    $\arccos x$ & $-\frac{1}{\sqrt{1-x^2}}$ \\
    \hline
    $\arctan x$ & $\frac{1}{1+x^2}$ \\
    \hline
  \end{tabular}
\end{center}

\section{高阶导数与高阶微分}
本小节我们讨论高阶导数和高阶微分。
需要注意高阶微分与一阶微分性质的差异。

\subsection{高阶导数}
对于函数$f$,如果它的一阶导数可导,那么对之求导能得到二阶导数。
如果它的二阶导数可导,那么对之求导能得到三阶导数。
以此类推,我们可以得到$n$阶导数,并称它$n$阶可导。

对于$n$阶可导的函数$f$,我们把它的$n$阶导数记为
\begin{displaymath}
  f^{(n)}\quad\text{或}\quad \frac{\md^n f}{\md x^n}
\end{displaymath}

下面是一些常见导数的高阶导数:
\begin{center}
  \begin{tabular}{|c|c|}
    \hline
    $f(x)$ & $f^{(n)}(x)$ \\
    \hline
    $a^x$ & $a^x\ln^n a$  \\
    \hline
    $x^\mu$ & $\mu(\mu-1)(\mu-2)\dots(\mu-n+1)x^{\mu-n }$ \\
    \hline
    $x^{-1}$ & $(-1)^n\frac{n!}{x^{n+1}}$ \\
    \hline
    $\ln(x+1)$ & $\left[(1+x)^{-1}\right]^{(n-1)}=(-1)^{n-1}\frac{(n-1)!}{(1+x)^n}$ \\
    \hline
    $\sin x$ & $\sin(x + \frac{n\pi}{2})$ \\
    \hline
  \end{tabular}
\end{center}

\begin{theorem}[高阶导数的运算公式]
  设函数$u,v$都$n$阶可导,则
  \begin{enumerate}
    \item
    $(u\pm v)^{(n)} = u^{(n)} \pm v^{(n)}$
    \item
    $(uv)^{(n)}=\sum_{i=0}^{n}\binom{n}{i}u^{(n-i)}v^{(i)}$
  \end{enumerate}
\end{theorem}

\subsection{高阶微分}
和高阶导数一样,对一阶微分$\md y$再求微分得到二阶微分,
记为$\md^2 y$,即
\begin{displaymath}
  \md^2 y = \md(\md y)
\end{displaymath}
类似可定义各阶微分。

对$y=f(x)$,当$x$是自变量时,$\md x = \Delta x$可任意变化,
与$x$无关,因此,对$x$求导时,$\md x$可看作常数,故有
\begin{align*}
  \md^2 y
  &= \md(f'(x)\md x) \\
  &= \md(f'(x))\md x \\
  &= f''(x)(\md x)^2
\end{align*}
为书写方便,$(\md x)^2$习惯上记作$\md x^2$,于是
\begin{displaymath}
  \md^2 y = f''(x)\md x^2
\end{displaymath}
以此类推,有
\begin{displaymath}
  \md^n y = f^{(n)}(x) \md x^n
\end{displaymath}
这样,前面引入的高阶导数记号$\frac{\md^n y}{\md x^n}$,
现在可以看成是函数$y$的$n$阶微分$\md^n y$
与自变量的一阶微分$\md x$的$n$次幂之商了。

需要注意的是,当$x$不是自变量而是中间变量时,即$y=f(x)$,$x=x(t)$,
$\md x$不可视作可任意变化的常数$\Delta x$了,
而是有$\md x = x'(t)\md t$,因此
\begin{align*}
  \md^2 y
  &= \md (f'(x)\md x) \\
  &= f''(x)\md x^2 + f'(x)\md^2 x \\
  &= f''(x)\left(u'(t)\md t\right)^2 + f'(x)(u''(t)\md t^2)\\
  &= \left(f''(x)u'(t)^2 + f'(x)u''(t)\right)\md t^2
\end{align*}
从上面第二行等式可以看出,高阶微分不满足形式不变性。

\section{微分中值定理}
本小节,我们会介绍费马引理、罗尔定理、拉格朗日中值定理和柯西中值定理。
因为这些定理的证明思路比较重要,并且也不是很难,所以我们会给出证明。

\subsection{两个引理}
\begin{definition}[极值]
  设$f$是定义在区间$I$上的一个实函数,$a\in I$。
  如果存在$a$的$\delta$邻域$U(a,\delta)$,使
  \begin{displaymath}
    f(x) \le a,\quad x\in U
  \end{displaymath}
  则称$f$在点$a$取\textbf{极大值}$f(a)$,
  称$a$为$f(x)$的\textbf{极大点}。
  如果
  \begin{displaymath}
    f(x) \ge a,\quad x\in U
  \end{displaymath}
  则称$f$在点$a$取\textbf{极小值}$f(a)$,
  称$a$为$f(x)$的\textbf{极小点}。
  极大、极小值称为\textbf{极值},
  达到极值的点统称为\textbf{极值点}。
\end{definition}

\begin{lemma}[费马(Fermat)定理] \label{thrm:fermat}
  设函数$f$在点$a$的某邻域有定义,在点$a$可导。
  如果$f(x)$在点$a$取极大或极小,则$f'(a)=0$。
\end{lemma}
\begin{proof}
  不妨设$f(x)$在点$a$取极大,则对$a+\Delta x\in U(a,\delta)$,有
  \begin{displaymath}
    f(a+\Delta x) \le f(a)
  \end{displaymath}
  当$\Delta x < 0$时,$[f(a+\Delta x) - f(a)]/\Delta x \ge 0$,
  所以根据极限的保号性(\ref{thrm:limit-sign-preserve})
  \begin{displaymath}
    f'(a) = f'_{-}(a) =\lim_{\Delta x\to 0-}
      \frac{f(a+\Delta x) - f(a)}{\Delta x} \ge 0
  \end{displaymath}
  当$\Delta x > 0$时,$[f(a+\Delta x) - f(a)]/\Delta x \le 0$,同样有
  \begin{displaymath}
  f'(a) = f'_{+}(a) =\lim_{\Delta x\to 0+}
  \frac{f(a+\Delta x) - f(a)}{\Delta x} \le 0
  \end{displaymath}
  因此只能等号成立,即$f'(a)=0$。
\end{proof}
\begin{remark}
  费马引理给出了极值的必要条件,但不是充分条件。
\end{remark}

\begin{lemma}[罗尔(Rolle)定理] \label{thrm:rolle}
  设函数$f$满足:
  \begin{enumerate}
    \item
    在闭区间$[a,b]$上连续;
    \item
    在开区间$(a,b)$上可导;
    \item
    $f(a)=f(b)$。
  \end{enumerate}
  则在$(a,b)$内至少存在一个点$\xi$,使得$f'(\xi)=0$。
\end{lemma}
\begin{proof}
  因为$f$在闭区间$[a,b]$上连续,
  所以根据闭区间上连续函数的性质(\ref{thrm:continuous-func-bounded}),
  $f$在$[a,b]$上能取到最大值$M$和最小值$m$。下面有两种可能:
  \begin{enumerate}
    \item
    若$M = m$,则$f$在$[a,b]$上为常数,
    因此对$(a,b)$内的任一点$x$都有$f'(x)=0$,
    即$\xi$可取$(a,b)$内任一点。
    \item
    若$M\neq m$,则$M$和$m$中至少有一个不等于$f(a)$,
    不妨设$M\neq f(a)=f(b)$,
    于是至少有一个点$\xi\in(a,b)$,使得$f(\xi)=M$。
    根据费马定理(\ref{thrm:fermat})得到$f'(\xi)=0$。
  \end{enumerate}
\end{proof}

\subsection{中值定理}
\begin{theorem}[拉格朗日(Lagrange)中值定理] \label{thrm-lagrange-mean-value}
  设函数$f$满足:
  \begin{enumerate}
    \item
    在闭区间$[a,b]$上连续;
    \item
    在开区间$(a,b)$上可导。
  \end{enumerate}
  则至少存在一点$\xi\in(a,b)$,使得
  \begin{displaymath}
    f'(\xi)=\frac{f(b)-f(a)}{b-a}
  \end{displaymath}
\end{theorem}

\begin{proof}
  作辅助函数
  \begin{displaymath}
    F(x) = f(x)-f(a)-\frac{f(b)-f(a)}{b-a}(x-a)
  \end{displaymath}
  它是由$f(x)$与端点为$(a,f(a))$、$(b,f(b))$的直线相减得到的。
  这样,就有$F(a)=F(b)=0$,
  且不难验证它满足罗尔定理(\ref{thrm:rolle})的其它条件。
  因此,至少存在一点$\xi\in(a,b)$,使得$F'(\xi)=0$,即
  \begin{displaymath}
    f'(\xi)-\frac{f(b)-f(a)}{b-a} = 0
  \end{displaymath}
  此即为求证的目标式。
\end{proof}

\begin{remark}
  需要留意辅助函数的构造方法。此外,拉格朗日中值定理还能写成
  \begin{align*}
    f(x_0 + \Delta x)
    &= f(x_0) + f'(\xi)\Delta x &\xi\in(0,x_0+\Delta x)\\
    &= f(x_0) + f'(x_0+\theta \Delta x)\Delta x &\theta\in(0,1)
  \end{align*}
\end{remark}

\begin{theorem}[柯西(Cauchy)中值定理] \label{thrm:cauchy-mean-value}
  设函数$f,g$满足
  \begin{enumerate}
     \item
    在闭区间$[a,b]$上连续;
    \item
    在开区间$(a,b)$上可导,且$g'(x)\neq 0$。
  \end{enumerate}
  则至少存在一点$\xi\in(a,b)$,使得
  \begin{displaymath}
    \frac{f(b)-f(a)}{g(b)-g(a)}=\frac{f'(\xi)}{g'(\xi)}
  \end{displaymath}
\end{theorem}

\begin{proof}
  因为$g'(x)\neq 0$,所以$g(b)-g(a)=g'(\xi)(b-a)\neq 0$,
  所以我们能构造有意义的函数
  \begin{displaymath}
    F(x)=f(x)-\frac{f(b)-f(a)}{g(b)-g(a)}g(x)
  \end{displaymath}
  不难发现$F(a)=F(b)$,
  且$F(x)$满足罗尔定理(\ref{thrm:rolle})的其它条件,
  所以至少存在一点$\xi\in(a,b)$,使得$F'(x)=0$,即
  \begin{displaymath}
    f'(\xi)-\frac{f(b)-f(a)}{g(b)-g(a)}g'(\xi)=0
  \end{displaymath}
  此即为求证的目标式
\end{proof}

\section{洛必达法则}
本小节介绍能够求$\frac{0}{0}$和$\frac{\infty}{\infty}$型不定型极限的有效方法
——洛必达(L'Hospital)法则。

\subsection{求$\frac{0}{0}$型不定型极限的法则}
\begin{theorem}
  设在去心邻域$U(a,r)-\{a\}$内函数$f,g$满足
  \begin{enumerate}
    \item
    $\lim_{x\to a} f(x) = \lim_{x\to a}g(x) = 0$,
    \item
    $f',g'$存在且$g'(x) \neq 0$,
    \item \label{item:deri-lim-exist}
    $\lim_{x\to a} \frac{f'(x)}{g'(x)}=A$(有限或$\pm\infty$)。
  \end{enumerate}
  则有
  \begin{displaymath}
    \lim_{x\to a}\frac{f(x)}{g(x)}=\lim_{x\to a} \frac{f'(x)}{g'(x)}
  \end{displaymath}
\end{theorem}

\begin{remark}
  一些注意点:
  \begin{enumerate}
    \item
    对单侧极限,该定理仍成立。
    \item
    对于$x\to\infty$,该定理也成立。
    \item
    定理的条件是充分的,不是必要的。
    即当条件\ref{item:deri-lim-exist}不满足时,
    不能断定极限$\lim \frac{f(x)}{g(x)}$不存在。
  \end{enumerate}
\end{remark}

\subsection{求$\frac{\infty}{\infty}$型不定型极限的法则}
\begin{theorem}
  设在去心邻域$U(a,r)-\{a\}$内函数$f,g$满足
  \begin{enumerate}
    \item
    $\lim_{x\to a} f(x) = \lim_{x\to a}g(x) = \infty$,
    \item
    $f',g'$存在且$g'(x) \neq 0$,
    \item
    $\lim_{x\to a} \frac{f'(x)}{g'(x)}=A$(有限或$\pm\infty$)。
  \end{enumerate}
  则有
  \begin{displaymath}
    \lim_{x\to a}\frac{f(x)}{g(x)}=\lim_{x\to a} \frac{f'(x)}{g'(x)}
  \end{displaymath}
\end{theorem}

\subsection{其它五种不定型极限}
其它五种不定型都可化为$\frac{0}{0}$或$\frac{\infty}{\infty}$型:
\begin{itemize}
  \item
  $0\cdot\infty$可以化成$\frac{0}{1/\infty}$。
  \item
  $\infty - \infty$可以进行通分。
  \item
  $0^0$,$1^{\infty}$,$\infty^0$可以取对数。
\end{itemize}

\section{泰勒公式}
本小节介绍泰勒公式,并给出常用的基本初等函数的麦克劳林公式。

\subsection{泰勒公式}
\begin{theorem}[泰勒(Taylor)公式]
  设函数$f$在点$a$的邻域$U$内存在直到$n+1$阶导数,
  则$f$在$U$上可展成$n$阶泰勒公式
  \begin{equation} \label{eq:talor}
     f(x) = \sum_{k=0}^{n}\frac{f^{(k)}(a)}{k!}(x-a)^k + r_n(x)
  \end{equation}
  其中$r_n(x)$可表达为\textbf{拉格朗日余项}形式
  \begin{displaymath}
    r_n(x)=\frac{f^{(n+1)}(\xi)}{(n+1)!}(x-a)^{n+1},
    \quad \xi\in(a,x)
  \end{displaymath}
  或
  \begin{displaymath}
    r_n(x)=\frac{f^{(n+1)}(a+\theta(x-a))}{(n+1)!}(x-a)^{n+1},
    \quad \theta\in(0,1)
  \end{displaymath}
  这时,称式\eqref{eq:talor}为\textbf{带拉格朗日余项的$n$阶泰勒公式}。
  $r_n(x)$也可表达为\textbf{皮亚诺(Peano)余项}形式
  \begin{displaymath}
    r_n(x)=o\left((x-a)^n\right),\quad((x-a)\to 0)
  \end{displaymath}
  这时,称式\eqref{eq:talor}为\textbf{带皮亚诺余项的$n$阶泰勒公式}。
  
  特别的,若$a=0$,分别称
  \begin{displaymath}
    f(x) = \sum_{k=0}^{n}\frac{f^{(k)}(0)}{k!}x^k
      + \frac{f^{(n+1)}(\theta x)}{(n+1)!}x^{n+1},
      \quad \theta\in (0,1)
  \end{displaymath}
  与
  \begin{displaymath}
    f(x) = \sum_{k=0}^{n}\frac{f^{(k)}(0)}{k!}x^k
      + o(x^n),\quad (x\to 0)
  \end{displaymath}
  为带拉格朗日余项与皮亚诺余项的\textbf{$n$阶麦克劳林(Maclaurin)公式}。
\end{theorem}

\subsection{基本初等函数的麦克劳林公式}
下面是带拉格朗日余项的初等函数的麦克劳林公式。其中,$\theta\in (0,1)$。
\begin{align*}
  e^x &= 1+x+\frac{x^2}{2!}+\dots+\frac{x^n}{n!}
    + \frac{e^{\theta x}}{(n+1)!}x^{n+1} \\
  \sin x &= x - \frac{x^3}{3!} + \frac{x^5}{5!} -\dots
    + \frac{(-1)^{m-1}}{(2m-1)!}x^{2m-1}\\
    &\quad + \frac{\sin(\theta x + (2m+1)\pi / 2)}{(2m+1)!}x^{2m+1}\\
  \cos x &= 1 - \frac{x^2}{2!} + \frac{x^4}{4!} -\dots
    + \frac{(-1)^{m}}{(2m)!}x^{2m}\\
    &\quad + \frac{\cos(\theta x + (m+1)\pi)}{(2m+2)!}x^{2m+2} \\
  \ln(1+x) &= x -\frac{x^2}{2} + \frac{x^3}{3}-\dots
    + \frac{(-1)^{n-1}}{n}x^n \\
    &\quad + \frac{(-1)^{n}}{(n+1)(1+\theta x)^{n+1}}x^{n+1} \\
  (1+x)^{\mu} &= 1 + \mu x + \frac{\mu(\mu-1)}{2}x^2+\dots
    + \frac{\mu(\mu-1)\dots(\mu-n+1)}{n!}x^n \\
    &\quad + \frac{\mu(\mu-1)\dots(\mu-n)}{(n+1)!}
    (1+\theta x)^{\mu-n-1}x^{n+1}
\end{align*}

\section{利用导数研究函数的性质}

\subsection{单调性}
高中内容。略。

\section{利用导数作函数的图形}
