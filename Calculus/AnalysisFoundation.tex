\chapter{分析基础}

\section{基础概念}
\subsection{邻域}
对于实数$a\in\mfR$,定义它的邻域为
$N(a,\delta) = \{ x: |x-a| < \delta \}$,
其中,实数$\delta > 0$。

同样,可以定义$a$的左邻域为
$N(a,\delta)_+ = \{ x: 0 \le x - a < \delta \}$,
右邻域为
$N(a,\delta)_- = \{ x: -\delta < x - a \le 0  \}$。

\subsection{常用等式与不等式}
\begin{enumerate}
  \item
  $1^2+2^2+\dots +n^2 = n(n+1)(2n+1)/6$
  \item
  $(\sum_{i}a_i b_i)^2 \le (\sum_{i}a_i^2)(\sum_{i}b_i)^2$
  \item
  $\sin x < x < \tan x \quad (0 < x < \pi/2)$
  \item
  若$x_1,\dots,x_n$符号相同且都大于$-1$,那么
  \[(1+x_1)(1+x_2)\dots(1+x_n)\ge 1+x_1+x_2+\dots+x_n\]
\end{enumerate}

\section{极限的概念}
\subsection{数列的极限}
数列的极限主要讨论自变量$n$无限增大时(记为$n\to+\infty$或$n\to\infty$),
因变量$a_n$的变化趋向。
当$n\to\infty$时,如果$a_n$无限接近于某一定值,
则称$\{a_n\}$为\textbf{收敛数列},
否则称$\{a_n\}$为\textbf{发散数列}。

对于收敛数列,我们用下面的$\epsilon$-$N$语言来定义它的极限。
\begin{definition}[数列极限的$\epsilon$-$N$定义]
  \begin{displaymath}
    \lim_{n\to\infty}a_n=A
    \iff \forall\epsilon>0,\exists N,\forall n>N,|a_n - A|<\epsilon
  \end{displaymath}
\end{definition}
\begin{remark}
  当要描述数列不收敛到某一个值时,只要把上面的定义取否命题即可。
\end{remark}

对于发散数列,它有四种情况:发散到正无穷,发散到负无穷,绝对值发散到正无穷,以及振荡。
我们用$G$-$N$语言来描述前三种情况。
\begin{definition}[数列极限的$G$-$N$的定义]
  \begin{align*}
    &\lim_{n\to\infty}a_n=+\infty
    \iff \forall G>0,\exists N,\forall n>N, a_n > G \\
    &\lim_{n\to\infty}a_n=-\infty
    \iff \forall G>0,\exists N,\forall n>N, a_n < -G \\
    &\lim_{n\to\infty}a_n=\infty
    \iff \forall G>0,\exists N,\forall n>N, |a_n| > G
  \end{align*}
\end{definition}

\subsection{函数的极限}
函数的极限研究函数在自变量的某变化过程中相应的因变量的变化趋向。
与数列极限不同的是,数列极限的自变量只有一种变化趋向,即$n\to\infty$,
而函数极限的自变量有六种变化趋向。

对于自变量$x$变化的趋向,我们定义下面两类共6种模板:
\begin{itemize}
  \item $\delta$模板:
  \begin{align*}
    x\to a
    &\iff \dots, \exists \delta > 0,\forall x, 0<|x-a|<\delta, \dots \\
    x\to a+0
    &\iff \dots, \exists \delta > 0,\forall x, 0<x-a<\delta, \dots \\
    x\to a-0
    &\iff \dots, \exists \delta > 0,\forall x, -\delta<x-a<0, \dots
  \end{align*}
    \item $H$模板:
  \begin{align*}
    x\to+\infty
    &\iff \dots, \exists H > 0, \forall x > H, \dots \\
    x\to-\infty
    &\iff \dots, \exists H > 0, \forall x < -H, \dots \\
    x\to\infty
    &\iff \dots, \exists H > 0, \forall |x| > H, \dots
  \end{align*}
\end{itemize}

对于因变量$f(x)$,我们定义下面两类共4种模板:
\begin{itemize}
  \item $\epsilon$模板
  \begin{displaymath}
    f(x)\to A
    \iff \forall\epsilon>0, \dots, |f(x)-A|<\epsilon
  \end{displaymath}
  \item $G$模板
  \begin{align*}
    f(x)\to+\infty
    &\iff \forall G>0,\dots, f(x) > G \\
    f(x)\to-\infty
    &\iff \forall G>0,\dots, f(x) < -G \\
    f(x)\to\infty
    &\iff \forall G>0,\dots, |f(x)| > G
  \end{align*}
\end{itemize}

以上模板可以组合成函数极限的
``$\epsilon$-$\delta$''、``$\epsilon$-$H$''、``$G$-$\delta$''、``$G$-$H$''定义。
以下面将要说明的左极限为例,我们可以组合出左极限的$\epsilon$-$\delta$定义:
\begin{displaymath}
  \lim_{x\to a-0}f(x)=A
  \iff \forall\epsilon>0,\exists\delta>0,\forall x, -\delta<x-a<0, |f(x)-A|<\epsilon
\end{displaymath}

\subsection{左极限\ 右极限}
函数$f$在$x\to a-0$和$x\to a+0$存在有限极限$A$的情况特别重要,
分别称为函数$f$在点$a$的\textbf{左}、\textbf{右极限},
统称为\textbf{单侧极限},记为
\begin{displaymath}
  \lim_{x\to a-0}f(x)=A\quad\text{或}\quad f(a-0)=A\quad\text{或}\quad f(a-)=A
\end{displaymath}
\begin{displaymath}
\lim_{x\to a+0}f(x)=A\quad\text{或}\quad f(a+0)=A\quad\text{或}\quad f(a+)=A
\end{displaymath}

\begin{theorem}[函数极限与单侧极限的关系]
  \begin{displaymath}
    \lim_{x\to a}f(x)=A \iff f(a-0)=f(a+0)=A
  \end{displaymath}
\end{theorem}

\begin{corollary}
  若函数$f$在点$a$存在左、右极限但不相等,
  或$f$在点$a$的左、右极限至少有一个不存在,
  则$f$在点$a$不存在极限。
\end{corollary}
\begin{remark}
  该推论可以用来证明函数极限不存在。
\end{remark}

\subsection{数列极限与函数极限的关系}
\begin{theorem}[数列极限与函数极限的关系]
  \begin{displaymath}
    \lim_{x\to a} f(x) = A
    \iff \forall \{x_n\}: x_n\neq a, x_n\to a(n\to\infty),
    \text{有}\lim_{n\to\infty} f(x_n) = A
  \end{displaymath}
\end{theorem}
\begin{remark}
  数列极限可以看成函数极限的特例,
  函数极限又可以转化为数列极限来讨论。
\end{remark}

\begin{corollary}
  设$\{x_n\}$和$\{y_n\}$是两个包含在$a$的去心邻域中且极限收敛到$a$的数列。
  若$\lim_{n\to\infty}f(x_n)$和$\lim_{n\to\infty}f(y_n)$极限存在但不相等,
  或者其中至少有一个不存在,那么$\lim_{x\to a}f(x)$不存在。
\end{corollary}
\begin{remark}
  该推论为证明函数极限不存在提供了简单的方法。
\end{remark}

\section{极限的性质与运算法则}
掌握极限的性质及四则运算法则.

\section{极限的存在准则}
掌握极限存在的两个准则,并会利用它们求极限,掌握利用两个重要极限求极限的方法.

\section{连续}
理解函数连续性的概念(含左连续与右连续),会判别函数间断点的类型.

了解连续函数的性质和初等函数的连续性,理解闭区间上连续函数的性质(有界性、最大值和最小值定理、介值定理),并会应用这些性质

\section{无穷小量}
理解无穷小量、无穷大量的概念,掌握无穷小量的比较方法,会用等价无穷小量求极限。
