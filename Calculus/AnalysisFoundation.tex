\chapter{分析基础}

\section{基础概念}
这一节主要介绍了邻域的概念和常用的不等式。
关于函数的基础知识,这里就不再赘述了。

\subsection{邻域}
对于实数$a\in\mfR$,定义它的邻域为
$N(a,\delta) = \{ x: |x-a| < \delta \}$,
其中,实数$\delta > 0$。

同样,可以定义$a$的左邻域为
$N(a,\delta)_+ = \{ x: 0 \le x - a < \delta \}$,
右邻域为
$N(a,\delta)_- = \{ x: -\delta < x - a \le 0  \}$。

\subsection{常用等式与不等式}
\begin{enumerate}
  \item
  $1^2+2^2+\dots +n^2 = n(n+1)(2n+1)/6$
  \item
  $(\sum_{i}a_i b_i)^2 \le (\sum_{i}a_i^2)(\sum_{i}b_i)^2$
  \item
  $\sin x < x < \tan x \quad (0 < x < \pi/2)$
  \item
  若$x_1,\dots,x_n$符号相同且都大于$-1$,那么
  \[(1+x_1)(1+x_2)\dots(1+x_n)\ge 1+x_1+x_2+\dots+x_n\]
\end{enumerate}

\section{极限的概念}
这一节首先介绍数列极限和函数极限的定义。微积分的大厦自此开始建立。
然后我们介绍比较重要的函数左极限与右极限的概念,
这是因为函数极限存在与左极限、右极限之间有一定关系。
最后,我们介绍函数极限和数列极限的关系,
从而获得另一个判定函数极限存在的方法。

\subsection{数列的极限}
数列的极限主要讨论自变量$n$无限增大时(记为$n\to+\infty$或$n\to\infty$),
因变量$a_n$的变化趋向。
当$n\to\infty$时,如果$a_n$无限接近于某一定值,
则称$\{a_n\}$为\textbf{收敛数列},
否则称$\{a_n\}$为\textbf{发散数列}。

对于收敛数列,我们用下面的$\epsilon$-$N$语言来定义它的极限。
\begin{definition}[数列极限的$\epsilon$-$N$定义]
  \begin{displaymath}
    \lim_{n\to\infty}a_n=A
    \iff \forall\epsilon>0,\exists N,\forall n>N,|a_n - A|<\epsilon
  \end{displaymath}
\end{definition}
\begin{remark}
  当要描述数列不收敛到某一个值时,只要把上面的定义取否命题即可。
\end{remark}

对于发散数列,它有四种情况:发散到正无穷,发散到负无穷,绝对值发散到正无穷,以及振荡。
我们用$G$-$N$语言来描述前三种情况。
\begin{definition}[数列极限的$G$-$N$的定义]
  \begin{align*}
    &\lim_{n\to\infty}a_n=+\infty
    \iff \forall G>0,\exists N,\forall n>N, a_n > G \\
    &\lim_{n\to\infty}a_n=-\infty
    \iff \forall G>0,\exists N,\forall n>N, a_n < -G \\
    &\lim_{n\to\infty}a_n=\infty
    \iff \forall G>0,\exists N,\forall n>N, |a_n| > G
  \end{align*}
\end{definition}

\subsection{函数的极限}
函数的极限研究函数在自变量的某变化过程中相应的因变量的变化趋向。
与数列极限不同的是,数列极限的自变量只有一种变化趋向,即$n\to\infty$,
而函数极限的自变量有六种变化趋向。

对于自变量$x$变化的趋向,我们定义下面两类共6种模板:
\begin{itemize}
  \item $\delta$模板:
  \begin{align*}
    x\to a
    &\iff \dots, \exists \delta > 0,\forall x, 0<|x-a|<\delta, \dots \\
    x\to a+0
    &\iff \dots, \exists \delta > 0,\forall x, 0<x-a<\delta, \dots \\
    x\to a-0
    &\iff \dots, \exists \delta > 0,\forall x, -\delta<x-a<0, \dots
  \end{align*}
    \item $H$模板:
  \begin{align*}
    x\to+\infty
    &\iff \dots, \exists H > 0, \forall x > H, \dots \\
    x\to-\infty
    &\iff \dots, \exists H > 0, \forall x < -H, \dots \\
    x\to\infty
    &\iff \dots, \exists H > 0, \forall |x| > H, \dots
  \end{align*}
\end{itemize}

对于因变量$f(x)$,我们定义下面两类共4种模板:
\begin{itemize}
  \item $\epsilon$模板
  \begin{displaymath}
    f(x)\to A
    \iff \forall\epsilon>0, \dots, |f(x)-A|<\epsilon
  \end{displaymath}
  \item $G$模板
  \begin{align*}
    f(x)\to+\infty
    &\iff \forall G>0,\dots, f(x) > G \\
    f(x)\to-\infty
    &\iff \forall G>0,\dots, f(x) < -G \\
    f(x)\to\infty
    &\iff \forall G>0,\dots, |f(x)| > G
  \end{align*}
\end{itemize}

以上模板可以组合成函数极限的
``$\epsilon$-$\delta$''、``$\epsilon$-$H$''、``$G$-$\delta$''、``$G$-$H$''定义。
以下面将要说明的左极限为例,我们可以组合出左极限的$\epsilon$-$\delta$定义:
\begin{displaymath}
  \lim_{x\to a-0}f(x)=A
  \iff \forall\epsilon>0,\exists\delta>0,\forall x, -\delta<x-a<0, |f(x)-A|<\epsilon
\end{displaymath}

\subsection{左极限\ 右极限}
函数$f$在$x\to a-0$和$x\to a+0$存在有限极限$A$的情况特别重要,
分别称为函数$f$在点$a$的\textbf{左}、\textbf{右极限},
统称为\textbf{单侧极限},记为
\begin{displaymath}
  \lim_{x\to a-0}f(x)=A\quad\text{或}\quad f(a-0)=A\quad\text{或}\quad f(a-)=A
\end{displaymath}
\begin{displaymath}
\lim_{x\to a+0}f(x)=A\quad\text{或}\quad f(a+0)=A\quad\text{或}\quad f(a+)=A
\end{displaymath}

\begin{theorem}[函数极限与单侧极限的关系]
  \begin{displaymath}
    \lim_{x\to a}f(x)=A \iff f(a-0)=f(a+0)=A
  \end{displaymath}
\end{theorem}

\begin{corollary}
  若函数$f$在点$a$存在左、右极限但不相等,
  或$f$在点$a$的左、右极限至少有一个不存在,
  则$f$在点$a$不存在极限。
\end{corollary}
\begin{remark}
  该推论可以用来证明函数极限不存在。
\end{remark}

\subsection{数列极限与函数极限的关系}
\begin{theorem}[数列极限与函数极限的关系]
  \begin{displaymath}
    \lim_{x\to a} f(x) = A
    \iff \forall \{x_n\}: x_n\neq a, x_n\to a(n\to\infty),
    \text{有}\lim_{n\to\infty} f(x_n) = A
  \end{displaymath}
\end{theorem}
\begin{remark}
  数列极限可以看成函数极限的特例,
  函数极限又可以转化为数列极限来讨论。
\end{remark}

\begin{corollary}
  设$\{x_n\}$和$\{y_n\}$是两个包含在$a$的去心邻域中且极限收敛到$a$的数列。
  若$\lim_{n\to\infty}f(x_n)$和$\lim_{n\to\infty}f(y_n)$极限存在但不相等,
  或者其中至少有一个不存在,那么$\lim_{x\to a}f(x)$不存在。
\end{corollary}
\begin{remark}
  该推论为证明函数极限不存在提供了简单的方法。
\end{remark}

\section{极限的性质与运算法则} \label{sec:limit-property}
当函数或数列存在有限极限时,极限具有唯一性、有界性、保序性和夹逼性。
同时,极限还可以与四则运算交换顺序。
必须指出的是,上述性质及运算法则对函数极限在其自变量$x$的六种极限过程及
数列极限在$n\to\infty$的极限过程都是同样成立的。
为了节省篇幅,我们仅就极限过程$x\to a$和$n\to\infty$叙述。

\subsection{极限的性质}
以下为统一写法,用$X,Y$代表因变量$x_n$或$f(x)$,而$\lim$代表某同一过程的极限。

\begin{theorem}[唯一性]
  设$\lim X=A,\lim Y=B$,$A,B$为有限数,则$A=B$。
\end{theorem}

\begin{theorem}[收敛函数的局部有界性]
  若$\lim_{x\to a}f(x)=A$(有限数),则函数$f$在$a$点附近有界。
\end{theorem}

\begin{theorem}[收敛数列的整体有界性]
  若$\lim_{n\to \infty}x_n=A$(有限数),则$\{x_n\}$是有界数列。
\end{theorem}

\begin{theorem}[函数极限的保序性]
  设$\lim_{x\to a} f(x)=A,\lim_{x\to a} g(x)=B$,$A,B$为有限数,则
  \begin{enumerate}
    \item
    若$A<B$,则存在$a$的某个去心邻域$N(a,\delta)-\{a\}$,
    在该邻域内,有$f(x) < g(x)$。
    \item
    若在$a$的某个去心邻域内有$f(x) \le g(x)$,
    则$A\le B$。
  \end{enumerate}
\end{theorem}

\begin{corollary}[函数极限的保号性质] \label{thrm:limit-sign-preserve}
  设$\lim_{x\to a} f(x)=A$(有限数),则
  \begin{enumerate}
    \item
    若$A>C$($A<C$),则存在$a$的某个去心邻域,
    在该邻域内,有$f(x)>C$($f(x)<C$)。
    \item
    若在$a$的某个去心邻域内有$f(x)\le C$($f(x)\ge C$),
    则$A\le C$($A\ge C$)。
  \end{enumerate}
\end{corollary}

\begin{remark}
  数列极限的保序性和保号性质的推论与函数极限类似。这里不再赘述了。
\end{remark}

根据保号性,就能得到夹逼定理:
\begin{theorem}[夹逼定理] \label{thrm:squeeze}
  设$X\le Z\le Y$。若$\lim X = \lim Y = A$,则$\lim Z = A$。
\end{theorem}

\subsection{运算法则}
设$\lim X=A,\lim Y=B$,$A,B$为有限数,则
\begin{itemize}
  \item
  $\lim (X\pm Y) = A \pm B$
  \item
  $\lim (cX) = cA$,$c$是常数。
  \item
  $\lim (XY) = AB$
  \item
  $\lim (X/Y) = A/B$,$B\neq 0$
\end{itemize}

\subsection{复合函数的极限}
\begin{theorem}[复合函数的极限]
  若函数$y=g(u)$与$u=\phi(x)$在点$a$的去心邻域上能构成复合函数,且满足
  \begin{enumerate}
    \item $\lim_{x\to a}\phi(x)=b$,且$x\neq a$时,$\phi(x)\neq b$,
    \item $\lim_{u\to b}g(u)=A$(有限),
  \end{enumerate}
  则$\lim_{x\to a}g(\phi(x))=A$。
\end{theorem}

\section{极限的存在准则}
本小节的主要内容是极限存在的两个准则和两个重要极限。
此外,我们还简单介绍了一下刻画实数系的连续性的方法。

\subsection{实数系的连续性}
实数系的连续性是使求极限这一分析运算能封闭的必备条件。
下面列举了刻画实数系的连续性的7个等价定理:
\begin{enumerate}
  \item
  确界存在定理:非空有界数集必有确界。
  \item
  单调有界定理:单调有界的数列必存在有限极限。
  \item
  区间套定理:闭区间套可以套出一个点。
  \item
  有限覆盖定理:闭区间上的任意开覆盖必有有限子覆盖。
  \item
  聚点定理:有界无限点列必有聚点。
  \item
  致密性定理:有界数列必有收敛子列。
  \item
  柯西收敛准则:若对任意$\epsilon > 0$,存在$N$,使得
  对任意$m,n\ge N$,有$|a_m-a_n|\le \epsilon$,
  则$\{a_n\}$有极限。
\end{enumerate}
以上内容只是做个小拓展。我也理解得不够深刻。

\subsection{极限的存在准则}
极限存在的两个准则分别是:夹逼定理(见定理\ref{thrm:squeeze})和单调有界定理。

\begin{theorem}[数列的单调有界定理]
  单调有界的数列必存在有限极限。
\end{theorem}

\begin{theorem}[函数的单调有界定理]
  设$f$是定义在区间$I$上的单调有界函数,
  则$f$在$I$的任意一点上都存在有限的单侧极限。
\end{theorem}

\subsection{两个重要极限}
\begin{displaymath}
  \lim_{x\to 0}\frac{\sin x}{x}=1
\end{displaymath}
\begin{displaymath}
  \lim_{x\to\infty}\left(1+x\right)^{\frac{1}{x}} = e
  \quad\text{或}\quad
  \lim_{x\to 0}\left(1+\frac{1}{x}\right)^{x} = e
\end{displaymath}

\section{连续}
本小节首先介绍函数连续性的概念(含左连续与右连续)与间断点的类型。
然后讨论连续函数的性质和初等函数的连续性。
最后,我们给出闭区间上连续函数的性质,
包括有界性定理、最值定理、介值定理和零点定理。

\subsection{连续与间断}
\begin{definition}[函数在点$a$连续与间断]
  设函数在点$a$的某个邻域上有定义。
  如果$\lim_{x\to a}f(x)=f(a)$,
  则称$f$在点$a$\textbf{连续},
  否则称$f$在点$a$\textbf{间断}。
  若$f(a-0)=f(a)$,则称$f$在点$a$\textbf{左连续}。
  若$f(a+0)=f(a)$,则称$f$在点$a$\textbf{右连续}。
\end{definition}
\begin{remark}
  显然,$f$在点$a$连续的充要条件有
  \begin{displaymath}
    f(a+0)=f(a-0)=f(a)
  \end{displaymath}
\end{remark}

函数$f$在点$a$连续有以下等价的说法:
\begin{align*}
  \text{$f$在点$a$连续}
  &\iff \text{$f$在$a$的某邻域有定义且$\lim_{x\to a}f(x)=a$} \\
  &\iff \lim_{\Delta x\to 0}f(a+\Delta x)=f(a) \\
  &\iff \lim_{\Delta x\to 0}\Delta f = 0
\end{align*}

如果函数$f$在点$a$间断,那么可以把间断点分成以下两类:
\begin{center}
  \begin{tabular}{|c|l|l|}
    \hline
    间断点类型 & \multicolumn{1}{c|}{$f(a+0)$与$f(a-0)$}
      & \multicolumn{1}{c|}{说明} \\
    \hline
    \multirow{2}{*}{第I类间断点}
      & \begin{tabular}[c]{@{}l@{}}存在且相等,\\ 极限值为$A$\end{tabular}
      & \begin{tabular}[c]{@{}l@{}}也称\textbf{可去间断点},因$f$在点$a$\\
      无定义或$f(a)\neq A$引起间断\end{tabular} \\
    \cline{2-3}
    & 存在但不相等 & 也称\textbf{跳跃型间断点} \\
    \hline
    \multirow{2}{*}{第II间断点} & 至少有一个是无穷大 & 也称\textbf{无穷型间断点} \\
    \cline{2-3}
    & 至少有一个不存在 &  \\
    \hline
  \end{tabular}
\end{center}

\subsection{连续函数的局部性质与运算法则}
若函数$f$在点$a$连续,根据\ref{sec:limit-property}节描述的
存在有限极限的函数的局部性质和运算法则,
能得到以下定理:

\begin{theorem}[局部有界性]
  若函数$f$在点$a$连续,则$f$在点$a$的某个邻域里有界。
\end{theorem}

\begin{theorem}[局部保号性]
  若函数$f$在点$a$连续,且$f(a)\neq 0$,
  则函数$f$在点$a$的某邻域内与$f(a)$同号。

  又若$f(a)>p(<q)$,则存在$a$的某个邻域$U$,在$U$内有
  $f(x)>p(<q)$。
\end{theorem}

\begin{theorem}[四则运算]
  若函数$f,g$都在点$a$连续,则它们的和、差、积、商
  \begin{displaymath}
    f\pm g,\ f\cdot g,\ \frac{f}{g}(g(a)\neq 0)
  \end{displaymath}
  也都在点$a$连续。
\end{theorem}

\begin{theorem}[复合函数的连续性]
   若函数$y=g(u)$与$u=\phi(x)$在点$a$的邻域上能构成复合函数,
   且函数$\phi$在点$a$连续,函数$g$在点$b=\phi(a)$连续,
   则复合函数$g\circ\phi$在点$a$连续。
\end{theorem}

\subsection{初等函数的连续性}
\begin{theorem}[初等函数的连续性]
  初等函数在其自然定义域内都是连续的。
\end{theorem}

\subsection{闭区间上连续函数的性质}
\begin{theorem}[有界性定理] \label{thrm:continuous-func-bounded}
  若$f$在闭区间$[a,b]$上连续,那么$f$在闭区间$[a,b]$上有界,即
  存在$m,M$,使得对任意$x\in[a,b]$,有$m\le f(x)\le M$。
\end{theorem}

\begin{theorem}[最值定理]
  若$f$在闭区间$[a,b]$上连续,那么$f$在闭区间$[a,b]$上有最大值和最小值,即
  存在$x_1,x_2\in[a,b]$,使得对任意$x\in[a,b]$,有
  $f(x_1)\le f(x)\le f(x_2)$。
\end{theorem}

\begin{theorem}[介值定理]
  若$f$在闭区间$[a,b]$上连续,
  记$m=\min_{x\in[a,b]}f(x)$,$M=\max_{x\in[a,b]}f(x)$,
  则$f$取到介于$m,M$之间的所有值,即对任意$m<\mu<M$,
  必然存在$\xi\in(a,b)$,使得$f(\xi)=\mu$。
\end{theorem}
\begin{remark}
  介值定理可以推广到在任意区间$I$上也成立。只要把$m$和$M$的重新记为
  $\inf_{x\in I}f(x)$和$\sup_{x\in I}f(x)$。
\end{remark}

\begin{theorem}[零点定理]
  若$f$在闭区间$[a,b]$上连续,且$f(a)\cdot f(b) < 0$,
  则必存在一点$\xi\in(a,b)$,使得$f(\xi)=0$。
\end{theorem}

\section{无穷小量}
我们首先给出无穷小量的定义及其性质。
然后介绍无穷小量的比较方法。
最后讨论不定型的极限。
以下介绍的内容都适用于无穷大量。

\subsection{无穷小量及其性质}
\begin{definition}[无穷小量]
  在某一极限过程中,以零为极限的变量$\alpha$(数列或函数)
  称为该极限过程中的\textbf{无穷小量},常称为无穷小,
  记为$\alpha=o(1)$。
\end{definition}
\begin{remark}
  无穷大量的定义类似。
\end{remark}

\begin{theorem}[无穷小量与极限]
  $\lim_{x\to a}f(x)=A$(有限)$\iff f(x)=A+\alpha(x)$,
  其中$\alpha(x)\to 0\ (x\to a)$是一个无穷小量。
\end{theorem}

\begin{theorem}[无穷小量的性质]
  设$\alpha,\beta$是某个过程中的两个无穷小量,那么
  \begin{enumerate}
    \item
    $\alpha\pm\beta$是无穷小量。
    \item
    $\alpha\cdot\beta$是无穷小量。
    \item
    若$\gamma$有界,则$\alpha\cdot\gamma$是无穷小量。
  \end{enumerate}
\end{theorem}

\subsection{无穷小量的比较}
\begin{definition}[无穷小量的比较]
  设$\alpha,\beta$是同一极限过程中的两个无穷小量。
  \begin{enumerate}
    \item
    若$\lim \frac{\alpha}{\beta}=0$,
    则称$\alpha$是关于$\beta$的\textbf{高阶无穷小量},
    记为$\alpha = o(\beta)$。

    \item
    若$\lim \frac{\alpha}{\beta}=A$(非零有限),
    则称$\alpha$与$\beta$是\textbf{同阶无穷小量},
    记为$\alpha = \Theta(\beta)$。

    \item
    若$\lim \frac{\alpha}{\beta}=1$,
    则称$\alpha$与$\beta$是\textbf{等价无穷小量},
    记为$\alpha \sim \beta$。

    以上三种情况,$\alpha$的阶都不低于$\beta$的阶,
    可以记为$\frac{\alpha}{\beta}=O(1)$或$\alpha=O(\beta)$。

    \item
    若$\lim \frac{\alpha}{\beta^s}=A$(非零有限),
    其中$\alpha$是大于0的常数,
    则称$\alpha$是关于$\beta$的\textbf{$s$阶无穷小量}。
  \end{enumerate}
\end{definition}

\begin{remark}
  关于无穷小量的比较,有下面一些注意点:
  \begin{enumerate}
    \item
    在使用$o,\Theta,O,\sim$记号的时候,必须指出自变量趋于什么,例如:
    \begin{displaymath}
      y_n=o(x_n), n\to\infty,\text{是指}\lim_{n\to\infty}\frac{y_n}{x_n}=0
    \end{displaymath}
    \item
    $\alpha=o(\beta)$,$\alpha=\Theta(\beta)$,$\alpha=O(\beta)$
    中的等号有着特殊的含义。
    实际上,$o(\beta)$,$\Theta(\beta)$,$O(\beta)$表示函数类,
    所以等号的含义是``属于''。
    \item
    $o,\Theta,O$记号在算法复杂度分析中也用到了呢。
  \end{enumerate}
\end{remark}

\begin{theorem}[替换定理]
  设$X_1,X_2,Y_1,Y_2$是同一极限过程中的无穷小量。
  若$X_1\sim X_2$,$Y_1\sim Y_2$,且$\lim Y_1/X_1$存在,则
  \begin{displaymath}
    \lim \frac{Y_2}{X_2}=\lim \frac{Y_1}{X_1}
  \end{displaymath}
\end{theorem}

\begin{theorem}[常用的等价无穷小]
  当$x\to 0$时,
  \begin{align*}
    & \sin x \sim x;\quad \tan x \sim x;\quad 1-\cos x \sim \frac{1}{2}x^2;  \\
    & \arcsin x \sim x;\quad \arctan x \sim x;\quad \ln(x+1)\sim x; \\
    & e^x-1\sim x;\quad (1+x)^\mu - 1 \sim \mu x
  \end{align*}
\end{theorem}

\subsection{不定型}
不定型主要包括以下7类:
\begin{displaymath}
  \frac{0}{0};\ \frac{\infty}{\infty};\ \infty-\infty;\ 0\cdot\infty;
  \ 1^\infty;\ 0^0;\ \infty^0.
\end{displaymath}