\chapter{随机变量的数字特征}

\section{数学期望}
本小节首先分别给出离散型和连续型随机变量的数学期望的定义,
然后讨论数学期望的性质,以及条件期望。

\subsection{离散型随机变量的数学期望}
\begin{definition}[离散型随机变量的数学期望]
  设$X$为离散型随机变量,分布律为$P(X=x_i)=p_i\ (i=1,2,\dots)$。
  若级数$\sum_{i=1}^{\infty}x_ip_i$绝对收敛,
  则称$\sum_{i=1}^{\infty}x_ip_i$为$X$的\textbf{数学期望},
  记为$\mexpect[X]$,即
  \begin{displaymath}
    \mexpect[X] = \sum_{i=1}^{\infty}x_ip_i
  \end{displaymath}
  若$\sum_{i=1}^{\infty}|x_i|p_i$发散,
  则称$X$的数学期望不存在。
\end{definition}

\begin{theorem}[离散非负随机变量的期望的其它计算方法]
  设$X$是取值为非负整数的离散随机变量,则
  \begin{displaymath}
    \mexpect[X] = \sum_{i=1}^{\infty} P(X\ge i)
  \end{displaymath}
\end{theorem}

\subsection{连续型随机变量的数学期望}
\begin{definition}[连续型随机变量的数学期望]
  设连续型随机变量$X$的概率密度为$p(x)$,
  若积分$\mintall xp(x)dx$绝对收敛,
  则称该积分为$X$的\textbf{数学期望},记为$\mexpect[X]$,即
  \begin{displaymath}
    \mexpect[X] = \mintall xp(x)dx
  \end{displaymath}
  若积分$\mintall |x|p(x)dx$发散,
  则称$X$的数学期望不存在。
\end{definition}

\subsection{数学期望的性质}
\begin{theorem}[数学期望的性质]
  数学期望有如下一些性质:
  \begin{enumerate}
    \item 线性性质:
    $\mexpect[X+Y] = \mexpect[X] + \mexpect[Y]$,
    $\mexpect[cX] = c\mexpect[X]$
    \item
    如果$f(x)$是凸函数,那么
    \begin{displaymath}
      \mexpect[f(X)] \ge f(\mexpect[X])
    \end{displaymath}
  \end{enumerate}
\end{theorem}

\begin{theorem}
  设连续型随机变量$X$的密度函数为$p(x)$。
  若$g(x)$连续,那么
  \begin{displaymath}
    \mexpect[g(X)] = \mintall g(x)p(x)dx
  \end{displaymath}
  类似地,设$X,Y$是连续型随机变量,联合密度为$p(x,y)$。
  若$g(x,y)$连续,那么
  \begin{displaymath}
    \mexpect[g(X,Y)] = \mintall\mintall g(x,y)p(x,y)dxdy
  \end{displaymath}
\end{theorem}

\subsection{条件期望}
\begin{definition}[离散型随机变量的条件期望]
  在事件$A$的条件下,离散型随机变量$X$的条件期望定义为
  \begin{displaymath}
    \mexpect[X|A] = \sum_{x} xP(X=x|A)
  \end{displaymath}
\end{definition}

\begin{theorem}[全期望公式]
  设事件$A_1,A_2,\dots,A_n$是概率空间的一个划分,
  则对于随机变量$X$,有
  \begin{displaymath}
    \mexpect[X] = \sum_{i=1}^n P(A_i)\mexpect[X|A_i]
  \end{displaymath}
\end{theorem}

\begin{theorem}[条件期望的线性性质]
  对于有限个随机变量$X_1,X_2,\dots,X_n$,
  以及常数$c_1,c_2,\dots,c_n$,有
  \begin{displaymath}
    \mexpect\left[\sum_{i=1}^{n}c_iX_i\bigg|A\right] =
      \sum_{i=1}^{n}c_i\mexpect[X|A]
  \end{displaymath}
\end{theorem}

\begin{theorem}[条件期望定义的随机变量]
  如果$X$和$Y$是随机变量,
  那么$\mexpect[X|Y]$是随机变量$Y$的函数。
  且拥有性质
  \begin{displaymath}
    \mexpect\left[\mexpect[X|Y]\right] = \mexpect[X]
  \end{displaymath}
\end{theorem}

\section{方差\ 协方差\ 相关系数}
本小节主要讨论:方差,协方差,相关系数与相关性。

\subsection{方差}
\begin{definition}[中位数]
  设$X$为随机变量。对于$m\in\mfR$,若
  $P(X\ge m) \ge 1/2$且$P(X\le m) \ge 1/2$。
  则称$m$为$X$的\textbf{中位数}。
\end{definition}

\begin{definition}[方差与标准差]
  设$X$是一个随机变量,
  若$\mexpect[(X-\mexpect[X])^2]$存在,
  则称$\mexpect[(X-\mexpect[X])^2]$为$X$的\textbf{方差},
  记为$\mvar(X)$,即
  \begin{displaymath}
    \mvar(X) = \mexpect\left[(X-\mexpect[X])^2\right]
  \end{displaymath}
  此外,称$\sqrt{\mvar(X)}$为\textbf{标准差},记为$\msdev(X)$。
\end{definition}

\begin{theorem}[方差的其它计算方法]
  设$X$是随机变量,则
  \begin{displaymath}
    \mvar(X) = \mexpect\left[X^2\right]-\mexpect[X]^2
  \end{displaymath}
\end{theorem}

\begin{theorem}[方差的性质]
  设$C$为常数,$X$为随机变量,则:
  \begin{enumerate}
    \item
    $\mvar(C)=0$
    \item
    $\mvar(CX)=C^2\mvar(X)$
  \end{enumerate}
\end{theorem}

\begin{remark}
  方差不具有线性性质。
\end{remark}

\subsection{协方差}
\begin{definition}[协方差]
  定义随机变量$X$和$Y$间的\textbf{协方差}为
  \begin{align*}
    \mcov(X,Y)
    &=\mexpect[(X-\mexpect[X])(Y-\mexpect[Y])] \\
    &= \mexpect[XY] - \mexpect[X]\mexpect[Y]
  \end{align*}
  特别地,$\mcov(X,X)=\mvar(X)$。
\end{definition}

\begin{theorem}[协方差的性质]
  协方差有如下一些性质:
  \begin{enumerate}
    \item
    $\mcov(X,C) = 0$,其中$C$为常数。
    \item
    $\mcov(X,Y) = \mcov(Y,X)$
    \item
    $\mcov(aX,bY)=ab\mcov(X,Y)$,其中$a,b$为常数。
    \item
    $\mcov(X_1+X_2,Y)=\mcov(X_1,Y)+\mcov(X_2,Y)$
    \item
    $\mvar(X\pm Y)=\mvar(X)+\mvar(Y)\pm 2\mcov(X,Y)$
  \end{enumerate}
\end{theorem}

\begin{theorem}[相互独立的随机变量的协方差]
  若随机变量$X,Y$独立,则
  \begin{displaymath}
    \mcov(X,Y) = 0
  \end{displaymath}
  亦即$\mexpect[XY]=\mexpect[X]\mexpect[Y]$,
  所以$\mvar(X\pm Y) = \mvar(X)\pm \mvar(Y)$。
\end{theorem}

\begin{remark}
  反之并不成立,即$\mcov(X,Y)=0$并不意味着$X,Y$相互独立。
\end{remark}

\begin{theorem}[随机变量和的方差]
  对于有限个随机变量$X_1,X_2,\dots,X_n$,
  \begin{align*}
    \mvar\left(\sum_{i=1}^{n}X_i\right)
    &= \sum_{i=1}^{n}\mvar(X) + 2\sum_{1\le i<j\le n}\mcov(X_i,X_j) \\
    &= \sum_{\substack{1\le i\le n \\ 1\le j\le n}}\mcov(X_i,X_j)
  \end{align*}
  特别地,若$X_1,X_2,\dots,X_n$两两独立,则
  \begin{displaymath}
    \mvar\left(\sum_{i=1}^{n}X_i\right) = \sum_{i=1}^{n}\mvar(X_i)
  \end{displaymath}
\end{theorem}

\subsection{相关系数} \label{subset-correlation-coefficent}
\begin{definition}[标准化随机变量]
  设随机变量$X$的期望$\mexpect[X]$和方差$\mvar(X)$,则
  \begin{displaymath}
    \widetilde{X} = X-\mexpect[X]
  \end{displaymath}
  被称为$X$的\textbf{中心化随机变量},
  \begin{displaymath}
    X^* = \frac{X-\mexpect[X]}{\sqrt{\mvar(X)}}
  \end{displaymath}
  被称为$X$的\textbf{标准化随机变量}。
\end{definition}

\begin{remark}
  随机变量标准化的目的是使期望为0,方差为1。
\end{remark}

\begin{definition}[相关系数]
  对随机变量$X,Y$,设$\mvar(X)>0,\mvar(Y)>0$均存在,则称
  \begin{displaymath}
    \rho_{XY}=\frac{\mcov(X,Y)}{\sqrt{\mvar(X)\mvar(Y)}}
  \end{displaymath}
  为$X$和$Y$的\textbf{相关系数}。
  在不引起混淆时,记$\rho_{XY}$为$\rho$。
\end{definition}

\begin{remark}
  记$X,Y$的标准化随机变量为$X^*,Y^*$,则有
  \begin{displaymath}
    \rho_{XY}=\mcov(X^*,Y^*)
  \end{displaymath}
\end{remark}

\begin{theorem}[柯西-施瓦兹(Cauchy-Schwartz)不等式]
  设$X,Y$为随机变量,则
  \begin{displaymath}
    \mexpect[XY]^2 \le \mexpect[X^2]\mexpect[Y^2]
  \end{displaymath}
\end{theorem}

\begin{theorem}[相关系数的性质]
  对随机变量$X,Y$,设$\rho_{XY}$为它们的相关系数,则
  $|\rho_{XY}|\le 1$,即$\mcov(X,Y)^2 \le \mvar(X)\mvar(Y)$。
\end{theorem}

\begin{remark}
  由柯西施瓦兹不等式直接得证。
\end{remark}

\begin{theorem}
  $|\rho_{XY}|=1$当且仅当存在常数$a,b$,使得
  \[ P(Y=aX+b) = 1 \]
  即$X$与$Y$有线性关系的概率为1。
\end{theorem}

\begin{theorem}[相关性]
  对随机变量$X,Y$,
  \begin{itemize}
    \item
    若$|\rho_{XY}|=1$,
    则称$X,Y$\textbf{线性相关}。
    \begin{itemize}
      \item
      若$\rho_{XY}=1$,则称$X,Y$\textbf{正相关}。
      \item
      若$\rho_{XY}=-1$,则称$X,Y$\textbf{负相关}。
    \end{itemize}
    \item
    $|\rho_{XY}|=0$表示$X$与$Y$不存在线性关系,称为\textbf{不相关}。
  \end{itemize}
\end{theorem}

\begin{remark}
  $\rho_{XY}$表示$X$与$X$存在线性关系的强弱程度。
  $|\rho_{XY}|$越大,则$X$与$Y$线性关系越强,反之越弱。
\end{remark}

\begin{theorem}[独立与不相关]
  设$X,Y$为随机变量,则
  \begin{center}
    \begin{tabular}{cccl}
      \multirow{4}{*}{$X,Y$独立} &
      \multirow{4}{*}{$\Longrightarrow$} &
      \multirow{4}{*}{$X,Y$不相关} &
      $\iff \rho_{XY}=0$ \\
      & & & $\iff \mcov(X,Y)=0$ \\
      & & & $\iff \mexpect[XY]=\mexpect[X]\mexpect[Y]$ \\
      & & & $\iff \mvar(X+Y)=\mvar(X)+\mvar(Y)$ \\
    \end{tabular}
  \end{center}
\end{theorem}

\begin{remark}
  $X,Y$不相关指的是$X,Y$不存在线性关系,不代表$X,Y$独立。
\end{remark}

\section{集中度(Concentration of measure)}
本小节主要介绍几个衡量尾分布的不等式。

\subsection{马尔可夫不等式}
\begin{theorem}[马尔科夫不等式]
  设随机变量$X$非负,则对任意$a>0$,有
  \begin{displaymath}
    P(X\ge a) \le \frac{\mexpect[X]}{a}
  \end{displaymath}
\end{theorem}

\begin{theorem}[马尔科夫不等式的推广]
  设$X$为随机变量,$f(X)$为取值非负的实函数,则对任意$a>0$,有
  \begin{displaymath}
    P(f(X)\ge a) \le \frac{\mexpect[f(X)]}{a}
  \end{displaymath}
\end{theorem}

\subsection{切比雪夫不等式}
\begin{theorem}[切比雪夫不等式]
  设$X$为随机变量,则对任意$a>0$,有
  \begin{displaymath}
    P\left(|X-\mexpect[X]|>a\right) \le \frac{\mvar(X)}{a^2}
  \end{displaymath}
\end{theorem}