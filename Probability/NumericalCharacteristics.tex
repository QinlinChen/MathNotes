\chapter{随机变量的数字特征}

\section{数学期望}

\subsection{离散型随机变量的数学期望}
\begin{definition}[离散型随机变量的数学期望]
  设$X$为离散型随机变量,分布律为$P(X=x_i)=p_i\ (i=1,2,\dots)$。
  若级数$\sum_{i=1}^{\infty}x_ip_i$绝对收敛,
  则称$\sum_{i=1}^{\infty}x_ip_i$为$X$的\textbf{数学期望},
  记为$\mexpect[X]$,即
  \begin{displaymath}
    \mexpect[X] = \sum_{i=1}^{\infty}x_ip_i
  \end{displaymath}
  若$\sum_{i=1}^{\infty}|x_i|p_i$发散,
  则称$X$的数学期望不存在。
\end{definition}

\begin{theorem}[性质]
  数学期望有如下一些性质:
  \begin{enumerate}
    \item 线性性质:
    $\mexpect[X+Y] = \mexpect[X] + \mexpect[Y]$,
    $\mexpect[cX] = c\mexpect[X]$
    \item 
    $\mexpect[X^2] \ge \mexpect[X]^2$  
  \end{enumerate}
\end{theorem}

\begin{theorem}[离散非负随机变量的期望的其它计算方法]
  设$X$是取值为非负整数的离散随机变量,则
  \begin{displaymath}
    \mexpect[X] = \sum_{i=1}^{\infty} P(X\ge i)
  \end{displaymath}
\end{theorem}

\subsection{连续型随机变量的期望}

\section{方差与标准差}

\section{协方差}

\section{相关系数}

\section{常见随机变量分布的数字特征}