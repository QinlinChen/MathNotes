\chapter{行列式}

\section{行列式}
本小节开始讨论行列式的概念。

首先我们给出了行列式的定义。
但是,如果我们根据行列式的定义来计算行列式,除了一些特殊矩阵,
用计算科学的说法,大部分$n$阶行列式的计算复杂度是$O(n!)$,
因此我们需要探索更高效的方法来计算行列式。

接下来,我们给出了行列式的一些基本性质。
有了这些性质,我们得到了新的行列式计算方法,
而且稍加分析,就能惊喜地看出这个方法的时间复杂度是$O(n^3)$。

\subsection{$n$阶行列式}
\begin{definition}[$n$级排列]
  $n$个自然数按任意固定的顺序构成的一个排列称为$n$\textbf{级排列}。
  所有$n$级排列构成的集合记作$\pi_n$。
\end{definition}

\begin{definition}[逆序数]
  在一个$n$级排列$(i_1,i_2,\dots,i_n)$中,
  如果$i_r>i_s$,但是$r<s$,即前面的数大于后面的数,
  就称这两个数构成\textbf{逆序对}。
  一个排列中逆序对的个数叫做这个排列的\textbf{逆序数},
  记作$\tau(i_1,i_2,\dots,i_n)$。
\end{definition}

\begin{definition}[奇、偶排列]
  逆序数为奇数的排列叫做\textbf{奇排列};
  逆序数为偶数的排列叫做\textbf{偶排列}。
\end{definition}

\begin{definition}[$n$阶行列式]
  设$\mmA=(a_{ij})_{n\times n}$是$n$阶方阵,它的$n$阶行列式为
  \begin{align*}
    |\mmA| &= \sum_{(j_1,j_2,\dots,j_n)\in\pi_n}
    (-1)^{\tau(j_1,j_2,\dots,j_n)}a_{1j_1}a_{2j_2}\dots a_{nj_n} \\
    &= \sum_{(i_1,i_2,\dots,i_n)\in\pi_n}
    (-1)^{\tau(i_1,i_2,\dots,i_n)}a_{i_1 1}a_{i_2 2}\dots a_{i_n n}
  \end{align*}
\end{definition}

特殊矩阵的行列式:
\begin{enumerate}
  \item $|\mdiag(d_1,d_2,\dots,d_n)| = d_1d_2\dots d_n$
  \item 三角阵的行列式为对角线之积
\end{enumerate}

\subsection{行列式的性质}
\begin{theorem}
  若$\mmA$为方阵,那么$|\mmA^\mT| = |\mmA|$
\end{theorem}

\begin{theorem}
  \begin{align*}
    &\quad\mdet{cccc}{
      a_{11} & a_{12} & \cdots & a_{1n} \\
      \vdots &        &        &        \\
      b_{i1} + c_{i1} & b_{i2} + c_{i2}  & \cdots & b_{in} + c_{in} \\
      \vdots &        &        &        \\
      a_{n1} & a_{n2} & \cdots & a_{nn} } \\
    &= \mdet{cccc}{
      a_{11} & a_{12} & \cdots & a_{1n} \\
      \vdots &        &        &        \\
      b_{i1} & b_{i2} & \cdots & b_{in} \\
      \vdots &        &        &        \\
      a_{n1} & a_{n2} & \cdots & a_{nn} }
    + \mdet{cccc}{
      a_{11} & a_{12} & \cdots & a_{1n} \\
      \vdots &        &        &        \\
      c_{i1} & c_{i2} & \cdots & c_{in} \\
      \vdots &        &        &        \\
      a_{n1} & a_{n2} & \cdots & a_{nn} }
    \end{align*}
\end{theorem}

\begin{theorem}
  设$\mmA$为$n$阶方阵。
  若$\mmA$有两行元素相等或对应成比例,
  那么$|\mmA| = 0$。
\end{theorem}

\begin{theorem}
  \begin{displaymath}
    \mdet{cccc}{
      a_{11} & a_{12} & \cdots & a_{1n} \\
      \vdots &        &        &        \\
      \lambda a_{i1} & \lambda a_{i2} & \cdots & \lambda a_{in} \\
      \vdots &        &        &        \\
      a_{n1} & a_{n2} & \cdots & a_{nn} }
    = \lambda \mdet{cccc}{
      a_{11} & a_{12} & \cdots & a_{1n} \\
      \vdots &        &        &        \\
      a_{i1} & a_{i2} & \cdots & a_{in} \\
      \vdots &        &        &        \\
      a_{n1} & a_{n2} & \cdots & a_{nn} }
  \end{displaymath}
\end{theorem}

\begin{theorem} \label{thrm:det-rowswi}
  交换方阵$\mmA$的两行,仅改变$\mmA$行列式的符号,
  即\[ |\mmRowSwi{i}{j}\mmA| = -|\mmA| \]
\end{theorem}

\begin{theorem} \label{thrm:det-rowadd}
  对方阵$\mmA$做$\mTfRowAdd{i}{+k}{j}$变换,不改变$\mmA$的行列式,
  即\[ |\mmRowAdd{i}{+k}{j}\mmA| = |\mmA| \]
\end{theorem}

\subsection{用行列式的性质求行列式}
使用行列式的性质把行列式转化成上三角矩阵的行列式。
后者的行列式即为对角线之积。

\section{行列式按行(列)展开}
本小节讨论另一种行列式的计算方法。

\subsection{行列式按一行(列)展开}
\begin{definition}[余子式]
  在$n$阶行列式$|\mmA|=|a_{ij}|_{n\times n}$中划去第$i$行第$j$列后
  所剩下的$(n-1)^2$个元素按照原来的相对位置排成的$n-1$阶子式$M_{ij}$
  叫做元素$a_{ij}$在$\mmA$中的\textbf{余子式}。
  而\[ A_{ij} = (-1)^{i+j}M_{ij} \]
  叫做元素$a_{ij}$在$\mmA$中的\textbf{代数余子式}。
  这里$1\le i, j \le n$。
\end{definition}

\begin{theorem}[按行(列)展开] \label{thrm:det-expansion}
  $n$阶行列式$|\mmA|=|a_{ij}|_{n\times n}$
  等于任一行(列)各元素与其代数余子式的乘积之和,即
  \begin{align*}
  |\mmA| &= a_{i1}A_{i1}+a_{i2}A_{i2}+\dots+a_{in}A_{in}\quad(i=1,2,\dots,n) \\
  |\mmA| &= a_{1j}A_{1j}+a_{2j}A_{2j}+\dots+a_{nj}A_{nj}\quad(j=1,2,\dots,n)
  \end{align*}
\end{theorem}

\begin{theorem} \label{thrm:det-expansion-zero}
  $n$阶行列式$|\mmA|=|a_{ij}|_{n\times n}$
  的任一行(列)元素与另一行(列)元素的代数余子式乘积之和等于零,即
  \begin{align*}
    a_{i1}A_{j1}+a_{i2}A_{j2}+\dots+a_{in}A_{jn} &= 0\quad(i\neq j) \\
    a_{1i}A_{1j}+a_{2i}A_{2j}+\dots+a_{ni}A_{nj} &= 0\quad(i\neq j)
  \end{align*}
\end{theorem}

\begin{remark}
  定理\ref{thrm:det-expansion}和\ref{thrm:det-expansion-zero}
  可以统一表示为(仅列出行的情况)
  \begin{displaymath}
  \sum_{k=1}^{n}a_{ik}A_{jk} = \begin{cases}
    |\mmA| & (i=j) \\
    0      & (i\neq j)
  \end{cases}
  \end{displaymath}
\end{remark}

\subsection{范德蒙(Vandermonde)行列式}
\begin{displaymath}
  V_n = \mdet{ccccc}{
    1         & 1         & 1         & \cdots & 1      \\
    x_1       & x_2       & x_3       & \cdots & x_n    \\
    x_1^2     & x_2^2     & x_3^2     & \cdots & x_n^2  \\
    \vdots    & \vdots    & \vdots    & \ddots & \vdots \\
    x_1^{n-1} & x_2^{n-1} & x_3^{n-1} & \cdots & x_n^{n-1}
  } = \prod_{1\le i<j\le n}(x_j - x_i)
\end{displaymath}

\section{用行列式求逆阵\ 克莱姆法则}
本小节通过引入伴随矩阵这个中间媒介,
来探索方阵可逆与行列式之间的关系,
从而得到方阵可逆的充要条件,以及通过行列式求逆阵方法。

有了上述方法,我们就能利用行列式来解线性方程组,
并总结出了克莱姆法则。但克莱姆法则也有其局限性。
后面会讲高斯消元来解一般线性方程组。

\subsection{用行列式求逆矩阵}
\begin{theorem}[行列式乘法规则] \label{thrm:det-mul}
  对任意$n$阶方阵$\mmA,\mmB$,有
  \[ |\mmA\mmB|=|\mmA||\mmB| \]
\end{theorem}

\begin{remark}
  证明思路是,首先证明引理:对任意$n$阶方阵$\mmA$和$n$阶初等阵$\mmP$,
  有$|\mmA\mmP|=|\mmP\mmA|=|\mmP||\mmA|$。
  然后利用定理\ref{thrm:inv-equiv-cond}对$\mmB$讨论:
  如果$\mmB$可逆,就能拆成一系列初等阵之积,从而利用引理得证;
  如果$\mmB$不可逆,只要证明等式两边都等于0。
\end{remark}

\begin{theorem}[行列式数乘规则] \label{thrm:det-num-mul}
  对任意$n$阶方阵$\mmA$和$\lambda\in\mfR$,有
  \[ |\lambda\mmA| = \lambda^n|\mmA| \]
\end{theorem}

\begin{definition}[伴随矩阵]
  $n$阶方阵$\mmA$的\textbf{伴随矩阵}为
  \begin{displaymath}
    \mmA^* = \mmat{cccc}{
      A_{11} & A_{12} & \cdots & A_{1n} \\
      A_{21} & A_{22} & \cdots & A_{2n} \\
      \vdots & \vdots & \ddots & \vdots \\
      A_{m1} & A_{m2} & \cdots & A_{mn} }^\mT
  \end{displaymath}
\end{definition}

\begin{theorem}[伴随矩阵的性质] \label{thrm:adjugate-mat-prop}
  设$\mmA$是$n$阶方阵,那么有
  \[ \mmA\mmA^*=\mmA^*\mmA=|\mmA|\mmI \]
\end{theorem}

\begin{remark}
  伴随矩阵就是为了这个性质而定义的。
\end{remark}

\begin{theorem}[伴随矩阵的行列式]
  设$\mmA$是$n$阶方阵,那么有
  \[ |\mmA^*| = |\mmA|^{n-1} \]
\end{theorem}

\begin{remark}
  该式对$|\mmA|$取任意值都成立。
  当$|\mmA|\neq 0$时,
  对伴随矩阵的性质\ref{thrm:adjugate-mat-prop}
  应用行列式乘法规则\ref{thrm:det-mul}
  以及数乘规则\ref{thrm:det-num-mul}即可证明该式。
  当$|\mmA|=0$时,
  则直接通过定义来证明。
\end{remark}

\begin{theorem}[方阵可逆的充要条件]
  设$\mmA$是$n$阶方阵,那么
  \[ \mmA\ \text{可逆} \iff |\mmA| \neq 0 \]
  并且对于可逆矩阵$\mmA$有
  \begin{enumerate}
    \item $\mmA^{-1} = \mmA^*/|\mmA|$
    \item $(\mmA^*)^{-1} = \mmA/|\mmA|$
  \end{enumerate}
\end{theorem}

\subsection{线性代数方程组与克莱姆(Cramer)法则}
$n$元方程组的矩阵形式为
\begin{equation} \label{eq:linear-eq:set}
  \mmA \mvx = \mvb
\end{equation}
其中$\mmA = (a_{ij})_{n\times n}$是\textbf{系数矩阵},
$\overline{\mmA}=(\mmA, \mvb)$是\textbf{增广矩阵},
$\mvx = (x_1,\dots,x_n)^\mT$与$\mvb = (b_1,\dots,b_n)^\mT$是$n$维列向量,
分别称为\textbf{未知向量}与\textbf{常数向量}。
若$\mvx$的一组取值$\hat{\mvx}$能满足\eqref{eq:linear-eq:set},
则称$\hat{\mvx}$为\eqref{eq:linear-eq:set}的一个\textbf{解向量}。

若常数向量不为$\mvZero$,则称\eqref{eq:linear-eq:set}为\textbf{非齐次线性方程组},
否则,叫做\textbf{齐次线性方程组}。

\begin{theorem}[克莱姆(Cramer)法则]
  若$n$元线性代数方程组的系数矩阵$\mmA$的行列式$|\mmA|\neq 0$,
  则方程组有唯一解$\mvx = (x_1,\dots,x_n)$,其中
  \begin{displaymath}
    x_j = \frac{D_j}{|\mmA|} \quad (j=1,\dots,n)
  \end{displaymath}
  $D_j$是将$|\mmA|$的第$j$列换成$\mvb$所得的行列式。
\end{theorem}

\begin{corollary}[齐次线性方程组有非零解的充要条件]
  对于$n$元齐次线性方程组$\mmA\mvx=\mvZero$,
  若$|\mmA|\neq 0$,则只有零解;
  若$|\mmA| = 0$,则有无穷多非零解。
\end{corollary}

\begin{remark}
  值得注意的是,克莱姆法则只适用于方程个数等于未知量个数的方程组,
  而且系数行列式不能为零。
  此外,克莱姆法则的计算复杂度也更高,为$O(n^4)$,
  不如后面要介绍的高斯消元法$O(n^3)$的复杂度。
\end{remark}