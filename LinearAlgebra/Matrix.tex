\chapter{矩阵}

\section{矩阵及其运算}
这一小节主要介绍了矩阵的基本概念和基本运算。

矩阵的概念对学过线性代数的人来说是稀松平常的了,
所以这里主要给出了重要概念和特殊矩阵的定义和符号,
方便后面的讨论。

矩阵的基本运算涉及了线性运算、乘积、转置。
大家应该对此都很熟悉,这里就简单给出了一些性质,并不作证明。
此外,我们还由转置运算给出了对称矩阵、反对称矩阵的概念。

\subsection{矩阵的概念}
\begin{definition}[矩阵]
  由$m\times n$个数排成$m$行$n$列的矩形数表
  \begin{equation} \label{eq:mat-def}
    \mmA = \mmat{cccc}{
      a_{11} & a_{12} & \cdots & a_{1n} \\
      a_{21} & a_{22} & \cdots & a_{2n} \\
      \vdots & \vdots & \ddots & \vdots \\
      a_{m1} & a_{m2} & \cdots & a_{mn} }
  \end{equation}
  称为一个$m\times n$矩阵。
  其中数$a_{ij}$称为矩阵$\mmA$的元素。
  $i$为行标,$j$为列标,
  $a_{ij}$是$\mmA$位于第$i$行第$j$列的元素,或简称$\mmA$的$(i,j)$元素。
  式\eqref{eq:mat-def}可简记为$\mmA=(a_{ij})_{m\times n}$或$(a_{ij})$。
\end{definition}

下面是一些特殊的矩阵:
\begin{itemize}
  \item
  $m\times 1$矩阵又称为\textbf{列矩阵}或$m$\textbf{维列向量}。
  \item
  $1\times n$矩阵又称为\textbf{行矩阵}或$n$\textbf{维行向量}。
  \item
  $n\times n$矩阵又称为$n$\textbf{阶方阵}。
  对于$n$阶方阵$\mmA=(a_{ij})$,
  其对角线上的元素$a_{11},\dots,a_{nn}$又称\textbf{主对角线元素}。
  \item
  如果一个$n$阶方阵除主对角线元素外,其余元素都为0,
  那么我们称这种矩阵为\textbf{对角矩阵},记为
  \[
    \mdiag(k_1, k_2\dots,k_n)= \mmat{cccc}{
      k_1 &     &        & \\
          & k_2 &        & \\
          &     & \ddots & \\
          &     &        & k_n }
  \]
  \item
  $\mmI = \mdiag(1,1,\dots,1)$被称为\textbf{单位矩阵}。
  \item
  \textbf{零矩阵}$\mmZero$是所有元素都为0的矩阵。
  \item
  设$\mmA=(a_{ij})$是$n$阶方阵。
  若当$i>j$时,$a_{ij}=0$,则称$\mmA$为\textbf{上三角阵}。
  若当$i<j$时,$a_{ij}=0$,则称$\mmA$为\textbf{下三角阵}。
  两者统称\textbf{三角阵}。
\end{itemize}

\begin{definition}[矩阵相等]
  如果$\mmA=(a_{ij})$与$\mmB=(b_{ij})$都是$m\times n$矩阵,
  且对$i=1,2,\dots,m,\ j=1,2,\dots,n$,有$a_{ij}=b_{ij}$,
  则称$\mmA$与$\mmB$相等,记作$\mmA=\mmB$。
\end{definition}

\subsection{矩阵的线性运算}
矩阵的线性运算包含\textbf{加法}和\textbf{数乘}。定义比较简单,在此不加赘述。

根据后面线性空间的定义\ref{def-linear-space},
矩阵空间也是线性空间。

\subsection{矩阵的乘法}
\begin{definition}[矩阵的乘法]
  设$\mmA=(a_{ik})_{m\times s}, \mmB=(b_{kj})_{s\times n}$,令
  \begin{equation}
  c_{ij}=\sum_{k=1}^{s}a_{ik}b_{kj}
  \end{equation}
  则$\mmC=(c_{ij})_{m\times n}$被称为$\mmA$和$\mmB$的乘积,
  记为$\mmC = \mmA\mmB$。
\end{definition}

矩阵的乘法满足分配率和结合律,但不满足交换律。

\subsection{矩阵的转置}
\begin{definition}[矩阵的转置]
    设$\mmA=(a_{ij})_{m\times n}$,
    令$b_{ij}=a_{ji}$,
    则$\mmB=(b_{ij})_{n\times m}$被称为$\mmA$的转置矩阵,
    记为$\mmA^T$。
\end{definition}

\begin{theorem}[转置矩阵的性质]
  转置矩阵具有以下性质:
  \begin{enumerate}
    \item $\left(\mmA^T\right)^T = \mmA$
    \item $(\mmA+\mmB)^T=\mmA^T + \mmB^T$
    \item $(k\mmA)^T=k\mmA^T$
    \item $(\mmA\mmB)^T=\mmB^T\mmA^T$
  \end{enumerate}
\end{theorem}

\begin{definition}[对称矩阵与反对称矩阵]
  对于$n$阶方阵$\mmA=(a_{ij})$,
  如果满足$\mmA^T=\mmA$,则称$\mmA$为\textbf{对称矩阵}。
  如果满足$\mmA^T=-\mmA$,则称$\mmA$为\textbf{反对称矩阵}。
\end{definition}

\begin{remark}
  不难看出,反对称矩阵的主对角线元素都为零。
\end{remark}

\section{矩阵的分块}
矩阵的分块是化简矩阵运算简单而又重要的思想。
其中最重要的是分块矩阵的乘法。

\subsection{分块矩阵的概念}
\begin{definition}[分块矩阵]
  一般地,将一个$m\times n$矩阵$\mmA$用横线划分成$r$块,
  用竖线划分成$s$块,就能得到一个$r\times s$分块矩阵。
  \begin{equation} \label{eq:mat-partition}
  \mmA = \mmat{cccc}{
    \mmA_{11} & \mmA_{12} & \cdots & \mmA_{1s} \\
    \mmA_{21} & \mmA_{22} & \cdots & \mmA_{2s} \\
    \vdots    & \vdots    & \ddots & \vdots    \\
    \mmA_{r1} & \mmA_{r2} & \cdots & \mmA_{rs} }
    = (\mmA_{ij})_{r\times s}
  \end{equation}
  其中,$\mmA_{ij}(i=1,\dots,r,\ j=1,\dots,s)$是$m_i\times n_j$矩阵,
  $\sum_{i=1}^{r}m_i = m$,$\sum_{j=1}^{s}n_j=n$。
\end{definition}

最常用的一种矩阵分块方法是把$m\times n$划分成$m$个行向量,
或者$n$个列向量。

\subsection{分块矩阵的运算}
分块矩阵的运算中需要注意的是转置和乘法。

转置运算不仅需要把矩阵的每一行转成列,而且内部子块也需要转置。
对于式\ref{eq:mat-partition}中的矩阵$\mmA$,它的转置是
\[
  \mmA^T = \mmat{cccc}{
    \mmA_{11}^T & \mmA_{21}^T & \cdots & \mmA_{r1}^T \\
    \mmA_{12}^T & \mmA_{22}^T & \cdots & \mmA_{r2}^T \\
    \vdots      & \vdots      & \ddots & \vdots      \\
    \mmA_{1s}^T & \mmA_{2s}^T & \cdots & \mmA_{sr}^T }
\]

对于乘法运算,
给定两个矩阵$\mmA=(a_{ij})_{m\times n}$,$\mmB=(b_{jk})_{n\times p}$,
只要使$\mmB$的行分法与$\mmA$的列分法一致,
就能把子块当作数一样按照矩阵乘法的规则进行计算。

\subsection{准对角矩阵}
\begin{definition}[准对角矩阵] 
  对于$n$阶方阵$\mmA$,
  如果有一种分法,使$\mmA$的主对角线以外的子块都是零矩阵,
  且主对角线上子块都是方阵,则称$\mmA$为准对角矩阵。
\end{definition}

\begin{remark}
  当然,准对角矩阵包含对角矩阵作为特殊情况。
\end{remark}

设$\mmA$和$\mmB$都是$n$阶方阵。
如果有相同的分法,使得
\[
\mmA = \mmat{cccc}{
  \mmA_1 &        &        & \\
         & \mmA_2 &        & \\
         &        & \ddots & \\
         &        &        & \mmA_n },\quad
\mmB = \mmat{cccc}{
  \mmB_1 &        &        & \\
         & \mmB_2 &        & \\
         &        & \ddots & \\
         &        &        & \mmB_n }
\]
都是准对角矩阵,那么显然有
\[
\mmA\mmB = \mmat{cccc}{
    \mmA_1\mmB_1 &              &        & \\
                 & \mmA_2\mmB_2 &        & \\
                 &              & \ddots & \\
                 &              &        & \mmA_n\mmB_n }
\]

\section{逆矩阵与初等变换}
本小节开始讨论逆矩阵,
并试图通过初等变换的概念,从更本质的视角来看待矩阵
——矩阵就是所谓标准型乘上有限个初等阵。
也因为初等阵具有良好的性质——可逆性,
所以矩阵可逆当且仅当它的标准型就是单位矩阵。

\subsection{逆矩阵}
\begin{definition}[逆矩阵]
  设$\mmA$是$n$阶方阵。如果存在$n$阶方阵$\mmB$,使
  \begin{equation}
    AB=BA=I
  \end{equation}
  则称$\mmB$为$\mmA$的\textbf{逆矩阵},记作$A^{-1}$,
  并说$\mmA$是\textbf{非奇异矩阵}或\textbf{可逆矩阵},
  否则便称$\mmA$是\textbf{奇异矩阵}或\textbf{不可逆矩阵}。
\end{definition}

\begin{theorem}[逆矩阵的唯一性]
  可逆矩阵的逆矩阵是唯一的。
\end{theorem}

\begin{theorem}[可逆矩阵的性质]
  若$\mmA,\mmB$是可逆矩阵,
  则$\mmA^{-1}$,$k\mmA\,(k\neq 0)$,$\mmA^T$,$\mmA\mmB$都是可逆矩阵,且
  \begin{enumerate}
    \item $(\mmA^{-1})^{-1} = \mmA$
    \item $(k\mmA)^{-1} = \frac{1}{k}\mmA^{-1}$
    \item $(\mmA^T)^{-1} = (\mmA^{-1})^T$
    \item $(\mmA\mmB)^{-1} = \mmB^{-1}\mmA^{-1}$
  \end{enumerate}
\end{theorem}

\subsection{初等变换与初等阵}
矩阵的\textbf{初等变换}可以分为\textbf{初等行变换}和\textbf{初等列变换}。
下面主要讨论初等行变换。初等列变换与之相似。

\begin{definition}[初等行变换]
  初等行变换有三种:
  \begin{enumerate}
    \item 交换矩阵的第$i$行和第$j$行,记为$\mTfRowSwi{i}{j}$。
    \item 将矩阵的第$i$行乘非零常数$k$,记为$\mTfRowMul{k}{j}$。
    \item 把矩阵的第$i$行加上第$j$行的$k$倍,记为$\mTfRowAdd{i}{+k}{j}$。
  \end{enumerate}
\end{definition}

\begin{definition}[初等阵]
  单位矩阵经过一次初等变换得到的矩阵叫做\textbf{初等矩阵},
  或简称\textbf{初等阵}。
\end{definition}

\begin{remark}
  显然,初等阵是方阵,并且都是可逆的。
\end{remark}

使用三种初等行变换能得到三个初等阵,
我们把它们分别记作
$\mmRowSwi{i}{j}$,$\mmRowMul{k}{i}$和$\mmRowAdd{i}{+k}{j}$。
对单位矩阵实行一次初等列变换,同样能得到上述三种形式的初等阵,
所以初等阵只有以上三种。

\begin{theorem}[一般矩阵的初等变换与初等阵的联系]
  对于一个$m\times n$矩阵$\mmA$做一次初等行变换,
  就相当于对$\mmA$左乘一个$m$阶初等矩阵;
  对$\mmA$做一次初等列变换,
  就相当于对$\mmA$右乘一个$n$阶初等矩阵。
\end{theorem}

\begin{definition}[矩阵等价]
  如果矩阵$\mmA$可以通过有限次初等变换化为矩阵$\mmB$,
  那么就说$\mmA$与$\mmB$等价。
\end{definition}

\begin{remark}
  矩阵等价是一个等价关系,即满足自反性、对称性、传递性。
\end{remark}

\begin{theorem}[标准型]
  任意非零$m\times n$矩阵$\mmA$都等价于矩阵
  \[
    \mmat{cc}{\mmI_r & \mmZero \\ \mmZero & \mmZero}
    \quad (1\le r \le \min(m,n))
  \]
  它称为矩阵$\mmA$的\textbf{标准型}。
  换句话说,任意非零$m\times n$矩阵$\mmA$,
  必有初等矩阵$\mmP_1,\dots,\mmP_s$,$\mmQ_1,\dots,\mmQ_t$,
  使得
  \[
    \mmP_s\dots\mmP_1\cdot\mmA\cdot\mmQ_1\dots\mmQ_t = 
      \mmat{cc}{\mmI_r & \mmZero \\ \mmZero & \mmZero}
  \]
\end{theorem}

因为初等矩阵可逆,所以有如下推论

\begin{corollary}
  对于任意非零$m\times n$矩阵$\mmA$,
  存在可逆$m\times m$矩阵$\mmP$和可逆$n\times n$矩阵$\mmQ$,
  使得
  \[
    \mmP\mmA\mmQ = \mmat{cc}{\mmI_r & \mmZero \\ \mmZero & \mmZero}
    \quad (1\le r \le \min(m,n))
  \]
  或写成
  \[
    \mmA = \mmP^{-1}\mmat{cc}{\mmI_r & \mmZero \\ \mmZero & \mmZero}\mmQ^{-1}
    \quad (1\le r \le \min(m,n))
  \]
\end{corollary}

\subsection{矩阵可逆的充要条件}

\begin{theorem}
  设$\mmA,\mmB$是$n$阶方阵,
  若$\mmA\mmB=\mmI_n$,
  则$\mmA,\mmB$都是可逆阵,且它们互为逆阵。
\end{theorem}

\begin{theorem} \label{thrm-inv-equiv-cond}
  设$\mmA$是$n$阶方阵,则下列结论等价:
  \begin{enumerate}
    \item $\mmA$是可逆矩阵。
    \item $\mmA$可以表示为有限个初等阵的乘积。
    \item $\mmA$可以经过有限次初等变换化为单位矩阵。
  \end{enumerate}
\end{theorem}

\subsection{用初等变换求逆矩阵}
\[
  \mmat{c|c}{\mmA & \mmI} \xrightarrow{\text{初等行变换}}
  \mmat{c|c}{\mmI & \mmA^{-1}}
\]
