\chapter{线性空间与线性变换}

\section{线性空间}
在这一小节,我们把先前向量的概念进一步抽象,得到了线性空间的概念。
向量组有线性相关、线性无关、极大线性无关组等概念,
这在线性空间中就转变成纬度、基、坐标的概念。

线性空间和物理的联系也是很紧密的。
因为向量是物理的常用量,而线性空间就是从向量抽象来的。
回忆一下物理学的知识,我们知道一个物理量在不同参考系的表示是不一样的。
对应到线性空间上来,我们会发现一个元素在不同基底下的坐标也是不一样的。
所以我们要研究这些坐标与基底的变换关系。

最后我们会简单讨论一下子空间的概念。

\subsection{线性空间}
\begin{definition}[线性空间] \label{def-linear-space}
  设$V$是非空集合,$\mfF$是一个数域。
  若在$V$上定义两种运算:加法$\oplus$和数乘$\otimes$,满足
  \begin{description}
    \item[加法]
    \begin{enumerate}
      \item 封闭性:
      $\forall\alpha,\beta\in V,\ \alpha\oplus\beta\in V$。
      \item 交换律:
      $\forall\alpha,\beta\in V,\ \alpha\oplus\beta = \beta\oplus\alpha$
      \item 结合律:
      $\forall\alpha,\beta,\gamma\in V,\ 
        (\alpha\oplus\beta)\oplus\gamma = \alpha\oplus(\beta\oplus\gamma)$。
      \item 零元存在:
      $\exists\theta\in V,\forall\alpha\in V,\ \alpha\oplus\theta=\alpha$。
      \item 逆元存在:
      $\forall\alpha\in V,\exists\beta\in V,\ \alpha\oplus\beta=\theta$。
      我们把$\beta$记为``$-\alpha$''。
    \end{enumerate}  
    \item[数乘]
    \begin{enumerate}
      \item 封闭性:
      $\forall\lambda\in\mfF,\alpha\in V,\ \lambda\otimes\alpha\in V$。
      \item 结合律:
      $\forall\lambda_1,\lambda_2\in\mfF,\alpha\in V,\ 
        \lambda_1\otimes(\lambda_2\otimes\alpha) =
        (\lambda_1\lambda_2)\otimes\alpha$。
      \item 分配律1:
      $\forall\lambda_1,\lambda_2\in\mfF,\alpha\in V,\ 
        (\lambda_1\oplus\lambda_2)\otimes\alpha =
        (\lambda_1\otimes\alpha)\oplus(\lambda_2\otimes\alpha)$
      \item 分配律2:
      $\forall\lambda\in\mfF,\alpha,\beta\in V,\ 
        \lambda\otimes(\alpha\oplus\beta) =
        (\lambda\otimes\alpha)\oplus(\lambda\otimes\beta)$。
      \item 单位元存在:
      $\forall\alpha\in V,\ 
        1\otimes\alpha = \alpha$。
    \end{enumerate}
  \end{description}
  则称$V$是数域$\mfF$上的一个\textbf{线性空间},又称\textbf{向量空间}。
  若$\mfF$是实数域$\mfR$,则称$V$是\textbf{实线性空间}。
  若$\mfF$是复数域$\mfC$,则称$V$是\textbf{复线性空间}。
\end{definition}

\begin{remark}
  如果有点抽象代数的背景,读者不难发现上述加法满足的是阿贝尔群的性质。
\end{remark}

\begin{theorem}[线性空间的性质]
  设$V$是$\mfF$上的线性空间,那么
  \begin{enumerate}
    \item 加法零元的具有唯一性
    \item 加法逆元的具有唯一性
    \item 加法零元的求法:
    $\forall\alpha\in V,\ 0\otimes\alpha=\theta$
    \item 加法逆元的求法:
    $\forall\alpha\in V,\ (-1)\otimes\alpha=-\alpha$
    \item $\forall k\in\mfF,\ k\otimes\theta=\theta$
    \item 若$\lambda\otimes\alpha=\theta$,则$\lambda=0$或$\alpha=\theta$。
  \end{enumerate}
\end{theorem}
为了书写与阅读的方便,以后``$\oplus$''用正常的加号来表示,
``$\otimes$''可省略。参考向量的记法。

\subsection{基与坐标}
向量组的线性相关、线性无关、线性表示的概念也适用于线性空间。

\begin{definition}[线性相关与线性无关]
  设$V$是线性空间,$\alpha_1,\alpha_2,\dots,\alpha_s\in V$。
  若数域中存在一组不全为0的数$k_1,k_2,\dots,k_s$,使得
  \[ k_1\alpha_1 + k_2\alpha_2 + \dots + k_s\alpha_s = \theta \]
  则称$\alpha_1,\alpha_2,\dots,\alpha_s$\textbf{线性相关}。
  否则,则称\textbf{线性无关}。
\end{definition}

\begin{definition}[线性表示]
  设$V$是线性空间,$\alpha,\alpha_1,\alpha_2,\dots,\alpha_s\in V$。
  若数域中存在一组数$k_1,k_2,\dots,k_s$,使得
  \[ \alpha = k_1\alpha_1 + k_2\alpha_2 + \dots + k_s\alpha_s \]
  则称$\alpha$为$\alpha_1,\alpha_2,\dots,\alpha_s$的\textbf{线性组合},
  也称$\alpha$可以由$\alpha_1,\alpha_2,\dots,\alpha_s$\textbf{线性表示}。
\end{definition}

\begin{definition}[维数]
  如果一个线性空间$V$中,线性无关的元素的最大个数是$n$,
  则称该线性空间是$n$维的,记作$\mdim V = n$。
  
  如果对任意正整数$N$,总存在$N$个线性无关的元素,
  则称该线性空间是\textbf{无穷维线性空间}。
  不是无穷维的线性空间叫做有穷维线性空间。
\end{definition}

\begin{remark}
  本章中我们不讨论无穷维线性空间。
\end{remark}

\begin{definition}[基与坐标]
  若$\alpha_1,\alpha_2,\dots,\alpha_n$是$n$维线性空间$V$中一组线性无关的元素,
  且$V$中任意元素$\alpha$都可由$\alpha_1,\alpha_2,\dots,\alpha_n$线性表示,
  即\[ \alpha = k_1\alpha_1 + k_2\alpha_2 + \dots + k_n\alpha_n \]
  则称$\alpha_1,\alpha_2,\dots,\alpha_n$是$V$的一组\textbf{基底}(简称\textbf{基})。
  其中,有序元组$(k_1,k_2,\dots,k_n)$称为$\alpha$在
  基底$\alpha_1,\alpha_2,\dots,\alpha_n$下的坐标。
\end{definition}

\begin{remark}
  线性空间的基对应的是极大线性无关组的概念。
\end{remark}

\subsection{基变换与坐标变换}
\begin{theorem}[基底变换公式]
  设$\alpha_1,\alpha_2,\dots,\alpha_n$与$\beta_1,\beta_2,\dots,\beta_n$是
  线性空间$V$的两组基底,那么可以写成
  \begin{equation} \label{eq-basis-transform}
    (\beta_1,\beta_2,\dots,\beta_n) = (\alpha_1,\alpha_2,\dots,\alpha_n)
    \mmat{cccc}{
      a_{11} & a_{12} & \cdots & a_{1n} \\
      a_{21} & a_{22} & \cdots & a_{2n} \\
      \vdots & \vdots & \ddots & \vdots \\
      a_{m1} & a_{m2} & \cdots & a_{mn} }
  \end{equation}
  我们把它简记为
  \begin{displaymath}
    \mmBasis{\beta} = \mmBasis{\alpha}\cdot\mmP
  \end{displaymath}
  我们称$\mmP$是$\mmBasis{\alpha}$到$\mmBasis{\beta}$的\textbf{过渡矩阵},
  式\ref{eq-basis-transform}是\textbf{基底变换公式}。
\end{theorem}

\begin{remark}
  过渡矩阵$\mmP$是非奇异矩阵。
  因为如果$\mmP$是奇异的,就存在$\mvx\neq\mvZero$,使得$\mmP\mvx=\mvZero$。
  从而有$\mmBasis{\beta}\mvx = \mmBasis{\alpha}\mmP\mvx = \mvZero$。
  于是$\mmBasis{\beta}$是奇异矩阵,这与``$\beta_1,\dots,\beta_n$是基底''是矛盾的。
\end{remark}

\begin{theorem}[坐标变换公式]
  设$\mmBasis{\alpha}$和$\mmBasis{\beta}$是线性空间$V$的两组基底,
  $\mmP$是从$\mmBasis{\alpha}$到$\mmBasis{\beta}$的过渡矩阵。
  若$\mvx=(x_1,\dots,x_n)^T$和$\mvy=(y_1,\dots,y_n)^T$
  分别是元素$\alpha\in V$在基底$\mmBasis{\alpha}$和$\mmBasis{\beta}$下的坐标,即
  \[ \alpha = \mmBasis{\beta}\mvy = \mmBasis{\alpha}\mvx \]
  那么有
  \begin{equation} \label{eq-coordinate-transform}
    \mvy = \mmP^{-1}\mvx
  \end{equation}
  式\ref{eq-coordinate-transform}被称为\textbf{坐标变换公式}。
\end{theorem}

\subsection{子空间}
\begin{definition}
  设$V$是数域$\mfF$上的线性空间。若$W\subset V$也是数域$\mfF$上的线性空间,
  则称$W$是$V$的\textbf{子空间}。
  $W=\{\theta\}$叫做\textbf{平凡子空间}。
\end{definition}

\begin{theorem}[子空间的充要条件]
  设$V$是数域$\mfF$上线性空间。
  $W\subset V$是$V$的子空间的充要条件是:
  1. 对加法封闭;2. 对乘法封闭。
  换句话说,即是$\forall\alpha,\beta\in W,\lambda,\mu\in\mfF$,
  \[ \lambda\alpha + \mu\beta \in W \]
\end{theorem}

\begin{definition}[子集张成的子空间]
    设$V$是数域$\mfF$上的线性空间,$\alpha_1,\alpha_2,\dots,\alpha_s\in V$。
    我们定义$V$的子空间
    \begin{displaymath}
    \mspan\{ \alpha_1,\alpha_2,\dots,\alpha_s \} =
    \{ \alpha: \alpha=\sum_{i=1}^{s}k_i\alpha_i, \forall k_1,\dots,k_s\in\mfF \}
    \end{displaymath}
    称作由向量组$\alpha_1,\alpha_2,\dots,\alpha_s$张成的子空间。
\end{definition}

\begin{definition}[线性空间的和]
  设$W_1,W_2$是线性空间的两个子空间,则它们的\textbf{和}定义为
  \begin{displaymath}
    W_1+W_2 = \{ u: u=\alpha+\beta, \forall\alpha\in W_1,\beta\in W_2 \}
  \end{displaymath}
\end{definition}

\begin{theorem}
  设$V$是线性空间,$W_1,W_2$是$V$的子空间,那么有以下结论:
  \begin{enumerate}
    \item $W_1\cap W_2$是$V$的子空间。
    \item $W_1\cup W_2$不是$V$的子空间。
    \item $W_1+W_2$是$V$的子空间。
  \end{enumerate}
\end{theorem}

\begin{theorem}[维数定理]
  设$W_1,W_2$是线性空间$V$的两个有限维子空间,则有
  \begin{displaymath}
  \mdim W_1 + \mdim W_2 = \mdim(W_1+W_2) + \dim(W_1\cap W_2)
  \end{displaymath}
\end{theorem}

\begin{definition}[直接和]
  设$V_1,V_2$是线性空间$V$的两个子空间。
  若对任意$\alpha\in V$,存在唯一的$\alpha_1\in V_1, \alpha_2\in V_2$,
  使得$\alpha=\alpha_1+\alpha_2$,
  则称$V$是$V_1$和$V_2$的\textbf{直接和}或\textbf{直和},
  记作$V=V_1\oplus V_2$。
  也称$V_1$和$V_2$是$V$内的\textbf{互补空间}。
\end{definition}

\begin{theorem}[直接和的等价条件]
  设$V_1,V_2$是线性空间$V$的两个子空间。
  \begin{align*}
    V=V_1\oplus V_2
    &\iff V = V_1 + V_2\ \text{且}\ V_1\cap V_2 = \{ \theta \} \\
    &\iff \mdim (V_1\oplus V_2) = \mdim V_1 + \mdim V_2
  \end{align*}
\end{theorem}

\section{线性变换}
如果说线性空间是对向量的抽象,那么线性变换就是对矩阵的抽象。
回想一下,我们在上一小节研究了基与坐标变换的公式,
公式中就是用矩阵来刻画了``变换''这个过程。
在介绍换线性变换的基本概念之后,
我们会证明,在取定一组基下,线性变换与矩阵有着一一对应的关系。

虽然线性变换在一组基下仅有唯一的矩阵与之对应,
但换个角度来看,有不同的基就会有不同的矩阵与之对应。
这些矩阵间又有什么关系呢?这就引出了相似矩阵的概念。

\subsection{线性变换的概念}
\begin{definition}[线性变换]
  设$V,W$是数域$\mfF$上的线性空间。
  函数$T: V\mapsto W$被称为$V$到$W$的\textbf{变换}。
  若对任意$\alpha,\beta\in V, k \in\mfF$,$T$满足
  \[ T(\alpha+\beta)=T(\alpha)+T(\beta),\quad T(k\alpha) = kT(\alpha) \]
  则称$T$是$V$到$W$的\textbf{线性变换}。
\end{definition}

\begin{theorem}[线性变换的性质]
  设$V,W$是数域$\mfF$上的线性空间,$T$是$V$到$W$的线性变换,
  $\theta_1$和$\theta_2$分别是$V$和$W$上的零元。
  $T$满足以下性质:
  \begin{enumerate}
    \item
    $T(\theta_1) = \theta_2$
    \item
    $\forall\alpha,\beta\in V,\lambda,\mu\in\mfF,\ 
      T(\lambda\alpha+\mu\beta)=\lambda T(\alpha)+\mu T(\beta)$
    \item
    若$V$中的$\alpha_1,\alpha_2,\dots,\alpha_s$线性相关,
    则$T(\alpha_1),T(\alpha_2),\dots,T(\alpha_s)$也线性相关。
  \end{enumerate}
\end{theorem}

\begin{definition}[特殊的线性变换]
  设$V$是数域$\mfF$上的线性空间,$T$是$V$上的线性变换。
  \begin{enumerate}
    \item
    若对任意$\alpha\in V$,$T(\alpha)=\theta$,
    则称$T$为\textbf{零变换},常用$T_0$来表示。
    \item
    若对任意$\alpha\in V$,$T(\alpha)=\alpha$,
    则称$T$为\textbf{恒等变换},常用$E$或$I$表示。
    \item 
    若对任意$\alpha\in V$,$T(\alpha)=k\alpha$,其中$k\in\mfF$,
    则称$T$为\textbf{数乘变换},常用$T_k$来表示。
  \end{enumerate}
\end{definition}

\begin{definition}[象空间与核空间]
  设$T$是线性空间$V$到$W$的线性变换,
  则$T$的\textbf{象空间}定义为:
  \[ \mim(T) = \{ \xi: \xi=T(\alpha), \alpha\in V\} \]
  $T$的\textbf{核空间}定义为:
  \[ \mker(T) = \{ \alpha: T(\alpha)=\theta, \alpha\in V \} \]
\end{definition}

\begin{theorem}[象空间与核空间的性质]
  设$T$是线性空间$V$到$W$的线性变换,那么
  \begin{enumerate}
    \item 
    $\mim(T)$是$V$的子空间,$\mker(T)$是$W$的子空间。
    \item
    $\mdim(\mim(T)) + \mdim(\ker(T)) = \dim(V)$
  \end{enumerate}
\end{theorem}

\subsection{线性变换的运算和可逆线性变换}
设$L(V)$是线性空间$V$上所有线性变换构成的集合。
类似矩阵,给定一个数域$\mfF$,
我们也可以在$L(V)$上定义加法、数乘和乘法运算:
\begin{description}
  \item[加法]
  对任意$T_1,T_2\in L(V),\alpha\in V$,$(T_1+T_2)(\alpha)=T_1(\alpha)+T_2(\alpha)$。
  \item[数乘]
  对任意$T\in L(V),\alpha\in V$,$(kT)(\alpha)=kT(\alpha)$,
  其中$k\in\mfF$。
  \item[乘法]
  因为线性变换定义上是个函数,所以两个线性变换相乘定义为两个函数的复合。
\end{description}

\begin{theorem}
  拥有上面定义的加法、数乘规则的$L(V)$是线性空间。
\end{theorem}

\begin{definition}[可逆线性变换]
  设$T$是线性空间$V$上的一个线性变换。
  如果$V$上存在一个变换$\sigma$,使得
  \[ T\sigma = \sigma T = E \]
  其中$E$是$V$上的恒等变换,则称$\sigma$是$T$的\textbf{逆变换}。
  不难看出,如果$T$的逆变换存在,那么它必然是唯一的,
  因此,我们把$T$的逆变换记作$T^{-1}$,
  并称$T$为\textbf{可逆线性变换}。
\end{definition}

\subsection{线性变换的矩阵表示}
接下来,我们讨论在给定一组基下,线性变换与矩阵有一一对应的关系。
需要注意的是,我们这里讨论的线性变换是从一个线性空间到自身的变换,
所以这里的矩阵也就是方阵。

首先,如果知道一个$n$维线性空间$V$上的线性变换$T$,
我们就能知道它在基$\epsilon_1,\epsilon_2\dots,\epsilon_n$下对应的矩阵:
对任意$i=1,2,\dots,n$,应该有一组坐标$a_{1i},a_{2i},\dots,a_{ni}$,使得
\begin{displaymath}
  T(\epsilon_i) = a_{1i}\epsilon_1+a_{2i}\epsilon_2+\dots+a_{ni}\epsilon_n
\end{displaymath}
写成矩阵的形式,即是
\begin{displaymath}
  (T(\epsilon_1), T(\epsilon_2),\dots,T(\epsilon_n)) =
    (\epsilon_1,\epsilon_2,\dots,\epsilon_n)\mmA
\end{displaymath}
其中$\mmA=(a_{ij})_{n\times n}$。
这样,对任意$\alpha=x_1\epsilon_1+x_2\epsilon_2+\dots+x_n\epsilon_n\in V$,
我们就能通过矩阵求出它经过线性变换后元素:
\begin{equation} \label{eq-coord-after-linear-trans}
  T(\alpha) = (\epsilon_1,\epsilon_2,\dots,\epsilon_n)\mmA
    (x_1,x_2,\dots,x_n)^T
\end{equation}
因此,我们称$\mmA$是
\textbf{线性变换$T$在基$\epsilon_1,\epsilon_2\dots,\epsilon_n$下的矩阵}。

反之,在线性空间$V$的一组基$\epsilon_1,\epsilon_2\dots,\epsilon_n$下,
给定矩阵$\mmA=(a_{ij})_{n\times n}$,
存在线性变换$T$,使得
\begin{displaymath}
  (T(\epsilon_1), T(\epsilon_2),\dots,T(\epsilon_n)) =
    (\epsilon_1,\epsilon_2,\dots,\epsilon_n)\mmA
\end{displaymath}
这个$T$是这样构造的:
对任意$\alpha=x_1\epsilon_1+x_2\epsilon_2+\dots+x_n\epsilon_n\in V$,
\begin{displaymath}
  T(\alpha) = (\epsilon_1,\epsilon_2,\dots,\epsilon_n)\mmA
    (x_1,x_2,\dots,x_n)^T
\end{displaymath}

总结一下上面的论述,有如下定理:
\begin{theorem}
  在线性空间$V$的一组基$\epsilon_1,\epsilon_2\dots,\epsilon_n$下,
  线性变换$T$与$n$阶方阵$\mmA$一一对应。
  $\mmA$的第$i$列就是$T(\epsilon_i)$在
  基$\epsilon_1,\epsilon_2\dots,\epsilon_n$下的坐标。
\end{theorem}

\subsection{相似矩阵}
\begin{theorem}
  设$n$为线性空间$V$的两组基底为$\epsilon_1,\epsilon_2\dots,\epsilon_n$以及
  $\eta_1,\eta_2,\dots,\eta_n$。由$\epsilon_1,\epsilon_2\dots,\epsilon_n$到
  $\eta_1,\eta_2,\dots,\eta_n$的过渡矩阵为$\mmP$。
  $V$上线性变换$T$在这两组基下的矩阵分别为$\mmA,\mmB$,那么有
  \[ \mmB = \mmP^{-1}\mmA\mmP \]  
\end{theorem}

\begin{remark}
  直观的理解就是,
  \begin{displaymath}
    (\epsilon_1,\epsilon_2\dots,\epsilon_n)\xrightarrow{\mmP}
    (\eta_1,\eta_2,\dots,\eta_n)\xrightarrow{\mmB}
    T(\eta_1,\eta_2,\dots,\eta_n)
  \end{displaymath}
  等效于
  \begin{displaymath}
  (\epsilon_1,\epsilon_2\dots,\epsilon_n)\xrightarrow{\mmA}
  T(\epsilon_1,\epsilon_2\dots,\epsilon_n)\xrightarrow{\mmP}
  T(\eta_1,\eta_2,\dots,\eta_n)
  \end{displaymath}
  所以有$\mmP\mmB=\mmA\mmP$。变形即是上面定理的结果。
  这也是这个定理的证明思路。
\end{remark}

\begin{definition}[相似矩阵]
  设$\mmA,\mmB$是两个同型矩阵。
  若存在满秩矩阵$P$,使得$\mmB=\mmP^{-1}\mmA\mmP$,
  则称$\mmB$是$\mmA$的\textbf{相似矩阵},记作$\mmA\sim\mmB$。
\end{definition}

\begin{remark}
  矩阵的相似是等价关系。
\end{remark}

\begin{theorem}[矩阵相似的等价条件] \label{thrm-mat-sim-equiv-cond}
  两个$n$阶方阵$\mmA,\mmB$相似当且仅当
  它们是$n$维线性空间$V$上的某一线性变换$T$在不同基下的矩阵。
\end{theorem}

\section{特征值与特征向量}
本小节首先会花一些篇幅来分析特征值,而不是简单给出求解特征值的方法。
这样是试图加强对特征值意义的理解。
然后会给出一些矩阵之间特征值的关联。
最后讨论一下矩阵的对角化问题。

\subsection{特征值与特征向量}
对于$n$维线性空间$V$内的一个线性变换$T$,
我们总希望找一组基$\xi_1,\xi_2,\dots,\xi_n$,
使得$T$在这组基下的矩阵具有最简单的形式,即有
\begin{equation} \label{eq-diagonalizable}
  (T(\xi_1), T(\xi_2),\dots,T(\xi_n)) = (\xi_1,\xi_2,\dots,\xi_n)
    \cdot \mdiag(\lambda_1,\lambda_2,\dots,\lambda_n)
\end{equation}
这样,根据式\ref{eq-coord-after-linear-trans},
任意$\alpha\in V$经过$T$变换后的坐标就是把它在
基$\xi_1,\xi_2,\dots,\xi_n$下的坐标放大常数倍。

根据上面的分析,我们需要找这样一组$\xi$和$\lambda$,使得$T(\xi) = \lambda\xi$。
这就引出了线性变换的特征值和特征向量的定义(注意,还不是矩阵的特征值与特征向量)。

\begin{definition}[线性变换的特征值和特征向量]
  设$V$是数域$\mfF$上的一个线性空间,$T$是$V$上的一个线性变换。
  如果对$\lambda\in\mfF$,存在非零向量$\xi\in V$,使得
  \[ T(\xi) = \lambda\xi \]
  则称$\lambda$是$T$的一个\textbf{特征值},
  而称$\xi$是$T$对应于$\lambda$的一个\textbf{特征向量}。
\end{definition}

\begin{theorem}[特征子空间]
  设$V$是数域$\mfF$上的线性空间,$T$是$V$上的线性变换,
  则给定一个特征值$\lambda\in\mfF$,
  \begin{displaymath}
    V_\lambda = \{ \xi\in V: T(\xi)=\lambda\xi \}
  \end{displaymath}
  构成了$V$的一个子空间。
  我们把它称为$T$对应于$\lambda$的特征子空间。
\end{theorem}

下面,我们要考虑如何求解线性变换的特征值与特征向量。
因为线性变换在给定一组基下与矩阵有着一一对应的关系,
所以我们尝试从矩阵开始入手。

我们设$\epsilon_1,\epsilon_2,\dots,\epsilon_n$是线性空间$V$的一组基,
$T$在这组基下的矩阵是$\mmA$。
设$\xi$是线性变换$T$的一个特征向量,它对应的特征值是$\lambda$,
它在基$\epsilon_1,\epsilon_2,\dots,\epsilon_n$下的坐标是
$\mvx=(x_1,\dots,x_n)^T$。
那么,根据式\ref{eq-coord-after-linear-trans},我们有
\[ T(\xi) = (\epsilon_1,\epsilon_2,\dots,\epsilon_n)\mmA\mvx \]
又因为
\[ T(\xi) = \lambda\xi = \lambda(\epsilon_1,\epsilon_2,\dots,\epsilon_n)\mvx \]
所以联立以上两个等式,我们就有
\begin{displaymath}
  (\epsilon_1,\epsilon_2,\dots,\epsilon_n)(\mmA\mvx - \lambda\mvx)=0
\end{displaymath}
因为$\epsilon_1,\epsilon_2,\dots,\epsilon_n$线性无关,
所以只可能是$\mmA\mvx - \lambda\mvx = \mvZero$,即
\begin{equation} \label{eq-eigen}
  (\mmA - \lambda\mmI)\mvx = \mvZero
\end{equation}

接下来我们需要解这个方程来得到特征值$\lambda$
和特征向量对应的坐标$\mvx$。
想要$\mvx$有非零解,充要条件是$|\mmA - \lambda\mmI| = 0$。
我们就能根据这一条件解出$\lambda$,
然后把$\lambda$代入上面的方程\ref{eq-eigen}中,
最后按照解方程的一般步骤,就能获得$\mvx$的基础解系。

我们还能发现,对式\ref{eq-eigen}变形,能得到类似于$T(\xi)=\lambda\xi$形式的等式
\begin{displaymath}
\mmA\mvx=\lambda\mvx
\end{displaymath}
而且上面一大堆分析都是针对矩阵$\mmA$的,和线性变换$T$基本上没关系。
所以,我们不如也定义矩阵的特征值与特征向量,先针对更加具体的矩阵进行研究,
再与$T$的特征值与特征向量联系起来。

\begin{definition}[矩阵的特征值和特征向量]
  对于$n$阶矩阵$\mmA$,记
  \[P(\lambda)=|\mmA-\lambda\mmI|\]
  $P(\lambda)$被称为$\mmA$的\textbf{特征多项式}。
  $P(\lambda)=0$的根称为$\mmA$的\textbf{特征值}或\textbf{特征根}。
  如果$\lambda$是$P(\lambda)=0$的$k$重根,
  那么又称$\lambda$是\textbf{$k$重特征值},$k$叫做\textbf{代数重数}。
  
  对于一个特征值$\lambda_0$,
  我们称方程$(\mmA-\lambda_0\mmI)\mvx=0$的非零解为
  $\mmA$对应于$\lambda_0$的\textbf{特征向量}。
  解空间线性无关的特征向量的个数叫做\textbf{几何重数}。
\end{definition}

\begin{remark}
  线性变换和矩阵的特征值、特征向量还是有区别的。
  线性变换的特征值属于线性空间的数域$\mfF$,而矩阵的特征值属于复数域。
  另外,矩阵的特征向量在线性变换眼里就是一个坐标,
  它需要与线性变换对应这个矩阵的基相乘,才是线性空间的特征向量。
\end{remark}

\begin{definition}[矩阵的谱]
  $n$阶矩阵$\mmA$的所有特征值$\lambda_1,\dots,\lambda_n$
  叫做矩阵$\mmA$的\textbf{谱}。
  $\max\{|\lambda_1|,\dots,|\lambda_n|\}$被称为\textbf{谱半径}。
\end{definition}

\begin{definition}[矩阵的迹]
  设矩阵$\mmA=(a_{ij})_{n\times n}$。
  定义$\sum_{i=1}^{n}a_{ii}$叫做矩阵$\mmA$的迹,记为$\mtr\mmA$。
\end{definition}

\begin{theorem}[特征值的性质]
  设$n$阶矩阵$\mmA$的所有特征值为$\lambda_1,\dots,\lambda_n$,则
  \begin{enumerate}
    \item $|\mmA| = \prod_{i=1}^{n}\lambda_i$
    \item $\mtr\mmA = \sum_{i=1}^{n}\lambda_i$
  \end{enumerate}
\end{theorem}

\subsection{矩阵之间特征值的关联}
\begin{theorem}[矩阵运算对特征值和特征向量的影响]
  设$\mmA$是$n$阶方阵,$\lambda_0$是$\mmA$的一个特征值,
  $\alpha_0$是对应$\lambda_0$的一个特征向量。
  \begin{enumerate}
    \item
    对于逆矩阵,有$\mmA^{-1}\alpha_0 = \frac{1}{\lambda_0}\alpha_0$
    \item
    对于伴随矩阵,有$\mmA^*\alpha_0 = \frac{|\mmA|}{\lambda_0}\alpha_0$
    \item
    对于转置矩阵,它的特征多项式与$\mmA$的特征多项式相同,
    所以它们有相同的特征值。但是特征向量未必相同。
    \item
    对于矩阵多项式$f(\mmA)=\sum_{i=1}^{k} c_i\mmA^i$,
    有$f(\mmA)\alpha_0=\left(\sum_{i=1}^{k} c_i\lambda_0^i\right)\alpha_0$
  \end{enumerate}
\end{theorem}

\begin{remark}
  如果知道$\mmA$的特征值和特征向量,就可以利用上述定理
  快速求出这些相关的矩阵的特征值与特征向量。
\end{remark}

\begin{theorem}[相似矩阵之间特征值的联系]
  相似矩阵有相同的特征多项式,从而也有相同的谱。
\end{theorem}

\begin{remark}
  这符合直观:相似矩阵是同一个线性变换在不同基下的矩阵,
  因此求出的特征值都应该是相同的。
\end{remark}

\begin{theorem}[分块对角矩阵的特征值]
  设$\mmA$是分块对角矩阵
  \begin{displaymath}
    \mmA = \mmat{cccc}{
      \mmA_1 & & & \\
      & \mmA_2 & & \\
      & & \ddots & \\
      & & & \mmA_n }
  \end{displaymath}
  则$\mmA$的特征多项式就是$\mmA_1,\dots,\mmA_n$的特征多项式的乘积。
  从而$\mmA_1,\dots,\mmA_n$的所有特征值就是$\mmA$的全部特征值。
\end{theorem}

\subsection{矩阵的对角化}
回想一下式\ref{eq-diagonalizable},
这是我们一开始就想要获得的性质,
我们把这个性质称为``可对角化''。
经过之前的讨论,我们现在知道,
式\ref{eq-diagonalizable}中对角矩阵上的对角线元素就是特征值,
那一组基就是特征向量。
然而,尽管我们现在会找特征值与特征向量了,
但是能找到的所有特征向量未必能构成一组基。
我们现在要研究在什么条件下,这个可对角化的性质可以被满足。

\begin{definition}[可对角化的矩阵]
  设$T$是$n$维线性空间上的一个线性变换。
  如果存在一组基,使$T$在这组基下的矩阵是对角形,
  则说$T$所对应的矩阵是\textbf{可对角化矩阵}。
\end{definition}

\begin{remark}
  根据定理\ref{thrm-mat-sim-equiv-cond},
  我们可以得到另一种说法:
  矩阵可对角化当且仅当它相似于对角矩阵。
\end{remark}

\begin{theorem}[矩阵可对角化的充要条件]
  $n$阶矩阵$\mmA$可对角化当且仅当它有$n$个线性无关的特征向量。
\end{theorem}

\begin{remark}
  如果$\xi_1,\dots,\xi_n$是$\mmA$的$n$个线性无关的特征向量,
  它们各自对应的特征值是$\lambda_1,\dots,\lambda_n$,
  那么,存在矩阵$\mmP=(\xi_1,\dots,\xi_n)$,使得
  \begin{displaymath}
    \mmP^{-1}\mmA\mmP = \mmat{ccc}{
      \lambda_1 & & \\
      & \ddots    & \\
      & & \lambda_n }
  \end{displaymath}
\end{remark}

接下来,我们再进一步细化这个充要条件。
\begin{lemma}
  矩阵互异的特征值所对应的特征向量线性无关。
\end{lemma}

\begin{lemma}
  设矩阵的两个互异的特征值为$\lambda_1,\lambda_2$,
  $\xi_1,\dots,\xi_m$是$\lambda_1$的特征向量,
  $\eta_1,\dots,\eta_n$是$\lambda_2$的特征向量,
  则$\xi_1,\dots,\xi_m,\eta_1,\dots,\eta_n$线性无关。
\end{lemma}

\begin{lemma}
  矩阵特征值的几何重数小于等于代数重数。
  换句话说,假设$\lambda$是矩阵的一个$k$重特征根,
  则对应于$\lambda$的特征向量中,
  极大线性无关组的大小不会超过$k$。
\end{lemma}

\begin{theorem}[矩阵可对角化的充要条件]
  $n$阶矩阵$\mmA$可对角化当且仅当
  对于每个$k_i$重特征根$\lambda_i$,
  $\mmA-\lambda_i\mmI$的秩为$n-k_i$
\end{theorem}

\begin{remark}
  换句话说,矩阵可对角化当且仅当
  所有特征根的几何重数等于代数重数。
\end{remark}
