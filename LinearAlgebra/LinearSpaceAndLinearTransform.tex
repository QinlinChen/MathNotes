\chapter{线性空间\ 线性变换\ 欧几里得空间}

\section{线性空间}

\subsection{线性空间}
\begin{definition}[线性空间]
  设$V$是非空集合,$\mfF$是一个数域。
  若在$V$上定义两种运算:加法$\oplus$和数乘$\otimes$,满足
  \begin{description}
    \item[加法]
    \begin{enumerate}
      \item 封闭性:
      $\forall\alpha,\beta\in V,\ \alpha\oplus\beta\in V$。
      \item 交换律:
      $\forall\alpha,\beta\in V,\ \alpha\oplus\beta = \beta\oplus\alpha$
      \item 结合律:
      $\forall\alpha,\beta,\gamma\in V,\ 
        (\alpha\oplus\beta)\oplus\gamma = \alpha\oplus(\beta\oplus\gamma)$。
      \item 零元存在:
      $\exists\theta\in V,\forall\alpha\in V,\ \alpha\oplus\theta=\alpha$。
      \item 逆元存在:
      $\forall\alpha\in V,\exists\beta\in V,\ \alpha\oplus\beta=\theta$。
      我们把$\beta$记为``$-\alpha$''。
    \end{enumerate}  
    \item[数乘]
    \begin{enumerate}
      \item 封闭性:
      $\forall\lambda\in\mfF,\alpha\in V,\ \lambda\otimes\alpha\in V$。
      \item 结合律:
      $\forall\lambda_1,\lambda_2\in\mfF,\alpha\in V,\ 
        \lambda_1\otimes(\lambda_2\otimes\alpha) =
        (\lambda_1\lambda_2)\otimes\alpha$。
      \item 分配律1:
      $\forall\lambda_1,\lambda_2\in\mfF,\alpha\in V,\ 
        (\lambda_1\oplus\lambda_2)\otimes\alpha =
        (\lambda_1\otimes\alpha)\oplus(\lambda_2\otimes\alpha)$
      \item 分配律2:
      $\forall\lambda\in\mfF,\alpha,\beta\in V,\ 
        \lambda\otimes(\alpha\oplus\beta) =
        (\lambda\otimes\alpha)\oplus(\lambda\otimes\beta)$。
      \item 单位元存在:
      $\forall\alpha\in V,\ 
        1\otimes\alpha = \alpha$。
    \end{enumerate}
  \end{description}
  则称$V$是数域$\mfF$上的一个线性空间。
  若$\mfF$是实数域$\mfR$,则称$V$是\textbf{实线性空间}。
  若$\mfF$是复数域$\mfC$,则称$V$是\textbf{复线性空间}。
\end{definition}

\begin{remark}
  如果有点抽象代数的背景,读者不难发现上述加法满足的是阿贝尔群的性质。
\end{remark}

\begin{theorem}[线性空间的性质]
  设$V$是$\mfF$上的线性空间,那么
  \begin{enumerate}
    \item 加法零元的具有唯一性
    \item 加法逆元的具有唯一性
    \item 加法零元的求法:
    $\forall\alpha\in V,\ 0\otimes\alpha=\theta$
    \item 加法逆元的求法:
    $\forall\alpha\in V,\ (-1)\otimes\alpha=-\alpha$
    \item $\forall k\in\mfF,\ k\otimes\theta=\theta$
    \item 若$\lambda\otimes\alpha=\theta$,则$\lambda=0$或$\alpha=\theta$。
  \end{enumerate}
\end{theorem}
为了书写与阅读的方便,以后``$\oplus$''用正常的加号来表示,
``$\otimes$''可省略。参考向量的记法。

\subsection{基与坐标}
向量组的线性相关、线性无关、线性表示的概念也适用于线性空间。

\begin{definition}[线性相关与线性无关]
  设$V$是线性空间,$\alpha_1,\alpha_2,\dots,\alpha_s\in V$。
  若数域中存在一组不全为0的数$k_1,k_2,\dots,k_s$,使得
  \[ k_1\alpha_1 + k_2\alpha_2 + \dots + k_s\alpha_s = \theta \]
  则称$\alpha_1,\alpha_2,\dots,\alpha_s$\textbf{线性相关}。
  否则,则称\textbf{线性无关}。
\end{definition}

\begin{definition}[线性表示]
  设$V$是线性空间,$\alpha,\alpha_1,\alpha_2,\dots,\alpha_s\in V$。
  若数域中存在一组数$k_1,k_2,\dots,k_s$,使得
  \[ \alpha = k_1\alpha_1 + k_2\alpha_2 + \dots + k_s\alpha_s \]
  则称$\alpha$为$\alpha_1,\alpha_2,\dots,\alpha_s$的\textbf{线性组合},
  也称$\alpha$可以由$\alpha_1,\alpha_2,\dots,\alpha_s$\textbf{线性表示}。
\end{definition}

\begin{definition}[维数]
  如果一个线性空间$V$中,线性无关的元素的最大个数是$n$,
  则称该线性空间是$n$维的,记作$\mdim V = n$。
  
  如果对任意正整数$N$,总存在$N$个线性无关的元素,
  则称该线性空间是\textbf{无穷维线性空间}。
  不是无穷维的线性空间叫做有穷维线性空间。
\end{definition}

\begin{remark}
  本章中我们不讨论无穷维线性空间。
\end{remark}

\begin{definition}[基与坐标]
  若$\alpha_1,\alpha_2,\dots,\alpha_n$是$n$维线性空间$V$中一组线性无关的元素,
  且$V$中任意元素$\alpha$都可由$\alpha_1,\alpha_2,\dots,\alpha_n$线性表示,
  即\[ \alpha = k_1\alpha_1 + k_2\alpha_2 + \dots + k_n\alpha_n \]
  则称$\alpha_1,\alpha_2,\dots,\alpha_n$是$V$的一组\textbf{基底}(简称\textbf{基})。
  其中,有序元组$(k_1,k_2,\dots,k_n)$称为$\alpha$在
  基底$\alpha_1,\alpha_2,\dots,\alpha_n$下的坐标。
\end{definition}

