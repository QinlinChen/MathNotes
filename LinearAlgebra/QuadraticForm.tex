\chapter{二次型}

\section{二次型的概念}
本小节首先介绍实二次型和标准型的概念,其中最重要的是二次型的矩阵表示。
然后我们会看到,二次型能通过对变量进行线性变换转化为另一个二次型。
这两个二次型与表示矩阵的合同有着紧密的联系。

\subsection{二次型与标准型}
\begin{definition}[实二次型及其矩阵表示]
  设多项式
  \begin{equation} \label{eq-quadratic-form}
    f(x_1,x_2,\dots,x_n)=\sum_{i=1}^n\sum_{j=1}^n a_{ij}x_ix_j,
    \quad (a_{ij} = a_{ji})
  \end{equation}
  其中,$x_1,x_2,\dots,x_n$是变量,$a_{ij}$是实数,
  那么我们称式\ref{eq-quadratic-form}为\textbf{实二次型}。
  如果令$\mmA=(a_{ij})_{n\times n}$,$\mvx=(x_1,\dots,x_n)^\mT$,
  那么$\mmA$是实对称矩阵,且式\ref{eq-quadratic-form}可以用矩阵表示为
  \begin{displaymath}
    f(x_1,x_2,\dots,x_n) = \mvx^\mT\mmA\mvx
  \end{displaymath}
  我们把$\mmA$叫做$f$的\textbf{表示矩阵}。
\end{definition}

\begin{remark}
  下面主要研究的是实二次型。所以下面提到二次型的地方都是指的实二次型。
\end{remark}

\begin{definition}[标准型]
  设$\mvx=(x_1,\dots,x_n)^\mT,\mvy=(y_1,\dots,y_n)^\mT$为两组变量,
  矩阵$\mmP$建立了它们的线性变换关系:$\mvx=\mmP\mvy$。
  如果$\mmP$为满秩矩阵,那么我们称这种变换为\textbf{满秩线性变换}。
  如果二次型能通过满秩线性变换变成新变量的平方和,
  那么我们称新的多项式为该二次型的\textbf{标准型}。
\end{definition}

\subsection{表示矩阵的合同与二次型的秩}
如果给定一个二次型$\mvx^\mT\mmA\mvx$和满秩线性变换$\mvx=\mmP\mvy$,
那么我们能得到
\begin{displaymath}
\mvx^\mT\mmA\mvx
= \mvy^\mT\mmP^\mT\mmA\mmP\mvy
= \mvy^\mT\mmB\mvy
\end{displaymath}
其中$\mmB=\mmP^\mT\mmA\mmP$是二次型$\mvy^\mT\mmB\mvy$的表示矩阵。
由此可见,经过满秩线性变换联系的两个二次型的表示矩阵$\mmB$和$\mmA$是合同的。
除此以外,我们不难证明$r(\mmB)=r(\mmA)$。
据此,我们能定义二次型的秩:

\begin{definition}[二次型的秩]
  我们称二次型的表示矩阵的秩为\textbf{二次型的秩}。
\end{definition}

\section{化二次型为标准型}
下面我们要考虑的问题自然是:如何把二次型化为它的标准型?
如果我们想要把二次型$\mvx^\mT\mmA\mvx$化为标准型$\mvy^\mT\mmB\mvy$,
那么$\mmB$必须是对角矩阵。
所以,把二次型化为标准型的问题就转化为,
给定一个对称矩阵$\mmA$,如何找到一个与它合同的对角矩阵$\mmB$。

\subsection{实对称矩阵的对角化——正交变换法}
\begin{theorem}
  实对称矩阵的特征值都是实数。
\end{theorem}

\begin{theorem}
  实对称矩阵不同特征值的特征向量正交。
\end{theorem}

\begin{theorem}[实对称矩阵的可对角化]
  对任意实对称矩阵$\mmA$,都存在正交矩阵$\mmQ$,
  使得$\mmQ^{-1}\mmA\mmQ$为对角矩阵。
\end{theorem}

以上定理说明了,实对称矩阵是可对角化的,
所以实对称矩阵必有$n$个线性无关的特征向量。
又因为实对称矩阵的特征值之间的特征向量都是正交的,
所以我们只要给特征值内部的特征向量正交化,
就能得到$n$个正交的特征向量。
正交矩阵$Q$就是这$n$个特征向量单位化的组合。
下面是$n$阶实对称矩阵$\mmA$对角化的一般步骤:
\begin{enumerate}
  \item
  求$\mmA$的特征值;
  \item
  求$n$个线性无关的特征向量;
  \item 
  将每个特征值内的特征向量按照施密特的方法正交化;
  \item
  把所有特征向量单位化,就能组合得到$\mmQ$。
\end{enumerate}

\begin{remark}
  因为正交矩阵有性质$\mmQ^{-1}=\mmQ$,
  所以对于实对称矩阵而言,相似于对角矩阵和合同于对角矩阵是一回事——
  这也就是我们想把二次型化为标准型要做的事情。
  以上化标准型的方法叫做\textbf{正交变换法}。
\end{remark}

\begin{theorem}[主轴定理]
  任何实二次型都可以用正交变换化为标准型。
\end{theorem}

\subsection{配方法}
配方法的步骤如下:
\begin{enumerate}
  \item 
  若二次项含有某个变量的平方,那么先集中含有此变量的乘积项,然后配方。
  \item
  对剩下的变量依次应用同样的过程。
  \item
  最后二次项会化为变量线性组合的平方和。
\end{enumerate}

\section{规范型与惯性定理}
从二次型的标准型出发继续做满秩线性变换,能得到平方项的系数只能为$\pm 1$的规范型。
惯性定理刻画了二次型的本质特征:无论经过怎么样的满秩线性变换,
最后得到的规范型的正系数和负系数的个数都是相同的。

\subsection{规范型}
\begin{definition}[规范型]
  对于二次型,如果能通过满秩线性变换得到形如
  \begin{displaymath}
    f = z_1^2+\dots+z_p^2-z_{p+1}^2-\dots-z_{r}^2
  \end{displaymath}
  的多项式,则称该多项式为二次型的\textbf{规范型}。
\end{definition}

\subsection{惯性定理}
\begin{theorem}
  秩为$r$的$n$元实二次型$f=\mvx^\mT\mmA\mvx$
  可以通过满秩线性变换$\mvx=\mmP\mvy$化为规范型
  \begin{displaymath}
    f = y_1^2+\dots+y_p^2-y_{p+1}^2-\dots-y_{r}^2
    + 0\cdot y_{r+1}^2 + \dots + 0\cdot y_{n}^2
  \end{displaymath}
  其中$0\le p \le r \le n$。
\end{theorem}

\begin{theorem}[惯性定理]
  若对秩为$r$的$n$元实二次型$f=\mvx^\mT\mmA\mvx$
  做两个满秩线性变换$\mvx=\mmB\mvy$和$\mvx=\mmC\mvz$
  分别把它们化为规范型
  \begin{displaymath}
  f = y_1^2+\dots+y_p^2-y_{p+1}^2-\dots-y_{r}^2
  \end{displaymath}
  和
  \begin{displaymath}
  f = z_1^2+\dots+z_p^2-z_{q+1}^2-\dots-y_{r}^2
  \end{displaymath}
  那么,$p=q$。
\end{theorem}

\subsection{惯性指数}
\begin{definition}[正负惯性指数]
  在二次型$f$的一个标准型中,
  正项的个数叫做$f$的\textbf{正惯性指数},
  负项的个数叫做$f$的\textbf{负惯性指数}。
\end{definition}

\begin{remark}
  正负惯性指数之和即是二次型的秩。
\end{remark}

\begin{theorem}
  两个$n$元实二次型能够用满秩线性变换相互转化的充要条件是
  正负惯性指数分别相等。
\end{theorem}

\section{正定二次型}
本小节主要讨论一类特殊的实二次型——正定二次型,
并给出判定正定二次型的充要条件。

\subsection{正定二次型与正定矩阵}
\begin{definition}[正定二次型与正定矩阵]
  设$f=\mvx^\mT\mmA\mvx$为实二次型。
  如果对任意$\mvx\neq\mvZero$,有$\mvx^\mT\mmA\mvx > 0(<0)$,
  则称这个二次型为\textbf{正定二次型}(\textbf{负定二次型})。
  $\mmA$则被称为\textbf{正定矩阵}。
\end{definition}

\begin{theorem}[正定二次型的充要条件]
  $n$元实二次型是正定二次型的充要条件是它的正惯性指数为$n$。
\end{theorem}

\begin{definition}[顺序主子式]
  设$\mmA$为$n$阶矩阵,行列式
  \begin{displaymath}
    D_i=\mdet{cccc}{
      a_{11} & a_{12} & \cdots & a_{1i} \\
      a_{21} & a_{22} & \cdots & a_{2i} \\
      \vdots & \vdots & \ddots & \vdots \\
      a_{i1} & a_{i2} & \cdots & a_{ii} }
  \end{displaymath}
  称为$\mmA$的\textbf{$i$阶顺序主子式}($i=1,\dots,n$)。
\end{definition}

\begin{theorem}[正定矩阵的充要条件]
  设$\mmA$是$n$阶方阵。以下命题等价:
  \begin{enumerate}
    \item
    $\mmA$是正定矩阵。
    \item
    $\mmA$的所有特征值为正。
    \item
    $\mmA$合同于单位矩阵。($\mmA$能写成$\mmP^\mT\mmP$)
    \item
    $\mmA$的各阶顺序主子式都为正。
  \end{enumerate}
\end{theorem}