\chapter{矩阵\ 行列式\ 线性方程组}

\section{矩阵及其运算}
这一小节主要介绍了矩阵的基本概念和基本运算。

矩阵的概念对学过线性代数的人来说是稀松平常的了,
所以这里主要给出了重要概念和特殊矩阵的定义和符号,
方便后面的讨论。

矩阵的基本运算涉及了线性运算、乘积、转置。
大家应该对此都很熟悉,这里就简单给出了一些性质,并不作证明。
此外,我们还由转置运算给出了对称矩阵、反对称矩阵的概念。

\subsection{矩阵的概念}
\begin{definition}[矩阵]
  由$m\times n$个数排成$m$行$n$列的矩形数表
  \begin{equation} \label{eq:mat-def}
    \mmA = \mmat{cccc}{
      a_{11} & a_{12} & \cdots & a_{1n} \\
      a_{21} & a_{22} & \cdots & a_{2n} \\
      \vdots & \vdots & \ddots & \vdots \\
      a_{m1} & a_{m2} & \cdots & a_{mn} }
  \end{equation}
  称为一个$m\times n$矩阵。
  其中数$a_{ij}$称为矩阵$\mmA$的元素。
  $i$为行标,$j$为列标,
  $a_{ij}$是$\mmA$位于第$i$行第$j$列的元素,或简称$\mmA$的$(i,j)$元素。
  式\eqref{eq:mat-def}可简记为$\mmA=(a_{ij})_{m\times n}$或$(a_{ij})$。
\end{definition}

下面是一些特殊的矩阵:
\begin{itemize}
  \item
  $m\times 1$矩阵又称为\textbf{列矩阵}或$m$\textbf{维列向量},
  下文主要会用$\beta$来表示。
  \item
  $1\times n$矩阵又称为\textbf{行矩阵}或$n$\textbf{维行向量},
  下文主要会用$\alpha$来表示。
  \item
  $n\times n$矩阵又称为$n$\textbf{阶方阵}。
  对于$n$阶方阵$\mmA=(a_{ij})$,
  其对角线上的元素$a_{11},\dots,a_{nn}$又称\textbf{主对角线元素}。
  \item
  如果一个$n$阶方阵除主对角线元素外,其余元素都为0,
  那么我们称这种矩阵为\textbf{对角矩阵},记为
  \[
    \mdiag(k_1, k_2\dots,k_n)= \mmat{cccc}{
      k_1 &     &        & \\
          & k_2 &        & \\
          &     & \ddots & \\
          &     &        & k_n }
  \]
  \item
  $\mmI = \mdiag(1,1,\dots,1)$被称为\textbf{单位矩阵}。
  \item
  \textbf{零矩阵}$\mmZero$是所有元素都为0的矩阵。
  \item
  设$\mmA=(a_{ij})$是$n$阶方阵。
  若当$i>j$时,$a_{ij}=0$,则称$\mmA$为\textbf{上三角阵}。
  若当$i<j$时,$a_{ij}=0$,则称$\mmA$为\textbf{下三角阵}。
  两者统称\textbf{三角阵}。
\end{itemize}

\begin{definition}[矩阵相等]
    如果$\mmA=(a_{ij})$与$\mmB=(b_{ij})$都是$m\times n$矩阵,
    且对$i=1,2,\dots,m,\ j=1,2,\dots,n$,有$a_{ij}=b_{ij}$,
    则称$\mmA$与$\mmB$相等,记作$\mmA=\mmB$。
\end{definition}

\subsection{矩阵的线性运算}
矩阵的线性运算包含\textbf{加法}和\textbf{数乘}。定义比较简单,在此不加赘述。

它们满足的性质与之后要定义的线性空间一致。

\subsection{矩阵的乘法}
\begin{definition}[矩阵的乘法]
  设$\mmA=(a_{ik})_{m\times s}, \mmB=(b_{kj})_{s\times n}$,令
  \begin{equation}
  c_{ij}=\sum_{k=1}^{s}a_{ik}b_{kj}
  \end{equation}
  则$\mmC=(c_{ij})_{m\times n}$被称为$\mmA$和$\mmB$的乘积,
  记为$\mmC = \mmA\mmB$。
\end{definition}

矩阵的乘法满足分配率和结合律,但不满足交换律。

\subsection{矩阵的转置}
\begin{definition}[矩阵的转置]
    设$\mmA=(a_{ij})_{m\times n}$,
    令$b_{ij}=a_{ji}$,
    则$\mmB=(b_{ij})_{n\times m}$被称为$\mmA$的转置矩阵,
    记为$\mmA^T$。
\end{definition}

\begin{theorem}[转置矩阵的性质]
  转置矩阵具有以下性质:
  \begin{enumerate}
    \item $\left(\mmA^T\right)^T = \mmA$
    \item $(\mmA+\mmB)^T=\mmA^T + \mmB^T$
    \item $(k\mmA)^T=k\mmA^T$
    \item $(\mmA\mmB)^T=\mmB^T\mmA^T$
  \end{enumerate}
\end{theorem}

\begin{definition}[对称矩阵与反对称矩阵]
  对于$n$阶方阵$\mmA=(a_{ij})$,
  如果满足$\mmA^T=\mmA$,则称$\mmA$为\textbf{对称矩阵}。
  如果满足$\mmA^T=-\mmA$,则称$\mmA$为\textbf{反对称矩阵}。
\end{definition}

\begin{remark}
  不难看出,反对称矩阵的主对角线元素都为零。
\end{remark}

\section{矩阵的分块}
矩阵的分块是化简矩阵运算简单而又重要的思想。
其中最重要的是分块矩阵的乘法。

\subsection{分块矩阵的概念}
\begin{definition}[分块矩阵]
  一般地,将一个$m\times n$矩阵$\mmA$用横线划分成$r$块,
  用竖线划分成$s$块,就能得到一个$r\times s$分块矩阵。
  \begin{equation} \label{eq:mat-partition}
  \mmA = \mmat{cccc}{
    \mmA_{11} & \mmA_{12} & \cdots & \mmA_{1s} \\
    \mmA_{21} & \mmA_{22} & \cdots & \mmA_{2s} \\
    \vdots    & \vdots    & \ddots & \vdots    \\
    \mmA_{r1} & \mmA_{r2} & \cdots & \mmA_{rs} }
    = (\mmA_{ij})_{r\times s}
  \end{equation}
  其中,$\mmA_{ij}(i=1,\dots,r,\ j=1,\dots,s)$是$m_i\times n_j$矩阵,
  $\sum_{i=1}^{r}m_i = m$,$\sum_{j=1}^{s}n_j=n$。
\end{definition}

最常用的一种矩阵分块方法是把$m\times n$划分成$m$个行向量,
或者$n$个列向量。

\subsection{分块矩阵的运算}
分块矩阵的运算中需要注意的是转置和乘法。

转置运算不仅需要把矩阵的每一行转成列,而且内部子块也需要转置。
对于式\ref{eq:mat-partition}中的矩阵$\mmA$,它的转置是
\[
  \mmA^T = \mmat{cccc}{
    \mmA_{11}^T & \mmA_{21}^T & \cdots & \mmA_{r1}^T \\
    \mmA_{12}^T & \mmA_{22}^T & \cdots & \mmA_{r2}^T \\
    \vdots      & \vdots      & \ddots & \vdots      \\
    \mmA_{1s}^T & \mmA_{2s}^T & \cdots & \mmA_{sr}^T }
\]

对于乘法运算,
给定两个矩阵$\mmA=(a_{ij})_{m\times n}$,$\mmB=(b_{jk})_{n\times p}$,
只要使$\mmB$的行分法与$\mmA$的列分法一致,
就能把子块当作数一样按照矩阵乘法的规则进行计算。

\subsection{准对角矩阵}
\begin{definition}[准对角矩阵] 
  对于$n$阶方阵$\mmA$,
  如果有一种分法,使$\mmA$的主对角线以外的子块都是零矩阵,
  且主对角线上子块都是方阵,则称$\mmA$为准对角矩阵。
\end{definition}

\begin{remark}
  当然,准对角矩阵包含对角矩阵作为特殊情况。
\end{remark}


设$\mmA$和$\mmB$都是$n$阶方阵。
如果有相同的分法,使得
\[
\mmA = \mmat{cccc}{
  \mmA_1 &        &        & \\
         & \mmA_2 &        & \\
         &        & \ddots & \\
         &        &        & \mmA_n },\quad
\mmB = \mmat{cccc}{
  \mmB_1 &        &        & \\
         & \mmB_2 &        & \\
         &        & \ddots & \\
         &        &        & \mmB_n }
\]
都是准对角矩阵,那么显然有
\[
\mmA\mmB = \mmat{cccc}{
    \mmA_1\mmB_1 &              &        & \\
                 & \mmA_2\mmB_2 &        & \\
                 &              & \ddots & \\
                 &              &        & \mmA_n\mmB_n }
\]

\section{逆矩阵与初等变换}
本小节开始讨论逆矩阵,
并试图通过初等变换的概念,从更本质的视角来看待矩阵
——矩阵就是所谓标准型乘上有限个初等阵。
也因为初等阵具有良好的性质——可逆性,
所以矩阵可逆当且仅当它的标准型就是单位矩阵。

\subsection{逆矩阵}
\begin{definition}[逆矩阵]
  设$\mmA$是$n$阶方阵。如果存在$n$阶方阵$\mmB$,使
  \begin{equation}
    AB=BA=I
  \end{equation}
  则称$\mmB$为$\mmA$的\textbf{逆矩阵},记作$A^{-1}$,
  并说$\mmA$是\textbf{非奇异矩阵}或\textbf{可逆矩阵},
  否则便称$\mmA$是\textbf{奇异矩阵}或\textbf{不可逆矩阵}。
\end{definition}

\begin{theorem}[逆矩阵的唯一性]
  可逆矩阵的逆矩阵是唯一的。
\end{theorem}

\begin{theorem}[可逆矩阵的性质]
  若$\mmA,\mmB$是可逆矩阵,
  则$\mmA^{-1}$,$k\mmA\,(k\neq 0)$,$\mmA^T$,$\mmA\mmB$都是可逆矩阵,且
  \begin{enumerate}
    \item $(\mmA^{-1})^{-1} = \mmA$
    \item $(k\mmA)^{-1} = \frac{1}{k}\mmA^{-1}$
    \item $(\mmA^T)^{-1} = (\mmA^{-1})^T$
    \item $(\mmA\mmB)^{-1} = \mmB^{-1}\mmA^{-1}$
  \end{enumerate}
\end{theorem}

\subsection{初等变换与初等阵}
矩阵的\textbf{初等变换}可以分为\textbf{初等行变换}和\textbf{初等列变换}。
下面主要讨论初等行变换。初等列变换与之相似。

\begin{definition}[初等行变换]
  初等行变换有三种:
  \begin{enumerate}
    \item 交换矩阵的第$i$行和第$j$行,记为$\mTfRowSwi{i}{j}$。
    \item 将矩阵的第$i$行乘非零常数$k$,记为$\mTfRowMul{k}{j}$。
    \item 把矩阵的第$i$行加上第$j$行的$k$倍,记为$\mTfRowAdd{i}{+k}{j}$。
  \end{enumerate}
\end{definition}

\begin{definition}[初等阵]
  单位矩阵经过一次初等变换得到的矩阵叫做\textbf{初等矩阵},
  或简称\textbf{初等阵}。
\end{definition}

\begin{remark}
  显然,初等阵是方阵,并且都是可逆的。
\end{remark}

使用三种初等行变换能得到三个初等阵,
我们把它们分别记作
$\mmRowSwi{i}{j}$,$\mmRowMul{k}{i}$和$\mmRowAdd{i}{+k}{j}$。
对单位矩阵实行一次初等列变换,同样能得到上述三种形式的初等阵,
所以初等阵只有以上三种。

\begin{theorem}[一般矩阵的初等变换与初等阵的联系]
  对于一个$m\times n$矩阵$\mmA$做一次初等行变换,
  就相当于对$\mmA$左乘一个$m$阶初等矩阵;
  对$\mmA$做一次初等列变换,
  就相当于对$\mmA$右乘一个$n$阶初等矩阵。
\end{theorem}

\begin{definition}[矩阵等价]
  如果矩阵$\mmA$可以通过有限次初等变换化为矩阵$\mmB$,
  那么就说$\mmA$与$\mmB$等价。
\end{definition}

\begin{remark}
  矩阵等价是一个等价关系,即满足自反性、对称性、传递性。
\end{remark}

\begin{theorem}[标准型]
  任意非零$m\times n$矩阵$\mmA$都等价于矩阵
  \[
    \mmat{cc}{\mmI_r & \mmZero \\ \mmZero & \mmZero}
    \quad (1\le r \le \min(m,n))
  \]
  它称为矩阵$\mmA$的\textbf{标准型}。
  换句话说,任意非零$m\times n$矩阵$\mmA$,
  必有初等矩阵$\mmP_1,\dots,\mmP_s$,$\mmQ_1,\dots,\mmQ_t$,
  使得
  \[
    \mmP_s\dots\mmP_1\cdot\mmA\cdot\mmQ_1\dots\mmQ_t = 
      \mmat{cc}{\mmI_r & \mmZero \\ \mmZero & \mmZero}
  \]
\end{theorem}

因为初等矩阵可逆,所以有如下推论

\begin{corollary}
  对于任意非零$m\times n$矩阵$\mmA$,
  存在可逆$m\times m$矩阵$\mmP$和可逆$n\times n$矩阵$\mmQ$,
  使得
  \[
    \mmP\mmA\mmQ = \mmat{cc}{\mmI_r & \mmZero \\ \mmZero & \mmZero}
    \quad (1\le r \le \min(m,n))
  \]
  或写成
  \[
    \mmA = \mmP^{-1}\mmat{cc}{\mmI_r & \mmZero \\ \mmZero & \mmZero}\mmQ^{-1}
    \quad (1\le r \le \min(m,n))
  \]
\end{corollary}

\subsection{矩阵可逆的充要条件}

\begin{theorem}
  设$\mmA,\mmB$是$n$阶方阵,
  若$\mmA\mmB=\mmI_n$,
  则$\mmA,\mmB$都是可逆阵,且它们互为逆阵。
\end{theorem}

\begin{theorem} \label{thrm-inv-equiv-cond}
  设$\mmA$是$n$阶方阵,则下列结论等价:
  \begin{enumerate}
    \item $\mmA$是可逆矩阵。
    \item $\mmA$可以表示为有限个初等阵的乘积。
    \item $\mmA$可以经过有限次初等变换化为单位矩阵。
  \end{enumerate}
\end{theorem}

\subsection{用初等变换求逆矩阵}
\[
  \mmat{c|c}{\mmA & \mmI} \xrightarrow{\text{初等行变换}}
  \mmat{c|c}{\mmI & \mmA^{-1}}
\]

\section{行列式}
本小节开始讨论行列式的概念。

首先我们给出了行列式的定义。
但是,如果我们根据行列式的定义来计算行列式,除了一些特殊矩阵,
用计算科学的说法,大部分$n$阶行列式的计算复杂度是$O(n!)$,
因此我们需要探索更高效的方法来计算行列式。

接下来,我们给出了行列式的一些基本性质。
有了这些性质,我们得到了新的行列式计算方法,
而且稍加分析,就能惊喜地看出这个方法的时间复杂度是$O(n^3)$。

\subsection{$n$阶行列式}
\begin{definition}[$n$级排列]
  $n$个自然数按任意固定的顺序构成的一个排列称为$n$\textbf{级排列}。
  所有$n$级排列构成的集合记作$\pi_n$。
\end{definition}

\begin{definition}[逆序数]
  在一个$n$级排列$(i_1,i_2,\dots,i_n)$中,
  如果$i_r>i_s$,但是$r<s$,即前面的数大于后面的数,
  就称这两个数构成\textbf{逆序对}。
  一个排列中逆序对的个数叫做这个排列的\textbf{逆序数},
  记作$\tau(i_1,i_2,\dots,i_n)$。
\end{definition}

\begin{definition}[奇、偶排列]
  逆序数为奇数的排列叫做\textbf{奇排列};
  逆序数为偶数的排列叫做\textbf{偶排列}。
\end{definition}

\begin{definition}[$n$阶行列式]
  设$\mmA=(a_{ij})_{n\times n}$是$n$阶方阵,它的$n$阶行列式为
  \begin{align*}
    |\mmA| &= \sum_{(j_1,j_2,\dots,j_n)\in\pi_n}
    (-1)^{\tau(j_1,j_2,\dots,j_n)}a_{1j_1}a_{2j_2}\dots a_{nj_n} \\
    &= \sum_{(i_1,i_2,\dots,i_n)\in\pi_n}
    (-1)^{\tau(i_1,i_2,\dots,i_n)}a_{i_1 1}a_{i_2 2}\dots a_{i_n n} 
  \end{align*}
\end{definition}

特殊矩阵的行列式:
\begin{enumerate}
  \item $|\mdiag(d_1,d_2,\dots,d_n)| = d_1d_2\dots d_n$
  \item 三角阵的行列式为对角线之积
\end{enumerate}

\subsection{行列式的性质}
\begin{theorem}
  若$\mmA$为方阵,那么$|\mmA^T| = |\mmA|$
\end{theorem}

\begin{theorem}
  \begin{align*}
    &\quad\mdet{cccc}{
      a_{11} & a_{12} & \cdots & a_{1n} \\
      \vdots &        &        &        \\
      b_{i1} + c_{i1} & b_{i2} + c_{i2}  & \cdots & b_{in} + c_{in} \\
      \vdots &        &        &        \\
      a_{n1} & a_{n2} & \cdots & a_{nn} } \\
    &= \mdet{cccc}{
      a_{11} & a_{12} & \cdots & a_{1n} \\
      \vdots &        &        &        \\
      b_{i1} & b_{i2} & \cdots & b_{in} \\
      \vdots &        &        &        \\
      a_{n1} & a_{n2} & \cdots & a_{nn} }
    + \mdet{cccc}{
      a_{11} & a_{12} & \cdots & a_{1n} \\
      \vdots &        &        &        \\
      c_{i1} & c_{i2} & \cdots & c_{in} \\
      \vdots &        &        &        \\
      a_{n1} & a_{n2} & \cdots & a_{nn} }
    \end{align*}
\end{theorem}

\begin{theorem}
  设$\mmA$为$n$阶方阵。
  若$\mmA$有两行元素相等或对应成比例,
  那么$|\mmA| = 0$。
\end{theorem}

\begin{theorem}
  \begin{displaymath}
    \mdet{cccc}{
      a_{11} & a_{12} & \cdots & a_{1n} \\
      \vdots &        &        &        \\
      \lambda a_{i1} & \lambda a_{i2} & \cdots & \lambda a_{in} \\
      \vdots &        &        &        \\
      a_{n1} & a_{n2} & \cdots & a_{nn} }
    = \lambda \mdet{cccc}{
      a_{11} & a_{12} & \cdots & a_{1n} \\
      \vdots &        &        &        \\
      a_{i1} & a_{i2} & \cdots & a_{in} \\
      \vdots &        &        &        \\
      a_{n1} & a_{n2} & \cdots & a_{nn} }
  \end{displaymath}
\end{theorem}

\begin{theorem} \label{thrm-det-rowswi}
  交换方阵$\mmA$的两行,仅改变$\mmA$行列式的符号,
  即\[ |\mmRowSwi{i}{j}\mmA| = -|\mmA| \]
\end{theorem}

\begin{theorem} \label{thrm-det-rowadd}
  对方阵$\mmA$做$\mTfRowAdd{i}{+k}{j}$变换,不改变$\mmA$的行列式,
  即\[ |\mmRowAdd{i}{+k}{j}\mmA| = |\mmA| \]
\end{theorem}

\subsection{用行列式的性质求行列式}
使用行列式的性质把行列式转化成上三角矩阵的行列式。
后者的行列式即为对角线之积。

\section{行列式按行(列)展开}
本小节讨论另一种行列式的计算方法。

\subsection{行列式按一行(列)展开}
\begin{definition}[余子式]
  在$n$阶行列式$|\mmA|=|a_{ij}|_{n\times n}$中划去第$i$行第$j$列后
  所剩下的$(n-1)^2$个元素按照原来的相对位置排成的$n-1$阶子式$M_{ij}$
  叫做元素$a_{ij}$在$\mmA$中的\textbf{余子式}。
  而\[ A_{ij} = (-1)^{i+j}M_{ij} \]
  叫做元素$a_{ij}$在$\mmA$中的\textbf{代数余子式}。
  这里$1\le i, j \le n$。
\end{definition}

\begin{theorem}[按行(列)展开] \label{thrm-det-expansion}
  $n$阶行列式$|\mmA|=|a_{ij}|_{n\times n}$
  等于任一行(列)各元素与其代数余子式的乘积之和,即
  \begin{align*}
  |\mmA| &= a_{i1}A_{i1}+a_{i2}A_{i2}+\dots+a_{in}A_{in}\quad(i=1,2,\dots,n) \\
  |\mmA| &= a_{1j}A_{1j}+a_{2j}A_{2j}+\dots+a_{nj}A_{nj}\quad(j=1,2,\dots,n)
  \end{align*}
\end{theorem}

\begin{theorem} \label{thrm-det-expansion-zero}
  $n$阶行列式$|\mmA|=|a_{ij}|_{n\times n}$
  的任一行(列)元素与另一行(列)元素的代数余子式乘积之和等于零,即
  \begin{align*}
    a_{i1}A_{j1}+a_{i2}A_{j2}+\dots+a_{in}A_{jn} &= 0\quad(i\neq j) \\
    a_{1i}A_{1j}+a_{2i}A_{2j}+\dots+a_{ni}A_{nj} &= 0\quad(i\neq j)
  \end{align*}
\end{theorem}

\begin{remark}
  定理\ref{thrm-det-expansion}和\ref{thrm-det-expansion-zero}
  可以统一表示为(仅列出行的情况)
  \begin{displaymath}
  \sum_{k=1}^{n}a_{ik}A_{jk} = \begin{cases}
    |\mmA| & (i=j) \\
    0      & (i\neq j)
  \end{cases}
  \end{displaymath}
\end{remark}

\subsection{范德蒙(Vandermonde)行列式}
\begin{displaymath}
  V_n = \mdet{ccccc}{
    1         & 1         & 1         & \cdots & 1      \\
    x_1       & x_2       & x_3       & \cdots & x_n    \\
    x_1^2     & x_2^2     & x_3^2     & \cdots & x_n^2  \\
    \vdots    & \vdots    & \vdots    & \ddots & \vdots \\
    x_1^{n-1} & x_2^{n-1} & x_3^{n-1} & \cdots & x_n^{n-1} 
  } = \prod_{1\le i<j\le n}(x_j - x_i)
\end{displaymath}

\section{用行列式求逆阵\ 克莱姆法则}
本小节通过引入伴随矩阵这个中间媒介,
来探索方阵可逆与行列式之间的关系,
从而得到方阵可逆的充要条件,以及通过行列式求逆阵方法。

有了上述方法,我们就能利用行列式来解线性方程组,
并总结出了克莱姆法则。但克莱姆法则也有其局限性。
后面会讲高斯消元来解一般线性方程组。

\subsection{用行列式求逆矩阵}
\begin{theorem}[行列式乘法规则] \label{thrm-det-mul}
  对任意$n$阶方阵$\mmA,\mmB$,有
  \[ |\mmA\mmB|=|\mmA||\mmB| \]
\end{theorem}

\begin{remark}
  证明思路是,首先证明引理:对任意$n$阶方阵$\mmA$和$n$阶初等阵$\mmP$,
  有$|\mmA\mmP|=|\mmP\mmA|=|\mmP||\mmA|$。
  然后利用定理\ref{thrm-inv-equiv-cond}对$\mmB$讨论:
  如果$\mmB$可逆,就能拆成一系列初等阵之积,从而利用引理得证;
  如果$\mmB$不可逆,只要证明等式两边都等于0。
\end{remark}

\begin{theorem}[行列式数乘规则] \label{thrm-det-num-mul}
  对任意$n$阶方阵$\mmA$和$\lambda\in\mfR$,有
  \[ |\lambda\mmA| = \lambda^n|\mmA| \]
\end{theorem}

\begin{definition}[伴随矩阵]
  $n$阶方阵$\mmA$的\textbf{伴随矩阵}为
  \begin{displaymath}
    \mmA^* = \mmat{cccc}{
      A_{11} & A_{12} & \cdots & A_{1n} \\
      A_{21} & A_{22} & \cdots & A_{2n} \\
      \vdots & \vdots & \ddots & \vdots \\
      A_{m1} & A_{m2} & \cdots & A_{mn} }^T
  \end{displaymath}
\end{definition}

\begin{theorem}[伴随矩阵的性质] \label{thrm-adjugate-mat-prop}
  设$\mmA$是$n$阶方阵,那么有
  \[ \mmA\mmA^*=\mmA^*\mmA=|\mmA|\mmI \]
\end{theorem}

\begin{remark}
  伴随矩阵就是为了这个性质而定义的。
\end{remark}

\begin{theorem}[伴随矩阵的行列式]
  设$\mmA$是$n$阶方阵,那么有
  \[ |\mmA^*| = |\mmA|^{n-1} \]
\end{theorem}

\begin{remark}
  该式对$|\mmA|$取任意值都成立。
  当$|\mmA|\neq 0$时,
  对伴随矩阵的性质\ref{thrm-adjugate-mat-prop}
  应用行列式乘法规则\ref{thrm-det-mul}
  以及数乘规则\ref{thrm-det-num-mul}即可证明该式。
  当$|\mmA|=0$时,
  则直接通过定义来证明。
\end{remark}

\begin{theorem}[方阵可逆的充要条件]
  设$\mmA$是$n$阶方阵,那么
  \[ \mmA\ \text{可逆} \iff |\mmA| \neq 0 \]
  并且对于可逆矩阵$\mmA$有
  \begin{enumerate}
    \item $\mmA^{-1} = \mmA^*/|\mmA|$
    \item $(\mmA^*)^{-1} = \mmA/|\mmA|$
  \end{enumerate}
\end{theorem}

\subsection{线性代数方程组与克莱姆(Cramer)法则}
$n$元方程组的矩阵形式为
\begin{equation} \label{eq-linear-eq-set}
  \mmA \mvx = \mvb
\end{equation}
其中$\mmA = (a_{ij})_{n\times n}$是\textbf{系数矩阵},
$\overline{\mmA}=(\mmA, \mvb)$是\textbf{增广矩阵},
$\mvx = (x_1,\dots,x_n)^T$与$\mvb = (b_1,\dots,b_n)^T$是$n$维列向量,
分别称为\textbf{未知向量}与\textbf{常数向量}。
若$\mvx$的一组取值$\hat{\mvx}$能满足\ref{eq-linear-eq-set},
则称$\hat{\mvx}$为\ref{eq-linear-eq-set}的一个\textbf{解向量}。

若常数向量不为$\mvZero$,则称\ref{eq-linear-eq-set}为\textbf{非齐次线性方程组},
否则,叫做\textbf{齐次线性方程组}。

\begin{theorem}[克莱姆(Cramer)法则]
  若$n$元线性代数方程组的系数矩阵$\mmA$的行列式$|\mmA|\neq 0$,
  则方程组有唯一解$\mvx = (x_1,\dots,x_n)$,其中
  \begin{displaymath}
    x_j = \frac{D_j}{|\mmA|} \quad (j=1,\dots,n)
  \end{displaymath}
  $D_j$是将$|\mmA|$的第$j$列换成$\mvb$所得的行列式。
\end{theorem}

\begin{corollary}
  对于$n$元齐次线性方程组$\mmA\mvx=\mvZero$,
  若$|\mmA|\neq 0$,则只有零解;
  若$|\mmA| = 0$,则有无穷多非零解。
\end{corollary}

\begin{remark}
  值得注意的是,克莱姆法则只适用于方程个数等于未知量个数的方程组,
  而且系数行列式不能为零。
  此外,克莱姆法则的计算复杂度也更高,为$O(n^4)$,
  不如后面要介绍的高斯消元法$O(n^3)$的复杂度。
\end{remark}

\section{向量组的线性无关}
在一个方程组中,会有一些方程是``无用的'',
也就是说,它能由其它方程通过线性运算表示出来。
向量组的线性无关与此概念紧密联系——向量组即是一个方程的系数矩阵。
向量组的极大线性无关组也就反应了在方程组中去掉那些无用的方程后得到的方程组。
向量组的秩即是这些有用的方程的个数。

\subsection{线性相关与线性无关}
\begin{definition}[线性表示]
  设$\alpha,\alpha_1,\alpha_2,\dots,\alpha_r$是一组$n$维向量。
  如果存在数$k_1,k_2,\dots,k_r$,使得
  \[ \alpha = k_1\alpha_1+k_2\alpha_2+\dots k_r\alpha_r \]
  则称$\alpha$是$\alpha_1,\alpha_2,\dots,\alpha_r$的\textbf{线性组合},
  或者说$\alpha$可由$\alpha_1,\alpha_2,\dots,\alpha_r$\textbf{线性表示},
  其中,$k_1,k_2,\dots,k_r$叫做\textbf{表示系数}。
\end{definition}

\begin{definition}[线性相关与线性无关]
  设$\alpha_1,\alpha_2,\dots,\alpha_r$是一组$n$维向量。
  如果存在一组\textbf{不全为零}的数$k_1,k_2,\dots,k_r$,使得
  \[ k_1\alpha_1+k_2\alpha_2+\dots k_r\alpha_r = \mvZero \]
  则称向量组$\alpha_1,\alpha_2,\dots,\alpha_r$\textbf{线性相关};
  否则,称\textbf{线性无关}。
\end{definition}

\begin{theorem}[线性相关与线性组合]
    $\alpha_1,\alpha_2,\dots,\alpha_r\ (r\ge 2)$线性相关
    当且仅当至少有一个向量是其它向量的线性组合。
\end{theorem}

\begin{theorem}[线性相关与线性无关的等价条件]
  设$\alpha_1,\alpha_2,\dots,\alpha_r$是一组$n$维列向量,
  $\mvk = (k_1,k_2,\dots,k_r)^T$,
  矩阵$\mmA = (\alpha_1,\alpha_2,\dots,\alpha_r)$,那么
  \begin{align*}
  \alpha_1,\alpha_2,\dots,\alpha_r\ \text{线性相关}
  &\iff \mmA\mvk = \mvZero\ \text{有非零解} \\
  &\iff |\mmA|=0 
  \end{align*}
  \begin{align*}
  \alpha_1,\alpha_2,\dots,\alpha_r\ \text{线性无关}
  &\iff \mmA\mvk = \mvZero\ \text{仅有零解} \\
  &\iff |\mmA|\neq 0 
  \end{align*}
\end{theorem}

\subsection{向量组组内关系}
\begin{theorem}[接长与补短]
  设向量组$S_1: \alpha_1,\alpha_2,\dots,\alpha_r$。
  若在每个向量中添加一个分量,把它变成$n+1$维向量组$S_2$,
  那么$S_1$线性无关能推出$S_2$线性无关,
  $S_2$线性相关能推出$S_1$线性相关。
\end{theorem}

\begin{theorem}[部分与整体]
  设向量组$S_1: \alpha_1,\alpha_2,\dots,\alpha_r$
  与向量组$S_2: \alpha_1,\alpha_2,\dots,\alpha_r,\alpha_{r+1},\dots,\alpha_s$
  是两个$n$维向量组。
  若$S_1$线性相关,则$S_2$线性相关;
  反之,若$S_2$线性无关,则$S_1$线性无关。
\end{theorem}

\begin{remark}
  直白地说,就是部分相关可以推出整体相关,整体无关可以推出部分无关。
\end{remark}

\begin{theorem}
  任意$n+1$个$n$维向量一定线性相关。
\end{theorem}

\subsection{向量组组间关系}
\begin{definition}[向量组等价]
  设向量组
  \[ S_1: \alpha_1,\alpha_2,\dots,\alpha_r \]
  \[ S_2: \beta_1,\beta_2,\dots,\beta_s \]
  若$S_1$中每个向量都可以被向量组$S_2$线性表示,
  则称向量组$S_1$可被$S_2$\textbf{线性表出}。
  若$S_1$与$S_2$能互相线性表出,那么称$S_1$和$S_2$\textbf{等价}。
\end{definition}

\begin{remark}
  向量组的等价是等价关系。
\end{remark}

\begin{theorem} \label{thrm-vector-set-size}
  设有两个向量组$S_1$和$S_2$,分别含有$r$和$s$个向量。
  若$S_1$能由$S_2$线性表出,且$r > s$,那么$S_1$线性相关。
  反之,若$S_1$线性无关且能由$S_2$线性表出,那么$r \le s$。
\end{theorem}

\begin{corollary} \label{thrm-vector-set-equiv}
  若两个线性无关的向量组$S_1$和$S_2$等价,那么它们包含的向量个数相同。
\end{corollary}

\subsection{极大线性无关组与向量组的秩}
\begin{definition}
  若一个向量组中有部分向量$\alpha_1,\alpha_2,\dots,\alpha_s$具有下面两个性质:
  \begin{enumerate}
    \item
    $\alpha_1,\alpha_2,\dots,\alpha_s$线性无关;
    \item
    从原向量组中任选一个新向量(如果还有的话)加入到这个向量组中,
    所得的部分向量组就线性相关了。
  \end{enumerate}
  那么称$\alpha_1,\alpha_2,\dots,\alpha_s$为原向量组的\textbf{极大线性无关组}。
\end{definition}

\begin{theorem} \label{thrm-vector-set-self-equiv}
   向量组的任意一个极大线性无关组都与向量组本身等价。
\end{theorem}

\begin{remark}
  一个向量组的极大线性无关组不一定是唯一的,
  但是根据定理\ref{thrm-vector-set-self-equiv}
  和推论\ref{thrm-vector-set-equiv}可以证明,
  它们的大小一定是相同的。
\end{remark}

\begin{definition}[向量组的秩]
  向量组$S$的极大线性无关组所含向量的个数称为这个向量组的\textbf{秩},
  记作$r(S)$。
\end{definition}

有的秩的概念,定理\ref{thrm-vector-set-size}
和推论\ref{thrm-vector-set-equiv}可以做出如下推广:
\begin{theorem}
  设有向量组$S_1$和$S_2$。
  \begin{enumerate}
    \item
    若$S_1$能由$S_2$线性表出,那么$r(S_1) \le r(S_2)$。
    \item
    若$S_1$与$S_2$等价,那么$r(S_1)=r(S_2)$。
  \end{enumerate}
\end{theorem}

\section{矩阵的秩}
矩阵如果按行划分,或者按列划分,其实都能看成一个向量组。
既然向量组有秩,那么矩阵的秩也可以因此定义出来。
我们会发现,不管是按行划分,还是按列划分,
行向量组与列向量组的秩都是相同的,
这个同一的值就是矩阵的秩。

\subsection{矩阵的行秩、列秩和秩}
\begin{definition}[行秩与列秩]
  一个矩阵$\mmA=(a_{ij})_{m\times n}$的行向量组的秩叫做$\mmA$的\textbf{行秩},
  列向量组的秩叫做$\mmA$的\textbf{列秩}。
\end{definition}

\begin{theorem}[行秩与列秩的不变性]
  一个矩阵的行秩和列秩在初等变换下保持不变。
\end{theorem}

\begin{remark}
  因为任何矩阵都能通过初等变换化为标准型,标准型的行秩与列秩相同,
  所以更进一步的结论呼之欲出。
\end{remark}

\begin{theorem}[行秩与列秩相等]
  一个矩阵的行秩等于列秩。
\end{theorem}

\begin{remark}
  有了这个定理,我们就能把行秩和列秩统称为秩。
\end{remark}

\begin{theorem}[秩]
  矩阵$\mmA$的行秩或列秩称为这个矩阵的\textbf{秩},记作$r(\mmA)$。
\end{theorem}

\subsection{矩阵的秩的性质}
\begin{theorem}
  设$\mmA=(a_{ij})_{m\times s}, \mmB=(b_{jk})_{s\times p}$,则
  \[ r(\mmA\mmB) \le \min\{ r(\mmA), r(\mmB) \} \]
\end{theorem}

\begin{remark}
  根据需要也可以写成$r(\mmA\mmB) \le r(\mmA)$,$r(\mmA\mmB)\le r(\mmB)$。
\end{remark}

\begin{theorem}
  设$\mmA$是$m\times n$阶矩阵,
  $\mmP$是$m$阶可逆方阵,$\mmQ$是$n$阶可逆方阵,则
  \[ r(\mmA) = r(\mmP\mmA) = r(\mmA\mmQ) \]
\end{theorem}

\begin{theorem}
  设$\mmA$和$\mmB$都是$m\times n$阶矩阵,则
  \[ r(\mmA+\mmB) \le r(\mmA) + r(\mmB) \]
\end{theorem}

\section{线性代数方程组}
本小节研究高斯消元解线性代数方程组。

\subsection{高斯消元}
对于线性代数方程组$\mmA\mvx=\mvb$,它的增广矩阵$\overline{\mmA}$是
\begin{displaymath}
  \mmat{cccc|c}{
    a_{11} & a_{12} & \cdots & a_{1n} & b_1 \\
    a_{21} & a_{22} & \cdots & a_{2n} & b_2 \\
    \vdots & \vdots & \ddots & \vdots & \vdots \\
    a_{m1} & a_{m2} & \cdots & a_{mn} & b_m 
  }
\end{displaymath}
通过高斯消元,可以转化为如下形式:
\begin{equation} \label{eq-gauss-elim}
  \mmat{cccccc|c}{
    c_{11} & c_{12} & \cdots & c_{1r} & \cdots & c_{1n} & c_1    \\
           & c_{22} & \cdots & c_{2r} & \cdots & c_{2n} & c_2    \\
           &        & \ddots & \vdots &        & \vdots & \vdots \\
           &        &        & c_{rr} & \cdots & c_{rn} & c_r    \\
           &        &        &        &        &        & c_{r+1}
  }
\end{equation}
其中$c_{ii}\neq 0\ (i=1,\dots,r)$。
当$c_{r+1}=0$时,方程组有解。
当$c_{r+1}\neq 0$时,方程组无解。
因此有如下结论:

\begin{theorem}[方程组有解的判定条件]
  线性代数方程组$\mmA\mvx=\mvb$有解当且仅当
  \[ r(\mmA) = r(\overline{\mmA}) \]
  若有解:
  \begin{enumerate}
    \item 若$r(\mmA)=n$,则有唯一解。
    \item 若$r(\mmA) < n$,则有无穷多解。
  \end{enumerate}
\end{theorem}

\begin{remark}
  对于齐次方程组$\mmA\mvx=\mvZero$,
  必有$r(\mmA)=r(\overline{\mmA})$,
  因此齐次方程组一定有解。
  当$r(\mmA)=n$时,则只有零解。
  当$r(\mmA)\neq n$时,则有无穷多解。
\end{remark}

\subsection{线性代数方程组解的结构}
经过高斯消元得到的式\ref{eq-gauss-elim}如果有解(即$c_{r+1}=0$),
那么可以进一步转化为
\begin{equation} \label{eq-gauss-elim-further}
  \mmat{ccccccc|c}{
    1 &        &        &   & d_{r1}   & \cdots & d_{r1} & d_1    \\
      & 1      &        &   & d_{r2}   & \cdots & d_{r2} & d_2    \\
      &        & \ddots &   & \vdots   & \ddots & \vdots & \vdots \\
      &        &        & 1 & d_{rr+1} & \cdots & d_{rn} & d_r
  }
\end{equation}
方程的解应该是一目了然的了。
据此,我们开始讨论线性代数方程组解的结构。
首先是齐次线性方程组。

\begin{definition}[基础解系]
    齐次线性方程组$\mmA\mvx=\mvZero$的解向量组(构成解空间)的一个极大线性无关组
    叫做它的一个\textbf{基础解系}。
\end{definition}

\begin{theorem}
    若齐次线性方程组的系数矩阵$\mmA$的秩小于$n$,
    则方程组必有基础解系。且基础解系所含解的个数等于$n-r$。
\end{theorem}

\begin{remark}
  把式\ref{eq-gauss-elim-further}中$d_1,\dots,d_r$设为0,
  则有如下形式的解:
  \begin{displaymath}
  \meqs{rcl}{
    x_1 &=& \xi_{11}x_{r+1}+\xi_{12}x_{r+2}+\cdots+\xi_{1n-r}x_{n} \\
    x_2 &=& \xi_{21}x_{r+1}+\xi_{22}x_{r+2}+\cdots+\xi_{2n-r}x_{n} \\
    & \vdots & \\
    x_r &=& \xi_{r1}x_{r+1}+\xi_{r2}x_{r+2}+\cdots+\xi_{rn-r}x_{n} \\
  }
  \end{displaymath}
  由此得到一个基础解系为
  \begin{displaymath}
  \meqs{rcl} {
    \xi_1 &=& (\xi_{11}, \xi_{21}, \dots, \xi_{r1}, 1, 0, \dots, 0)^T \\
    \xi_2 &=& (\xi_{12}, \xi_{22}, \dots, \xi_{r2}, 0, 1, \dots, 0)^T \\
    & \vdots & \\
    \xi_{n-r} &=& (\xi_{1n-r}, \xi_{2n-r}, \dots, \xi_{rn-r}, 0, 0, \dots, 1)^T
  }
  \end{displaymath}
  方程组任何解$\xi$都可以由$\xi_1,\dots,\xi_{n-r}$线性表示。
\end{remark}

接下来是非齐次线性方程组的解的结构。
设非齐次线性方程组为$\mmA\mvx=\mvb$。
它对应的齐次线性方程组为$\mmA\mvx=\mvZero$,
称为\textbf{导出组}。它们解的结构有着密切的联系:
\begin{enumerate}
  \item
  若$\eta_1,\eta_2$是$\mmA\mvx=\mvb$的解,
  那么$\eta_1-\eta_2$是$\mmA\mvx=\mvZero$的解。
  \item
  若$\eta$是$\mmA\mvx=\mvb$的解,$\xi$是$\mmA\mvx=\mvZero$的解,
  那么$\eta+\xi$是$\mmA\mvx=\mvb$的解。
\end{enumerate}

\begin{theorem}
  设$\eta_0$是$\mmA\mvx=\mvb$的一个解,那么它任意解$\eta$都能表示为
  \[ \eta = \eta_0 + k_1\xi_1 + \cdots + k_{n-r}\xi_{n-r} \]
  其中$\xi_1,\dots,\xi_{n-r}$是它导出组$\mmA\mvx=\mvZero$的基础解系,
  $k_1,\dots,k_{n-r} \in \mfR$。
\end{theorem}
