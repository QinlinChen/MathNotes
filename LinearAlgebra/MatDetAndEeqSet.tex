\chapter{矩阵\ 行列式\ 线性方程组}

\section{矩阵及其运算}

\subsection{矩阵的概念}
\begin{definition}[矩阵]
  由$m\times n$个数排成$m$行$n$列的矩形数表
  \begin{equation} \label{eq:mat-def}
    \mmA = \mmat{cccc}{
      a_{11} & a_{12} & \cdots & a_{1n} \\
      a_{21} & a_{22} & \cdots & a_{2n} \\
      \vdots & \vdots & \ddots & \vdots \\
      a_{m1} & a_{m2} & \cdots & a_{mn} }
  \end{equation}
  称为一个$m\times n$矩阵。
  其中数$a_{ij}$称为矩阵$\mmA$的元素。
  $i$为行标,$j$为列标,
  $a_{ij}$是$\mmA$位于第$i$行第$j$列的元素,或简称$\mmA$的$(i,j)$元素。
  式\eqref{eq:mat-def}可简记为$\mmA=(a_{ij})_{m\times n}$或$(a_{ij})$。
\end{definition}

下面是一些特殊的矩阵:
\begin{itemize}
  \item
  $m\times 1$矩阵又称为\textbf{列矩阵}或$m$\textbf{维列向量},
  下文主要会用$\beta$来表示。
  \item
  $1\times n$矩阵又称为\textbf{行矩阵}或$n$\textbf{维行向量},
  下文主要会用$\alpha$来表示。
  \item
  $n\times n$矩阵又称为$n$\textbf{阶方阵}。
  对于$n$阶方阵$\mmA=(a_{ij})$,
  其对角线上的元素$a_{11},\dots,a_{nn}$又称\textbf{主对角线元素}。
  \item
  如果一个$n$阶方阵除主对角线元素外,其余元素都为0,
  那么我们称这种矩阵为\textbf{对角矩阵},记为
  \[ \mdiag(k_1, k_2\dots,k_n)= \mmat{cccc}{
    k_1 &     &        & \\
        & k_2 &        & \\
        &     & \ddots & \\
        &     &        & k_n } \]
  \item
  $\mmI = \mdiag(1,1,\dots,1)$被称为\textbf{单位矩阵}。
\end{itemize}

\begin{definition}[矩阵相等]
    如果$\mmA=(a_{ij})$与$\mmB=(b_{ij})$都是$m\times n$矩阵,
    且对$i=1,2,\dots,m,\ j=1,2,\dots,n$,有$a_{ij}=b_{ij}$,
    则称$\mmA$与$\mmB$相等,记作$\mmA=\mmB$。
\end{definition}

\subsection{矩阵的线性运算}
矩阵的线性运算包含\textbf{加法}和\textbf{数乘}。定义比较简单,在此不加赘述。

它们满足的性质与之后要定义的线性空间一致。

\subsection{矩阵的乘法}
\begin{definition}[矩阵的乘法]
  设$\mmA=(a_{ik})_{m\times s}, \mmB=(b_{kj})_{s\times n}$,令
  \begin{equation}
  c_{ij}=\sum_{k=1}^{s}a_{ik}b_{kj}
  \end{equation}
  则$\mmC=(c_{ij})_{m\times n}$被称为$\mmA$和$\mmB$的乘积,
  记为$\mmC = \mmA\mmB$。
\end{definition}

矩阵的乘法满足分配率和结合律,但不满足交换律。

\subsection{矩阵的转置}
\begin{definition}[矩阵的转置]
    设$\mmA=(a_{ij})_{m\times n}$,
    令$b_{ij}=a_{ji}$,
    则$\mmB=(b_{ij})_{n\times m}$被称为$\mmA$的转置矩阵,
    记为$\mmA^T$。
\end{definition}

转置矩阵具有以下性质:
\begin{enumerate}
  \item $\left(\mmA^T\right)^T = \mmA$
  \item $(\mmA+\mmB)^T=\mmA^T + \mmB^T$
  \item $(k\mmA)^T=k\mmA^T$
  \item $(\mmA\mmB)^T=\mmB^T\mmA^T$
\end{enumerate}

\begin{definition}[对称矩阵与反对称矩阵]
  对于$n$阶方阵$\mmA=(a_{ij})$,
  如果满足$\mmA^T=\mmA$,则称$\mmA$为\textbf{对称矩阵}。
  如果满足$\mmA^T=-\mmA$,则称$\mmA$为\textbf{反对称矩阵}。
\end{definition}

\begin{remark}
  不难看出,反对称矩阵的主对角线元素都为零。
\end{remark}

\section{矩阵的分块}
TODO