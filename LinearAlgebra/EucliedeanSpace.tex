\chapter{内积空间与正交性}

\section{内积空间}
我们知道线性空间是对向量的抽象,而向量还有内积的概念没有出现在线性空间中。
所以,本小节在线性空间的基础上引入内积操作来定义内积空间,
从而建立了长度、交角、正交性的概念。
根据线性空间数域的不同,我们会在有限维上分别讨论欧几里得空间和酉空间。

\subsection{实内积空间与欧几里得空间}
\begin{definition}[实内积空间与欧几里得空间]
  设$V$是实数域$\mfR$上的一个线性空间。
  对于任意$\alpha,\beta\in V$,
  定义一个返回实数的运算$(\alpha,\beta)$,满足:
  \begin{enumerate}
    \item 对称性:
    $(\alpha,\beta)=(\beta,\alpha)$
    \item 线性性:
    $(k\alpha,\beta)=k(\alpha,\beta),\quad
      (\alpha+\beta,\gamma)=(\alpha,\gamma)+(\beta,\gamma)$
    \item 正定性:
    $(\alpha,\alpha)\ge 0$。
    当且仅当$\alpha=\theta$时,$(\alpha,\alpha)=0$
  \end{enumerate}
  其中$\alpha,\beta,\gamma\in V, k\in\mfR$。
  我们称$(\alpha,\beta)$为$\alpha,\beta$的\textbf{内积},
  $V$为\textbf{实内积空间}。
  有限维的实内积空间又称\textbf{欧几里得空间}。
\end{definition}

\begin{theorem}[实内积空间的性质] \label{thrm-real-inner-prod-space-prop}
  设$V$是实内积空间。对于$\alpha,\beta,\gamma\in V, k\in\mfR$,有
  \begin{enumerate}
    \item \label{thrm-RIPS-prop1}
    $(\alpha,\beta+\gamma)=(\alpha,\gamma)+(\beta,\gamma)$
    \item
    $(\alpha,k\beta)=k(\alpha,\beta)$
    \item
    $(\theta,\beta)=0=(\alpha,\theta)$
    \item
    $(\alpha,\beta)=
      \left(\sum_{i=1}^{m}a_i\alpha_i,\sum_{j=1}{n}b_j\beta_j\right) $
    \begin{equation} \label{eq-metric-mat}
      = (a_1,a_2,\dots,a_m) \mmat{cccc}{
        (\alpha_1,\beta_1)&(\alpha_1,\beta_2)&\cdots&(\alpha_1,\beta_n)\\
        (\alpha_2,\beta_1)&(\alpha_2,\beta_2)&\cdots&(\alpha_2,\beta_n)\\
        \vdots            &\vdots            &\ddots&\vdots            \\
        (\alpha_m,\beta_1)&(\alpha_m,\beta_2)&\cdots&(\alpha_m,\beta_n)}
        \mmat{c}{b_1 \\ b_2 \\ \vdots \\ b_n}
    \end{equation}
  \end{enumerate}
\end{theorem}

\begin{definition}[长度]
  设$V$是实内积空间。对于$\alpha\in V$,
  $\| \alpha \| = \sqrt{(\alpha,\alpha)}$
  称为$\alpha$的\textbf{长度}或\textbf{模}。
\end{definition}

\begin{theorem}[柯西-施瓦兹(Cauchy-Schwartz)不等式]
  设$V$是实内积空间。对于任意$\alpha,\beta\in V$,有
  \begin{displaymath}
    |(\alpha,\beta)| \le \|\alpha\|\cdot\|\beta\|
  \end{displaymath}
  等号仅在$\alpha,\beta$线性相关时成立。
\end{theorem}

\begin{definition}[交角]
  实内积空间中的两个非零向量$\alpha,\beta$的交角$\theta$定义为
  \begin{displaymath}
    \theta = \arccos\frac{(\alpha,\beta)}{\|\alpha\|\cdot\|\beta\|},
    \quad  0\le\theta\le\pi 
  \end{displaymath}
\end{definition}

\begin{definition}[正交]
  设$\alpha,\beta$是实内积空间中的两个向量。
  若$(\alpha,\beta)=0$,则称$\alpha,\beta$\textbf{正交},
  记作$\alpha\perp\beta$。
\end{definition}

\begin{remark}
  零向量与任意向量正交。
\end{remark}

\begin{theorem}[正交的充要条件]
  设$\alpha,\beta$是实内积空间中的两个向量,则
  \begin{displaymath}
    \alpha\perp\beta \iff \|\alpha+\beta\|^2=\|\alpha\|^2+\|\beta\|^2
  \end{displaymath}
\end{theorem}

\subsection{复内积空间与酉空间}
\begin{definition}[复内积空间与酉空间]
  设$V$是复数域$\mfC$上的一个线性空间。
  对于任意$\alpha,\beta\in V$,
  定义一个返回复数的运算$(\alpha,\beta)$,满足:
  \begin{enumerate}
    \item 共轭对称性:
    $(\alpha,\beta)=\mconj{(\beta,\alpha)}$
    \item 线性性:
    $(k\alpha,\beta)=k(\alpha,\beta),\quad
      (\alpha+\beta,\gamma)=(\alpha,\gamma)+(\beta,\gamma)$
    \item 正定性:
    $(\alpha,\alpha)$是非负实数。
    当且仅当$\alpha=\theta$时,$(\alpha,\alpha)=0$
  \end{enumerate}
  其中$\alpha,\beta,\gamma\in V, k\in\mfC$。
  我们称$(\alpha,\beta)$为$\alpha,\beta$的\textbf{内积},
  $V$为\textbf{复内积空间}。
  有限维的复内积空间又称\textbf{酉空间}。
\end{definition}

\begin{theorem}[复内积空间的性质]
  复内积空间满足定理\ref{thrm-real-inner-prod-space-prop}列出的
  除了性质\ref{thrm-RIPS-prop1}以外的所有性质。
  性质\ref{thrm-RIPS-prop1}应该改写为:
  \[ (\alpha,k\beta)=\mconj{k}(\alpha,\beta) \]
\end{theorem}

复内积空间的长度、正交性同实内积空间定义。
柯西-施瓦兹不等式依旧成立。
只有交角要重新定义为
\begin{displaymath}
  \theta = \arccos\frac{|(\alpha,\beta)|}{\|\alpha\|\cdot\|\beta\|},
  \quad 0\le\theta\le\frac{\pi}{2}
\end{displaymath}

下面是复矩阵相关的三个概念。
只用了解一下,不会深入讨论。

\begin{definition}[共轭矩阵]
  设$\mmA$是$n$阶复矩阵。
  把$\mmA$的元素都取共轭得到的矩阵叫做\textbf{共轭矩阵},
  记作$\mconj{\mmA}$。
\end{definition}

\begin{definition}[酉矩阵]
  设$\mmA$是$n$阶复矩阵。
  若$\mmA$满足$\mconj{\mmA^\mT}\mmA=\mmA\mconj{\mmA^\mT}=\mmI$,
  则称$\mmA$是\textbf{酉矩阵}。
\end{definition}

\begin{remark}
  酉矩阵是对正交矩阵的推广。正交矩阵会在下面讨论。
\end{remark}

\begin{definition}[厄米特(Hermite)矩阵]
  设$\mmA$是$n$阶复矩阵。
  若$\mmA$满足$\mconj{\mmA^\mT}=\mmA$,
  则称$\mmA$是\textbf{厄米特(Hermite)矩阵}
\end{definition}

\begin{remark}
  厄米特矩阵是对对称矩阵的推广。
\end{remark}

\section{正交性}

\subsection{欧几里得空间的标准正交基}
\begin{definition}[度量矩阵]
  设$V$是$n$维欧几里得空间,$\alpha_1,\dots,\alpha_n$是一组基底。
  对任意$\alpha,\beta\in V$,
  有$\alpha=\mmBasis{\alpha}\mvx$,$\beta=\mmBasis{\alpha}\mvy$。
  根据式\ref{eq-metric-mat},我们有
  \begin{displaymath}
    (\alpha,\beta) = \mvx^\mT\mmA\mvy
  \end{displaymath}
\end{definition}
